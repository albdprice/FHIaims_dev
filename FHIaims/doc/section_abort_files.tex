\section{Stopping a run: Files \texttt{abort\_scf} and \texttt{abort\_opt}}
\label{section:abort}

Sometimes, you may wish to stop a running FHI-aims calculation
prematurely, but in an organized way.

Of course, with any running instance of FHI-aims, there is always the
option to stop a run by invoking the Unix 'kill' command on every
single running 'aims' process, and this will normally end the run
right where it is.

To obtain a slightly more civilized stop (to allow the code to finish
in a defined location and stop after writing some more output), you
may instead create one of two specific files:
\begin{enumerate}
  \item \keyword{abort\_scf}
  \item \keyword{abort\_opt}
\end{enumerate}
The code simply checks for the existence of either of these files
periodically. No input is needed. Thus, simply change to the directory
in which the code is running, and type (at the command line)
\begin{verbatim}
  touch abort_scf
\end{verbatim}
or
\begin{verbatim}
  touch abort_opt
\end{verbatim}
After a while, the run will stop.

The existence of \keyword{abort\_scf} will stop the code after the
current s.c.f. \emph{iteration} is finished, i.e., the solution of the
Kohn-Sham equations will not be self-consistent even for the present
geometry.

The existence of \keyword{abort\_opt} will stop the code after the
current s.c.f. \emph{cycle} is converged during a geometry relaxation,
i.e., the electronic structure will be converged for the present
geometry, but the forces will not be zero.

In either case, the stop of FHI-aims will not happen
immediately. Depending on the nature of the run, it may take quite
some time until the 'abort' takes effect, since the code needs to
reach the appropriate state first. If you are interested in an
immediate stop, the Unix 'kill' command is still your best bet.

One can also envision numerous refinements or alternative
scenarios where an 'abort' file could be useful. If you really need
such a case, please create the appropriate check where you need it. If
it does the trick for you, we will be happy to incorporate the change
into the mainline version of FHI-aims.

