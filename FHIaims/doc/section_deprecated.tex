\section{Deprecated keywords}

The following section lists a number of keywords in FHI-aims which
exist, but which may go away in future versions of FHI-aims. In some
cases, this is because the relevant modifications proved successful,
and there is no sense in maintaining some old, obsolete extra
functionality without any use in production settings. In other cases,
the relevant keywords were experiments that did not yield the
anticipated success, and / or functionality that may be superseded in
a different, more comprehensive way in the future.

\newpage

\subsection*{Tags for general section of \texttt{control.in}:}

\keydefinition{Adams\_Moulton\_integrator}{control.in}
{
  \noindent
  Usage: \keyword{Adams\_Moulton\_integrator} \option{flag} \\[1.0ex]
  Purpose: Allows to switch between a simple integrator and the
    higher-order Adams-Moulton integration scheme to determine the
    Hartree potential components from classical
    electrostatics. \\[1.0ex] 
  \option{flag} is a logical string, either \texttt{.false.} or
    \texttt{.true.} Default: \texttt{.true.} \\
}

\keydefinition{batch\_distribution\_method}{control.in}
{
  \noindent
  Usage: \keyword{batch\_distribution\_method} \option{method} \\[1.0ex]
  Purpose: Parallel distribution of integration grid batches
    \emph{only} in the case that the external qhull and METIS
    libraries are configured. \\[1.0ex]
  \option{method} is a string, the only possible value being
    \texttt{qhull+metis} at this point. \\
}
Outsources the distribution of integration grid batches to the
external qhull and METIS libraries. Only relevant if these libraries
are compiled into the code. However, the associated
\keyword{grid\_partitioning\_method}s are less useful than the
default \texttt{maxmin} algorithm, and the internal work distribution
method of FHI-aims usually performs rather well. Therefore, this
option is deprecated and kept only for experimental purposes, for
now. 


\keydefinition{communication\_type}{control.in}
{
  \noindent
  Usage: \keyword{communication\_type} \option{type} \\[1.0ex]
  Purpose: Determines the type of calculation / storage of
    per-atom spline arrays of the Hartree potential for a parallel
  run. \\[1.0ex] 
  \option{type} is a string, see below. Default: \option{calc\_hartree} . \\
}
In a parallel run of FHI-aims, each processor holds a certain part of the
real-space integration grid, which in turn are each touched by all
atom-centered multipole components (splined) of the real-space Hartree
potential. So, to construct the electrostatic (Hartree) potential on each grid
point, an array of splined atom-centered multipole components
$\delta\tilde{v}_{\text{at},lm}(r)$ must be available on every MPI sub-process
(see Sec. \ref{Sec:Hartree} for details regarding the electrostatic
potential). The memory use to store these components grows rapidly and with a
large prefactor with system size. Thus, keeping a local copy of all the
splined multipole components of the Hartree potential on each CPU is not
advisable.

The actual handling of these components is instead controlled
by keyword \keyword{communication\_type}. The following choices for
  \texttt{option} are possible: \\
\texttt{calc\_hartree} : The default, and usually very efficient. Hartree
  potential components for each atom are integrated on the fly on each CPU when
  needed (usually less CPU time than communication time). \\
\texttt{shmem} : If compiled with shared memory support (see Section
  \ref{Sec:Makefiles}), each compute node of a parallel run keeps the
  components of the Hartree potential in a separate shared memory segment,
  only internode communication is needed.  Performance test show hardly any
  benefet over \texttt{calc\_hartree}.
Note that the (legacy) keyword \keyword{distributed\_spline\_storage} should
be false for \texttt{shmem}, at least.

\keydefinition{distribute\_leftover\_charge}{control.in}
{
	\noindent
	Usage: \keyword{distribute\_leftover\_charge} \option{.true./.false.}
%	Mostly relevant for surfaces, in particular when calculating
%	electrostatic potentials, interface dipoles, or work functions.
%	Due to the finite summation in the multipole expansion, the electronic and the
%	nuclear charge may not cancel each other exaclty. The remaining, apparent charge
%	of the system is plotted in the output as
%	\textit{RMS charge density error from multipole expansion: (value)}
%	and causes a slightly curved potential in vacuum region. As a result, e.g., the
%	energy of the vacuum level depends on its positon (which can be changed using the tag
%	\keyword{set\_vacuum\_level}). If set to .true., this keyword distributes the apparent charge
%	evenly among all atoms in the system, causing a perfectly flat potential in the vacuum region.
}
This keyword is superseded by the \keyword{compensate\_multipole\_errors} keyword which
solves the problem of small residual charge integration errors in slab calculations much
better than \keyword{distribute\_leftover\_charge} .

Keyword \keyword{distribute\_leftover\_charge} could introduce noticeable total energy 
inaccuracies especially for ``light'' integration grid settings.


\keydefinition{force\_new\_functional}{control.in}
{
  \noindent
  Usage: \keyword{force\_new\_functional} \option{flag} \\[1.0ex]
  Purpose: For test purposes, allows to switch to an energy functional
    form that treats the electronic and nuclear electrostatic energy
    terms separately. \\[1.0ex]
  \option{flag} is a logical string, either \texttt{.false.} or
    \texttt{.true.} Default: \texttt{.true.} \\
}
See Ref. \cite{Blum08} regarding the correct shape of the energy
functional that treats the nuclear and electronic parts of the
electrostatic energy together on a per-atom basis.

\keydefinition{force\_smooth\_cutoff}{control.in}
{
  \noindent
  Usage: \keyword{force\_smooth\_cutoff} \option{tolerance} \\[1.0ex]
  Purpose: Optionally, enforces smoothness of all basis functions near
    the cutoff radius. \\[1.0ex]
  \option{tolerance} is a small positive real number. Default: No
    check. \\
}
If requested, keyword \keyword{force\_smooth\_cutoff} ensures that the
radial function $u(r)$ and its first and second derivatives remain
below \option{tolerance} at the outermost point of the logarithmic
grid where any of them is non-zero at all. The code stops if the onset
of the radial function is too abrupt. 

It would be a good idea to switch this option on if reducing the 
\texttt{width} parameter of keyword \subkeyword{species}{cut\_pot} to a very low
value (say, below 1 {\AA}).

\keydefinition{grouping\_factor}{control.in}
{
  \noindent
  Usage: \keyword{grouping\_factor} \option{factor} \\[1.0ex]
  Purpose: Grouping factor for the (experimental, and not recommended!) 
    \texttt{group} \keyword{grid\_partitioning\_method}. \\[1.0ex]
  \option{factor} is an integer number, describing how close-by grid
    points are grouped together. Default: 2. \\
}
This keyword is retained for experimental purposes only, for the
moment. Since the related \keyword{grid\_partitioning\_method}
\texttt{group} was a proof-of-concept to show that the default
\texttt{maxmin} performs better, this keyword is now deprecated. See
Ref. \cite{Havu08} for more details if interested.

\keydefinition{hartree\_worksize}{control.in}
{
  \noindent
  Usage: \keyword{hartree\_worksize} \option{megabytes} \\[1.0ex]
  Purpose: Limits the size of workspace arrays used to construct the
    Hartree potential on each CPU. \\[1.0ex]
  \option{megabytes} : The maximum allowed work space size to
    construct the Hartree potential, in megabytes. Default: 200
  MB. \\
}
Several large work space arrays across the integration grid are used
in the construction of the Hartree potential. Their size can grow
quite large, especially when forces are computed for large structures
(then, three arrays per atom are required for all atoms).

FHI-aims can circumvent this by computing the final output (integrated
energies and forces) in ``chunks'' of the whole integration grid,
limiting the work space used for each chunk. This modification is
especially important on low-memory-per-processor architectures such as
the BlueGene.

\keydefinition{KH\_post\_correction}{control.in}
{
  \noindent
  Usage: \keyword{KH\_post\_correction} \option{flag} \\[1.0ex]
  Purpose: \emph{Under construction -- do not use} A way to replace
    the scaled ZORA post-processing correction for scalar relativity
    by a Koelling-Harmon type scalar-relativistic
    correction. \\[1.0ex]
  \option{flag} is a logical string, either \texttt{.false.} or
    \texttt{.true.} Default: \texttt{.false.} \\
}
This keyword is no longer supported, do not use it. It will be
superseded by an improved handling of scalar relativity during the
s.c.f. cycle in the future. 

\keydefinition{mixer\_swap\_boundary}{control.in}
{
  Usage: \keyword{mixer\_swap\_boundary} \option{bytes} \\[1.0ex]
  Purpose: Ignored; never swap.
  Used to allow to swap the stored density components
  for Pulay mixing to disk if they exceed a certain memory boundary. \\[1.0ex] 
}
On few-CPU systems and for mid-sized systems (several hundred atoms),
the stored electron density components from past iterations are a 
large part of the memory used. 
If this becomes a bottleneck, the stored Pulay arrays can in principle be
swapped to disk, instead, to be read only during Pulay mixing.

\emph{If anyone has a strong need for this currently unsupported
  feature, please contact us.}


\keydefinition{multiplicity}{control.in}
{
 \noindent
 Usage: \keyword{multiplicity} \option{value} \\[1.0ex]
 Purpose: If set, specifies the multiplicity of the system. \\[1.0ex]
 Restriction: Currently available for non-periodic geometries
   only. Use \keyword{fixed\_spin\_moment} instead. \\[1.0ex] 
  \option{value} : integer number, sets the overall multiplicity as
 $2S+1$. \\ 
} 
Meaningful only in the spin-polarized case (\texttt{spin collinear} in
\texttt{control.in}). On a technical level, this is a special case of
the more general, locally spin-constrained DFT formalism available
within FHI-aims (see Sec. \ref{Sec:constraint}). Note that the underlying
\keyword{constraint\_electrons} keyword can be used to enforce a
\emph{non-integer} fixed spin moment, in addition to allowing to fix
electron or spin numbers in given subsystems  

Also, be sure to check what the Kohn-Sham eigenvalues mean if you need
them, do not use them blindly. \keyword{multiplicity} shifts the
eigenvalues! 

\keydefinition{occupation\_thr}{control.in}
{
  \noindent
  Usage: \keyword{occupation\_thr} \option{value} \\[1.0ex]
  Purpose: Any occupation numbers below \option{value} will be treated
    as zero.  
  \option{value} is a small positive real number. Default: 0.d0 .
}

\keydefinition{recompute\_batches\_in\_relaxation}{control.in}
{
  \noindent
  Usage: \keyword{recompute\_batches\_in\_relaxation} \option{flag}
    \\[1.0ex]
  Purpose: Allows to switch off the redistribution of atom-centered
    grid points into new integration batches after a relaxaton
    step. \\[1.0ex] 
  \option{flag} is a logical string, either \texttt{.false.} or
    \texttt{.true.} Default: \texttt{.true.} \\
}
For practical purposes, the integration grid should always be
repartitioned after a relaxation step; the associated overhead is low,
and the shape of the batches will remain optimal in the face of
individual points that ``move'' with different atoms. 

\keydefinition{squeeze\_memory}{control.in}
{
  \noindent
  Usage: \keyword{squeeze\_memory} \option{flag} \\[1.0ex]
  Purpose: Used to allow one combined workspace for three different
    purposes. \\[1.0ex]
  This option is no longer necessary due to optimizations by Rainer
  Johanni.
  \option{flag} is a logical string, either \texttt{.false.} or
  \texttt{.true.} Default: \texttt{.false.} \\  
}


\keydefinition{use\_angular\_division}{control.in}
{
  \noindent
  Usage: \keyword{use\_angular\_division} \option{flag} \\[1.0ex]
  Purpose: If radial grid shells are used as integration batches,
    allows to switch off their subdivision into ``octant''
    batches. \\[1.0ex] 
  \option{flag} is a logical string, either \texttt{.false.} or
    \texttt{.true.} Default: \texttt{.true.} \\
}
This flag currently only applies to the initialization iteration, in
case that self-adapting angular grids are used (not recommended; see
keyword \subkeyword{species}{angular\_acc} if interested). Even then, switching
off the subdivision of radial shells can only decrease the
performance. 


\newpage

\subsection*{Subtags for \emph{species} tag in \texttt{control.in}:}

\subkeydefinition{species}{cut\_core}{control.in}
{
  \noindent
  Usage: \subkeyword{species}{cut\_core} \option{type} [\option{radius}] \\[1.0ex]
  Purpose: Can be used to define a separate (tighter) onset of the
    cutoff potential for all \subkeyword{species}{core} radial functions. \\[1.0ex]  
  \option{type} : A string, either \texttt{finite} or
    \texttt{infinite}. Default: \texttt{finite} . \\
  \option{radius} : A real number, in {\AA}: Onset radius for the
    cutoff potential, as defined in the \subkeyword{species}{cut\_pot}
    tag. Default: same as \texttt{onset} in \subkeyword{species}{cut\_pot}.\\ 
}
Deprecated flag because \subkeyword{species}{basis\_dep\_cutoff} should
supersede this functionality in a more organized way. Having a
separate, tighter cutoff for core radial functions sounds like a good
idea, but core radial functions are already rather localized
anyway. Our experience is that the separate core setting either does
not matter for CPU time, or already introduces noticeable total energy
changes when it matters.
