\section{Molecular dynamics}
\label{Sec:examples-MD}

This section describes a short set of example runs for molecular dynamics.

In general, molecular dynamics aims to reproduce the statistical
averages of an ensemble of molecules, solids, etc. of some kind, for
specific environmental conditions (isolation, contact with a bath,
constant temperature, pressure, etc.).

This means that `molecular dynamics' is really not a singke technique,
but rather a collection of possible simulation goals -- i.e., not
black box. To understand which specific input parameters etc. are
needed for a given case, a good physical understanding of the intended
outcome is required.

Thus, ``molecular dynamics'' in FHI-aims offers many different
options, and we can not give examples for the use of every single
conceivable combination -- that is the subject for text books. We will
try to give a few simple examples, though.

... Integrators: Verlet vs higher order, and why Verlet is wat we
support in practice (but please feel free to play with higher order
and implement missing functionality if needed)

... What thermostats exists

\subsection*{Constant energy run, classical nuclei (microcanonical ensemble)}

Non-periodic.

Molecule.

Describe choice of defaults made, in particular cleaning of velocities.

\subsection*{Thermostatted run, classical nuclei (canonical ensemble)}

Non-periodic.

Molecule.

Describe choice of defaults made, in particular NO cleaning of velocities.

