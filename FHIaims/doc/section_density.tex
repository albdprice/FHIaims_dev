\section{Electron density update}
\label{Sec:density_update}

In FHI-aims, the first step of a new iteration is the update of the
electron density based on the output Kohn-Sham orbitals produced by a
previous step. 

The present section covers only the \emph{actual}
density update. Techniques relevant for the self-consistent
\emph{convergence} of the whole calculation (electron density mixing,
preconditioning, etc.) are covered separately in Sec. \ref{Sec:scf}.

\newpage

\subsection*{Tags for general section of \texttt{control.in}:}

\keydefinition{density\_update\_method}{control.in}
{
  \noindent
  Usage: \keyword{density\_update\_method} \option{type} \\[1.0ex]
  Purpose: Governs the selection of the density update
  type.\\[1.0ex]
  Restriction: For periodic boundary conditions, only the density-matrix 
    based electron density update is supported. \\[1.0ex] 
  Default: Cluster case: \texttt{automatic}. 
    Periodic case: \texttt{density\_matrix} \\ 
}

Choices for \option{type}:
\begin{itemize}
  \item \texttt{orbital} : Use Kohn-Sham orbitals based update
  \item \texttt{density\_matrix} : Use density-matrix based update
    method. Required for periodic systems. 
  \item \texttt{automatic} : Selects the best
    update method automatically, based on the expected amount of
    work. 
  \item \texttt{split\_update\_methods} : Charge density is
     updated via Kohn-Sham orbitals and force is updated via density-matrix
\end{itemize} 
If not specified, default for cluster geometries is the automatic selection of the
density update method.

See Ref. \cite{Blum08} for details regarding density update
mechanisms. In general, FHI-aims offers an electron density update
based on Kohn-Sham orbitals [$O(N^2)$ with system size, but faster for
finite systems up to $\approx$100-500 atoms depending on basis
size], and an $O(N)$ rewrite based on the density matrix. This should
be used for large systems, and is the default for periodic systems.

For the non-periodic case, the current code version determines the
switching point between the orbital-based update and the
density-matrix based update automatically through some
heuristics. This procedure guarantees that accidental $O(N^2)$
calculations will not happen for very large systems, but the
optimum cross-over point may not always be exactly found. If you are
planning long runs of essentially the same geometry (molecular
dynamics trajectories are a good example), you may save some time by
performing some explicit benchmarks first. You can then specify the
optimum density update method for \emph{your} own case, instead of
relying on our heuristics.
