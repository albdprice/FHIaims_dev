\section{Embedding in external fields}

To simulate the effect of external field (for instance, to connect to
a QM/MM embedding formalism), FHI-aims allows to add the effect of
a homogeneous electrical field and/or point charges surrounding the
molecule in question.

Note that these embedding charges are \emph{in addition} to any
\keyword{charge} specified in \texttt{control.in}, and \emph{not} already
included there. \keyword{charge} should equal only to the sum of charges of all
nuclei in \texttt{geometry.in} minus the overall number of electrons in the
system, but does not count any embedding charges specified by keyword
\keyword{multipole}. 

\emph{This functionality is not yet available for periodic systems.}

\textbf{Warning: When using a multipole, e.g., an external charge with
  no basis functions etc., you are creating a Coulomb singularity. 
  If this singularity is inside the radius of a basis function of
  another atom, it will lead to numerical noise in integrals, up to
  near-infinities.} 

To test and/or overcome this problem, all you need to do is to place
an integration grid on any multipole that is within the radius of a
basis function of any atom. This radius is given by the cutoff radius
plus width in the \subkeyword{species}{cut\_pot} keyword of each
\keyword{species}. 

Such a grid can be placed by creating an \keyword{empty} site with no basis
functions (\subkeyword{species}{include\_min\_basis} .false.) and placing this empty
site on the same site as the multipole in question in
\texttt{geometry.in}. Simply taking the species defaults for a H atom
(light settings) and adjusting them to have no basis functions should
create the necessary definition of the empty site in question (see Fig.\ref{Fig:empty_site_for_multipole} for an example).

\begin{figure}[hb]
  \small
  \begin{verbatim}


[...]
  species        empty_site
#     global species definitions
    nucleus             1
    mass                1.00794
#
    l_hartree           4
#
    cut_pot             3.5  1.5  1.0
    basis_dep_cutoff    1e-4
#     
    include_min_basis .false.

    radial_base         24 5.0
    radial_multiplier   1
    angular_grids       specified
      division   0.2421   50
      division   0.3822  110
      division   0.4799  194
      division   0.5341  302
      outer_grid  302
################################################################################
#
#  Definition of "minimal" basis
#
################################################################################
#     valence basis states
    valence      1  s   1.
#     ion occupancy
    ion_occ      1  s   0.5
################################################################################




[...]

  \end{verbatim}
  \normalsize

  \vspace*{-4.0ex}

  \caption{\label{Fig:empty_site_for_multipole}
Species data for what could be used as an empty site on top of monopole.
  }
\end{figure}

\newpage

\subsection*{Tags for \texttt{geometry.in}:}

\keydefinition{homogeneous\_field}{geometry.in}
{
  \noindent
  Usage: \keyword{homogeneous\_field} \option{E\_x} \option{E\_y}
    \option{E\_z} \\[1.0ex]  
  Purpose: Allows to perform a calculation for a system in a
    homogeneous electrical field $\boldE$. \\[1.0ex]
  \option{E\_x} is a real number, the $x$ component of $\boldE$ in
    V/{\AA}. \\
  \option{E\_y} is a real number, the $y$ component of $\boldE$ in
    V/{\AA}. \\
  \option{E\_z} is a real number, the $z$ component of $\boldE$ in
    V/{\AA}. \\

   Please note: The electrical field is usually defined to point in the direction
   of a force exerted on a \emph{positive} probe charge. Historically grown, FHIaims
   uses the opposite sign convention. Although this behaviour might
 be considered a bug, we decided to leave it this way in order not to break any scripts
people are already using.

}

\keydefinition{multipole}{geometry.in}
{
  \noindent
  Usage: \keyword{multipole} \option{x} \option{y} \option{z}
    \option{order} \option{charge} \\[1.0ex] 
  Purpose: Places the center of an electrostatic multipole field at a
    specified location, to simulate an embedding potential. \\[1.0ex]
  \option{x} : $x$ coordinate of the multipole. \\
  \option{y} : $y$ coordinate of the multipole. \\
  \option{z} : $z$ coordinate of the multipole. \\
  \option{order} : Integer number, specifies the order of the
    multipole (0 or 1 $\equiv$ monopole or dipole). \\
  \option{charge} : Real number, specifies the charge associated with
    the multipole. \\
}
If the order of the multipole is greater than zero (presently, only
 monopoles or dipoles are supported), a dipole moment must be 
specified in addition to the data provided with the
\keyword{multipole} tag itself. To that end, a line must
immediately follow the original \keyword{multipole} line, adhering to
the following format: \\[1.0ex]
  \texttt{data} \texttt{m\_x} \texttt{m\_y} \texttt{m\_z}
\\[1.0ex]
Here, \texttt{m\_x}, \texttt{m\_y}, \texttt{m\_z} are the $x$, $y$,
and $z$ components of the dipole moment, in $e\cdot${\AA}. 

\emph{Warning: Note that monopoles amount to Coulomb singularities. When
  inside the basis function radius of any atom, such monopoles should
  be covered with an integration grid in \texttt{geometry.in}, as
  explained in the beginning of this section.}

\newpage

\subsection*{Tags for general section of \texttt{control.in}:}

\keydefinition{full\_embedding}{control.in}
{
  \noindent
  Usage: \keyword{full\_embedding} \option{flag} \\[1.0ex]
  Purpose: Allows to switch between embedding of the full electronic
    structure (affecting the Kohn-Sham equations) or the embedding of
    an electronic density that is calculated without knowledge of the 
    embedding potential. \\[1.0ex]
  \option{flag} is a logical string, \texttt{.true.} or
    \texttt{.false.} Default: \texttt{.true.} \\
}
For most purposes, embedding into an external potential will involve a
change to the electronic structure of the structure which is
embedded. However, in some instances one may wish to embed a given
charge density \emph{non-selfconsistently}, i.e., by calculating the
electron density without an external field and then computing the
energy of that unperturbed electron density in the external field.

This feature is useful if multipoles are located too close
to the quantum-mechanical region of the calculation. These act as
Coulomb-like potentials, just like any other potential in the
systems. If there are basis functions that cover the location of that 
potential, some electronic charge may artificially become trapped
there, creating a bad approximation to the core / valence
electrons of an atom with a nucleus of charge \option{charge}. In most
cases, this is clearly undesirable behaviour, and apart from that will
create unwanted numerical noise since the electronic structure near
the Coulomb-like singularity of the multipole will be represented
solely by basis functions that are inadequate for this purpose in the
first place. 

\keydefinition{qmmm}{control.in}
{
  \noindent
  Usage: \keyword{qmmm} \\[1.0ex]
  Purpose: Allows to compute Hellmann-Feynman like forces \emph{from}
    the quantum-mechanical part of a structure \emph{exerted on} the
    multipoles on an external embedding field. \\[1.0ex]
  Restriction: Works only for external monopole potentials. \\[1.0ex]
}
For quantum-mechanics / molecular-mechanics (QM/MM) ``hybrid''
molecular dynamics simulations, one must evolve \emph{both} the
quantum and classical subsystems with time. In that case, it is
necessary to know the derivatives of the quantum-mechanical total
energy with respect to the positions of the \emph{classical}
multipoles (see keyword \keyword{multipole}), i.e., the forces on the
multipoles that originate from the quantum-mechanical region. 

The
computation of these forces is switched on by adding the
\keyword{qmmm} keyword to \texttt{control.in}. The actual QM/MM
simulation must still be performed using an outside framework, for
example ChemShell \cite{Sherwood03} that uses the energies and forces
provided by FHI-aims as a ``plugin''. 

