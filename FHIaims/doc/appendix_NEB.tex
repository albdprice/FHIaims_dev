% some keywords occurring here are the same as in the original AIMS,
% their labels should not be doubly defined!
\newcommand{\doublekeydefinition}[3]
{{\textbf{Tag: \texttt{#1}}{ \color{filename_color} (#2)}}
#3

}
\section{Transition state search: Nudged Elastic Band method}
\label{appendix_NEB}

\emph{The functionality described in the present section is no longer
  the preferred option to calculate transition states in FHI-aims. We
  here keep it as legacy only. Please try the 'aimschain'
  functionality of the previous section first. Let us know if you
  are still using the old NEB.f90 version. Else, it will be removed in
  the near future.}

\subsection{Theory and methods}

\subsection{Usage}
The transition state search in AIMS at present is implemented in 
form of a PERL-script \option{NEB.pl} that calls a FORTRAN-90 code 
\option{NEB.f90}, both are located in the \option{src}-subdirectory
\option{NEB}. The Fortran code still has to be compiled with 
\option{LAPACK} and \option{BLAS} routines linked in. 

The job description of a TS-search is contained in a separate control 
file with a number of keywords defined below. That control file 
should be called \option{control\_NEB.in}. You may change the name
to anything else you like, just specify the name of the new control
file as an argument when calling \option{NEB.pl}. Note that (at present)
a lot of keywords are simply assumed to be there, and not enough 
testing is done to capture any missing options. This might lead to 
unexpected runtime behaviour. Please check to make sure everything 
you (think you) need is contained in the file \option{control\_NEB.in}.

The interface to this program is via multi-frame \option{xyz} geometry 
files, which are described under the keyword \keyword{input\_template}. 
You also need a control file template for the AIMS calls (see keyword 
\keyword{aims\_control\_file\_template}), which must contain a line of the 
form \\
\option{restart <}\keyword{aims\_restart\_string}\option{>} \\ 
where the restart file for the different images can be specified and 
passed on. Towards the end of a run, this functionality greatly
accelerates the DFT calculations. Also make sure that the AIMS 
control file template contains the two lines \\
\keyword{compute\_forces}  \option{.true.} \\
\keyword{final\_forces\_cleaned} \option{.true.}

The keywords used in the NEB-control file are:

\keydefinition{aims\_control\_file\_template}{control\_NEB.in}
{\begin{itemize}
  \item Usage: \keyword{aims\_control\_file\_template} \option{filename}
  \item Location of the AIMS template file to be used for the TS search.
  \end{itemize}}

\keydefinition{aims\_restart\_string}{control\_NEB.in}
{\begin{itemize}
  \item Usage: \keyword{aims\_restart\_string} \option{template\_string}
  \item Marker in the control file to be replaced by the actual restart file of a given iteration. 
  \end{itemize}}

\keydefinition{dft\_exe}{control\_NEB.in}
{\begin{itemize}
    \item Usage: \keyword{dft\_exe} \option{filename}
    \item \option{filename} is the AIMS executable including the full path.
  \end{itemize}}

\keydefinition{force\_convergence\_criterion}{control\_NEB.in}
{\begin{itemize}
    \item Usage: \keyword{force\_convergence\_criterion} \option{value}
    \item \option{value} is the convergence force component on any image coordinate 
                         after it has been corrected by the transition state search.
  \end{itemize}}

\keydefinition{input\_template}{control\_NEB.in}
{\begin{itemize}
    \item Usage: \keyword{input\_template} \option{filename}
    \item The location of the initial guess for the TS-search. The file name contains 
      the number of the iteration, followed by \option{.xyz}, while 
      \keyword{input\_template} should be without those two points. 
    \item Example: An initial guess called \option{space\_dog\_0.xyz} 
      would be described with \\
      \option{input\_template space\_dog\_}
    \item This file has to be a multi-frame xyz file with the following format:\\
      \option{<n\_atoms>} \\
      \option{start} \\
      \option{<species1> <xstart> <ystart> <zstart> }\\
      $\ldots$\\
      \option{<n\_atoms>}\\
      \option{image 1}\\
      \option{<species1> <x1> <y1> <z1> }\\
      $\ldots$\\
      \option{<n\_atoms>}\\
      \option{image <n\_images>}\\
      \option{<species1> <x1> <y1> <z1> }\\
      $\ldots$\\
      \option{<n\_atoms>}\\
      \option{END}\\
      \option{<species1> <x1> <y1> <z1> }\\
      $\ldots$\\      
  \end{itemize}}

\doublekeydefinition{max\_atomic\_move}{control\_NEB.in}
{\begin{itemize}
    \item Usage: \option{max\_atomic\_move} \option{value}
    \item The maximally allowed displacement during a single NEB step.
    \item This value should be relatively small (default is 0.1\AA) as the
      algorithm might literally tear apart some of the images if the 
      force constants are not set ideally and the transition path is not 
      extremely close to the actual path. 
    \item Same as keyword \keyword{max\_atomic\_move} in AIMS
  \end{itemize}}

\doublekeydefinition{line\_step\_reduce}{control\_NEB.in}
{\begin{itemize}
    \item Usage: \option{line\_step\_reduce} \option{value}
    \item for BFGS: factor by which a line step will be reduced if the 
      object function increased more than \keyword{object\_function\_tolerance}
      between two successive iterations. 
    \item at present only useful for \keyword{method} \option{PEB}, as there
      is no implementation of an object function for the other methods.
    \item This is the same as the original AIMS keyword \keyword{line\_step\_reduce}
  \end{itemize}}

\keydefinition{method}{control\_NEB.in}
{\begin{itemize}
    \item Usage: \keyword{method} \option{flag} \\
    \item \option{flag} can be either \option{PEB} or \option{NEB} for the pure 
      elastic band and the nudged elastic band methods respectively.
  \end{itemize}}

\doublekeydefinition{min\_line\_step}{control\_NEB.in}
{\begin{itemize}
    \item Usage: \option{min\_line\_step} \option{value}
    \item \option{value} is the minimal line step for the BFGS solver 
      below which the code simply executes a step in order to get out of 
      any numerical noise. 
    \item same as the keyword \keyword{min\_line\_step} in the original 
      AIMS code.
  \end{itemize}}

\keydefinition{n\_atoms}{control\_NEB.in}
{\begin{itemize}
    \item Usage: \keyword{n\_atoms} \option{value}
    \item \option{value} is the number of atoms in each image. 
\end{itemize}}

\keydefinition{n\_images}{control\_NEB.in}
{\begin{itemize}
    \item Usage: \keyword{n\_images} \option{value}
    \item \option{value} is the total number of images, not counting the start and end points. 
  \end{itemize}}

\keydefinition{n\_iteration\_start}{control\_NEB.in}
{\begin{itemize}
    \item Usage: \keyword{n\_iteration\_start} \option{value}
    \item Default \keyword{n\_iteration\_start} \option{= 0}
    \item If there is data available from a previous run of the same system,
      it might start at iteration \option{value} rather than from the beginning.
    \item Careful that you use the proper \keyword{save\_data} file if using this feature!
  \end{itemize}}

\keydefinition{n\_max\_iteration}{control\_NEB.in}
{\begin{itemize}
    \item Usage: \keyword{n\_max\_iteration} \option{value}
    \item \option{value} is the maximal number of iteration of the search.
  \end{itemize}}

\keydefinition{object\_function\_tolerance}{control\_NEB.in}
{\begin{itemize}
  \item Usage: \keyword{object\_function\_tolerance} \option{value}
  \item An iteration is rejected when the object function increases
    more than \option{value} between two successive iteration. 
  \item This only works for the pure elastic band \keyword{method}
    as there is no object function in the current NEB implementation.
  \item Same function as keyword \keyword{energy\_tolerance} in the main AIMS code.
\end{itemize}}

\keydefinition{save\_data}{control\_NEB.in}
{\begin{itemize}
    \item Usage: \keyword{save\_data} \option{filename}
    \item Gives a place to store BFGS and other information between successive iteration. 
    \item Be careful to delete this file when starting a new NEB run.
\end{itemize} }

\keydefinition{spring\_constant}{control\_NEB.in}
{\begin{itemize}
    \item Usage: \keyword{spring\_constant} \option{value}
    \item \option{value} is the single fixed spring constant throughout a single NEB run.
    \item Using this is either default or it should be done in conjunction with \\
      \keyword{use\_variable\_spring\_constants} \option{.false.}
  \end{itemize}}

\keydefinition{spring\_constant\_max}{control\_NEB.in}
{\begin{itemize}
    \item Usage: \keyword{spring\_constant\_max} \option{value}
    \item Maximal value in a range of possible spring constants
    \item can only be used together with the option \\
      \keyword{use\_variable\_spring\_constants} \option{.false.}
\end{itemize}}

\keydefinition{spring\_constant\_min}{control\_NEB.in}
{\begin{itemize}
    \item Usage: \keyword{spring\_constant\_min} \option{value}
    \item Minimal value in a range of possible spring constants
    \item can only be used together with the option \\
      \keyword{use\_variable\_spring\_constants} \option{.false.}
\end{itemize}}

\keydefinition{start\_BFGS}{control\_NEB.in}
{\begin{itemize}
    \item Usage: \keyword{start\_BFGS} \option{value}
    \item Force convergence criterion below which the algorithm switches
          from a steepest-descent type approach to BFGS algorithm. 
 \end{itemize}}

\keydefinition{switch\_to\_CI-NEB}{control\_NEB.in}
{\begin{itemize}
    \item Usage: \keyword{switch\_to\_CI-NEB} \option{value}
    \item will switch to using the climbing image nudged elastic band method
      after reaching a NEB force convergence of \option{value}
\end{itemize}}

\keydefinition{use\_variable\_spring\_constants}{control\_NEB.in}
{\begin{itemize}
    \item Usage: \keyword{use\_variable\_spring\_constants} \option{.true./.false.}
    \item If \option{.true.} a range of spring constants that depend linearly on the 
      energy \cite{HenkelmanJCP2000a} is used. 
    \item \option{.true.} Requires setting of the keywords \keyword{spring\_constant\_min} and 
      \keyword{spring\_constant\_max}
\end{itemize}}

