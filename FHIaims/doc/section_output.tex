
\section{Output options}
\label{section output}

The primary (and most important) output of FHI-aims is written to the
standard output channel, and can / should be captured in a file from
there. However, FHI-aims provides a host of further output options
that can be activated to yield more specialized data not ordinarily
required from a standard calculation, but highly useful for specific
purposes.

The majority of these output options is activated by invoking the
\keyword{output} option in file \texttt{control.in}. The individual
subkeywords to this keyword are therefore listed separately, towards
the end of this section. In addition, some particularly important
output options are revisited with examples in Chapter
\ref{CH:running}.

\newpage

\subsection*{Tags for \texttt{geometry.in}:}

% WPH 23 April 2018: I've inlined the keydefinition macro here, removing the
%     portion related to hyperlinking, as the original version completely broke
%     the flow of the text.
\centerline{\rule{1.0\textwidth}{1pt}}
\textbf{Tag: \texttt{verbatim\_writeout}}{ \color{filename_color} (geometry.in)}
\\[2ex] \hspace*{0.05\textwidth}
\begin{minipage}{0.92\textwidth}
  \noindent
  Usage: \texttt{verbatim\_writeout} \option{flag} \\[1.0ex]
  Purpose: Enables or suppresses the writing of \texttt{geometry.in} to
  the FHI-aims standard output stream exactly as it is read the first
  time. \\[1.0ex]
  \option{flag} is a logical variable (.true. or .false.). Default:
  \texttt{.true.} . \\
\end{minipage} \\
}
By default, \texttt{geometry.in} is now written (copied) verbatim into
the FHI-aims standard output as it is parsed for the first time,
allowing to reproduce exactly any FHI-aims calculation simply by
copy-pasting that part to a new \texttt{geometry.in} file.

If \keyword{verbatim\_writeout} is set to false anywhere in
\texttt{geometry.in}, no writing will occur from that point on
forward.

The exact same option (same keyword / syntax) can also be used in
\texttt{control.in}, producing the same effect there.

Note that the keyword has the same name in \texttt{geometry.in} and
\texttt{control.in}, and is therefore only documented as a clickable
link for \texttt{control.in}. Apologies for this omission.

\subsection*{Tags for general section of \texttt{control.in}:}

\keydefinition{dos\_kgrid\_factors}{control.in}
{
  \noindent
  Usage: \keyword{dos\_kgrid\_factors} $n_1 \, n_2 \, n_3$ \\[1.0ex]
  Purpose: If set, a post-scf density of states is computed with a
    denser k-point grid than used in the s.c.f. cycle. \\[1.0ex]
  Restriction: Works only for periodic systems. Does not work when keyword
    \keyword{use\_local\_index} is set. \\
}
%
Only useful in conjunction with the keywords \keyword{output}
\subkeyword{output}{dos} or \keyword{output}
\subkeyword{output}{postscf\_eigenvalues}, and only for periodic systems.

In a periodic calculation, one usually specifies the basic
\keyword{k\_grid} used to obtain the self-consistent electron density,
total energy etc. Such $k$-space grids are usually fairly sparse, and
if a density of states (DOS) is calculated directly from the eigenvalues
stored at these $k$-points only, the DOS will either look choppy, or
(after significant broadening), smooth, broad, and blurred.

A simple remedy is to use the original \keyword{k\_grid} while
approaching self-consistency as usual, but then compute the DOS using
an auxiliary $k$-grid that is made \emph{denser} by factors $n_1$,
$n_2$, $n_3$, respectively. For example, the settings \\
  \keyword{k\_grid} 10 10 10 \\
  \keyword{output} \subkeyword{output}{dos} [...] \\
  \keyword{dos\_kgrid\_factors} 8 8 8 \\
mean that the s.c.f.-cycle itself is run with a 10$\times$10$\times$10
$k$-point grid, but subsequently, a density of states is computed with
an 80$\times$80$\times$80 $k$-point grid.

Note that this additional calculation is done using serial lapack solutions
of the eigenvalue problems for individual k-points on individual CPU cores.
This always works but will create memory problems as the system size increases,
simply because local copies of all matrices are kept on single CPUs. For large
systems, our usual, more sophisticated parallelization strategies have not yet
been copied over to this routine.

\keydefinition{evaluate\_work\_function}{control.in}
{
  \noindent
  Usage: \keyword{evaluate\_work\_function}  \\[1.0ex]
  Purpose: Surface slab calculations only -- if true, the work functions of
  both slab surfaces will be evaluated.
 \\[1.0ex]
}
%
This option requires that a reference $z$ coordinate for the electrostatic
potential evaluation deep in the vacuum be provided by hand, through the
keyword \keyword{set\_vacuum\_level}. The surface must be parallel to the $xy$
plane.

The output for the ``upper' and ``lower'' surface of the slab (larger and
smaller $z$ value, respectively), will be printed separately. Note that for
\emph{non-symmetric} slabs, these work functions should generally be
different. In practice, this behavior is correctly reproduced only if the
\keyword{use\_dipole\_correction} is additionally specified.

If the work function output is requested, keyword
\keyword{compensate\_multipole\_errors} is now automatically switched
on by default. This will change total energies slightly compared to
the uncompensating case, and -- we believe -- even for the better. It will
certainly lead to a better description of the long-range Hartree potential.

However, it must be possible to find a vacuum plane $z$, where the
surface dipole is compensated, that is further than 6 {\AA} away from
the nearest atom. Otherwise, the calculation will stop and alert the user.

Specifically, the reference Hartree potential component for the work function
evaluation is only the long-range (reciprocal-space) Hartree potential term of
the Ewald sum, not the full electrostatic potential. Thus, the vacuum level
\emph{must} be specified in a region where all real-space components of the
electrostatic potential have safely died away to zero. One may achieve this by
increasing the vacuum layer thickness, which can be done at very small
overhead cost in FHI-aims.

\keydefinition{output}{control.in}
{
  \noindent
  Usage: \keyword{output} \option{type} [\option{further options}]\\[1.0ex]
  Purpose: This is the central keyword that controls most of the
    non-standard output that can be written by FHI-aims. \\[1.0ex]
  \option{type} is a string that specifies the kind of requested
    output; any further needed options, or possibly additional lines,
    depend on \option{type}. \\
}
The list of additional output \option{type}s is given as a separate
subsubsection below.

\keydefinition{output\_boys\_centers}{control.in}
{
  \noindent
  Usage: \keyword{output\_boys\_centers} \\[1.0ex]
  Purpose: Calculates and outputs the maximally localized Boys centers
  (equivalent of Wannier centers for the isolated molecule case). \\
}
The maximization procedure follows JCP 135, 134107 (2011). Currently
only the cartesian position of the centers is outputted in xyz format in
\texttt{geometry\_boys.xyz}.  The transformation matrix is also calculated,
but not applied to the KS orbitals.

\keydefinition{output\_cube\_nth\_iteration}{control.in}
{
  \noindent
  Usage: \keyword{output\_cube\_nth\_iteration} \option{n} \\[1.0ex]
  Purpose: Writes all cube files specified in \texttt{control.in} every
   n$^{th}$ SCF iteration. \\[1.0ex]
  \option{n} is an integer greater than or equal to 1.  Default: N/A (cube
   files will be output after the SCF cycle has converged.) \\
}
By default, cube files are written once, after the SCF cycle has converged.
With this keyword,all cube files specified in \texttt{control.in} will be
written each $n^{th}$ iteration, where \texttt{n} is an integer greater than
zero. This keyword should be helpful to analyse what is going on during
subsequent SCF cycles. However, the output of cubes is usually quite slow, so
choosing this option will slow down the calculation a lot.

This keyword does not support spin-orbit coupling, as spin-orbit coupling is
applied after the SCF cycle has converged.

This keyword is only applicable when the
\keyword{output} \subkeyword{output}{cube} keyword(s) are being used; please see
the manual entry for \keyword{output} \subkeyword{output}{cube} for more
information.

\keydefinition{output\_in\_original\_unit\_cell}{control.in}
{
  \noindent
  Usage: \keyword{output\_in\_original\_unit\_cell} \option{flag} \\[1.0ex]
  Purpose: Shifts the atoms in a periodic calculation back into the
    first unit cell before printing them out at the beginning of a new
    geometry step. \\[1.0ex]
  \option{flag} is a logical string, either \texttt{.true.} or
    \texttt{.false.} Default: \texttt{.true.} \\
}
In some, atoms in FHI-aims can unexpectedly ``switch'' unit cells
during relaxation. This has no effect on the calculation, but the
output geometry coordinates (written to the standard output stream)
will look strangely detached when visualized. By default, FHI-aims
maps its coordinates back to the first unit cell anyway, but this
behavior can be forcibly switched off if so requested (makes nicer
movies).

\keydefinition{output\_level}{control.in}
{
  \noindent
  Usage: \keyword{output\_level} \option{level} \\[1.0ex]
  Purpose: Allows to increase the amount of output written to the
    standard output of FHI-aims. \\[1.0ex]
  \option{level} is a string that determines the amount of output
    written. Default: \texttt{normal} . \\
}
If increased to \texttt{full}, the Kohn-Sham eigenvalues of every
s.c.f. iteration are written to the standard output file. For
single-point calculations, this may be quite desirable, but leads to
unmanageable file sizes for long relaxation or molecular dymnamics
runs. \\
Another option, useful for long molecular dynamics (MD) runs, is \texttt{MD\_light}.
It writes standard output only in the initialization part and at the end of each
MD step, while a minimal output is written for the single scf cycle.

\keydefinition{overwrite\_existing\_cube\_files}{control.in}
{
  \noindent
  Usage: \keyword{overwrite\_existing\_cube\_files} \option{flag} \\[1.0ex]
  Purpose: Allows overwriting of pre-existing cube files with new cube files of
    the same name, instead of preserving the pre-existing cube files by
    appending numbers to the end of the new file names. \\[1.0ex]
  \option{flag} is a logical string, either \texttt{.true.} or
    \texttt{.false.} Default: \texttt{.false.} \\
}
If set to .false., FHI-aims will check during the output of cube files whether a
file with the selected file name already exists. If such a file is found, the
file name will be changed by adding a number (1,2,3...) to the end of the file
name. This is very useful when relying on default names or when plotting cubes
during the SCF cycle.

If set to .true. FHI-aims will not check during output whether a file with the
selected name already exists. If it does exist, it will be simply overwritten!

This keyword is only applicable when the
\keyword{output} \subkeyword{output}{cube} keyword(s) are being used; please see
the manual entry for \keyword{output} \subkeyword{output}{cube} for more
information.

\keydefinition{verbatim\_writeout}{control.in}
{
  \noindent
  Usage: \keyword{verbatim\_writeout} \option{flag} \\[1.0ex]
  Purpose: Enables or suppresses the writing of \texttt{control.in} to
  the FHI-aims standard output stream exactly as it is read the first
  time. \\[1.0ex]
  \option{flag} is a logical variable (.true. or .false.). Default: \texttt{.true.} . \\
}
By default, \texttt{control.in} is now written (copied) verbatim into
the FHI-aims standard output as it is parsed for the first time,
allowing to reproduce exactly any FHI-aims calculation simply by
copy-pasting that part to a new \texttt{control.in} file.

If \keyword{verbatim\_writeout} is set to false anywhere in
\texttt{control.in}, no writing will occur from that point on
forward.

The exact same option (same keyword / syntax) can also be used in
\texttt{geometry.in}, producing the same effect there.

\newpage

\subsection*{Specific options \texttt{output} keyword:}

\subkeydefinition{output}{aitranss}{control.in}
{
  \noindent
  Usage: \keyword{output} \subkeyword{output}{aitranss} \\[1.0ex]
  Purpose: Writes Kohn-Sham eigenvectors $c_{il}$ and energies $\varepsilon_l$
    (where $i$ is a basis function index and $l$ is an eigenstate index) of each
    spin channel and overlap matrix $s_{ij}$ into separate ASCII-files in a
    format compatible with \textsc{aitranss} (\textit{ab initio} transport
    simulations) package. \\[1.0ex]
  Restrictions: This functionality is available only for
    \emph{non-periodic} systems. If \keyword{KS\_method} \texttt{scalapack} is
    used, \keyword{packed\_matrix\_format} is not supported. Flag
    \keyword{use\_local\_index} is not supported either.\\
}
\emph{Please, look at Chapter~\ref{Ch:aitranss} for
  a comprehensive description on how to perform transport simulations across
  nanoscale objects}. \\


\subkeydefinition{output}{atom\_proj\_dos}{control.in}
{
  Usage: \keyword{output} \subkeyword{output}{atom\_proj\_dos}
  \option{Estart Eend n\_points broadening} \\[1.0ex]
  Purpose: Writes an atom-projected, angular-momentum resolved partial
    density of states (pDOS). \\[1.0ex]
  \option{Estart} : Lower bound of the single-particle energy range
    for which the pDOS are given. \\
  \option{Eend} : Upper bound of the single-particle energy range
    for which the pDOS are given. \\
  \option{n\_points} : Number of energy data points for which the
    pDOS are given. \\
  \option{broadening} : Gaussian broadening applied to obtain a smooth
    partial density of states based on the peaks produced by
    individual states. \\
}
This option is based on a Mulliken Analysis and shares its syntax with
\keyword{output} \subkeyword{output}{dos} and \keyword{output}
\subkeyword{output}{species\_proj\_dos}. See also section \ref{band
and dos plotting} for more details.

There are two types of output files for each atom:
\begin{itemize}
  \item \texttt{atom\_proj\_dos\_\emph{number}\_raw.dat}, where \emph{number}
    denotes the atom number in the order of \texttt{geometry.in}. This file
    contains the total and angular-momentum resolved DOS components as a
    function of eigenvalue energy (first column) as used internally in
    FHI-aims. The energy zero is then given by the vacuum level (non-periodic
    systems) or by the $\boldG$=0 component of the long-range Hartree
    potential (periodic systems).
  \item \texttt{atom\_projected\_dos\_\emph{number}.dat}, which gives the same
    information, except that the energy zero is shifted to the Fermi energy
    (metallic systems) or valence band maximum (insulators), respectively.
\end{itemize}
Note that projected densities of states such as given here must be
based on some kind of projection orbitals, the choice of which is
somewhat arbitrary by necessity. This is thus a tool for
\emph{qualitative} analyses.

In FHI-aims, we project directly on
the atom-centered angular-momentum components as defined by the
\emph{overlapping} basis set. This definition becomes the more arbitrary the
larger the basis set, just like a \subkeyword{output}{mulliken}
analysis. The closer the full basis comes to completeness, the
more ambiguous will a \subkeyword{output}{mulliken}-like analysis become,
since it may not be \emph{a priori} clear which electrons should be counted
towards the basis functions of one atom vs. those of another atom. Thus, do
\emph{not} expect a pDOS to simply converge as the basis set size is
increased; use it as a qualitative indicator of trends, but nothing more.

This keyword supports spin-orbit coupling.  When spin-orbit coupling is
enabled, the file(s) containing the spin-orbit-coupled DOS will have the
default filename(s) and the file(s) containing the scalar-relativistic
(i.e. no SOC) DOS will have an additional suffix ".no\_soc".


\subkeydefinition{output}{band}{control.in}
{
  \noindent
  Usage: \keyword{output} \subkeyword{output}{band} \option{kstart1
    kstart2 kstart3 kend1 kend2 kend3 n\_points name\_start
    name\_end}\\[1.0ex]
  Purpose: Plots a band along the line from
    \option{<kstart1,kstart2,kstart3>} to
  \option{<kend1,kend2,kend3>} at \option{n\_points} equally
  spaced points. The $k$-vectors are written in relative
  coordinates of the reciprocal basis vectors. \\
}
Several bands can be plotted; FHI-aims outputs one file
per specified \keyword{output band} line.

The band structure output files are named
\option{bandXYYY.out}, where the letters X and YYY are replaced with
numbers in the actual output file names:
\begin{itemize}
\item The letter X encode the spin
channel. In a non-spinpolarized or in a spin-orbit coupled
calculation, X will always be 1. In a spin-polarized calculation, X=1
indicates the first spin channel, X=2 indicates the second spin
channel.
\item The letters YYY are the consecutive numbers of the bands
(starting from 001) requested in the \texttt{control.in} file, that
is, the bands given in the order of \keyword{output} \subkeyword{band}
lines in \texttt{control.in}.
\end{itemize}
The files \option{bandXYYY.out} have the format\\[1.0ex]
%
\option{ipoint k1 k2 k3 E1 occ1 E2 occ2 ... EN occN}, \\[1.0ex]
%
i.e. they specify not only the bands for each $k$-point but also
the occupation number for this particular point.

A safe starting value for \option{n\_points} when performing semi-local
calculations is 21.  We have found that this generally samples the fine
features of the k-path reasonably well, even for small unit cells with
correspondingly large Brillouin zones.  For hybrid-functional calculations,
due to the computational expense one should consider using a smaller value.
We also note that \option{n\_points} includes the end-points, i.e.
\option{n\_points}=21 will give 20 intervals for a given branch.  For
comparison with results calculated by other DFT codes, it's recommended to
use values of form 1, 6, 11, 16, 21, ... to ensure that the reciprocal
coordinates are nice, simple fractions.

Note that a fully occupied band has the occupation number 2.0 in a
non-spinpolarized calculation. In a spin-polarized or
spin-orbit-coupled calculation, the maximum occupation number is 1.0.

Since this format contains all the important information,
but is not particularly useful for actually plotting the band
structure, we provide a small script which is described in
section \ref{band and dos plotting}.

Note: the last two input options are technically not needed by the
FHI-aims main program, but they are seriously helpful when
turning this data into a plot and are used by the band plotting
script provided along with this distribution, see Section
\ref{band and dos plotting} for details.

For periodic calculations, the eigenvectors, overlap matrices, and
hamiltonian matrices at each k-point requested by
\keyword{output} \subkeyword{output} {band}
can be written out using the
\keyword{output} \subkeyword{output}{eigenvectors},
\keyword{output} \subkeyword{output}{overlap\_matrix}, and
\keyword{output} \subkeyword{output}{hamiltonian\_matrix} keywords,
respectively.

For periodic band structure output, the
\keyword{exx\_band\_structure\_version} keyword must be set -- see the
respective Section \ref{Sec:periodic_hf} for a brief explanation of
the background.

This keyword supports spin-orbit coupling.  When spin-orbit coupling is
enabled, the file(s) containing the spin-orbit-coupled band structures
will have the default filename(s) and the file(s) containing the
scalar-relativistic (i.e. no SOC) band structures will have an additional
suffix ".no\_soc".

\subkeydefinition{output}{band\_during\_scf}{control.in}
{
  Usage: \keyword{output} \subkeyword{output}{band\_during\_scf} \option{kstart1
    kstart2 kstart3 kend1 kend2 kend3 name\_start
    name\_end} \\[1.0ex]
  Purpose: Plots a band along the line from
    \option{<kstart1,kstart2,kstart3>} to
  \option{<kend1,kend2,kend3>} but only at those $k$ points that are
  already part of the normal s.c.f. \keyword{k\_grid}. The
  $k$-vectors are written in relative
  coordinates of the reciprocal basis vectors. \\
}
This keyword allows to get the band structure along a certain
reciprocal-space direction, but \emph{only} at those $k$-points that
are already used during the s.c.f. calculation. If there are no
appropriate $k$-points, no band structure is printed. If there are
appropriate $k$-points, they are printed in the same format as the
normal band structure from \keyword{output}\subkeyword{output}{band},
although some additional editing may be required to get a clean plot.

The purpose of this keyword is to allow to extract a band structure
even in cases when the normal
\keyword{output}\subkeyword{output}{band} functionality is
experimental or, for some reason, not available.


\subkeydefinition{output}{band\_mulliken}{control.in}
{
	Usage: \keyword{output} \subkeyword{output}{band\_mulliken} \option{kstart1
		kstart2 kstart3 kend1 kend2 kend3 n\_points name\_start
		name\_end}\\[1.0ex]
	Purpose: Plots a band along the line from
	\option{<kstart1,kstart2,kstart3>} to
	\option{<kend1,kend2,kend3>} at \option{n\_points} equally
	spaced points. The $k$-vectors are written in relative
	coordinates of the reciprocal basis vectors. \\
}

This keyword allows to calculate the mulliken charge analysis at all K points along the band K path. The file name is named as bandmlk1001.out, bandmlk1002.out, ..., etc. In the output file, the mulliken charge data is written first by K point, then by eigenstates. In each eigenstate, the mulliken charge on all atoms is written line by line. In each line, the following information is written: eigenstate ID, eigenvalue, occupation number, atom ID, spin ID, total mulliken charge, mulliken charge for l = 0, 1, 2,.. etc. For each state at a given point, the total mulliken charge for all atoms should sum up to 1 or 2, for non-SOC and SOC, respectively.

There is a keyword called \keyword{band\_mulliken\_orbit\_num} specifying how many orbitals (states) to be written out. If the number following the keyword \keyword{band\_mulliken\_orbit\_num} is I, then the orbitals in the range of HOMO - I + 1 and LUMO + I would be written out in the output file. The default value of \keyword{band\_mulliken\_orbit\_num} is 50 for non-SOC and 100 for SOC.

To plot the band structures with Mulliken decomposition, two python script in the utilities directory in FHI-aims distribution can be used,
i.e., band\_mlk.py and band\_mlk\_soc.py for non-SOC and SOC, respectively. Running of these scripts is instructed in the first lines of these python files.

\subkeydefinition{output}{basis}{control.in}
{
  \noindent
  Usage: \keyword{output} \subkeyword{output}{basis} \\[1.0ex]
  Purpose: Writes radial functions before and after
    orthonormalization, as well as second derivatives and the
    basis-defining potentials to separate files. \\[1.0ex]
}
This output option allows to visualize the basis functions used, as
well as some of the other defining pieces of the basis set. Note
that the output is written for each grid point of the dense
1-dimensional \subkeyword{species}{logarithmic} grid, with the radius
given in bohr.

Specifically, this option produces the following types of files:
\begin{itemize}
  \item \texttt{A$i$\_$j$\_$nl$\_base.dat} : Atomic (minimal basis)
    radial function $u(r)$ \emph{after} the basis-confining potential
    was applied, for \keyword{species} number $i$, atomic-like
    (minimal) radial function  number $j$, radial and angular quantum
    numbers $nl$.
  \item \texttt{A$i$\_$j$\_$nl$\_base\_kin.dat} : Corresponding
    kinetic energy expression $[\epsilon - v(r)]\cdot u(r)$
  \item \texttt{C$i$\_$j$\_$nl$\_base.dat} :
    \subkeyword{species}{confined} free-atom like radial function $u(r)$
      number $j$ for \keyword{species} number $i$, radial and angular
      quantum numbers $nl$.
  \item \texttt{C$i$\_$j$\_$nl$\_base\_pot.dat} : Corresponding
    basis-defining potential including confining potential
  \item \texttt{\emph{El}\_base\_$n$\_$l$.dat} : Radial function
    $u(r)$ \emph{after} the basis-confining potential was applied for
    the \keyword{species} named \emph{El}, radial and angular
      quantum numbers $nl$ (same as
    \texttt{A$i$\_$j$\_$nl$\_base.dat}).
  \item \texttt{\emph{El}\_base\_pot.dat} : Basis-defining potential
    for atomic (minimal) radial functions of the \keyword{species}
    named \emph{El}, after addition of the confining potential (as
    defined by \subkeyword{species}{cut\_pot}).
  \item \texttt{\emph{El}\_base\_pot.dat} Free-atom density of
    \keyword{species} named \emph{El} (same as
    \texttt{\emph{El}\_base\_pot.dat}).
  \item  \texttt{\emph{El}\_free\_$n$\_$l$.dat}: Radial function
    $u(r)$ \emph{before} the basis-confining potential was applied for
    the \keyword{species} named \emph{El}, radial and angular
    quantum numbers $nl$
  \item \texttt{\emph{El}\_free\_pot.dat} Spherical self-consistent
    free-atom potential of the \keyword{species} named \emph{El}
    (implicitly confined by the \subkeyword{species}{cut\_free\_atom}
    potential, but this artificial part is here not included)
  \item \texttt{\emph{El}\_free\_rho.dat} Free-atom density of
    the \keyword{species} named \emph{El}.
  \item \texttt{H$i$\_$j$\_$nl$\_base.dat} :
    \subkeyword{species}{hydro} radial function $u(r)$
      number $j$ for \keyword{species} number $i$, radial and angular
      quantum numbers $nl$.
  \item \texttt{H$i$\_$j$\_$nl$\_base\_kin.dat} : Corresponding
    kinetic energy expression $[\epsilon - v(r)]\cdot u(r)$
  \item \texttt{I$i$\_$j$\_$nl$\_base.dat} :
    \subkeyword{species}{ionic} radial function $u(r)$
      number $j$ for \keyword{species} number $i$, radial and angular
      quantum numbers $nl$.
  \item \texttt{I$i$\_$j$\_$nl$\_base\_pot.dat} : Corresponding
    basis-defining potential including confining potential
  \item \texttt{\emph{ty}\_$i$\_$j$\_$n$\_$l$.dat} : \emph{After}
    on-site orthonormalization, radial function $u(r)$ of type
    \emph{ty} (atomic, \subkeyword{species}{confined},
    \subkeyword{species}{hydro}, \subkeyword{species}{ionic}, ...),
    for \keyword{species} number $i$, radial function number $j$,
    radial and angular quantum numbers $n,l$.
  \item \texttt{kin\_\emph{ty}\_$i$\_$j$\_$n$\_$l$.dat} :
    Corresponding kinetic energy expression after on-site
    orthonormalization.
  \item \texttt{S$i$\_$j$\_$nl$\_base.dat} : Slater-type orbital
    radial function $u(r)$ for \keyword{species} number $i$, radial
    function number $j$, radial and angular quantum numbers $nl$.
  \item \texttt{S$i$\_$j$\_$nl$\_base\_kin.dat} : Corresponding
    kinetic energy expression $[\epsilon - v(r)]\cdot u(r)$
\end{itemize}

\subkeydefinition{output}{batch\_statistics}{control.in}
{
  \noindent
  Usage: \keyword{output} \subkeyword{output}{batch\_statistics} \\[1.0ex]
  Purpose: Write out statistics for each batch of points used in the evaluation of
           real-space quantities, organized by associated MPI task (one file per MPI
           task) \\[1.0ex]
}

This keyword outputs information about the batch distribution used by FHI-aims to evaluate
real-space quantities (real-space Hamiltonian, charge density update, etc.)  Every MPI task
creates a \texttt{batch\_statistics\_task\_\#\#\#.dat} file, where \#\#\# is the MPI task's
rank, and statistics for each batch are output sequentially to file.  Statistics output for
each batch include:
\begin{itemize}
  \item Number of points in the batch
  \item Minimum and maximum number of radial basis functions evaluated on batch
  \item Maximum number of basis functions evaluated on batch
  \item Minmum and maximum number of atoms whose basis functions are evaluated on the batch
  \item Minimum and maximum values for the integration weights for points in batch
\end{itemize}
Note:  As of this writing, the output files will be rewritten with every SCF restart, including
SCF reinitialization, geometry relaxation steps, and MD steps.

\subkeydefinition{output}{cube}{control.in}
{
  \noindent
  Usage: \keyword{output} \subkeyword{output}{cube} \option{type} \\[1.0ex]
  Purpose: Writes a quantity (density, eigenfunction, ...) to a
    uniform three-dimensional grid, using the ASCII-based cube file
    format established by the Gaussian code and accepted by numerous
    visualization tools. \\[1.0ex]
  \option{type} is a string, indicating the quantity to be plotted. \\
}
The ``cube'' file format originates from the Gaussian code, but
publically available descriptions exist, for example here: \\[1.0ex]
 \url{http://paulbourke.net/dataformats/cube/} \\[1.0ex]
It is accepted by multiple visualization tools; three visualization
tools which may be used to plot cube file and are available for all
major operating systems are Avogadro (\url{https://avogadro.cc}), jmol
(\url{http://www.jmol.org}), and VMD
(\url{http://www.ks.uiuc.edu/Research/vmd/}).  Please see the documentation
of those programs for more information on plotting the resulting cube files.

By default, the cube files will be output once, after the SCF cycle has
converged, and FHI-aims will avoid overwriting pre-existing cube files it finds
by appending numbers to the end of new file names.  To output the cube files at
regular intervals during the SCF cycle, use the
\keyword{output\_cube\_nth\_iteration} keyword, and to overwrite pre-existing
cube files with new files, use the \keyword{overwrite\_existing\_cube\_files}
keyword.

This keyword supports spin-orbit coupling, but only when the \texttt{type} is
either \texttt{eigenstate\_density} or \texttt{eigenstate}.  The large-scale
\keyword{use\_local\_index} and \keyword{load\_balancing} keywords
are only supported when the \texttt{type} is either \texttt{eigenstate\_density}
or \texttt{eigenstate}.

% The specification of the cube format was removed because it had no clear
% target audience: it was too technical for the average user trying to use
% the code and not technical enough for the average deveoper trying to extend
% the code.

Before we move on to the supported options for \texttt{type}, as well as
other keywords related to cube output, there is an important note about
specifying the dimensions of the cube that all users should know.  In some
cases, such as separated molecules or surfaces with large amounts of vacuum,
the cube dimensions that FHI-aims would silently default to may not be ideal
and may result in excessively large files.  Presently, the code will stop if
the defaults imply cube files with more than 650 MB. Also, many viewers do not
implement non-rectangular cube edges, which would result from non-rectangular
unit cells by default.

In short: Users are \emph{always} strongly encouraged to specify cube
geometries directly.

A list of all keywords related to cube plotting is given below.  After this
list of keywords, we have provided an example set of lines for plotting the
total density of a molecule as well as the densities for individual eigenstates.
This example should be adaptable for other \texttt{types} of cube files.
Units for densities and eigenstates are {\AA}$^{-3}$ and {\AA}$^{-3/2}$, respectively.
The unit for the long range and hartree potential is Hartree [Ha].

\begin{itemize}
\item \keyword{output} \subkeyword{output}{cube} \option{type}
This is the only mandatory line, specifying which type of cube file should be produced.
FHI-aims presently allows the following options for \option{type}:
\begin{enumerate}
  \item \texttt{delta\_density} : Writes the difference between the
    initial (superposition of free atoms) and the final
    self-consistent density to a file.
  \item \texttt{eigenstate\_density $n$}. Writes the electron density of the
    $n-th$ eigenstate to a file.  The eigenstate density is obtained as the
    square of the wavefunction.  In periodic calculations, this includes the
    contribution from the imaginary part of the wave function, and thus the
    output of this \texttt{type} is more physically relevant than the
    \texttt{eigenstate} \texttt{type}, which outputs only the real part of the
    wave function.  By default, the first spin channel and the first k-point
    is printed out, see also \texttt{cube spinstate}.
  \item \texttt{eigenstate $n$} :  Writes the \emph{real part} of the wave function
    of the $n-th$ eigenstate to a file.  $n$ must be an integer number. For
    non-periodic non-spin-orbit coupled calculations, the wave function has
    no complex part, so this \texttt{type} can be used safely.  However, periodic
    and/or spin-orbit-coupled calculations produce wave functions with both a real
    part and a complex part.  Accordingly, it is highly recommended that you use the
    \texttt{eigenstate\_density} \texttt{type} for periodic calculations (and non-periodic
    calculations with spin-orbit coupling) instead, as the output for this \texttt{type}
    will be missing the contribution from the complex part of the wavefunction and
    thus have limited physical significance.  By default, the first spin channel
    and the first k-point is printed out, see also \texttt{cube spinstate} and
    \texttt{cube kpoint}.
  \item \texttt{spin\_density} : The spin density $n^\uparrow(\boldr)
    - n^\downarrow(\boldr)$ is written to a file named
    \texttt{spin\_density.cube}. Only available for \keyword{spin}
    \texttt{collinear}.
  \item \texttt{stm} :
    Must be followed by a real number $V$. Calculates
    3D tunneling current map (more precisely, the tunneling current
    is proportional to the printed values)
    which can be used to
    plot STM images for a given voltage $V$ (in Volts) in the frame
    of the Tersoff-Hamann model.
    This is done by summing up eigenstate densities
    for all eigenstates between the Fermi level and $V$ (in eV),
    and the result is multiplied by $V$.
    In addition, a file \texttt{cube\_xxx\_stm\_z\_map.cube}
    will be printed. It contains values of the z-coordinate at the
    vertices of the cube, and can be used along with the tunneling current map
    to color the constant current isosurfaces according to their extent in
    the z-direction (to mimic the constant current image contrast in STM
    imaging). The output of stm-cubes works only for periodic systems.
  \item \texttt{total\_density} : The full electron density is
    recomputed and written to the a. In case of a periodic calculation,
    electron density from all unit cells that overlap with the cube
    output region will be printed.
  \item \texttt{long\_range\_potential} : Prints the long range
    electrostatic potential of the Ewald summation. This result is
    useful in regions where no electron density is found and is much
    faster than the output of the full potential.
  \item \texttt{hartree\_potential}: The whole (i.e, short-range and
    long-range) electrostatic potential is recomputed on a cube grid
    and written out. Be careful with keyword
    \texttt{potential}. Please report errors.
  \item \texttt{xc\_potential}: The xc potential is recomputed on a
    cube grid and written out. \textbf{PBE only},  \textbf{spin unpolarized only}.
  \item \texttt{potential}: Legacy. The whole (i.e, short-range and
    long-range) electrostatic potential is recomputed on a cube grid
    and written out. The keyword is considered
    broken/experimental. Use \texttt{hartree\_potential}.
  \item \texttt{delta\_v}: Output $\delta v = v - v^\mathrm{free}$. This is especially tested for MPB solvation effects and is also an experimental feature especially for vacuum calculations.
  \item \texttt{ion\_dens}: Ionic charge density $n_\mathrm{ion}^\mathrm{MPB}$ as obtained from an MPB-DFT calculation. Still experimental.
  \item \texttt{dielec\_func}: Dielectric function $\varepsilon[n_\mathrm{el}]$ as used in the MPB-DFT calculation.
  \item \texttt{elf}: Electron localization function. Different options are available
for spin-polarized systems, see keyword \texttt{cube elf\_type}.
Currently under testing, please report any errors. The implementation is not yet compatible
with spin-orbit calculations.
\end{enumerate}

\item \subkeyword{output}{cube} \texttt{spinstate} \textit{spin} \\
This keyword allows the user to choose whether to print spin-channel 1 or 2. The
default value is 1.  An example for how to use the \texttt{cube spinstate} keyword
to print eigenstates in both spin channels is given below.  This keyword is only
useful for spin-polarized calculations (keyword\keyword{spin} \texttt{collinear}), as
spin-non-polarized calculations (keyword\keyword{spin} \texttt{none}) have degenerate
spin channels by definition.  This keyword is not supported by spin-orbit-coupled
calculations (keyword\keyword{include\_spin\_orbit}), as spin-orbit-coupled systems do not
have spin channels; by definition, spin and orbit are coupled.

\item \subkeyword{output}{cube} \texttt{kpoint} \textit{kpoint} \\
This keyword allows to choose the k-point to be printed.
$kpoint$ is an integer number following the same ordering
as the k-points within the SCF-cycle. It is presently not
possible to output cube files at a k-point not included in the scf.
Keyword \keyword{output} \subkeyword{output}{k\_point\_list} may be
used to print out the entire list of k-points used in the s.c.f. cycle.
This will help identify which k-point is printed in a cube file.
Default: 1

\item \subkeyword{output}{cube} \texttt{state} \textit{spin} \textit{k-point} \\
Deprecated keyword to choose spin and k-point.
The \texttt{cube spinstate} keyword
should be used instead as given in the example below.
If nothing else is specified, this keyword defaults to
\texttt{cube state $1$ $1$}. Note that for cluster
calculations, $k-point$ must always be 1.

\item \subkeyword{output}{cube} \texttt{filename} \textit{name\_of\_the\_file} \\
Allows to customize the name of the cubefile. If this line is not given, FHI-AIMS
will default to a file name which contains the number of the cube requested, its
type, and, if applicable, the corresponding spin and k-point of the data.

\item \subkeyword{output}{cube} \texttt{format} \textit{format} \\
Apart from the default cube format, FHI-AIMS also supports output
in the formats of the gOpenMol and XCrysden software. This is requested by the line
\texttt{cube format $format$}. The options for $format$ are cube,
gOpenMol, and xsf (the XCrysden format). Default: cube

\item \subkeyword{output}{cube} \texttt{divisor} \textit{number} \\
This is a technical settings which governs the paralellization of the
cube output. The whole cube is divided into smaller, so-called minicubes,
which are then treated one after the other. The value governs the number of points in each
directed to be used for each minicube, i.e., its size. A larger setting therefore results
in less minicubes (and it thus potentially faster), but also a larger demand for memory.
Unless there are problems with memory, this setting usually does not
need to be touched, with the exception of output potential, where it
should be set to its maximal value (45), independent of the number of
processors used. Default: 10

\item \subkeyword{output}{cube} \texttt{spinmask} $i$ $j$ \\
For total\_density cube files, the line \texttt{cube spinmask}
    $i$ $j$ with integer $i$ and $j$ allows to manipulate the spin channels
    independently according to the following formula: $i\cdot n^\uparrow(\boldr)
    - j\cdot n^\downarrow(\boldr)$. If $i$ = 1 and $j$ = 1 (which is the default),
    the total density will be computed as normal.
    This keyword replaces the earlier keyword \subkeyword{output}{cube} \texttt{spin} .

\item \subkeyword{output}{cube} \texttt{elf\_type} $i$ \\
Specifies the type of electron localization function (ELF) to be calculated.
The default $i=0$ (and the only available option for spin-unpolarized calculations)
is the Savin {\em et al.} formula~\cite{Savin96}. For spin-polarized systems,
$i=1$ and $i=2$ correspond to the original formulation by
Becke and Edgecombe~\cite{Becke90} for spin channels 1 and 2, respectively.
If $i$ is not 0, 1, or 2, and the system is spin-polarized,
the Kohout-Savin variant of ELF~\cite{Kohout96} will be calculated.

\item \subkeyword{output}{cube} \texttt{origin} $x$ $y$ $z$ \\
Single line which specifies the origin , i.e., the center of the region to be plotted. Values are given in \AA
If omitted, the same origin as for the previous cube file is used. If
no origin has been given yet, it defaults to the geometric center of
the molecule in the cluster case or (0,0,0) for periodic
calculations. For slab type calculations (when using
\texttt{dipole\_correction .true.}), the origin is set to the center
of the slab.

\item \subkeyword{output}{cube} \texttt{edge} $n$ $dx$ $dy$ $dz$ \\
Specifies the edges of the volumetric data to be plotted. Separate
lines have to be given for each of the three edges of the cube. In
each line, $n$ indicates the number of steps a particular edge
(voxel), and $dx$, $dy$, $dz$ indicate the length of each individual
step [i.e., the full cube edge length is ($n\cdot dx$, $n\cdot dy$,
  $n\cdot dz$). If omitted, the same edges as for the previous cube
  file are used. If no edges have been specified yet, FHI-aims
  defaults (in the cluster case) to orthogonal grids of 0.1 {\AA}
  length, which span the whole molecule plus 14 Bohr beyond the
  outermost nucleii. For periodic calculations, the default edges
  are the same as the lattice vectors, again with 0.1 {\AA} step
  length. \emph{Note: In some cases, such as separated molecules or
    surfaces with large amounts of vacuum, the defaults might be far
    from ideal and result in excessively large files.  Presently, the
    code will stop if the defaults imply cube files with more than
    650MB. Users are \emph{always} strongly encouraged to specify
    cube geometries themselves. }
\end{itemize}

\textbf{NOTE:} A word of warning here: Historically grown, FHI-aims outputs the density and the wave function
in units of 1/\AA$^3$, although the cube voxels are in atomic units. This must be accounted for
when doing any kind of postprocessing! Although this behaviour might be considered a bug, we
decided to leave it this way in order not to break any scripts people are already using.
Instead, the keyword \texttt{cube\_content\_unit} was introduced. \newline
\texttt{cube\_content\_unit legacy} is the default and gives the output in SI units, while \newline
\texttt{cube\_content\_unit bohr} switches to a consistent output in atomic units.
(Note that the definition of cube file usually is in bohr, which is why we do not provide
a switch to \AA).

As promised, an example set of lines for the \texttt{control.in} file showing how
to plot the total density and densities for individual eigenstates of a hypothetical
spin-polarized non-periodic system.  This example will not exactly correspond
to your system and should not be blindly copy-paste'd, nevertheless it should give
you an idea how the keywords presented previously work together.
\begin{verbatim}
    output cube total_density
    cube origin 1.59 9.85 12.80
    cube edge 101 0.15 0.0 0.0
    cube edge 101 0.0 0.15 0.0
    cube edge 101 0.0 0.0 0.15
    output cube eigenstate_density 151
    cube spinstate 1
    output cube eigenstate_density 151
    cube spinstate 2
    output cube eigenstate_density 152
    cube spinstate 1
    output cube eigenstate_density 152
    cube spinstate 2
    output cube eigenstate_density 153
    cube spinstate 1
    output cube eigenstate_density 153
    cube spinstate 2
    output cube eigenstate_density 154
    cube spinstate 1
    output cube eigenstate_density 154
    cube spinstate 2
\end{verbatim}
The first line requests cube output for the total density. See the
details above on how to get individual versions of the spin
density. Then, the \textbf{center} of the cube is specified at (1.59,
9.5, 12.80) {\AA}. Each cube direction has 101 points that are spaced
apart by 0.15 {\AA}, giving a total cube edge length of 15 {\AA} in
each direction. Finally, output is requested for the densities correspond to
the Kohn-Sham wave functions associated with eigenstates number 151-154. For
each eigenstate, the additional \texttt{cube spinstate} \emph{i} line requests
first the spin-up channel ($i$=1), then spin-down ($i$=2).

\textbf{NOTE} that for periodic systems, the \texttt{cube spinstate} lines
may have to be followed by \texttt{cube kpoint} lines to specify the k-point
at which an eigenstate is printed. By default, only $k$-point number 1 is printed.
For unshifted $k$ grids, that is the $\Gamma$ point.

Thus ends our brief treatise on cube plotting.  We return you to your regularly
scheduled programming.

\subkeydefinition{output}{density}{control.in}
{
  \noindent
  Usage: \keyword{output} \subkeyword{output}{density} \\[1.0ex]
  Purpose: Writes the electron density $n(\boldr)$
    at each integration grid point $\boldr$ to a file
  \texttt{density.dat}. \\[1.0ex]
}
Note that this density output is \emph{not} given on a uniform grid,
but simply on the overlapping atom-centered grid used for all internal
operations of FHI-aims. For a density on a uniform grid, see the
\keyword{output} \subkeyword{output}{cube} subkeyword.

\keyword{output} \subkeyword{output}{density} additionally writes the
difference between the current density and the
superposition-of-free-atom reference density to a file \texttt{diff-density.dat}

\subkeydefinition{output}{dipole}{control.in}
{
  \noindent
  Usage: \keyword{output} \subkeyword{output}{dipole} \\[1.0ex]
  Purpose: Calculates and writes the electrical dipole moment of the
    structure to the FHI-aims standard output as a post-processing step. \\[1.0ex]
}
This is generally useful, but the dipole moment is particularly needed
to compute molecular oscillator strengths for individual vibrational
frequencies.

Note that, for charged systems, the electrical dipole is defined with
respect to the (possibly arbitrary) origin of the file
\texttt{geometry.in}, rather than an origin within the system
itself. Thus, charged systems will yield different dipole moments
for different choices of origin, but the important dipole
\emph{differences} needed to compute, e.g., oscillator strengths via
finite diffences remain well-defined.

\subkeydefinition{output}{dos}{control.in}
{
  Usage: \keyword{output} \subkeyword{output}{dos} \option{Estart Eend
    n\_points broadening} \\[1.0ex]
  Purpose: Writes the density of states (DOS) to an external file for
    plotting purposes.\\[1.0ex]
  \option{Estart} : Lower bound of the single-particle energy range
    for which the DOS is given. \\
  \option{Eend} : Upper bound of the single-particle energy range
    for which the DOS is given. \\
  \option{n\_points} : Number of energy data points for which the
    DOS is given. \\
  \option{broadening} : Gaussian broadening applied to obtain a smooth
    density of states based on the peaks produced by
    individual states. \\
}
This keyword shares its syntax with \keyword{output}
\subkeyword{output}{atom\_proj\_dos} and \keyword{output}
\subkeyword{output}{species\_proj\_dos}. See also section \ref{band
  and dos plotting} for more details.

Two output files emerge from this option:
\begin{itemize}
  \item \texttt{KS\_DOS\_total\_raw.dat} contains the total DOS components as
    a function of the eigenvalue energy (first column) as used internally in
    FHI-aims. The energy zero is then given by the vacuum level (non-periodic
    systems) or by the $\boldG$=0 component of the long-range Hartree
    potential (periodic systems).
  \item \texttt{KS\_DOS\_total.dat} contains the same
    information, except that the energy zero is shifted to the Fermi energy
    (metallic systems) or valence band maximum (insulators), respectively.
\end{itemize}

When followed by the following keyword

\keyword{dos\_kgrid\_factors}  $n_1$  $n_2$  $n_3$

where $n_1$, $n_2$ and $n_3$ are integers, the dimensions of the k-point grid along
the first, second and third lattice vectors are multiplied by $n_1$, $n_2$ and $n_3$,
 respectively. New eigenvalues are re-calculated
non-selfconsistently on the new denser k-point grid. The new eigenvalues are
then used to plot an improved (so-called perturbative) density of states.

If no \keyword{dos\_kgrid\_factors} are specified, the original k-point grid is used.

The unit of output for the density of states is $(eV \cdot
V_\text{unit cell})^{-1}$, i.e., number of states per energy unit (eV)
and unit cell volume.

This keyword supports spin-orbit coupling.  When spin-orbit coupling is
enabled, the file(s) containing the spin-orbit-coupled DOS will have the
default filename(s) and the file(s) containing the scalar-relativistic
(i.e. no SOC) band structure will have an additional suffix ".no\_soc".

\subkeydefinition{output}{eigenvectors}{control.in}
{
  \noindent
  Usage: \keyword{output} \subkeyword{output}{eigenvectors} \\[1.0ex]
  Purpose: Writes the actual Kohn-Sham eigenvectors $c_{il}$ into
    separate files for each spin channel. \\[1.0ex]
  Restriction: This functionality is best tested for periodic systems
  with \keyword{KS\_method} \texttt{lapack} at present. For periodic systems,
  it has no effect unless specific band structure output is requested through
  \keyword{output} \subkeyword{output}{band}. \\[1.0ex]
}
For \emph{non-periodic} systems, this option causes the Kohn-Sham eigenvectors
$c_{il}$ (for basis functions $i$, eigenstates $l$) to be written out for each
  state, whenever the Kohn-Sham eigenvalues
are written. However, at present the non-periodic version is mostly of 'debug'
character. In particular, \keyword{KS\_method} \texttt{scalapack} is not
supported under all circumstances. Please test carefully.

For \emph{periodic} systems, eigenvector output must be requested together
with the \keyword{output} \subkeyword{output}{band} functionality described
above. If \keyword{output} \subkeyword{output}{eigenvectors} is set in
addition, the Kohn-Sham eigenvectors $c_{il}(\mbox{\boldmath$k$})$ will be
written into separate files for each spin channel, \emph{only} for each
$k$-point for which band output was requested.

For each state $l$, the real and imaginary part of
$c_{il}(\mbox{\boldmath$k$})$ will then be written out for each basis function
$i$. Output file names are assigned automatically based on the number of the
output band, and to the specific k-point in that band, to which they pertain,
e.g.:
\\[1.0ex]
\texttt{KS\_eigenvectors.band\_$\ast$.kpt\_$\ast$.out} \\[1.0ex]
In addition to the eigenvectors themselves, these files also contain as header
information:\\[-2.0ex]
\begin{itemize}
  \item the \emph{relative} coordinates of the $k$-point in question (in
    units of the reciprocal lattice vectors of the structure in question) \\[-2.0ex]
  \item Information on the identity (atom number and angular momentum) for
    each basis function \\[-2.0ex]
  \item The Kohn-Sham eigenvalue and occupation number of each state. \\[-2.0ex]
\end{itemize}
The listed eigenvectors pertain to the superimposed, Bloch-like basis
functions (with phase factors!) $\chi_{i,k}(\mbox{\boldmath$r$})$ as defined
through Eq. (22) of the FHI-aims Computer Physics Communications description,
Ref. \cite{Blum08}.

Note that the \keyword{output} \subkeyword{output}{aitranss} keyword
provides one further option to print eigenvectors and other
information for any non-periodic system, serial or parallel (scalapack
or elpa).

This keyword will output scalar-relativistic eigenvectors.  For spin-orbit-coupled
eigenvectors, please see the \subkeyword{output}{soc\_eigenvectors} keyword.


\subkeydefinition{output}{elpa\_timings}{control.in}
{
  \noindent
  Usage: \keyword{output} \subkeyword{output}{elpa\_timings}  \\[1.0ex]
  %
  Purpose: Writes timings for the parallel Elpa eigenvalue solver to standard
  output.  \\
}

\subkeydefinition{output}{elsi\_log}{control.in}
{
  \noindent
  Usage: \keyword{output} \subkeyword{output}{elsi\_log}  \\[1.0ex]
  %
  Purpose: Output ELSI runtime information in a JSON format. \\
}
When this flag is enabled, the ELectronic Structure Infrastructure
(ELSI) will output information every time it is used to solve the Kohn-Sham
eigenvalue problem, including timings for the ELSI invocation, a list of
values for relevant variables, and versioning information.

This information is stored in a JSON format and is written to the file
\texttt{elsi\_log.json} using the FortJSON library, which is distributed with
FHI-aims and ELSI and is built automatically.  Only task 0 will output this
file, and it will be overwritten every time the SCF cycle is reinitialized
(including geometry relaxations and MD steps.)

To output information from an FHI-aims calculation in a JSON format, please
use the \keyword{output} \subkeyword{output}{json\_log} keyword.

\subkeydefinition{output}{grids}{control.in}
{
  \noindent
  Usage: \keyword{output} \subkeyword{output}{grids} \\[1.0ex]
  Purpose: Writes to separate files: (i) the
    \subkeyword{species}{radial\_base} integration grid shells for each
    species incl. integration weights, and (ii) the full
    three-dimensional grid point locations incl. integration weights. \\[1.0ex]
}

\subkeydefinition{output}{h\_s\_matrices}{control.in}
{
  \noindent
  Usage: \keyword{output} \subkeyword{output}{h\_s\_matrices} \\[1.0ex]
  Purpose: Writes the Hamiltonian and overlap matrices $s_{ij}$ and
  $h_{ij}$ into separate files. The output format has one line per
  matrix entry. On this line the first column is the row index of the
  entry, the second column the column index of the entry and the third
  column is the value of the entry. Only the upper triangle and the
  diagonal of the symmetric matrix is written. \\[1.0ex]
  This functionality behaves slightly differently in periodic vs.~cluster
  calculations. In the cluster case $s_{ij}$ and $h_{ij}$ are written
  out in keeping with the \keyword{packed\_matrix\_format}. In the
  periodic case only the Gamma-point Hamiltonian is written out and only as
  a dense matrix regardless of \keyword{packed\_matrix\_format} settings.\\[1.0ex]
}

\subkeydefinition{output}{hamiltonian\_matrix}{control.in}
{
  \noindent
  Usage: \keyword{output} \subkeyword{output}{hamiltonian\_matrix} \\[1.0ex]
  Purpose: Writes out the k-point dependent complex hamiltonian matrix
    for a periodic system for those k-points for which band structure
    output was requested. \\[1.0ex]
  Restriction: For periodic systems only, and scalapack is not
  supported. Specific band structure output must be requested through
  \keyword{output} \subkeyword{output}{band}. \\[1.0ex]
}
For \emph{periodic} systems, this option allows to write out the
k-dependent (Bloch) hamiltonian matrices that correspond to the set of
k-points requested with the \keyword{output} \subkeyword{output}{band}
keyword. Both spin channels are written into the same file.

\subkeydefinition{output}{ks\_coulomb\_integral}{control.in}
{
  \noindent
  Usage: \keyword{output} \subkeyword{output}{ks\_coulomb\_integral} \\[1.0ex]
  Purpose: Writes the Coulomb integrals matrices $<ij|V|kl>$ into the file named
  \rm{coulomb\_integrals\_mo.out}. The output format has one line per
  matrix entry. On this line the four columns are the indices $i, j, k, l$, denoting
  the KS molecular orbitals.
  The last column is the value of the entry. The whole matrix is written out.
  This functionality works only for cluster calculations at this point.
}

\subkeydefinition{output}{hessian}{control.in}
{
  \noindent
  Usage: \keyword{output} \subkeyword{output}{hessian} \\[1.0ex]
  Purpose: Writes the initial Hessian approximation for the structure
  optimizer into the file \texttt{hessian.out}. \\[1.0ex]
}

\subkeydefinition{output}{hirshfeld}{control.in}
{
  \noindent
  Usage: \keyword{output} \subkeyword{output}{hirshfeld} \\[1.0ex]
  Purpose: Produces a Hirshfeld analysis of partial charges and
    moments on each atom. \\[1.0ex]
    Restriction: Currently disabled for Hartree-Fock and some other
    functionals. Support for this can easily be added by commenting
    out the ``stop'' line in the source code, except that the
    Hirshfeld analysis is then based on the DFT-LDA free atom density
    $n_\text{at}^\text{free}(r)$.
    For the hybrid functionals HSE, PBE0, and B3LYP, the underlying
    densities used for the partitioning are free-atom PBE and BLYP
    densities, respectively.  \\[1.0ex]
}
Defining ``atoms-in-molecules'' is a classic, intuition based problem;
one would like to associate individual (partial) charges or moments
with individual atoms in a bonded structure. This process is by
necessity not uniquely definable (molecules \emph{are} not atoms, and
the are no rigorously defined boundaries between atoms). Nonetheless,
much chemical intuition is based on such a concept.

Hirshfeld's \cite{Hirshfeld77} ``atoms-in-molecules'' partitioning
relies on the same idea as the partitioning of our charge density for
the electrostatic potential (Eq. \ref{Eq:mp}), using the free-atom
electron density $n_\text{at}^\text{free}(r)$ as the weight function
$g_\text{at}(\boldr)$ in Eq. (\ref{Eq:part}). Since (for a given
functional), we know the spherical $n_\text{at}^\text{free}(r)$
exactly, this analysis remains well-defined even as external
circumstances such as the basis set change. However, the resulting
values are still qualitative in the sense that Hirshfeld's
\cite{Hirshfeld77} ``atoms-in-molecules'' partitioning is just one
among many other prescriptions that have been suggested in the
literature. Any ghost atoms in the system are skipped for this analysis.
This could lead to ``missing'' charge if the ghost atoms carried any
nonnegligible charge, but that should happen only in abnormal systems.

Note that the \keyword{output} \subkeyword{output}{hirshfeld} keyword itself
only writes a Hirshfeld analysis for the final geometry of an FHI-aims
run, not, for
instance, for intermediate molecular dynamics steps. This ensures that the
Hirshfeld analysis does not accidentally clutter an output file with large
amounts of data if the \keyword{output\_level} \texttt{MD\_light}
output level is set. Output for every geometry can be accomplished
with the \keyword{output} \subkeyword{output}{hirshfeld\_always}
keyword below.

\subkeydefinition{output}{hirshfeld\_always}{control.in}
{
  \noindent
  Usage: \keyword{output} \subkeyword{output}{hirshfeld\_always} \\[1.0ex]
  Purpose: Writes out a Hirshfeld analysis at every geometry step of a run. \\[1.0ex]
}

This keyword ensures that a Hirshfeld analysis is written at every
geometry step of a FHI-aims run, not just after the final step.
If \keyword{output} \subkeyword{output}{hirshfeld-I} is requested
together with \keyword{output} \subkeyword{output}{hirshfeld\_always},
results from both the normal and the iterative Hirshfeld analysis are
written at every step.

\subkeydefinition{output}{hirshfeld-I}{control.in}
{
  \noindent
  Usage: \keyword{output} \subkeyword{output}{hirshfeld-I} \\[1.0ex]
  Purpose: Produces an ``iterative Hirshfeld'' analysis of partial charges and
    moments on each atom. \\[1.0ex]
}
Similar functionality to the normal \keyword{output}
\subkeyword{output}{hirshfeld} Hirshfeld analysis -- see the comments
for that keyword -- except that in the ``iterative Hirshfeld'' \cite{Bultinck07}
analysis the partition weights are changed. Here, the partitioning
densities are not those of neutral atoms but rather those of ions with
the same formal charge as the formal charge determined by the
``iterative Hirshfeld'' analysis.

This keyword is implemented but has not seen much production use. It
is therefore not guaranteed that it will always work or that the
results will always make sense or even be in line with the original
``iterative Hirshfeld'' publication.\cite{Bultinck07} All may be well, but if you do
use the functionality, please check very carefully that the results
appear to be correct.

\subkeydefinition{output}{json\_log}{control.in}
{
  \noindent
  Usage: \keyword{output} \subkeyword{output}{json\_log}  \\[1.0ex]
  %
  Purpose: Output FHI-aims runtime information in a JSON format. \\
}

When this flag is enabled, FHI-aims will output information in a JSON format,
which may be easily parsed by your favorite post-processing language (or Python).

This feature was added in 2018, and much of the functionality in FHI-aims
outside of the main SCF cycle will not write out any information.  A partial
list of quantities which will be written includes:
\begin{itemize}
  \item Initial/final geometries
  \item Versioning information
  \item Runtime settings: number of basis functions, number of k-points, etc.
  \item SCF convergence evolution
  \item Mulliken decompositions (when calculated)
  \item Domain decomposition (a.k.a. batch partitioning) statistics
  \item Total energies
  \item HOMO/LUMO levels (on the SCF k-grid)
  \item Fundamental gap (on the SCF k-grid)
  \item Timings
\end{itemize}

This information is written to the file \texttt{aims.json} using the FortJSON
library, which is distributed with FHI-aims and built automatically.  Only task
0 will output this file.  Unlike the ELSI JSON log (which is written out using
the \keyword{output} \subkeyword{output}{elsi\_log} keyword), this JSON log will
persist through SCF reinitialization, geometry relaxation steps, and MD steps;
it will only be overwritten when a new FHI-aims calculation is performed.

To output information from ELSI directly in a JSON format, please use the
\keyword{output} \subkeyword{output}{elsi\_log} keyword.

\subkeydefinition{output}{k\_eigenvalue}{control.in}
{
  \noindent
  Usage: \keyword{output} \subkeyword{output}{k\_eigenvalue}
  \option{number} \\[1.0ex]
  Purpose: For periodic structures, determines for how many $k$ points
    FHI-aims will write the electronic eigenvalues
    $\epsilon_l(\boldk)$. \\[1.0ex]
  \option{number} is an integer number. Default: 1. \\
}
Eigenvalues will only be written for the first \option{number}
$k$-points by default. For dense $\boldk$-grids, the sheer number of
$k$-points simply gets too large to allow for a full output.

\subkeydefinition{output}{k\_point\_list}{control.in}
{
  \noindent
  Usage: \keyword{output} \subkeyword{output}{k\_point\_list} \\[1.0ex]
  Purpose: For periodic geometries only, this option writes a complete
  listing of the reciprocal space coordinates of all $k$-points in the
  calculation to the standard output file. \\
}
The $k$-point coordinates are written in units of the reciprocal
lattice vectors of the structure. This option is also the default when
using \keyword{output\_level} \texttt{full} .

\subkeydefinition{output}{matrices\_2005}{control.in}
{
  \noindent
  Usage: \keyword{output} \subkeyword{output}{matrices\_2005} \\[1.0ex]
  Purpose: Writes the Hamiltonian and overlap matrices $s_{ij}$ and
  $h_{ij}$ into separate files. The output format is the legacy format
  where the entries of the matrix are written out in five-by-five
  blocks. \\[1.0ex]
  Restriction: This functionality is unavailable for periodic systems,
    or if a \keyword{packed\_matrix\_format} is used. \\[1.0ex]
}

\subkeydefinition{output}{matrices\_parallel}{control.in}
{
  \noindent
  Usage: \texttt{output matrices\_parallel} \option{types} [\option{format}]
  \\[1.0ex]
  %
  Purpose: Writes Scalapack/Elpa matrices into separate files.  \\[1.0ex]
  %
  \option{types} is a string that determines the type of matrices that are
  written.  \\[1.0ex]
  %
  \option{format} (optional) is a string that determines the format of the
  output.  \\
}
Depending on option ``\option{types}'', the upper part of up to three different
matrices is written into separate files. If \option{types} is ``\texttt{n}''
(without quotes), no matrices are written. If \option{types} is set to
``\option{h}'', then the Hamiltonian is written. The choice ``\option{o}''
causes the overlap matrix to be output. Finally, ``\option{s}'' refers to the
system matrix from which the eigenvalues are calculated. The last three options
can be combined. For example, ``\option{ho}'' means Hamiltonian and overlap.

The format of the files can be controlled with the optional parameter
\option{format}. Possible values are ``\option{asc}'' for ASCII output (default)
and ``\option{bin}'' for binary output.


\subkeydefinition{output}{memory\_tracking}{control.in}
{
  \noindent
  Usage: \keyword{output} \subkeyword{output}{memory\_tracking} \\[1.0ex]
  Purpose: Outputs all tracked allocations and deallocations. \\[1.0ex]
  Restriction: A large number of allocations and deallocations in FHI-aims are
    currently not tracked. \\[1.0ex]
}


\subkeydefinition{output}{mulliken}{control.in}
{
  \noindent
  Usage: \keyword{output} \subkeyword{output}{mulliken} \\[1.0ex]
  Purpose: Produces a Mulliken analysis of the occupation of each atom
    and its angular momentum channels in terms of the basis
  used. \\[1.0ex]
  Restriction: This option is available as a per-atom summary only if
    ScaLapack is used. \\[1.0ex]
}
Defining ``atoms-in-molecules'' is a classic, intuition based problem;
one would like to associate individual (partial) charges or moments
with individual atoms in a bonded structure. This process is by
necessity not uniquely definable (molecules \emph{are} not atoms, and
the are no rigorously defined boundaries between atoms). Nonetheless,
much chemical intuition is based on such a concept.

A classic ``atoms-in-molecules'' concept is the Mulliken analysis
\cite{Mulliken55}, which defines electronic occupations of atoms by
projected occupations into the localized basis functions associated
with them (see the standard literature for exact definitions and use).

In short, when so requested, FHI-aims will provide a decomposition of
the electronic density per atom, angular momentum channel, and
possibly spin channel, allowing to deduce approximate partial
charges. The summarized Mulliken analysis is written into the standard
output stream, while a separate file \texttt{Mulliken.out} contains
detailed state-by-state projected electron occupations.

Note that a Mulliken analysis is somewhat ill-defined because of basis
function overlap; thus electrons can be counted to one atom or another
at will. For small basis sets, a Mulliken analysis may still yield
qualitatively reasonable numbers, but it becomes increasingly
ill-defined as the atom-centered basis sets approach basis set
completeness.

This keyword supports spin-orbit coupling.  When spin-orbit coupling is
enabled, the file(s) containing the spin-orbit-coupled Mulliken analysis
will have the default filename(s) and the file(s) containing the
scalar-relativistic (i.e. no SOC) Mulliken analysis will have an additional
suffix ".no\_soc".

\subkeydefinition{output}{nuclear\_potential\_matrix}{control.in}
{
  \noindent
  Usage: \keyword{output} \subkeyword{output}{nuclear\_potential\_matrix} \\[1.0ex]
  Purpose: Writes the matrix elements of only the bare electron-nuclear
  potential in the current basis sto to a file
  \\[1.0ex]
  Restriction: This functionality is unavailable for periodic systems,
    or if a \keyword{packed\_matrix\_format} is used. \\[1.0ex]
}
This can be useful if the FHI-aims basis functions are needed for a further,
  separate purpose (Quantum Monte Carlo etc) but please note that the
  integration accuracy for the Coulomb singularity near the nuclei must be
  higher than in our standard calculations (where the singularity is cancelled
  by the kinetic energy), so increasing
  \subkeyword{species}{radial\_multiplier} is in order.

\subkeydefinition{output}{onsite\_integrands}{control.in}
{
  \noindent
  Usage: \keyword{output} \subkeyword{output}{onsite\_integrands} \\[1.0ex]
  Purpose: Writes out onsite integrands for all radial functions on
           the code's internal 'radial' and 'logarithmic' grids. \\[1.0ex]
}
Since August 2013, FHI-aims verifies the accuracy of its 'radial' integration
grid (the sparse grid of atom-centered radial shells around each atom which
is part of the definition of its three-dimensional, overlapping atom-centered
integration grids) in comparison to integrals on the dense 'logarithmic' grid
which is used to set up the spherical free atom, all radial functions etc. in
one dimension.

These integrals take the form
\begin{equation}
 \int d^3r \phi_{i}(\boldr) \hat{H} \phi_{i}(\boldr) = \int dr
 \left[ f(r) \right] \times \text{angular integral} .
\end{equation}
With our usual definition of basis functions,
\begin{equation}
  \phi_i(r) = \frac{u_i(r)}{r} Y_{lm}(\Omega) \quad ,
\end{equation}
we get:
\begin{equation}
  f(r) = u_i(r) \cdot \left[ -\frac{1}{2} u_i^{\prime\prime}(r) +
    \frac{1}{2} \frac{l(l+1)}{r^2} u_i(r) + v(r)u_i(r) \right]
\end{equation}
in the non-relativistic case. In the case of scaled ZORA or atomic
ZORA scalar relativity, the kinetic energy part is modified and the
integrand reads:
\begin{eqnarray}
  f(r) = u_i(r) & \cdot & \left [
     \frac{2 c^2}{2
      c^2 - v(r)} \cdot \left( -\frac{1}{2} u_i^{\prime\prime}(r) +
    \frac{1}{2} \frac{l(l+1)}{r^2} u_i(r) \right)  \right. \\ \nonumber
    & \, & \left. - \frac{c^2}{(2 c^2 - v(r))^2} \cdot v^\prime(r)
    \cdot \left(u_i^\prime(r) - \frac{u_i(r)}{r} \right) + v(r)u_i(r) \right] \, .
\end{eqnarray}
These are the integrands to test both the 'logarithmic' grid and the
'radial' grid around each atom, where $v(r)$ is set to be the
one-dimensional potential of a spherical free atom as calculated at
the outset of each run.

If \keyword{output} \subkeyword{output}{onsite\_integrands} is set to
be true, the actual integrands $f(r)$ and various of their parts are
printed for each radial function. This is mainly useful for debugging
purposes, to understand what we are integrating for a given basis
function choice. Especially for contracted Gaussian basis functions,
$f(r)$ can look quite ugly near the nucleus.

The files that contain the actual integrand $f(r)$ defined above are
called \\
\texttt{Onsite\_r2\_phi\_h\_phi\_log.(function).dat} and \\
\texttt{Onsite\_r2\_phi\_h\_phi\_rad.(function).dat} for the logarithmic and
radial grids, respectively, with ``(function)'' indicating the element
and the radial function number in the order used by FHI-aims (for
instance, the \keyword{output} \subkeyword{output}{basis} keyword uses
the same order to output the radial functions used in the code). Units
are {\AA} for the radial coordinate, but atomic units (Ha/bohr$^3$)
for the integrand itself.

\subkeydefinition{output}{overlap\_matrix}{control.in}
{
  \noindent
  Usage: \keyword{output} \subkeyword{output}{overlap\_matrix} \\[1.0ex]
  Purpose: Writes out the k-point dependent complex overlap matrices
    for a periodic system for those k-points for which band structure
    output was requested. \\[1.0ex]
  Restriction: For periodic systems only, and scalapack is not
  supported. Specific band structure output must be requested through
  \keyword{output} \subkeyword{output}{band}. \\[1.0ex]
}
For \emph{periodic} systems, this option allows to write out the
k-dependent (Bloch) overlap matrices that correspond to the set of
k-points requested with the \keyword{output} \subkeyword{output}{band}
keyword. If \keyword{output} \subkeyword{output}{eigenvectors} is set in
addition, the Kohn-Sham eigenvectors $c_{il}(\mbox{\boldmath$k$})$ will be
written into separate files for each spin channel, \emph{only} for each
$k$-point for which band output was requested.

\subkeydefinition{output}{ovlp\_spectrum}{control.in}
{
  \noindent
  Usage: \keyword{output} \subkeyword{output}{ovlp\_spectrum} \\[1.0ex]
  Purpose: Writes the non-singular part of the eigenvalue spectrum of
    the overlap matrix to the FHI-aims standard output. \\[1.0ex]
  Restriction: Works only for the cluster case, and only for
    \keyword{KS\_method} \texttt{lapack}. \\[1.0ex]
}
This option can help show if (or if not) the chosen basis set for the full
system is close to ill-conditioning (see the comments for keyword
\keyword{basis\_threshold}).

\subkeydefinition{output}{postscf\_eigenvalues}{control.in}
{
  \noindent
  Usage: \keyword{output} \subkeyword{output}{postscf\_eigenvalues} \\[1.0ex]
  Purpose: For periodic systems, writes all Kohn-Sham eigenvalues on a
    potentially dense k-space grid to an ASCII file 'Final\_KS\_eigenvalues.dat'. \\[1.0ex]
  Restriction: Works only for periodic systems. Does not work when keyword
    \keyword{use\_local\_index} is set. \\[1.0ex]
}
If this keyword is set, the eigenvalues and occupation numbers for a periodic system are
recomputed and written to a file 'Final\_KS\_eigenvalues.dat' after the s.c.f. calculation
(and, possibly, relaxation, dynamics etc.) is complete. A denser $k$-space grid than during
the original s.c.f. calculation can be specified using the \keyword{dos\_kgrid\_factors} keyword.

Note that the resulting output file can become very large. See the header of the
'Final\_KS\_eigenvalues.dat' for details and for units used.

Note that this additional calculation is done using serial lapack solutions
of the eigenvalue problems for individual k-points on individual CPU cores.
This always works but will create memory problems as the system size increases,
simply because local copies of all matrices are kept on single CPUs. For large
systems, our usual, more sophisticated parallelization strategies have not yet
been copied over to this routine.

\subkeydefinition{output}{quadrupole}{control.in}
{
  \noindent
  Usage: \keyword{output} \subkeyword{output}{quadrupole} \\[1.0ex]
  Purpose: Calculates and writes the electrical quadrupole moment of the
    structure to the FHI-aims standard output as a post-processing step. \\[1.0ex]
}

\subkeydefinition{output}{rho\_multipole}{control.in}
{
  \noindent
  Usage: \keyword{output} \subkeyword{output}{rho\_multipole} \\[1.0ex]
  Purpose: Writes the partitioned atom-centered charge multipole
    components $\delta\tilde{n}_{\text{at},lm}(r)$ to individual files
    for each atom, $l$, and $m$ (see Eq. \ref{Eq:mp}). \\[1.0ex]
}

\subkeydefinition{output}{soc\_eigenvalues}{control.in}
{
  \noindent
  Usage: \keyword{output} \subkeyword{output}{soc\_eigenvalues} \\[1.0ex]
  Purpose:  Writes the SOC-perturbed eigenvalues at every k-point of the SCF k-grid
    to an output file named \texttt{SOC\_eigenvalues.dat}.  This keyword will not enable
    spin-orbit coupling;  if spin-orbit coupling is not enabled via the
    \keyword{include\_spin\_orbit} keyword, this keyword will be ignored.  \\[1.0ex]
}

\subkeydefinition{output}{species\_proj\_dos}{control.in}
{
  Usage: \keyword{output} \subkeyword{output}{species\_proj\_dos} \option{Estart Eend
    n\_points broadening} \\[1.0ex]
  Purpose: Writes an projected, angular-momentum resolved partial
    density of states (pDOS) averaged over all atoms of each
    \keyword{species}. \\[1.0ex]
  \option{Estart} : Lower bound of the single-particle energy range
    for which the pDOS are given. \\
  \option{Eend} : Upper bound of the single-particle energy range
    for which the pDOS are given. \\
  \option{n\_points} : Number of energy data points for which the
    pDOS are given. \\
  \option{broadening} : Gaussian broadening applied to obtain a smooth
    partial density of states based on the peaks produced by
    individual states. \\
}
This option is based on a Mulliken Analysis and shares its syntax with
\keyword{output} \subkeyword{output}{dos} and \keyword{output}
\subkeyword{output}{atom\_proj\_dos}. See also section \ref{band
and dos plotting} for more details.

Different from the
\subkeyword{output}{atom\_proj\_dos} option, this option writes its
output averaged over all atoms of each \keyword{species} defined in
\texttt{control.in} and used in \texttt{geometry.in}. This provides a
quick handle to obtain averaged pDOS's for well-defined subgroups of
individual atoms, e.g., those of a given layer of a slab, by simply
defining a separate \keyword{species} for the desired atoms in
\texttt{control.in}.

There are two types of output files for each atom:
\begin{itemize}
  \item \texttt{\emph{species}\_l\_proj\_dos\_raw.dat}, where \emph{species}
    denotes the \keyword{species} name used in \texttt{geometry.in} and
    \texttt{control.in}. This file
    contains the total and angular-momentum resolved DOS components as a
    function of eigenvalue energy (first column) as used internally in
    FHI-aims. The energy zero is then given by the vacuum level (non-periodic
    systems) or by the $\boldG$=0 component of the long-range Hartree
    potential (periodic systems).
  \item \texttt{\emph{species}\_l\_proj\_dos.dat}, which gives the same
    information, except that the energy zero is shifted to the Fermi energy
    (metallic systems) or valence band maximum (insulators), respectively.
\end{itemize}
Note that projected densities of states such as given here must be
based on some kind of projection orbitals, the choice of which is
somewhat arbitrary by necessity. This is thus a tool for
\emph{qualitative} analyses.

In FHI-aims, we project directly on
the atom-centered angular-momentum components as defined by the
\emph{overlapping} basis set. This definition becomes the more arbitrary te
larger the basis set, just like a \subkeyword{output}{mulliken}
analysis. The closer the full basis comes to completeness, the
more ambiguous will a \subkeyword{output}{mulliken}-like analysis become,
since it may not be \emph{a priori} clear which electrons should be counted
towards the basis functions of one atom vs. those of another atom. Thus, do
\emph{not} expect a pDOS to simply converge as the basis set size is
increased; use it as a qualitative indicator of trends, but nothing more.


This keyword supports spin-orbit coupling.  When spin-orbit coupling is
enabled, the file(s) containing the spin-orbit-coupled DOS will have the
default filename(s) and the file(s) containing the scalar-relativistic
(i.e. no SOC) DOS will have an additional suffix ".no\_soc".

\subkeydefinition{output}{v\_eff}{control.in}
{
  \noindent
  Usage: \keyword{output} \subkeyword{output}{v\_eff} \\[1.0ex]
  Purpose: Writes the local effective potential $v_\text{eff}(\boldr)$
    at each integration grid point $\boldr$ to a file
    \texttt{v\_eff.dat}. \\[1.0ex]
}
Note that the meaning of this effective potential depends on the
\keyword{xc} option used. For DFT-LDA, this is simply the full local
potential. For gradient-corrected (GGA) functionals, the gradient
partial derivatives of the exchange-correlation functional are
\emph{not} included in the potential, as they are treated separately
by integration by parts (see Ref. \cite{Blum08}). For hybrid
functionals or Hartree-Fock, the exchange part of the potential is of
course not included.

Note also that this output does \emph{not} happen on a uniform
grid. For further processing, a proper visualization tool is needed,
and/or an interpolation onto a uniform grid must be done.

\subkeydefinition{output}{v\_hartree}{control.in}
{
  \noindent
  Usage: \keyword{output} \subkeyword{output}{v\_hartree} \\[1.0ex]
  Purpose: Writes the electrostatic (Hartree) potential multipole
    components $\delta\tilde{v}_{\text{at},lm}(r)$ to individual files
    for each atom, $l$, and $m$. \\[1.0ex]
}

%\subkeydefinition{output}{vacuum\_potential}{control.in}
%{
%  \noindent
%  Usage: \keyword{output} \subkeyword{output}{vacuum\_potential}
%    \option{z} \option{Nx} \option{Ny} \\[1.0ex]
%  Purpose: For the purpose of defining the work function of a periodic
%    slab, requests the $x$-$y$ averaged value of the electrostatic
%    potential (reciprocal-space part of the Ewald method only!) at a
%    fixed $z$ value. \\[1.0ex]
%  \option{z} is the requested $z$ value (in {\AA}). \\
%  \option{Nx} is the number of $x$ divisions of the projected unit
%    cell at $z$ for the averaging. \\
%  \option{Ny} is the number of $y$ divisions of the projected unit
%    cell at $z$ for the averaging. \\
%}
%In a surface (or slab) calculation, knowing the ``vacuum''
%electrostatic potential far away from the slab is useful since this
%can be used to determine the work function with respect to the Fermi
%level of the slab. Internally, FHI-aims only determines the
%electrostatic potential $v_\text{es}(\boldr)$  at the points of the
%overlapping, atom-centered integration grid, which yields
%electrostatic potential values at somewhat arbitrary coordinate
%values in the vacuum of a slab.

%Ideally, what one would like (in the absence of any dipolar fields
%between slab surfaces) is the value of the electrostatic potential far
%into the vacuum, at a point where the real-space part of the (Ewald)
%electrostatic potential has already died down. This can be found by
%the present option, in the following way:
%\begin{itemize}
%  \item Define a slab geometry with a sufficiently large vacuum, so
%    that the (exponentially damped) real-space part of the Hartree
%    potential safely does not extend all the way through.
%  \item The slab surface must be periodic in the $x$-$y$ direction.
%  \item Choose a \option{z} value safely in the far-field limit of the
%    vacuum, e.g., at the center.
%  \item Choose \option{Nx} and \option{Ny} values that are
%    sufficiently large to define a grid at $z$ that safely averages
%    out any residual fluctuations of the electrostatic potential at
%    \option{z}
%  \item The code then produces \emph{only} the $x$-$y$-averaged
%    reciprocal-space part of the electrostatic potential at
%    the coordinate $z$ in the vacuum.
%\end{itemize}



\subkeydefinition{output}{zero\_multipoles}{control.in}
{
  \noindent
  Usage: \keyword{output} \subkeyword{output}{zero\_multipoles} \\[1.0ex]
  Purpose: Prints out the partial charges associated with the
    multipole electrostatic potential of each atoms in each
    s.c.f. iteration. \\[1.0ex]
}
The resulting values are ``atoms-in-molecules'' like partial charges
assigned to each atom, similar to a \subkeyword{output}{hirshfeld}
partitioning (but not identical because a different partitioning
function may be used as the \keyword{hartree\_partition\_type}).


\subkeydefinition{output}{dgrid}{control.in}
{
  \noindent
  Usage: \keyword{output} \subkeyword{output}{dgrid} \\[1.0ex]
  Purpose: Dumps the Wave function of the final s.c.f. cycle on
  disk into the file \emph{dgrid\_aims.dat}. This files serves
  as an interface to the DGrid program.\\[1.0ex]
}
This option serves as interface to the DGrid program (written by Miroslav Kohout),
which is available free of charge at \url{https://www2.cpfs.mpg.de/~kohout/dgrid.html}.
DGrid allows to employ various electronic structure and chemical bonding analysis
alogorithms in real space,~e.g.,~the QTAIM (Quantum Theory of Atoms in Molecules
of R.F.W. Bader) or the ELI-D/ELF method. After installation of DGrid (Version $\ge 5.0$)
the command \emph{dgrid dgrid\_aims.dat} converts the interface wave function file
\emph{dgrid\_aims.dat} into a new file \emph{dgrid\_aims.fhi} with native DGrid file
format. This file provides the basis to all capabilities of the DGrid program
described in detail in its manual at \url{https://www2.cpfs.mpg.de/~kohout/Documents/dgrid-html/dgrid.html}.


