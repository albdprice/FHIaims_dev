\section{Defaults for chemical elements: species\_defaults}
\label{Sec:species}

FHI-aims requires exactly two input files, located in the same directory where
a calculation is started: \texttt{control.in} and \texttt{geometry.in}. Both
files can
in principle be specified from scratch for every new calculation, using the
keywords listed in Chapter \ref{Ch:full}. However, choosing the central
computational settings consistently for series of calculations greatly
enhances the accuracy of any resulting energy differences (error cancellation).

In FHI-aims, the key parameters regarding computational accuracy are actually
subkeywords of the \keyword{species} keyword of \texttt{control.in},
controlling the basis set, all integration grids, and the accuracy of the
Hartree potential. These settings should of course not be retyped from scratch
for every single calculation; on the other hand, they should remain obvious
to the user, since these are the central handles to determine the accuracy and
efficiency of a given calculation. 

FHI-aims therefore provides preconstructed default definitions for the
important subkeywords associated with different \keyword{species} (chemical
elements) from $Z$=1-102 (H-Md). These can be found in the
\emph{species\_defaults} subdirectory of the distribution, and are built for
inclusion into a \texttt{control.in} file by simple copy-paste. 

For all elements, FHI-aims offers three or four different levels of
\emph{species\_defaults}: 
\begin{itemize}
  \item \emph{light} :
    Out-of-the-box settings for fast prerelaxations, structure searches,
    etc. In our own work, no obvious geometry / convergence errors resulted
    from these settings, and we now recommend them for many household
    tasks. For ``final'' results (meV-level converged energy differences between
    large molecular structures etc), any results from the \emph{light} level
    should be verified with more accurate post-processing calculations,
    e.g. \emph{tight}. 
  \item \emph{intermediate} :
    This level is presently only available for a few elements, but can
    play an important role for large and expensive calculations,
    especially for hybrid functionals. \emph{Intermediate} settings
    use most of the  
    numerical settings from \emph{tight}, but includes basis functions
    between \emph{light} and \emph{tight}. The cost of hybrid
    functionals scales heavily with the number of basis functions
    found on a given atom. Full \emph{tight} settings, which were
    designed with the much cheaper semilocal functionals in mind, can
    be prohibitively expensive for large structures and hybrid density
    functionals. Hybrid DFT results from \emph{intermediate} settings
    are typically completely sufficient for production results, much
    cheaper, and we hope to produce \emph{intermediate} defaults for a
    wider range of elements in coming years.
  \item \emph{tight} : 
    Regarding the integration grids, Hartree
    potential, and basis cutoff potentials, the settings specified here are
    rather safe, intended to provide meV-level accurate energy
    \emph{differences} also for large structures. In the \emph{tight} settings,
    the basis set level is set to \emph{tier} 2 for the light elements 1-10, a
    modified \emph{tier} 1 for the slightly heavier Al, Si, P, S, Cl (the first
    \emph{spdfgd} radial functions are enabled by default),
    and \emph{tier} 1 for all other elements. This reflects the fact that, for
    heavy elements, \emph{tier} 1 is sufficient for tightly converged ground
    state properties in DFT-LDA/GGA, but for the light elements (H-Ne),
    \emph{tier} 2 is, e.g., required for meV-level converged energy
    differences. For convergence
    purposes, the specification of the basis set itself (\emph{tier}
    1, \emph{tier} 2, etc.) may still be 
    decreased / increased as needed. Note that especially for hybrid functionals,
    \emph{tight} can already be very expensive and specific reductions
    of the number of radial functions may still provide essentially
    converged results at a much more affordable cost (see
    \emph{intermediate} settings).
  \item \emph{really\_tight} : 
    Same basis sets (radial functions and cutoff radii) as in the
    \emph{tight} settings, but for the other numerical aspects (grids,
    Hartree potential), settings that are \emph{strongly
    overconverged} settings for most purposes. The idea is that
    \emph{really\_tight} can be used for very specific, manual
    convergence tests of the basis set and other settings -- if really
    needed. \\ 
    Note that the \emph{``tight''} settings are intended to
    provide reliably accurate results for most DFT production
    purposes; and they are not cheap. The absolute total energies for
    \emph{tight} and DFT are in practice converged to some tens of
    meV/atom for most elements. To go beyond, take the
    \emph{really\_tight} settings and increase the basis set or other
    numerical aspects step by step. (Radial function by radial 
    function may often be a good strategy to go.) \emph{We emphasize
    that the \emph{really\_tight} settings should only ever be
    needed for individual, specific tests. They should not be
    needed for any standard production tasks unless you have
    seriously too much CPU time to spend.} \\ Specific differences
    between \emph{tight} 
    and the unmodified \emph{really\_tight} settings: The
    \subkeyword{species}{basis\_dep\_cutoff} keyword is set to zero, a
    prerequisite to approach the converged basis limit. Regarding the
    Hartree potential, \subkeyword{species}{l\_hartree} is set to 8,
    and the maximum number of angular grid points per radial
    integration shell is increased to 590. \\  
    Note that there can still
    be corner cases where you may want to test some numerical setting
    beyond \emph{really\_tight}. Mostly, these are custom scenarios or
    things beyond standard FHI-aims calculations of DFT total
    energies. Examples include: The confinement radius for surface
    work functions (should be checked), use of very extended or
    extremely tight Gaussian-type orbital basis functions (e.g., from
    very large Dunning-type basis sets -- the density of the radial
    and angular grids should be checked), or RPA and MP2 calculations,
    which can need very different and often much larger basis sets
    (again, radial and angular grids should be checked). 
\end{itemize}

A separate group of species defaults for light elements (H-Ar) is
available especially for calculations involving explictly correlated
methods (methods other than semilocal and hybrid density functionals): 
\begin{itemize}
  \item \emph{NAO-VCC-nZ} : NAO type basis sets for H-Ar by Igor Ying
    Zhang and coworkers \cite{Zhang2013}. These basis sets
    are constructed according to Dunning's ``correlation consistent''
    recipe. Their intended application is for methods that invoke
    the continuum of unoccupied orbitals explicitly, for instance MP2,
    RPA or $GW$. Note that they were constructed for valence-only
    correlation (hence ``VCC'', valence correlation consistent), i.e.,
    they work best in frozen-core correlated approaches following a
    full s.c.f. cycle (core and valence) to generate the
    orbitals. While NAO-VCC-nZ can be used for ``normal'' density
    functional theory (LDA, GGA, or hybrid functionals), the normal
    ``light'', ``intermediate'', ``tight'' and ``really\_tight'' species defaults are
    more effective in those cases. The advantage of NAO-VCC-nZ over
    GTO basis sets such as the Dunning ``cc'' basis sets is that with
    NAOs, both the behaviour near the nucleus as well as that for the
    tails of orbitals far away from atoms is much more physical. This
    means that we can use more efficient integration grids than for
    GTO basis sets to obtain systematic convergence of the unoccupied
    state space. 
\end{itemize}

The \textit{NAO-J-n} basis sets are designed for the calculation of indirect spin-spin coupling constants (J-couplings):
\begin{itemize}
\item \textit{NAO-J-n} : The basis sets are available for most light elements from H to Cl. Since these are more expensive (tighter grids) than other basis sets, they should only be used for J-couplings. Even then, they should only be placed on atoms of interest, while cheaper basis sets can be used on other atoms. They are constructed by adding tight Gaussian orbitals to the NAO-VCC-nZ basis sets. In order to describe the Gaussian orbitals correctly near the nucleus, tighter grids than normally are required (with \keyword{radial\_multiplier} 8 and \keyword{l\_hartree} 8, among other parameters). Other stages of the calculation, such as geometry relaxation, should be performed with basis sets more suitable for the particular task (using, e.g., the default tight settings).
\end{itemize}

In addition, the \emph{species\_defaults} directory contains a few
more sets of species defaults for special purposes. These can be found
in the \emph{non-standard} subdirectory and include:
\begin{itemize}
  \item \emph{gaussian\_tight\_770} : Species defaults that allow to
    perform calculation with some standard published Gaussian-type
    orbital (GTO) basis sets for elements H-Ar (including basis sets due to
    Pople, Dunning, Ahlrichs and their coworkers). These species
    defaults are meant to allow for exact benchmarks against
    GTO codes such as NWChem. The other
    numerical settings (especially grids) are thus much tighter than
    needed for ``normal'' NAO-type calculations. Note that FHI-aims is
    not optimized for GTO basis sets. We recommend
    NAO-type basis sets, not GTO basis sets, for production
    calculations -- NAO-type basis sets are much easier to handle
    with our techniques and give better accuracy at lower cost. That
    said -- the grid settings in the \emph{gaussian\_tight\_770}
    species defaults are rather overconverged for benchmark
    purposes. One could create much more efficient species defaults for
    GTO basis sets -- but GTOs still would not be as efficient as NAO
    basis sets (at the same level of accuracy).
  \item \emph{Tier2\_aug2} : Example, pioneered by Jan Kloppenburg, of
    basis sets that merge FHI-aims' tier2 basis sets with a very
    reduced set of Gaussian augmentation functions taken from
    Dunning's augmented correlation-consistent basis sets. This
    prescription appears to provide a remarkably accurate but affordable
    foundation to compute neutral (optical) vertical molecular
    excitation energies by linear-response time-dependent density
    functional theory, as well as (thanks to Chi (Garnett) Liu) the
    Bethe-Salpeter Equation.
  \item \emph{light\_194} : This is just an example of how to tune
    down the normal ``light'' basis sets of FHI-aims by reducing the
    integration grid even further. For things like fast
    molecular-dynamics type screening of many structures, this is a
    perfectly viable approach. Examples are provided for
    H-Ne. Obviously, do test the impact of such modifications for your
    own purposes.
\end{itemize}

For calculations that involve the excited state spectrum directly (this
includes $GW$, MP2, or RPA, among others), the numerical settings
from \emph{tight} still perform rather well \emph{if} a
counterpoise correction is performed (i.e., for energy
differences). Still, the basis set size and/or cutoff radii
\emph{must} be converged and carefully verified beyond the settings
specified in \emph{tight}.

To extrapolate the absolute total energy of methods which rely on the
unoccupied state continuum explicitly, e.g., RPA or MP2, 
we recommend using the NAO-VCC-nZ basis sets. These basis sets are
presently available for light elements (H-Ar). 
A popular completeness-basis-set extrapolation scheme is two-point
extrapolation:
\begin{equation*}
	E[\infty]=\frac{E[n_1]n_1^3-E[n_2]n_2^3}{n_1^3-n_2^3}
\end{equation*}
where ``n$_1$'' and ``n$_2$'' are the indicies of NAO-VCC-nZ.
This $1/n^3$ formula was originally proposed for the correlation energy,
but was also used directly for the total energy.
