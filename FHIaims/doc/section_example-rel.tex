\section{Heavy elements ($Z\gtrsim$30): Modifications for scalar relativity}
\label{Sec:rel-example}

Actually, this section could easily apply to all elements. As outlined
in Section \ref{Sec:rel}, it seems that the \keyword{relativistic}
\texttt{atomic\_zora scalar} keyword and the underlying ``atomic
ZORA'' approach as implemented in FHI-aims (Equations (55) and (56) of
Ref. \cite{Blum08}) perform on par with the best available
scalar-relativistic benchmark methods (see
Refs. \cite{Lejaeghereaad3000,Huhn2017_SOC}). Just use this approach
in production, unless there is a specific reason not do do so
(a few methods in FHI-aims that are still being implemented may
initially only support a non-relativistic treatment).

Note that total energy \emph{differences} will be very large between
different relativistic treatments, so it is best to stick to a single
level of scalar relativity for all calculations.

With H and O, the simple H$_2$O testcase of the preceding Section
\ref{Sec:example-Etot} involves only 
elements so light that relativistic effects on their total energies
can still be ignored in production calculations. 

The rule of thumb that relativistic effects do not matter much for
quantities derived from total energies (binding energies, geometries)
holds up to elements number $Z$=20-30 (Ca or Zn), depending on the
reqired accuracy. For example, the DFT-LDA lattice parameter for fcc
Cu is 3.54~{\AA} in the non-relativistic case, but 3.52~{\AA} if
relativistic effects are included. In any case, relativistic effects
should be accounted for in accurate calculations \emph{beyond} these
elements. As described in more detail in Ref. \cite{Blum08}, this
process is handled by FHI-aims at different levels of approximation,
with some overhead resulting compared to the simple non-relativistic
case. 

In a nutshell, the physical impact of relativity for heavy elements
\emph{is} noticeable not just as a total-energy offset, but
importantly in total energy differences, and in particular impacts
also geometries. The underlying reason is that core and 
valence orbitals ``see'' the near-nuclear region (where relativistic
effects are most important) differently, depending on their 
angular momentum $l$. Somewhat simplistically put, the increased
relativistic mass of the electron near the nucleus results in a
tendency for all orbitals to ``shrink'' compared to the
non-relativistic case, but to a different degree for different
orbitals. The shrinkage thus changes not just atom sizes, but also the
nature of bonding itself. A detailed discussion of relativistic
effects is beyond the scope of this manual (and can be found in many
excellent reviews, e.g., Ref. \cite{Pyykko88}) -- but bear in mind
that relativistic effects should not simply be shrugged off!

As noted in Section \ref{Sec:rel}, we recommend ``atomic ZORA'' as the
blanket default where possible, invoked by the keyword

\keyword{relativistic} \texttt{atomic\_zora scalar}

Additionally, the effects of spin-orbit coupling may be included in
energy levels / band structures, densities of states,
etc. post-selfconsistently, i.e., after the self-consistent
calculation is complete. The keyword to do this is:

\keyword{include\_spin\_orbit}

That's it.

We illustrate the practical use of atomic ZORA and spin-orbit coupling
for the Au dimer in DFT-LDA, found in the \emph{testcases/Au\_dimer} 
directory. Importantly, this follows the always recommended two-step
approach for relaxations: First, a \textit{light} prerelaxation (saves
much time for steps of the relaxation algorithm that will be
completely irrelevant for the final result) and second, a
\textit{tight} post-relaxation for final results.

The testcase here uses semilocal DFT, which is relatively cheap. For
the much more costly hybrid density functionals, 
\textit{tight} settings can be prohibitively expensive, and
\textit{intermediate} settings (where available) are often completely
sufficient. 

In the subdirectory \emph{relax\_light}, a quick but
sufficiently accurate prerelaxation is set up, at the \texttt{atomic\_zora}
level. The \texttt{control.in}
file is set up to use \keyword{relativistic} \texttt{atomic\_zora}
\texttt{scalar} and \keyword{relax\_geometry} \texttt{bfgs}
\texttt{1.e-2} to converge forces down to 10$^{-2}$ eV/{\AA}. Since
this is intended to be a quick prerelaxation run (starting with an
arbitrary separation of 3 {\AA} of both atoms), the ``light'' species 
default settings for Au are used for the \keyword{species}
subsettings. Compared to the much more accurate ``tight'' settings,
\begin{itemize}
\item the radial and angular integration grids are significantly reduced, 
\item the Hartree potential expansion is capped at
  \subkeyword{species}{l\_hartree}=4,
\item the cutoff potential onset and width in \subkeyword{species}{cut\_pot}
  are reduced to a minimum that we still consider safe (3.5 {\AA} / 1.5 {\AA},
  respectively),
\item the basis set is the \emph{spdf} section of \emph{tier} 1 only.
\end{itemize}.
Among these settings, the reduction of the basis set to \emph{spdf} has the
biggest impact on the resulting equilibrium geometry,
$d$=2.464~{\AA}. (The difference to the \textit{tight} result below is
quite minor -- 0.012~{\AA}.) 

After this relaxation run is complete, a file \texttt{geometry.in.next\_step} is
written out by FHI-aims. This file contains the final, relaxed geometry from
the ``light'' prerelaxation run just performed, as well as the Hessian matrix
estimate created by the \keyword{relax\_geometry} \texttt{bfgs} relaxation algorithm.
This file can serve as an improved guess for \texttt{geometry.in} in the next, 
typically more expensive ``tight'' post-relaxation step. 

In the subdirectory \emph{postrelax\_tight}, the final geometry from the 
previous step is used as the starting point for an accurate post-relaxation
using the ``tight'' \emph{species\_defaults} settings. Note that,
while all other settings are tightly converged at this level, the basis set
convergence should normally still be tested explicitly. The \emph{tier} 1
level (without the $h$ function) used here for Au is quite sufficient
for an accurate result. The binding 
distance converges after two additional relaxation steps, at $d$=2.452 {\AA}.

We also added in spin-orbit coupling for completeness using the
\keyword{include\_spin\_orbit} keyword.

In conclusion, this section illustrates how a quick pre-relaxation followed 
by a safe calculation of the relevant energy differences can be
combined. The key point in this endeavour is to ensure that the final 
relaxed geometry is accurate with as little computational overhead as
possible. 
