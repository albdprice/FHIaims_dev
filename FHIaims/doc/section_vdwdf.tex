\section{Calculating nonlocal correlation energy within density functional approach}
\label{Sec:vdwdf}

\emph{Warning: This functionality is available in FHI-aims, but has not seen
  extensive testing by ourselves. It should therefore be treated as
  experimental. If you intend to use this functionality, by all means
  ensure that literature results obtained using this functional are
  reproducible using the implementation presented here.} 

There are currently \textbf{two} separate working implementations of
van der Waals DF in FHI-aims:
\begin{itemize}
  \item Sec. \ref{vdwdf-MC}: One version that allows post-processing 
    only (i.e., compute the self-consistent density by another XC
    functional, the evaluate only the vdw-DF energy term again after
    the fact), using a Monte Carlo integration scheme. This version
    works for non-periodic as well as periodic systems. The original
    code was developed in the group of Claudia Ambrosch-Draxl at
    University of Leoben, Austria.
  \item Sec. \ref{vdwdf-TKK}: A second version that relies on an
    analytical integration scheme developed by Simiam Ghan, Andris
    Gulans and Ville Havu at Aalto University in Helsinki. This
    version allows non-self-consistent \emph{and} self-consistent
    usage, as well as gradients (forces). It also has a number of
    numerical convergence parameters that can be adjusted.
\end{itemize}

\subsection{Monte Carlo integration based vdW-DF}
\label{vdwdf-MC}

As a postprocessing step after a self-consistent calculation, the nonlocal
part of the correlation energy can be calculated using the van der Waals
density functional proposed by M. Dion et al.~\cite{Dion04}. This task follows
exactly the recipe presented in the original paper~\cite{Dion04}. This
calculation can be performed by choosing \texttt{ll\_vdwdf} as a
\keyword{total\_energy\_method} (please see Section \ref{Sec:xc}).

\textbf{Important acknowledgment:} The Langreth-Lundqvist functional and basic
Monte Carlo integration scheme used here was made available to us by the group
of Claudia Ambrosch-Draxl and coworkers at University of Leoben, Austria. If
you use this functionality successfully, please cite their work (currently,
Ref. \cite{Nabok08} and ``unpublished'').

In order to perform the calculation, one should define an even
spacing grid (followed by the \texttt{vdwdf} tag explained below), where
the total charge densities of a system obtained after the scf cycle
are projected.

In order to effectively solve the nonlocal correlation energy part,
presented in Equation(1) of ~\cite{Dion04}, the Monte Carlo
integration scheme of Divonne integration method of CUBA library
(Please visit http://www.feynarts.de/cuba for details). 

This means that you should (yourself) compile the CUBA library as an external
dependency to FHI-aims. The alternative Makefile \texttt{Makefile.cuba} the
contains examples of how to link FHI-aims to the CUBA library, and enable the
vdW-LL functional

Parameters
for tuning the performance of Monte Carlo integration are defined
under the \texttt{mc\_int} tag explained below. Also, the kernel values are
tabulated in terms of the parameters in the formula of Equation(14)
of ~\cite{Dion04}. The aims package comes with the tabulated kernel
data (a file called \texttt{kernel.dat}) and the name of the file should be
included in control.in.

\textbf{Necessary input file:} In your FHI-aims distribution, a version of the
\texttt{kernel.dat} file as well as an example control.in file should be located in
subdirectory \emph{src/ll\_vdwdf} . For an actual FHI-aims program run, the
\texttt{kernel.dat} file (currently, \texttt{kernel\_my.dat} is provided) must
be copied to your working directory and must be referenced in your
\texttt{control.in} file, using the \subkeyword{mc\_int}{kernel\_data}
sub-keyword of the \keyword{mc\_int} keyword (see below).

\newpage

\newpage

\subsection*{Tags for general section of \texttt{control.in}:}

\keydefinition{mc\_int}{control.in}
{
  \noindent
  Usage: \keyword{mc\_int} \option{subkeyword(s)} \\[1.0ex]
  Purpose: A line that begins with \texttt{vdwdf} is associated with the Monte
  Carlo integration performed to evaluate the non-local Langreth-Lundqvist
  functional. The \texttt{mc\_int} keyword must be followed \emph{on the same
  line} by a subkeyword that indicates the specific setting made here. \\[1.0ex]  
  \option{subkeyword(s)} are one or more subkeywords or data for the
  Monte Carlo integration. \\
}
To use the Monte Carlo integrated Langreth-Lundqvist functional, more than one
  subkeyword for \keyword{mc\_int} must be specified in the
  \texttt{control.in} file. See below for valid / necessary subkeywords. You
  may also want to check the documentation for the CUBA Monte Carlo
  integration library (http://www.feynarts.de/cuba) that you must have built
  and linked to the FHI-aims code in order to use the Langreth Lundqvist
  functional. 


\keydefinition{vdwdf}{control.in}
{
  \noindent
  Usage: \keyword{vdwdf} \option{subkeyword(s)} \\[1.0ex]
  Purpose: A line that begins with \texttt{vdwdf} is associated with the
  non-local Langreth-Lundqvist functional. The \texttt{vdwdf} keyword must be
  followed \emph{on the same line} by a subkeyword that indicates the specific
  setting made here. \\[1.0ex]  
  \option{subkeyword(s)} are one or more subkeywords or data for the
  Langreth-Lundqvist functional. \\
}
To use the Langreth-Lundqvist functional, more than one subkeyword for
\keyword{vdwdf} must be specified in the \texttt{control.in} file. See below
for valid / necessary subkeywords.


\subsection*{Subtags for \emph{vdwdf} tag in \texttt{control.in}:}
 \subkeydefinition{vdwdf}{cell\_origin}{control.in} {
  \noindent
  Usage: \subkeyword{vdwdf}{cell\_origin} \option{$x$ $y$ $z$} \\[1.0ex]
  \option{$x$ $y$ $z$} indicates the cartesian coordinates (in {\AA})
  of the origin of an even-spacing cubic cell. Default: \texttt{0.0
    0.0
    0.0}.\\
} This option is only valid for a cluster calculation. For the case of
periodic system, a cell origin is automatically determined at the
center of a supercell.

\subkeydefinition{vdwdf}{cell\_edge\_steps}{control.in} {
  \noindent
  Usage: \subkeyword{vdwdf}{cell\_edge\_steps} \option{$N_x$ $N_y$ $N_z$} \\[1.0ex]
  Purpose: the total number of grids in each direction are defined
  by integer numbers, \option{$x$ $y$ $z$}.\\
}

\subkeydefinition{vdwdf}{cell\_edge\_units}{control.in} {
  \noindent
  Usage: \subkeyword{vdwdf}{cell\_edge\_units} \option{$d_{x}$ $d_{y}$ $d_{z}$}. \\[1.0ex]
  Purpose: The real numbers \option{$d_{x}$ $d_{y}$ $d_{z}$} (in
  {\AA}) define the length of grid units in each direction. Therefore,
  the full grid length is ($N_x d_x$,$N_y d_y$, $N_z d_z$).  }

\subkeydefinition{vdwdf}{cell\_size}{control.in} {
  \noindent
  Usage: \subkeyword{vdwdf}{cell\_size} \option{$L_{x}$ $L_{y}$ $L_{z}$} \\[1.0ex]
  Purpose: this defines number of interacting cells in x, y, z
  directions for van der Waals interactions. This option is meaningful
  for periodic calculation.\\[1.0ex] \option{$L_{x}$ $L_{y}$ $L_{z}$}
  are
  integer. Default: \texttt{0 0 0}.\\
}

Note: As a temporary restriction, FHI-aims currently supports only
grids with vectors aligning along $x$, $y$, and $z$ axes.

Calculations for periodic systems: In defining even spacing grid of
a periodic system, only information of cell\_edge\_steps and
cell\_size (if needed) is necessary and other parameters will be
automatically determined from that.

\newpage

\subsection*{Subtags for \emph{mc\_int} tag in \texttt{control.in}:}

\subkeydefinition{mc\_int}{kernel\_data}{control.in} {
  \noindent
  Usage: \subkeyword{mc\_int}{kernel\_data} \option{kernel.dat} \\[1.0ex]
  Purpose: the name of the tabulated kernel file.  }

\subkeydefinition{mc\_int}{output\_flag}{control.in} {
  \noindent
  Usage: \subkeyword{mc\_int}{output\_flag} \option{flag} \\[1.0ex]
  Purpose: this controls output of Monte Carlo integration process,
  level 0 for no output, level 1 for ``reasonable'', and level 3
  prints further the subregion results(if applicable).
  \option{flag} is an integer number. Default: \texttt{0} \\
}

\subkeydefinition{mc\_int}{number\_of\_MC}{control.in} {
  \noindent
  Usage: \subkeyword{mc\_int}{number\_of\_MC} \option{N} \\[1.0ex]
  Purpose: the total number of Monte-Carlo integration steps.\\[1.0ex]
  \option{N} is an integer number. Default: \texttt{5E5}\\
}

\subkeydefinition{mc\_int}{relative\_accuracy}{control.in} {
  \noindent
  Usage: \subkeyword{mc\_int}{relative\_accuracy} \option{E$_{acc}$} \\[1.0ex]
  Purpose: control the accuracy of Monte-Carlo integration performed by Cuba library.\\[1.0ex]
  \option{E$_{acc}$} is a real number. Default: \texttt{1E-16}\\
}

 \subkeydefinition{mc\_int}{absolute\_accuracy}{control.in} {
  \noindent
  Usage: \subkeyword{mc\_int}{absolute\_accuracy} \option{E$_{abs}$} \\[1.0ex]
  Purpose: control the error bar of nonlocal correlation energy.\\[1.0ex]
  \option{E$_{abs}$} is a real number (in the unit of Hartree). Default: \texttt{1E-2}\\
}

\newpage

\subsection{Analytic integration scheme for non-selfconsistent and
  self-consistent vdW-DF}
\label{vdwdf-TKK}

This method calculates the non-local part of the correlation energy as
described in~\cite{Dion04} allowing for both non-self-consistent and
self-consistent treatment. It works for both cluster and periodic
geometries and can be used to compute forces. The implementation as
well as the kernel function are from~\cite{Gulans09}. At each
scf-cycle the following steps are performed:
\begin{itemize}
\item An octree is built to interpolate the current electron density
  to the new integration grid below.
\item To each grid point of the main integration grid another grid of
  similar form is attached. The non-local correlation is then
  integrated on this grid using density and its gradient interpolated
  from the octree. In each node of the tree a tricubic interpolation
  is used.
\end{itemize}
The first step of building the octree is not parallel but the second
step of integration is MPI-parallel the usual way.

\newpage

\subsection*{Tags for general section of \texttt{control.in}:}

\keydefinition{nlcorr\_nrad}{control.in}
{
  \noindent
  Usage: \keyword{nlcorr\_nrad} \option{number}\\[1.0ex] 
  Purpose: Sets
  the number of radial shells used in the integration of the non-local
  correlation potential and energy. Default: \texttt{10}\\
}
\keydefinition{nlcorr\_i\_leb}{control.in} 
{
  \noindent
  Usage: \keyword{nlcorr\_i\_leb} \option{number}\\[1.0ex] 
  Purpose:
  Sets the index of angular Lebedev grid used in the integration of
  the non-local correlation potential and energy. Maximum value
  available is \texttt{15}. Default: \texttt{7}\\ 
}
\keydefinition{vdw\_method}{control.in} 
{
  \noindent
  Usage: \keyword{vdw\_method} \option{type}
  \option{accuracy}\\[1.0ex] 
  Purpose: Sets the method for density
  interpolation for the integration of the non-local correlation
  potential and energy.\\[1.0ex] 
  \option{type} is the method selected,
  either \texttt{octree}, \texttt{mixed}, or \texttt{multipoles}.
  Default: \texttt{octree}\\[1.0ex] 
  \option{accuracy} applies to
  methods \texttt{octree} and \texttt{mixed}. It is the targer
  accuracy of the interpolation. In case of \texttt{multipoles} all
  available multipoles are used. Default: \texttt{1E-6}\\
}
The point of providing three different options for \texttt{type} is
simply that any prospective user should test which one is fastest for
a given problem. The difference is simply the style of integration of
the non-local part. Ideally, the results should be the same. However,
as always, please check in case of doubt.
