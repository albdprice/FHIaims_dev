\addcontentsline{toc}{chapter}{How to use this manual} 
\chapter*{How to use this manual}

If you are reading this introduction, you are likely reading the
manual for the first time. In that case, please read on. There is,
however, a strategy to use this manual most effectively to find
keywords used in the input files to FHI-aims. This is it:
\begin{itemize}
  \item Open the manual (pdf)
  \item Go to the table of contents
  \item At the bottom of the table of contents, click on ``Index''
  \item Find the keyword you are looking for in the index
  \item Click on it.
\end{itemize}
Using the manual in this way may greatly reduce the barrier to looking up
what a keyword actually does.

To first build FHI-aims, please also read this manual. You cannot simply type 'make'.
Chapter \ref{Ch:quickstart}, particularly sections \ref{Sec:installation}--\ref{Sec:build-cmake},
are what you need to read.

And now, for the actual ...

\chapter*{Introduction}
\addcontentsline{toc}{chapter}{Introduction} 

FHI-aims (``Fritz Haber Institute \emph{ab initio} molecular simulations'') is
a computer program package for computational materials science based only on
quantum-mechanical first principles. The main production method is density
functional theory (DFT) \cite{Hohenberg64,Kohn65,Dreizler90} to compute the total energy and
derived quantities of molecular or solid condensed matter in its electronic
ground state. In addition, FHI-aims allows to describe electronic
single-quasiparticle excitations in molecules using different self-energy
formalisms (e.g., $GW$ and MP2), and wave-function based molecular total energy
calculation based on Hartree-Fock and many-body perturbation theory (e.g., MP2,
RPA, SOSEX, or the more encompassing renormalized second-order
perturbation theory, RPT2). 

The basic physical algorithms in FHI-aims concerning ground state DFT and
applications are described in

\begin{center}
\parbox[c]{0.8\textwidth}
{\small
Volker Blum, Ralf Gehrke, Felix Hanke, Paula Havu, Ville Havu, Xinguo Ren,
Karsten Reuter, and Matthias Scheffler, Computer Physics Communications
\textbf{180}, 2175-2196 (2009). 
}
\end{center}

A copy of this paper can also be obtained from our web site: \\
\url{http://www.fhi-berlin.mpg.de/aims/} . \\ Please cite this reference if you use
FHI-aims. 

However, FHI-aims is not just a product of this basic reference.
Many more developments make this code a reality. For each individual
FHI-aims  run, a list of references describing the specific methods
used is given at the end of the FHI-aims standard output. Please give
credit in your publications if you can. FHI-aims is
a scientific code, written by and for scientists. The primary
recognition for their work is credit in the form of appropriate
reference to their work. 

Some particularly important papers (also worth reading!) follow
below. When making use of / reference to scalability, please refer to
and cite 

\begin{center}
\parbox[c]{0.8\textwidth}
{\small
Ville Havu, Volker Blum, Paula Havu, and Matthias Scheffler, Journal of
Computational Physics \textbf{228}, 8367-8379 (2009).
}
\end{center}

and also to the large-scale eigenvalue solver ELPA:

\begin{center}
\parbox[c]{0.8\textwidth}
{\small
A. Marek, V. Blum, R. Johanni, V. Havu, B. Lang, T. Auckenthaler, 
A. Heinecke, H.-J. Bungartz, and H. Lederer, 
The Journal of Physics: Condensed Matter \textbf{26}, 213201 (2014). 
}
\end{center}

Any application making use of functionality beyond LDA, GGA, or mGGA
-- i.e., Hartree-Fock, hybrid functionals, MP2, RPA, $GW$, etc. --
should please refer to and cite

\begin{center}
\parbox[c]{0.8\textwidth}
{\small
Xinguo Ren, Patrick Rinke, Volker Blum, J\"urgen Wieferink, Alex
Tkatchenko, Andrea Sanfilippo, Karsten Reuter, and Matthias Scheffler,
New Journal of Physics 14, 053020 (2012).
}
\end{center}

Further methodological publications for specific methods 
in FHI-aims can also be found at

\url{https://aimsclub.fhi-berlin.mpg.de/aims_publications.php}

Finally, we're quite proud that FHI-aims performed extremely well in the 
precision benchmark of 15 leading electronic structure codes known as
the ``Delta Project'', https://molmod.ugent.be/deltacodesdft -- see 
Reference \cite{Lejaeghereaad3000} in Science Magazine for
details.  
Numerical reliability -- high precision -- in everyday applications,
applicable up to very large production problems -- continues to be a top
priority and is, in fact, one of the key reasons why FHI-aims was
written in the first place. 

In the present documentation, we do not repeat the basic physical
algorithms; rather, the focus is on the actual \emph{use} of the methods in
FHI-aims for a given task, including a full description of all input and
output possibilities.  

The rest of this document is organized as follows:
\begin{itemize}
  \item In Chapter \ref{Ch:quickstart}, a ``quickstart'' 
    description attempts to give you all the necessary
    (but not more) information to get FHI-aims up and running on your own
    computer system, up to the first test run.
  \item Chapter \ref{Ch:basic} explains the basic input files and input
    philosophy very briefly. Some important remarks on choosing the numerical
    accuracy are summarized here.
  \item Chapter \ref{Ch:full} gets into the gory details, summarizing
    \emph{all} available input keywords and their meaning, sorted roughly by
    their expected use. 
  \item A large chapter \ref{CH:running} is dedicated to some frequently required
    ``meta-tasks'' of electronic structure theory: Not just setting up a
    specific set of input files for a given run, but actually extracting some
    of the frequently required information from those runs. For the more
    complex tasks (e.g., a transition state search), we attempt to provide
    scripts that perform a series of well-defined runs automatically, the 
    use of an external visualization tool, etc.  
  \item In chapter \ref{Ch:aitranss} we provide a description of the 
    \textsc{aitranss} ({\it ab initio} transport simulations) package which is
    a project under continuous development at the Institute of Nanotechnology  
    of the Karlsruhe Institute of Technology (KIT), Germany, since 2002. 
    When combined with FHI-aims, \textsc{aitranss} provides a
    post-processor module that enables calculation of the electron
    transport characteristics of molecular junctions based on a Landauer  
    formalism in a Green's function formulation.
  \item In the appendices, we suggest further reading, more on building
    the code from source, and we also address
    some issues (``troubleshooting'') that are either beyond
    our control (operating-system related issues come to mind), or simply
    require some level of experience to address.  
\end{itemize}
    Electronic structure
    theory (and FHI-aims) is extremely versatile but many of the most
    interesting applications require complex workflows. We cannot
    possibly document them all on our own. Please consider sending us
    hands-on descriptions of any complex workflows that worked for
    you, and we would gladly include them in this manual (obviously,
    we'll happily include references to your work).

In any case, we hope that this manual will be helpful for your specific
purposes. We welcome feedback, in particular regarding issues from
production settings that we might not yet have thought of / experienced
ourselves. In any event: Happy computing with FHI-aims! 

