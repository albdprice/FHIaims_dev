\section{Linear macroscopic dielectric function and Kubo-Greenwood transport}

The linear macroscopic dielectric tensor $\epsilon_{ij}(\omega)$, the ratio between the average of the total potential in one unit cell and the external field, 
within the RPA is calculated. For derivation from the microscopic dielectric function $\epsilon(\vec{r},t;\vec{r'},t')$ and references see \cite{Draxl06} 
(Chapter 1, 2 and appendix). From the complex frequency dependent dieletric function all other optical constants can be determined, e.g. optical conductivity, 
Loss function, Reflectivity, etc. Also see Appendix K in Ashcroft/Mermin: solid state physics.\\
The frequency dependent real and imaginary part of the inter- (eq.~\ref{inter}) and intra- (eq.~\ref{intra}) band contribution to the linear dielectric tensor is calculated as post-processing after 
convergence of the SCF-cycle from (atomic units):
\begin{align}
 \epsilon_{ij}(\omega)=\delta_{i,j}&-\frac{4\pi}{ V_{cell}\omega^2}\sum_{n,\mathbf{k}}\left(-\frac{\partial f_0}{\partial \epsilon}\right)_{\epsilon_{n,\mathbf{k}}}p_{i;n,n,\mathbf{k}}p_{j;n,n,\mathbf{k}}\label{intra}\\
                       &+\frac{4\pi}{V_{cell}}\sum_{\mathbf{k}}\sum_{c,v}\frac{p_{i;c,v,\mathbf{k}}p_{j;c,v,\mathbf{k}}}{\left(\epsilon_{c,\mathbf{k}}-\epsilon_{v,\mathbf{k}}-\omega\right)\left(\epsilon_{c,\mathbf{k}}-\epsilon_{v,\mathbf{k}}\right)^2}\label{inter}
 \end{align}
$f$ is the Fermi-function and $V_{cell}$ the unit cell volume. The intra band part is singular at $\omega=0$. Here the plasma frequency $\omega_{pl;ij}$ is calculated:
\begin{equation}
 \omega^2_{pl;ij}=\frac{1}{\pi }\sum_n\int_{\vec{k}}p_{i;n,n,\vec{k}}p_{j;n,n,\vec{k}}\delta(\epsilon_{n,\vec{k}}-\epsilon_{F})
\end{equation}
By adopting a Drude-like shape for the intra-band contributions a lifetime broadening $\Gamma$ is introduced and $\epsilon_{ij}^{inter}(\omega)$ becomes:
\begin{align}
 &\mathrm{Im}(\epsilon_{ij}^{intra}(\omega))=\frac{\Gamma\omega^2_{pl;ij}}{\omega(\omega^2+\Gamma^2)}\\
&\mathrm{Re}(\epsilon_{ij}^{intra}(\omega))=1-\frac{\omega^2_{pl;ij}}{\omega^2+\Gamma^2}
\end{align}
In the case of a spin unpolarized calculation eq.~\ref{inter} and eq.~\ref{intra} have to be multiplied by $2$ to yield the correct occupation.\\
The following quantities are needed/calculated:
\begin{itemize}
 \item[-] $\epsilon_{n,\vec{k}}$ - the eigenvalue of the Kohn-Sham eigenstate $(n,\vec{k})$.
 \item[-] $\delta(\epsilon_{n^{'},\vec{k}}-\epsilon_{n,\vec{k}}-\omega)=\frac{1}{\sqrt{2\pi}width}exp\left(-\frac{1}{2}\frac{(\epsilon_{n^{'},\vec{k}}-\epsilon_{n,\vec{k}}-\omega)^2}{\Gamma^2}\right)$ - Gauss function with width $\Gamma$=$width$ for calculating the plasma frequency. In eq~\ref{inter} $\omega$ is replaced by $\omega+\mathrm{i}\Gamma$, introducing Lorentzian broadenig.
 \item[-] $p_{j;n^{'},n,\vec{k}}=\left\langle\psi_{n^{'}\vec{k}}\vert\nabla_j\vert\psi_{n\vec{k}}\right\rangle$ - the momentum matrix elements calculated form the real space basis 
functions in k-space and KS-eigenstate basis
\end{itemize}
\begin{equation}
 \left\langle\psi_{n^{'}\vec{k}}\vert\nabla_j\vert\psi_{n\vec{k}}\right\rangle=\sum_{ij}c_{in^{'}}^{*\vec{k}}c_{jn}^{\vec{k}}\sum_{\vec{N},\vec{M}}\mathrm{e}^{\mathrm{i}\vec{k}\left[\vec{T}\left(\vec{N}\right)-\vec{T}\left(\vec{M}\right)\right]}\left\langle\phi_{i\vec{M}}\vert\nabla_j\vert\phi_{j\vec{N}}\right\rangle
\end{equation}
\begin{equation}
 \left\langle\phi_{i\vec{M}}\vert\nabla\vert\phi_{i\vec{N}}\right\rangle=\int_{\text{unit cell}}\mathrm{d}^3 r \phi_{i,\vec{M}}\vec{\nabla}\phi_{j,\vec{N}}
\end{equation}
with the real space functions $\phi_i(\vec{r})$ centered in unit cells shifted by $\vec{T}\left(\vec{N}\right),~\vec{N}=(N_1, N_2, N_3)$, see \cite{Blum08} for details.

Building up on the expressions for the dielectric function also the corresponding Kubo-Greenwood transport properties expressed via the Onsager coefficients $L_{ij}$: 

\begin{equation}
 L_{ij}(\omega) = \frac{2\pi e^{4-i-j}}{3V m^2_e \omega} \sum_{\bm{k},m,n} | \langle \Psi_m |\bm{\hat{p}} | \Psi_n \rangle|^2 \cdot   \left( \frac{\epsilon_{\bm{k}m} + \epsilon_{\bm{k}n}}{2} -\epsilon_{Fermi}   \right)^{i+j-2}   \cdot   \left(f_{\bm{k}m}-f_{\bm{k}n}\right)  \delta\left(\epsilon_{\bm{k}n}-\epsilon_{\bm{k}m}-\hbar \omega \right) 
\end{equation}

In this representation $L_{11}$ corresonds to the optical conductivity $\sigma(\omega)$. Furthermore the (frequency dependent) Seebeck coefficient is easily obtainable: 

\begin{equation}
S = \frac{L_{12}}{TL_{11}}
\end{equation}
\newpage


\subsection*{Tags for general section of \texttt{control.in}:}
\keydefinition{compute\_dielectric}{control.in}
{
  \noindent
  Usage: \keyword{compute\_dielectric}  \option{$\omega_{max}$} \option{$n_\omega$} \\[1.0ex]
  Purpose: Sets basic parameters for calculating the imaginary and real part of the inter-band and intra-band contribution to the linear dielectric tensor within the RPA approximation.  This keyword is specified once to set parameters for the dielectric tensor calculation.\\ 
  
  By setting this keyword, the whole dielectric tensor components (in directions: x\_x, y\_y, z\_z, x\_z, y\_z, x\_y) would be output automatically within the corresponding absorption coefficients (in directions: x\_x, y\_y, z\_z) for the diagonal parts.  \\ 
  The momentum matrix elements in the energy window [ VBM-(\option{$\omega_{max}$} + 10.0 eV), CBM+(\option{$\omega_{max}$} + 10.0 eV)] (in eV) relative to the internal zero will be summed. \\ 
  The default broadening type and broadening width used for the delta function is 0.1 eV in Lorentzian type. These settings can also be changed via the \keyword{dielectric\_broadening} keyword. \\
  In order to avoid numerical integration  errors caused by including 0 eV energy, the minimum energy ($\omega_{min}$) are automatically setting to a spefic value corresponding to the broadening width you used (broadening width/ 10.0). \\
  The resulting output files will be named dielectric\_function\_(directions).out (e.g. dielectric\_function\_x\_x.out for the x\_x direction) and absorption\_(directions).out. \\ 
  In order to test the impacts of different broadening type and broadening parameters, the individual tensor component for the dielectric constants can also be specified via the \keyword{output dielectric} keyword. By setting this keyword, the code would only output the dielectric functions in the directions you listed in the \keyword{output dielectric} keyword, instead of automatically outputing the whole tensor components. 
    The calculation is very sensitive to the k-point grid, an extremely high number of k-points might be needed for convergence, especially for metals.\\[1.0ex]
}
\keydefinition{dielectric\_broadening}{control.in}
{
   \noindent 
   Usage: \keyword{dielectric\_broadening} \option{broadening\_method} \option{width} \\[1.0ex]
   
   Purpose: Changing the broadening function type and broadening parameters used in the dielectric calculation. To use this keyword, the \keyword{compute\_dielectric} keyword must be specified in \texttt{control.in} \\
   The Delta-distribution is approximated by a broadening function specified by the \option{broadening\_method} option with a defined width specified by the \option{width} option (in eV). Gaussian (\option{gaussian}) and lorentz (\option{lorentzian}) broadenings are supported.   
}

\keydefinition{output dielectric}{control.in}
{
   \noindent
   Usage: \keyword{output dielectric} \option{broadening\_method} \option{width} \option{i} \option{j}\\[1.0ex]

   Purpose: Output the \option{i}\option{j} tensor components (choices: i,j = x, y, z) of the imaginary and real parts of the inter-band and intra-band contribution to the linear dielectric tensor.  This keyword may be specified multiple times in \texttt{control.in} to output more than one tensor component (possibly with different broadenings) per calculation.  The RPA approximation (i.e. Lindhard theory) is used.  To use this keyword, the \keyword{compute\_dielectric} keyword must be specified in \texttt{control.in}.
   
   Note: This keyword are just setted for testing purpose. By setting this keyword, only the directions you listed will be output. 
   The Delta-distribution is approximated by a broadening function specified by the \option{broadening\_method} option with a defined width specified by the \option{width} option (in eV).   Gaussian (\option{gaussian}) and Lorentz (\option{lorentzian}) broadenings are supported.  Good starting choices for parameters are \option{broadening\_method} = gaussian and \option{width}=$0.05eV$.  We encourage the user to try out different broadenings by specifying this keyword multiple times. \\[1.0ex]
}   
\keydefinition{compute\_absorption}{control.in}
{
  \noindent
  Usage: \keyword{compute\_absorption} \option{width} \option{$E_{min}$} \option{$E_{max}$} \option{$\omega_{min}$} \option{$\omega_{max}$} \option{$n_\omega$} \option{i} \option{use\_gauss}\\[1.0ex]
  Purpose: Calculate and output the \option{i} component (choices: x, y or z) of the linear absorption. 
  \begin{equation}
   \alpha_{i}(\omega)=\frac{8\pi^2}{\omega V_{cell}}\sum_{c,v}\sum_{\vec{k}}\left|p_{i;c,v,\vec{k}}\right|^2\delta(\epsilon_{c,\vec{k}}-\epsilon_{v,\vec{k}}-\omega)\mathrm{d}\vec{k}
  \end{equation}
  The momentum matrix elements in the energy window [\option{$E_{min}$},\option{$E_{max}$}] (in eV) relative to the internal zero are summed up. The Delta-distribution is 
  represented by a Gaussian function (\option{use\_gauss} = .true.) or a Lorentz function (\option{use\_gauss} = .false.) with width \option{width} (in eV) 
  for \option{$n_\omega$} $\omega$-values in the interval [\option{$\omega_{min}$}, \option{$\omega_{max}$}] (in eV). The output file is named \textit{absorption\_}\option{i}\textit{.out}. 
  A good choice is: \option{width}=$0.1eV$, usually it is enough to include only a few states below and above the fermi level in the 
  energy window [\option{$E_{min}$},\option{$E_{max}$}]. The unit is $a_0^{-1}$. $V_{cell}$ the unit cell volume. (see page 38, eq. 3.13 and 3.14 of http://www.tddft.org/bmg/files/papers/85619.pdf, \cite{Botti2012})\\
  The calculation is very sensitive to the k-point grid, an extremely high number of k-points might be needed for convergence, especially for metals.\\[1.0ex]
}
\keydefinition{compute\_momentummatrix}{control.in}
{
  \noindent
  Usage: \keyword{compute\_momentummatrix} \option{$E_{min}$} \option{$E_{max}$} \option{k-point}\\[1.0ex]
  Purpose: Calculate and output the momentum matrix elements $\left\langle\psi_{n^{'}\vec{k}}\vert\nabla_x\vert\psi_{n\vec{k}}\right\rangle$, 
  $\left\langle\psi_{n^{'}\vec{k}}\vert\nabla_y\vert\psi_{n\vec{k}}\right\rangle$, $\left\langle\psi_{n^{'}\vec{k}}\vert\nabla_z\vert\psi_{n\vec{k}}\right\rangle$
  for the k-point with the number \option{k-point} for KS-eigenstates that are within the energy window [\option{$E_{min}$},\option{$E_{max}$}] (in eV) relative to the 
  internal zero. The output file is named \textit{element\_k\_}\option{k-point}\textit{.dat}. The unit is $a_0^{-1}$ (bohr$^{-1}$).\\
  If you are conducting a cluster calculation (no periodic boundary conditions) make sure to set \option{k-point} to 1.\\
  Setting \option{k-point} to 0 will output the momentum matrix elements for all k-points into one container file (mommat.h5). To use this functionality, FHI-aims 
  has to be compiled with the external hdf5 module (see Sec.~\ref{sec:CMake_variables} for details).\\[1.0ex]
}
\keydefinition{compute\_dipolematrix}{control.in}
{
  \noindent
  Usage: \keyword{compute\_dipolematrix} \option{$E_{min}$} \option{$E_{max}$} \option{k-point}\\[1.0ex]
  Purpose: Calculate and output the dipole matrix elements $\left\langle\psi_{n^{'}\vec{k}}\vert x\vert\psi_{n\vec{k}}\right\rangle$, 
  $\left\langle\psi_{n^{'}\vec{k}}\vert y\vert\psi_{n\vec{k}}\right\rangle$, $\left\langle\psi_{n^{'}\vec{k}}\vert z\vert\psi_{n\vec{k}}\right\rangle$
  (position operator!) for the k-point with the number \option{k-point} for KS-eigenstates that are within the energy window [\option{$E_{min}$},\option{$E_{max}$}] (in eV) relative to the 
  internal zero. The output file is named \textit{element\_k\_}\option{k-point}\textit{.dat}. The unit is $a_0$ (bohr).\\
  If you are conducting a cluster calculation (no periodic boundary conditions) make sure to set \option{k-point} to 1.\\
  Setting \option{k-point} to 0 will output the dipole matrix elements for all k-points into one container file (dipmat.h5). To use this functionality, FHI-aims 
  has to be compiled with the external hdf5 module (see Sec.~\ref{sec:CMake_variables} for details).\\[1.0ex]
}
\keydefinition{compute\_dipolematrix\_k\_k}{control.in}
{
  \noindent
  Usage: \keyword{compute\_dipolematrix\_k\_k} \option{$E_{min}$} \option{$E_{max}$} \option{k\_k\_method}\\[1.0ex]
  Purpose: Calculate and output the dipole matrix elements $\left\langle\psi_{n^{'}\vec{k'}}\vert x\vert\psi_{n\vec{k}}\right\rangle$, 
  $\left\langle\psi_{n^{'}\vec{k'}}\vert y\vert\psi_{n\vec{k}}\right\rangle$, $\left\langle\psi_{n^{'}\vec{k'}}\vert z\vert\psi_{n\vec{k}}\right\rangle$
  (position operator!) for all (k, k')-points for KS-eigenstates that are within the energy window [\option{$E_{min}$},\option{$E_{max}$}] (in eV) relative to the 
  internal zero. The output file is named \textit{dipmat\_k}\textit{.h5}. The unit is $a_0$ (bohr). With the option \option{k\_k\_method} (choises: 1, 2, 3) 
  you can chose the method for calculating the matrix elements. \textit{1} will first calculate the matrix elements (k, k')!!! for the atomic basis before 
  transforming to the KS-basis (biggest array size: n\_basis*n\_k\_points*n\_k\_points). \textit{2} will calculate the matrix elements (k) for the atomic basis and transform to 
  the KS-basis for all (k') (n\_k\_points times) (biggest array size: n\_basis*n\_k\_points). \textit{3} will calculate the matrix elements in the atomic basis and transform to 
  the KS-basis for all (k, k') (n\_k\_points*n\_k\_points times) (biggest array size: n\_basis). \textit{1} needs the most memory, \textit{3} does the most (repetitive) 
  calculations.\\
  The dipole matrix elements for all (k,k')-points are written into one container file (\textit{dipmat\_k}\textit{.h5}). To use this functionality, FHI-aims 
  has to be compiled with the external hdf5 module (see Sec.~\ref{sec:CMake_variables} for details).\\[1.0ex]
}

\keydefinition{compute\_kubo\_greenwood}{control.in}
{
  \noindent
  Usage: \keyword{compute\_kubo\_greenwood} \option{width} \option{FD\_Temp} \option{$E_{min}$} \option{$E_{max}$} \option{$\omega_{min}$} \option{$\omega_{max}$} \option{$n_\omega$} \option{i} \option{j}\\[1.0ex]
  Purpose: Calculate and output the \option{i} \option{j} component (choices: (x, y or z) or (a a)) of the the Kubo-Greenwood transport properties optical conductivity $\sigma(\omega)$ and Seebeck coefficient 
  S in units $(\Omega \cdot cm)^{-1}$ and $\mu V /K$, respectively. The (a a) choice triggers the calculation of the average of the three diagonal elements $L_{ij,xx}$, $\sigma_{ij,yy}$ and $\sigma_{ij,zz}$.
  As before, [\option{$E_{min}$},\option{$E_{max}$}] (in eV) determine the energy window (relative to the internal zero) within which the momentum-matrix elements are considered. The Delta-distribution is 
  represented by a Gaussian function with width \option{width} (in eV) and the Fermi occupations are calculated at an electronic temperature \option{FD\_Temp} (in eV). The spectra contain \option{$n_\omega$} $\omega$-values 
  in the interval [\option{$\omega_{min}$}, \option{$\omega_{max}$}] (in eV). The output file is named \textit{dielectric\_}\option{i}\textit{\_}\option{j}\textit{<Fermi-level>}\textit{.out} and consists of the dielectric function,
  the optical conductivity and the Seebeck coefficient. 
  Usually it is enough to include only a few states below and above the fermi level in the 
  energy window [\option{$E_{min}$},\option{$E_{max}$}].\\
  The calculation is very sensitive to the k-point grid, an extremely high number of k-points might be needed for convergence, especially for metals.\\[1.0ex]
}

\keydefinition{greenwood\_method}{control.in}
{
  \noindent
  Usage: \keyword{greenwood\_method} \option{method}\\[1.0ex]
  Purpose: Switches between the use of sparse or full transition matrices. (Choices: \textit{sparse} or \textit{full}). As of now, the keyword has to be set in case the \keyword{compute\_kubo\_greenwood} is used. The use of \textit{sparse} matrices is recommended.
  (Keyword will be deprecated in the near future)[1.0ex]
}
