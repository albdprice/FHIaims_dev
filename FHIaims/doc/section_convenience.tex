

 \section{Usability (convenience)}
\label{section:usability}

This section is only intended for functionality that fits none of the
other categories (which are all scientifically / technically
motivated). These files and keywords affect the general user
convenience / experience for FHI-aims.

As an exception, this section also lists any \emph{files}
which may be used to interact with a \emph{running} instance of
FHI-aims. Currently, only two such files exist, but in principle, more
could be envisioned.

\subsection*{Files that interacting with the running code:}

\keydefinition{abort\_scf}{file}
{
  \noindent
  Usage: At the command line, use the Unix command \texttt{touch
    abort\_scf} in the current working directory of a running instance
    of FHI-aims to trigger a controlled stop of the run later. \\[1.0ex] 
  Purpose: If the file \keyword{abort\_scf} is found in the current
  working directory of FHI-aims, the present run will be aborted after
  the next s.c.f. iteration is complete (but importantly without
  achieving self-consistency). \\
}
This functionality allows FHI-aims to stop in a controlled fashion,
but not instantly. If you are interested in an instant stop, the Unix
'kill' command (or its equivalent in the queueing system of a
production machine) is the best way to proceed. See
Sec. \ref{section:abort} for some further remarks.

\keydefinition{abort\_opt}{file}
{
  \noindent
  Usage: At the command line, use the Unix command \texttt{touch
    abort\_opt} in the current working directory of a running instance
    of FHI-aims to trigger a controlled stop of the run later. \\[1.0ex] 
  Purpose: If the file \keyword{abort\_opt} is found during a geometry
  relaxation in the current working directory of FHI-aims, the present
  run will be aborted after the next s.c.f. cycle is complete (i.e.,
  afer achieving self-consistency for the present geometry, but
  without fully optimizing the structure). \\
}
This functionality allows FHI-aims to stop in a controlled fashion,
but not instantly. If you are interested in an instant stop, the Unix
'kill' command (or its equivalent in the queueing system of a
production machine) is the best way to proceed. See
Sec. \ref{section:abort} for some further remarks.

\keydefinition{control.update.in}{file}
{
  \noindent
  Usage: Allowed content of this file is a fairly limited subset of what is allowed and parsed in \keyword{control.in}. Details below.\\[1.0ex] 
  Purpose: FHI-aims checks for presence of this file in the current working directory at the end of each individual iteration of the SCF cycle. If the file is found, it is parsed, and found settings are updated. Note that if you do not remove the file manually, it will be parsed after each iteration. But with the current limited functionality, this should not pose any problems.\\
}
This file allows to modify some of the parameters of a calculation during runtime of FHI-aims. At present, this is limited to the settings of the convergence of the SCF cycle, namely: \keyword{sc\_accuracy\_rho}, \keyword{sc\_accuracy\_eev}, \keyword{sc\_accuracy\_etot}, \keyword{sc\_accuracy\_potjump}, \keyword{sc\_accuracy\_forces}, \keyword{sc\_accuracy\_stress}.

\newpage

\subsection*{Tags for general section of \texttt{control.in}:}

\keydefinition{check\_cpu\_consistency}{control.in}
{
  \noindent
  Usage: \keyword{check\_cpu\_consistency} \option{flag} \\[1.0ex]
  Purpose: In parallel runs, determines whether the consistency of geometry-related
           arrays is verified explicitly between different MPI tasks. \\[1.0ex]
  \option{flag} is a logical string, either \texttt{.false.} or
    \texttt{.true.} Default: \texttt{.true.} \\
}
%
This flag is introduced as default purely to monitor and possibly undo errors
that should not happen. Theoretically, all MPI tasks of a given FHI-aims run
should have the same atomic coordinates and lattice vectors. In practice, it 
appears that certain hardware and/or compilers/libraries introduce bit flips
between different instances of what is formally the same variable on different
CPUs. 

If \keyword{check\_cpu\_consistency} is \texttt{.true.}, the code checks
for deviations. 

If the discrepancy is numerically negligible (below the
value set by the tolerance parameter \keyword{cpu\_consistency\_threshold}, 
the code will work based on the assumption that the observed discrepancy is
a platform-dependent artifact, will set all instances of the geometry to
that stored on MPI task myid=0, and continue the run. Nonetheless, a warning
will be printed in the output file and near the end of the output.

If the discrepancy is larger than the tolerance parameter
\keyword{cpu\_consistency\_threshold}, the code will stop and inform
the user.

\keydefinition{cpu\_consistency\_threshold}{control.in}
{
  \noindent
  Usage: \keyword{cpu\_consistency\_threshold} \option{tolerance} \\[1.0ex]
  Purpose: In parallel runs, determines the degree to which
    inconsistencies of geometry-related arrays will be tolerated
    between different MPI tasks. \\[1.0ex] 
  \option{tolerance} : A small real numerical value, positive or zero. Default:
    10$^{-11}$. 
}
%
See keyword \keyword{check\_cpu\_consistency}. If
\keyword{check\_cpu\_consistency} is \texttt{.true.}, then keyword
\keyword{cpu\_consistency\_threshold} determines the maximum value
to which discrepancies of geometry-related quantities between
different MPI tasks will be tolerated (they will, however, be
set to identical values even if the run continues).

\keydefinition{dry\_run}{control.in}
{
  \noindent
  Usage: \keyword{dry\_run} \\[1.0ex]
  Purpose: If set in \texttt{control.in}, the FHI-aims run will only
  pass through all preparatory work to ensure the basic consistency of
  all input files, but will stop before any real work is done.
 \\[1.0ex]
}
%
This keyword is useful to check the consistency of input files with
the same exact binary that may later be used in a series of (perhaps
queued) production runs. If there are trivial errors in the input
files, no need to wait for the queue. The same effect can be achieved
by building a 'parser' binary, but this version saves the
recompilation. The price is that one must not forget to comment out
the \keyword{dry\_run} option in the actual, queued input files.

\subsubsection*{Subtags for \emph{species} tag in \texttt{control.in}:}

\subkeydefinition{species}{cite\_reference}{control.in}
{
  \noindent
  Usage: \subkeyword{species}{cite\_reference} \option{string} \\[1.0ex]
  Purpose: Triggers the addition of a specific citation to the end
           of the FHI-aims standard output for a given run. \\[1.0ex] 
  \option{string} is a string that identifies the reference in question. \\
}

This feature is useful, e.g., to make sure that the literature reference for
a given basis set (encoded in the species\_defaults input file) is available
at the end of an FHI-aims run. 

Each citation must, however, be coded into the FHI-aims source code
in module \texttt{applicable\_citations.f90} to ensure that the requested 
output is actually available. Note that the practical format for such references
can vary widely -- from a simple string (explanation who did the work)
all the way to the more usual case of a journal reference.

At the time of writing, species-related legitimate values of \texttt{string} are:
\begin{itemize}
  \item \texttt{NAO-VCC-2013} for reference \cite{Zhang2013},
    describing the NAO-VCC-nZ basis sets for valence-correlated
    calculations of elements H-Ar (useful for basis set extrapolation
    for many-body perturbation methods, e.g., MP2, RPA, RPT2, or
    $GW$). 
\end{itemize}
