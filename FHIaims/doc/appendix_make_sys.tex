\chapter{Building FHI-aims with a make.sys}
\label{Sec:makesys}

This section contains a quick and practical explanation
of the main steps using a \texttt{make.sys} file. This how FHI-aims used to be compiled in the past and is here for legacy reasons.

1. In the \emph{src} directory, create a file called \texttt{make.sys}
and open it with a text editor. Make sure you did \textit{not} edit
the file called \texttt{Makefile} as provided with the original
distribution of FHI-aims if you choose to use and edit the
\texttt{make.sys} file (which is recommended).

2. In order to build FHI-aims, you will need to inform the computer
about which particular compilers, libraries, optimization flags and
possible optional parts of the build process you intend to use. This
is the purpose of \texttt{make.sys}. We here only cover a few most
important keywords (variables) to be included in
\texttt{make.sys}. Many more are available, often documented in the
actual \texttt{Makefile} or, if nothing else, in the more detailed
\texttt{Makefile.backend}, which controls the detailed pieces of the
build process. Note that the syntax, particularly the spaces around
the `` = `` signs, in \texttt{make.sys} are important since this file
will be included in the \texttt{Makefile} and will have to be read
by the \texttt{make} command further below.

3. The following is what a typical \texttt{make.sys} file could look like (see
the aimsclub wiki for other examples for specific platforms). The
explanation of all keywords follows below. Note that this is the copy
of \texttt{make.sys} on the author's (VB's) laptop. You \text{will}
need to edit every single variable -- the directories to be used on
other computers \emph{will} be different. Blind copying and hoping
for the best will not work.

\begin{verbatim}
 FC = ifort
 FFLAGS = -O3 -ip -fp-model precise -module $(MODDIR)
 FMINFLAGS = -O0  -fp-model precise -module $(MODDIR)
 F90MINFLAGS = -O0  -fp-model precise -module $(MODDIR)
 F90FLAGS = $(FFLAGS)
 ARCHITECTURE = Generic
 LAPACKBLAS = -L/opt/intel/mkl/lib -I/opt/intel/mkl/include \ 
              -lmkl_intel_lp64 -lmkl_sequential -lmkl_core 
 USE_MPI = yes
 MPIFC = mpif90
 SCALAPACK = /usr/local/scalapack-2.0.2/libscalapack.a
 USE_C_FILES = yes
 CC = gcc
 CCFLAGS =
 USE_SPGLIB = yes
 USE_LIBXC = yes
\end{verbatim}

4. Here is a list of each of these keywords' meanings:
\begin{itemize}
  \item \texttt{FC} : The name of the Fortran compiler you intend to
    use. This choice is not unimportant. On x86 platforms, Intel
    Fortran usually produces fast code, whereas other compilers
    (unfortunately, particularly free compilers such as gfortran) can
    lead to significantly slower (factor 2-3) FHI-aims runs
    later.
  \item \texttt{FFLAGS} : These are compile-time and linker flags that
    control the optimization level that the compiler will use. Finding
    out which optimization level is fastest is worth your time, but
    note that real-world compilers can have bugs. In the worst case,
    this can mean numerically wrong results, something you should
    definitely care about. One way to test the broader correctness of a given
    FHI-aims build (later) is to run FHI-aims' regression tests on the
    computer you intend to use and make sure that all results are
    marked as correct. For example,
    for Intel Fortran, 
    \texttt{-fp-model precise} is highly recommended. Unfortunately,
    we have no way to foresee all possible compiler bugs across all
    future platforms and compilers -- testing is best. Please ask if
    needed (see Sec. \ref{Sec:community} for where to find help).
  \item \texttt{FMINFLAGS} specifies a lower optimization level for
    some subroutines that do not need
    optimization. \texttt{read\_control.f90}, the subroutine that
    reads one of FHI-aims' main input files, is one such file that
    does not need high levels of optimization but could take very long
    to compile if a high optimization level were requested for it.
  \item \texttt{F90MINFLAGS} and \texttt{F90FLAGS} are usually just
    copies of \texttt{FMINFLAGS} and \texttt{FFLAGS}, except for the
    few compilers (IBM's xlf) that might treat Fortran .f90 and
    (legacy) .f files differently.
  \item \texttt{ARCHITECTURE} can have multiple meanings, including
    specific handling of a few compilers' quirks (the pgi compiler,
    for example, needs a different call to erf()) and potentially
    optimization levels for CPU-specific extensions (e.g., AVX - this
    can be worthwhile). For many purposes, ``Generic'' is good enough
    but do take the time to look into CPU-specific optimizations if
    you intend to run very large, demanding calculations.
  \item \texttt{LAPACKBLAS} specifies the locations of numerical
    linear algebra subroutines, particularly the Basic Linear Algebra
    Subroutines (BLAS) and the higher-level Lapack
    subroutines. The location and names of these libraries will vary
    from computer to computer, but it is \textbf{VERY} important to
    select well-performing BLAS subroutines for a given computer --
    the effect on performance will be drastic. An additional item to
    ensure is that these BLAS libraries should \textbf{NEVER} try to
    use any internal multithreading (for example, the
    \texttt{mkl\_sequential} library quoted above is inherently
    single-threaded, which is normally what we want).
    FHI-aims is already very efficiently parallized
    for multiple processors. Requesting (say) 16 threads for each of
    (say) 16 parallel tasks on a parallel computer with 16 physical
    CPU cores would have the effect of trying to balance 256 threads
    within the computer, typically slowing execution down to a
    crawl. With FHI-aims, only ever use only a single thread per
    parallel task unless you have a special reason and know exactly
    what you are doing.  
  \item \texttt{USE\_MPI} will make sure that the code knows and will use
    the process-based Message Passing Interface (MPI) parallelization,
    which makes sure that FHI-aims can run in parallel both inside a
    single compute node as well as across a large number of
    nodes. In later production runs and unless you have a good reason
    not to do so, always use as many MPI tasks as there are physical
    processor cores available (no more, no less).
  \item \texttt{MPIFC} is the name of the wrapper command that ensures a
    correct compilation with a given Fortran compiler and a given MPI
    library. This command (often called \texttt{mpif90}) is also
    specific to a given computer system and to the installed MPI
    library.
  \item \texttt{SCALAPACK} specifies the location of the library that
    contains scalapack's parallel linear algebra subroutines and the
    so-called basic linear algebra communications (BLACS)
    subroutines. The author (VB) built his own version of this libary,
    but usually these subroutines are also supplied with standard
    linear algebra libraries such as Intel's Math Kernel Library
    (mkl).
  \item \texttt{USE\_C\_FILES} specifies whether a small number of
    subroutines written in C should be compiler into FHI-aims. These
    routines are not performance-critical but can access environment
    variables. They can thus provide very useful additional output
    about the computer system environment used and also make a few
    other useful libraries accessible (like spglib for symmetry
    handling).
  \item \texttt{CC} is the C compiler to be used. Usually, gcc is fine
    and actually recommended even if the Fortran compiler used is not
    a Gnu compiler.
  \item \texttt{CCFLAGS} could house any compiler flags needed for the
    C compiler. It is not worth doing this for performance reasons
    (very little impact) but some compilers may need other special
    instructions to work with Fortran.
  \item \texttt{USE\_SPGLIB} decides whether the \texttt{spglib}
    library for symmetry handling will be used. We are very much
    indebted to the authors of this and other libraries that are
    available as open source -- please respect their open-source
    licenses and cite the authors if you use their tools.
  \item \texttt{USE\_LIBXC} decides whether additional subroutines for
    exchange correlation functionals, provided in the \texttt{libxc}
    library, should be used. As for \texttt{spglib}, we are very much
    indebted to the authors of this library. Please respect their
    open-source license and cite them if you use their tools.
\end{itemize}

5. \textit{Phew.} That was a lot of keywords. But this is computational
science, and having a reasonable command of these pieces is worth our
while. If you did figure them all out, close the \texttt{make.sys}
file and continue to ...

6. ... build the code by typing \texttt{make -j scalapack.mpi}.

7. Do not despair. If the process above worked well, proceed to try a
testrun and then, if you are up for it, the regression tests. If you received an error  
message during the build (that may well be the case), do not despair
-- try again and, if needed, seek help. This process is ultimately not
rocket science and only a finite amount of pieces are needed. Look up
examples on aimsclub or seek help through one of the channels
mentioned in Sec. \ref{Sec:community} if needed.

8. There are other pieces that can help improve a build on a specific
platform. For example, it can be quite desirable to build and link
instead to a separate (standalone build) of the ELPA library
(high-performance eigenvalue solver) and of the ELSI electronic
structure infrastructure. For time and space reasons, this is not
covered here presently, but it's worth investigating these libraries.

\emph{In general, the ``aimsclub'' Wiki is the appropriate place to
  look for detailed compiler settings for specific platforms. If you
  have a successful 'make.sys' file for your own setup, please add 
  it there. The information given in this section is essential as it
  explains the process, but the platform specific remarks in the Wiki
  may help you save some time.} 

\section{A more measured approach to building FHI-aims}
\label{Sec:build}

This is a slower and step by step explanation of the build process,
slightly different and somewhat redundant with
Sec. \ref{Sec:makesys}. Ultimately, your build process should ideally
look somewhat like what is covered in Sec. \ref{Sec:makesys}, and in
particular, never edit the \texttt{Makefile} if you already have a
file \texttt{make.sys} around. What follows is based on direct editing
of the \texttt{Makefile} and should only be needed for practice
purposes. 

Starting from the end of Sec. \ref{Sec:prerequisites} and once all
prerequisites are in place, change directory to the \emph{src/} 
directory, and open the \texttt{Makefile} in a text editor. 

\begin{center}
  \parbox[c]{0.8\textwidth}
  {
  \emph{You must
  adjust at least some system-specific portions of the} \texttt{Makefile}\emph{---simply typing
  ``make'' and hoping that the problem will go away will not work.}
  }
\end{center}

Usually, all you will have to do is to decide on one of the preconfigured make
targets -- ``serial'', ``mpi'', or ``scalapack.mpi''. Near the top of the
\texttt{Makefile}, a number of \emph{mandatory} settings are commented for each
target. Uncomment \emph{only} the block of settings relevant to your chosen
make target, and fill in the correct values of each variable (\texttt{FC},
\texttt{FFLAGS}, \texttt{LAPACKBLAS}, ...) for your computer
system. \emph{Note} that the Makefile itself contains detailed instructions
and explanations regarding the meaning of these variables. Often, simply
adjusting the compiler name, the location of your libraries
(\texttt{LAPACKBLAS}, possibly \texttt{MPIFC} or \texttt{SCALAPACK}) will be
sufficient. In addition, we \emph{strongly} recommend that you consult the
documentation of  your compiler, in order to find out which optimization
options \emph{beyond} the generic ``-O3'' optimization level suggested preset in the
\texttt{Makefile} will make a  difference on your computer.

Finally, this brings us to the key step of the build process: Building the
code. After the \texttt{Makefile} is adjusted, type

\begin{verbatim}
  make <target>
\end{verbatim}

at the command line, where ``$\langle$target$\rangle$'' should be replaced by
the target of your choice: 
``serial'', ``mpi'', or ``scalapack.mpi''.\footnote{There is an
  additional target, \texttt{parser}, which builds an executable 
that stops after parsing the input. This binary can be used to check the
validity of input files. (The \keyword{dry\_run} keyword achieves the
almost same effect with the full binary.)
} 
If successful, this should build
the desired FHI-aims binary (the compilation will take a while) and place it
in the \emph{bin/} directory mentioned above.

Building FHI-aims can take a while nowadays. If you have more than one
processor on the machine for building FHI-aims, try

\begin{verbatim}
  make -j <number_of_processors> <target>
\end{verbatim}

This choice should speed up the process greatly.

You may also wish to keep your own copy of the \texttt{Makefile}, for
instance to be able to work directly with the FHI-aims git repository
without overwriting the general \texttt{Makefile}. In that case, just
copy the standard \texttt{Makefile} to something like
\texttt{Makefile.myname}, and use 

\begin{verbatim}
  make -j <number_of_processors> -f Makefile.myname <target>
\end{verbatim}

Finally, we do note that there is some support for more sophisticated
tasks, such as cross-platform builds (building binaries for different
architectures from the same home directory) among the non-standard
environment variables in the Makefile (see there). 

\subsection{Cross-Compiling with a C Compiler}
\label{cross-compile-c}

As noted in Section \ref{Sec:makesys}, FHI-aims does provide some
functionality that can only be accessed by compiling part of the code
base with a C compiler, in some cases simply because a system call in
question is not available from a Fortran interface. It is possible to
build FHI-aims without C support and the code will happily run based
only on Fortran, but if possible, cross-compiling with C is preferable. 

Cross-compiling with a C compiler can be simple \emph{if} your Fortran
and C compiler use compatible interfaces. (This may not be always the
case.) In the simplest case, adding the following two variables to
your \texttt{Makefile} or to your \texttt{make.sys} file will enable
some basic functionality that can only be accessed with a C
cross-compiled build:

\begin{verbatim}
 USE_C_FILES = yes
 CC = gcc
\end{verbatim}

The CC variable should specify a C compiler compatible with your
Fortran compiler on your specific computer system. \texttt{gcc} is
often the right choice. Again, C cross-compiling is not
necessary, but if you do, please make sure that the right C compiler
is chosen for your system.

An additional variable \texttt{CCFLAGS} can be set to specify C
compiler flags along with the \texttt{CC} variable above and will add
C compiler flags that might be necessary for your particular computer setup.

\section{Compilation options beyond the standard Makefile}
\label{Sec:Makefiles}

The build provided by the standard \texttt{Makefile} in FHI-aims is designed
for minimal complexity to obtain the full functionality that most users should
have. Separate ``basic linear algebra subroutines'' (BLACS), Lapack,
parallel builds (if a parallel machine is available, nowadays almost always),
and scalapack support are so performance-critical that every user should spend
the time to investigate them in detail before doing serious production work
with FHI-aims. These dependencies of FHI-aims on external libraries are
therefore kept in the main \texttt{Makefile}.

In addition, FHI-aims provides further functionality that can be achieved by
linking to other external libraries. However, this functionality will not be
needed by all users and/or could seriously complicate the build process for
everyone. Such functionality is therefore available through separate, amended
versions of the \texttt{Makefile}. We encourage everyone to try these builds
(they are not so difficult after all), but they should not become stumbling
blocks. 

We also note that not all of these builds are routinely tested. At this
time, it is not certain that all of them will still work out of the
box. The information is kept here to make sure it is available. Please
ask (see Section \ref{Sec:community}) if you encounter problems.

In particular, several optional Makefile with additional functionality
exist (and could even be combined):
\begin{itemize}
  \item \texttt{Makefile.cuba} : Allows to compile in the separate ``CUBA''
    Monte Carlo integration library, which enables the Langreth-Lundqvist van
    der Waals functional based on noloco as a post-processing step. See Section
    \ref{Sec:vdwdf} for more details.
  \item \texttt{Makefile.meta} : \emph{still experimental!} Allows to
    interface FHI-aims to the \texttt{PLUMED} library for free-energy
    calculations for molecular dynamics. \\ (see
    \url{http://merlino.mi.infn.it/\~plumed/PLUMED/Home.html} , the \texttt{PLUMED}
    project homepage). Currently, a copy of the \texttt{PLUMED} library is
    kept in the \emph{external} directory of the FHI-aims source code, and
    must be compiled separately using a C compiler. Note that
     the token ``-DFHIAIMS'' must be included in the CCFLAGS (at present, this is already specified in \texttt{plumed.inc}). 
     For compiling on IBM power machines, the flags -mpowerpc64 -maix64 should
    be included. In the future, this will be corrected by housing the
    respective FHI-aims plugin directly in the \texttt{PLUMED} library. We
    apologize that the linking process is not yet further documented,
    but if this functionality is of interest to you, please contact us.\\
    By selecting the correct target in \texttt{Makefile.meta}, the code can be compiled 
    with \texttt{lapack} or \texttt{scalapack} libraries, or with shared memory support (see next item).
  \item \texttt{Makefile.ipi} : Allows to interface with the i-PI path integral molecular dynamics wrapper\cite{CeriottiMoreMano_2013}.
    See the \keyword{use\_pimd\_wrapper} keyword for a few more details.
  \item \texttt{Makefile.shm} : Another example of a Makefile that
    cross-links C and Fortran based functionality, although the
    actual functionality that Makefile.shm provides --  access to
    shared-memory arrays in the Hartree potential -- is no longer
    needed at this point.
  \item \texttt{Makefile.amd64\_SSE} : This Makefile contains an
    example how to use FHI-aims together with a version of the 
    ELPA library which is especially tuned for SSE vector instructions.
    Note that the GNU C compiler must be installed in order to produce
    running code from the specific assembler files. As the name
    ``amd64'' indicates this optimization is only reasonable on
    processors which support the amd64 instruction set.
    Note that ''ELPA\_ST'' is set, to use the single-threaded version
    of ELPA.
  \item \texttt{Makefile.amd64\_SSE\_mt} : The same as
    Makefile.amd64\_SSE, but by setting ''ELPA\_MT'' the hybrid MPI
    OpenMP version of ELPA is used.
  \item \texttt{Makefile.amd64\_AVX} : This Makefile contains an
    example how to use FHI-aims together with a version of the 
    ELPA library which is especially tuned for AVX vector instructions.
    Note that the GNU C and C++ compiler must be installed in order to produce
    running code from the specific ELPA files which contain gcc
    intrinsic functions. As the name  ``amd64'' indicates this
    optimization is only reasonable on  processors which support the 
    amd64 instruction set and which already ''understand'' AVX,
    i.e. Intel Sandybridge or newer. Note that ''ELPA\_ST'' is set, to 
    use the single-threaded version of ELPA.
  \item \texttt{Makefile.amd64\_AVX\_mt} : The same as
    Makefile.amd64\_AVX, but by setting ''ELPA\_MT'' the hybrid MPI
    OpenMP version of ELPA is used.
  \item \texttt{Makefile.hdf5} : The HDF5 module provides functions and routines to efficently make use of parallel writing and reading of data (mpi-io). The usual compiler options have to be set in the same way as in the normal Makefie. Additionally the path to your installation of hdf5 must be provided. If the environment variable \textit{HDF5\_HOME} is already set on your system nothing else has to be changed otherwise it has to be set in \textit{Makefile.hdf5}.

\end{itemize}
