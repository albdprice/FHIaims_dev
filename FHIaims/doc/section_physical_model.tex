\section{Physical model: Geometry, charge, spin, etc.}

The present section summarizes all keywords in FHI-aims that are directly
concerned with the \emph{physical model} of the problem to be
tackled. Importantly, this includes some specific subtags that you
\emph{cannot} ignore, because they define the physical question that you are
trying to address -- and no one else but you can do that. The present section
thus includes such things as atomic positions or unit cells, but also the
level of theory to be used (exchange-correlation, relativistic treatment), or
a potential charge of the system. 

\subsection*{Tags for \texttt{geometry.in}:}

\keydefinition{atom}{geometry.in}
{
  \noindent
  Usage: \keyword{atom} \option{x} \option{y} \option{z}
  \option{species\_name} \\[1.0ex]
  Purpose: Specifies the initial location and type of an atom. \\[1.0ex]
  \option{x}, \option{y}, \option{z} are real numbers (in \AA) which
  specify the atomic position. \\
  \option{species\_name} is a string descriptor which names the element on
    this atomic position; it must match with one of the species descriptions
    given in \texttt{control.in}. \\
}

\keydefinition{atom\_frac}{geometry.in}
{
  \noindent
  Usage: \keyword{atom\_frac} \option{$n_1$} \option{$n_2$} \option{$n_3$}
  \option{species\_name} \\[1.0ex]
  Purpose: Specifies the initial location and type of an atom in fractional coordinates. \\[1.0ex]
  \option{$n_i$} is a real multiple of lattice vector $i$.
  \option{species\_name} is a string descriptor which names the element on
    this atomic position; it must match with one of the species descriptions
    given in \texttt{control.in}. \\
}
Fractional coordinates are only meaningful in periodic calculations.

\keydefinition{lattice\_vector}{geometry.in}
{
  \noindent
  Usage: \keyword{lattice\_vector} \option{x} \option{y} \option{z} \\[1.0ex]
  Purpose: Specifies one lattice vector for periodic boundary conditions. \\[1.0ex]
  \option{x}, \option{y}, \option{z} are real numbers (in \AA) which
  specify the direction and length of a unit cell vector. \\
}
If up to three lattice vectors are specified, FHI-aims automatically assumes
periodic boundary conditions in those directions. \emph{Note} that the
order of lattice vectors matters, as the order of $k$ space divisions (given
in \texttt{control.in}) depends on it!

\newpage

\subsection*{Tags for general section of \texttt{control.in}:}

\keydefinition{charge}{control.in}
{
 \noindent
 Usage: \keyword{charge} \option{q} \\[1.0ex]
 Purpose: If set, specifies an overall charge in the system. \\[1.0ex]
 \option{q} is a real number that specifies a positive or negative total
   charge in the system. \\
}
For most normal systems, this definition is unambiguous (sum of all nuclear
charges in \texttt{geometry.in} minus number of electrons in the system). Note
specifically that the same definition continues to hold also in systems with
external embedding charges (specified by keyword \keyword{multipole} in
\texttt{geometry.in}). The charges of the external embedding charges are in
addition to the \keyword{charge} keyword in \texttt{control.in}, and
\emph{not} included.

\keydefinition{fixed\_spin\_moment}{control.in}
{
 \noindent
 Usage: \keyword{fixed\_spin\_moment} \option{value} \\[1.0ex]
 Purpose: If set, allows to enforce a fixed overall spin moment throughout the calculation. \\[1.0ex]
  \option{value} : real-valued number, specifies the difference of electrons
  between spin channels, $2S = N_\text{up}-N_\text{down}$. \\ 
} 
Meaningful only in the spin-polarized case (\texttt{spin collinear} in
\texttt{control.in}). 

This keyword replaces the earlier keyword
\keyword{multiplicity}. Note that the value that must be given for
\keyword{fixed\_spin\_moment} is $2S$, which corresponds to a
\keyword{multiplicity} $(2S+1)$ ---i.e., the values are not the same.
Keyword \keyword{fixed\_spin\_moment} works for periodic and cluster
systems alike, and uses two different chemical potentials (Fermi
levels) for the spin channels.

\keydefinition{species}{control.in}
{
  \noindent
  Usage: \keyword{species} \option{species\_name} \\[1.0ex]
  Purpose: Defines the name of a species (element) for possible use with atoms
    in \texttt{geometry.in}\\[1.0ex] 
  \option{species\_name} is a string descriptor (e.g. C, N, O, Cu, Zn,
    Zn\_tight, ...). \\
}
Every \option{species\_name} used in an atom descriptor in
\texttt{geometry.in} must correspond to a \keyword{species}
given in \texttt{control.in}. Following the \keyword{species} tag, all
sub-tags describing that species must follow in one block. (No particular
order is enforced within that block). For example, the choice of the basis
set, the atom-centered integration grid, or the multipole decomposition of
the atom-centered Hartree potential are all specified per \keyword{species}. 

\keydefinition{spin}{control.in}
{
 \noindent
 Usage: \keyword{spin} \option{type} \\[1.0ex]
 Purpose: Specifies whether or not spin-polarization is used. \\[1.0ex]
 \option{type} is a string, either \texttt{none} or
 \texttt{collinear}, depending on whether an unpolarized (spin-restricted) or
 spin-polarized (unrestricted) calculation is performed. \\
}
In the \texttt{collinear} case, defining the moments used to create the
initial spin density is required (see the beginning of Sec. \ref{Sec:scf} for
an explanation). This means that an overall \keyword{default\_initial\_moment}
(in \texttt{control.in}), or at least one individual \keyword{initial\_moment}
tag in \texttt{geometry.in}, or both, must be set. Else, the code will stop
with a warning. (It is not necessary to specify \keyword{initial\_moment} for
every atom in \texttt{geometry.in}. A single one will do.) Choosing the right
initial spin density can be performance-critical, and critical for the
resulting physics. FHI-aims should not make this choice for you.

\emph{Warning:} It is not a good idea to run each and every calculation
with \keyword{spin} \texttt{collinear} just because that seems to be
the more general case. In a system that will safely be non-magnetic,
using something other than \keyword{spin} \texttt{none} \emph{will}
roughly double the CPU time needed in the best case, and it will most
likely lead to much worse s.c.f. convergence (i.e., more iterations
needed to find the self-consistent electronic solution). There is no
fundamental problem with running \keyword{spin} \texttt{collinear}, but
again: just doing this out of some sense of impartiality may not be a
wise use of resources.

This keyword is completely independent of spin-orbit coupling, which is
applied as a post-processed correction after the SCF cycle has converged.  
For more information on spin-orbit coupling, please see the 
\keyword{include\_spin\_orbit} keyword and the discussion in the associated
chapter.

\newpage

\subsection*{Subtags for \emph{species} tag in \texttt{control.in}:}

\subkeydefinition{species}{mass}{control.in}
{
  \noindent
  Usage: \subkeyword{species}{mass} \option{M} \\[1.0ex]
  Purpose: Atomic mass\\[1.0ex]
  \option{M} is a real number that specifies the atomic mass in atomic
  mass units (amu).\\
}
This tag is used only for molecular dynamics. The preconstructed
\texttt{species\_defaults} files supplied with FHI-aims contain
the mass \emph{average} over the different isotopes of each natural
element. 

\subkeydefinition{species}{nucleus}{control.in}
{
  \noindent
  Usage: \subkeyword{species}{nucleus} \option{Z} \\[1.0ex]
  Purpose: Via the nuclear charge defines the chemical element associated with
    the present species.\\[1.0ex]
  \option{Z} is a real number (the nuclear charge). \\
}
\option{Z} is usually an integer number. However, partial (non-integer) charges
are also possible. 

Fractional \option{Z} can be useful, for example, for a stoichiometric hydrogen-like 
termination of a compound semiconductor slab (e.g., in a III-V compound, the valence of
the connecting element would be mimicked by a fractionally charged H
of charge 0.75 or 1.25). 

If the difference between the specified nuclear charge and the nearest integer is
greater than 0.34, keyword \subkeyword{species}{element} may be needed to be set
explicitly in the species definition to designate an unambiguous chemical identity.

Fractional \option{Z} can also be useful to
distribute a compensating charge for an electronically charged
periodic supercell calculation. In electronic charged periodic systems, a
compensating background charge is always implicit. This is often
accomplished by introducing an implicit homegeneous charged background
density. However, the choice of such a jellium background is often
anything but ideal. For instance, in a surface slab calculation, part
of this compensating charge will be located in the vacuum. In such
cases, it may be better to place the compensating charge in the system
explicitly and ``by hand''. One good way to do this is to place the
compensating charges on certain nuclei.\cite{Richter2013}

If you add a fractional \option{Z} to a species\_default, you will
have to take care to modify the \subkeyword{species}{valence} tags to
reflect the exact opposite charge, creating an overall neutral
spherical free atom as far as the \subkeyword{species}{valence}
occupation numbers in the species definition go. 

\subkeydefinition{species}{element}{control.in}
{
  \noindent
  Usage: \subkeyword{species}{element} \option{symb} \\[1.0ex]
  Purpose: Chemical element associated with the species.\\[1.0ex]
  \option{symb} is a string (max.\ 2 characters) that corresponds to the symbol
  of the chemical element. Default: see below.\\
}
The purpose of this tag is to specify the chemical identity of a species in the
rare cases when it cannot be determined from \option{Z} because it has been set
to a non-integer value. In particular, when \option{Z} is more than 0.34 from
the nearest integer number, the species element must be set with the
\subkeyword{species}{element} tag. Currently, this value is used only by the
van der Waals routines, but the requirement above must be satisfied for any
calculation.
