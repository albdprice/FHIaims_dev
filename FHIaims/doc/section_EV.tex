\section{Eigenvalue solver and (fractional) occupation numbers}
\label{Sec:EV}

With an updated Hamiltonian matrix $h_{ij}$ and overlap matrix $s_{ij}$
available at the end of an s.c.f. iteration ($i,j$ run over all basis
functions), or in the post-processing step of the calculation, FHI-aims updates
the Kohn-Sham orbitals $l$ (wave function coefficients $c_{jl}$) by solving the
following eigenvalue problem:
\begin{equation}\label{Eq:EVP}
  \sum_j h_{ij} c_{jl} = \epsilon_l \sum_j s_{ij} c_{jl} \quad .
\end{equation}
In periodic boundary conditions, this eigenvalue problem is solved at every
$k$-point, and $\boldk$ is implictly included in the eigenstate index $l$ above.

FHI-aims now uses the open-source ELSI infrastructure
\url{http://elsi-interchange.org} -- and most often the efficient, massively
parallel ELPA eigensolver (\url{http://elpa.mpcdf.mpg.de}) -- to handle all
aspects of this problem.

Since the basis size needed even for meV-converged accuracy in FHI-aims is
rather small, and this size determines the dimension of $h_{ij}$ and $s_{ij}$,
the recommended eigenvalue solver(s) in FHI-aims are customized conventional
solvers (publicly available as the ELPA library since 2011), employing the same
basic algorithms as LAPACK or the parallel ScaLAPACK implementation, but with
significant scalability enhancements. Although these solvers scale strictly as
$O(N^3)$ with system size, their application becomes dominant only for systems
above $\approx$1000 atoms (light elements) or $\approx$500 atoms (heavy
elements, e.g. Au) in our experience. For large systems, there are alternative
methods available through the ELSI library, including the orbital minimization
method (libOMM), the pole expansion and selected inversion method (PEXSI), the
shift-and-invert parallel spectral transformation eigensolver (SLEPc-SIPs), and
the density matrix purification algorithms using sparse matrix linear algebra
from the NTPoly library. Note that PEXSI and SLEPc-SIPs are not installed with
FHI-aims by default.

The present section describes the available eigensolvers and density matrix
solvers and relevant options in FHI-aims, including the determination of a Fermi
level and occupation numbers for all orbitals following the process. Keywords
starting with a prefix \texttt{elsi\_} are ELSI-specific. The key ideas of using
ELSI and its supported solvers are briefly introduced here. For more
information, please refer to the ELSI documentation available at
\url{http://elsi-interchange.org}.

FHI-aims also offers the possibility to solve a \emph{constrained} eigenvalue
problem, e.g., in order to restrict the number of spin-up or spin-down electrons
in the basis functions of a given set of atoms. Since this functionality is
experimental and for experienced users only, it is documented separately in
Sec. \ref{Sec:constraint}.

Finally, we emphasize that the basis set in FHI-aims \emph{is} non-orthogonal.
For all practical production settings, this is not a problem, and in fact taken
care of through the overlap matrix $s_{ij}$ in Eq. (\ref{Eq:EVP}) above. It is,
however, still possible to generate an overcomplete, nearly ill-conditioned
basis set in practical calculations, usually by specific, \emph{deliberate}
user action. The signature of such ill-conditioning are near-zero eigenvalues of
$s_{ij}$ (e.g., 10$^{-5}$ and below). Possible reasons include: systematically
constructed, deliberately overconverged basis sets for non-periodic calculations
(not easy); excessively large cutoff radii in dense periodic structures together
with very large basis sets (the density of non-zero basis functions per volume
element increases as $r_\text{cut}^3$); or, badly integrated, very extended
basis functions (diffuse Gaussian basis functions without increasing
\subkeyword{species}{radial\_multiplier} appropriately).

FHI-aims does include a number of safeguards against an ill-conditioned overlap
matrix, most importantly the \keyword{basis\_threshold} keyword that projects
out the eigenvectors of the overlap matrix that correspond to its smallest
eigenvalues, usually enabling a meaningful calculation anyway. However, to alert
every user to the fact that their chosen basis set may be ill-conditioned, the
code now stops when it encounters an overlap matrix with too low eigenvalues---
unless the keyword \keyword{override\_illconditioning} is deliberately set,
indicating that the user knows what they are doing and wishes to continue
regardless.

\newpage

\subsection*{Tags for general section of \texttt{control.in}:}

\keydefinition{basis\_threshold}{control.in}
{
  \noindent
  Usage: \keyword{basis\_threshold} \option{threshold} \\[1.0ex]
  Purpose: Threshold to prevent any accidental ill-conditioning of the basis
    set. \\[1.0ex]
  \option{threshold} is a small positive threshold for the eigenvalues of the
    overlap matrix. Default: 10$^{-5}$. \\
}
Since NAO basis functions are situated at different atomic centers in a
structure, they form a non-orthogonal basis set by construction. Usually, this
is not a problem, since the non-orthogonality is naturally accounted for by
inserting the overlap matrix $s_{ij}$ into the Kohn-Sham eigenvalue problem,
Eq. (\ref{Eq:EVP}). For very large basis sets, this can lead to accidental
ill-conditioning (some basis functions may be exactly expressable as linear
combinations of some others).

This behavior is detected by directly inverting the overlap matrix, and
computing its eigenvalues. If one or more eigenvalues are smaller than
\option{threshold}, the corresponding eigenvectors are projected out of the
basis before solving the Kohn-Sham eigenvalue problem, and the latter is solved
after transforming to the reduced eigenbasis-set of the overlap matrix.

\emph{Important change:} Even when \keyword{basis\_threshold} is set, FHI-aims
will automatically stop when a near-singular overlap matrix is detected. The
user can still override this safeguard by setting the
\keyword{override\_illconditioning} keyword in \texttt{control.in} explicitly,
but we do now do our best to alert the user to this condition.

\keydefinition{elpa\_settings}{control.in}
{
  \noindent
  Usage: \keyword{elpa\_settings} \option{setting} \\[1.0ex]
  Purpose: Allows to determine the exact algorithm used in the ELPA eigensolver
    by hand. \\[1.0ex]
  \option{setting} is a descriptor (string) that selects certain aspects of
    ELPA. Default: \texttt{auto} \\
}
If the parallel ELPA eigensolver is used (see keyword \keyword{KS\_method}), a
number of choices are made automatically by default. The
\keyword{elpa\_settings} keyword allows to set some of these aspects by hand.
Allowed choices for \texttt{setting} are:
\begin{itemize}
  \item \texttt{auto} : The default. ELPA makes all its choices on the fly.
  \item \texttt{one\_step\_solver} : Only the one-step tridiagonalization (and
    corresponding back transformation) are used. This is usually the slower
    choice, but not always ...
  \item \texttt{two\_step\_solver} : Only the two-step tridiagonalization (and
    corresponding back transformation) are used. This is usually the faster
    choice, but not always ...
\end{itemize}
The \keyword{elpa\_settings} keyword is particularly useful \\
(i) if you already know what the faster choice is, and you wish to eliminate the
extra test of the slower solver from your calculations, or \\
(ii) if you suspect that one of the two solvers links to a buggy external(!)
library. LAPACK and BLAS implementations (still used in ELPA) come from many
vendors, they are often precompiled, and of course they \emph{always} work---the
computer vendor hopes so, after all. We have seen our share of bugs in external
libraries (outside the control of FHI-aims), and sometimes, switching the
algorithm to change the exact subroutines used can be a helpful backup check.

\keydefinition{empty\_states}{control.in}
{
  \noindent
  Usage: \keyword{empty\_states} \option{number} \\[1.0ex]
  Purpose: Specifies how many Kohn-Sham states \emph{beyond} the occupied levels
    are computed by the eigensolver. \\[1.0ex]
  \option{number} is the integer number of empty Kohn-Sham states
    \emph{per atom} to be computed beyond the occupied levels. \\
}
For DFT-LDA/GGA, typically only a small (but non-zero) number of empty states is
required to allow a complete determination of the Fermi level.

By default, ($l$+1)$^2$+2 states are added \emph{for each} atom in the
structure, where $l$ is the maximum valence angular momentum in the valence of
that atom ($l$=0 for hydrogen, but $l$=3 for $f$-electron atoms and beyond).

For correlated methods including excited states (MP2, RPA, $GW$, ...),
\emph{all} available states should be included. To achieve this, set
\keyword{empty\_states} to a large number (safely larger than your basis set) or
use the \keyword{calculate\_all\_eigenstates} keyword.

\keydefinition{calculate\_all\_eigenstates}{control.in}
{
  \noindent
  Usage: \keyword{calculate\_all\_eigenstates} \\[1.0ex]
  Purpose: Specifies that all possible eigenstates obtainable from the basis set
    used (after ill-conditioning has been accounted for) should be calculated
    and stored. \\[1.0ex]
}

This keyword instructs FHI-aims to calculate and store all possible eigenstates
obtainable from the solution of the Kohn-Sham eigenvalue problem. It functions
identically to setting the \keyword{empty\_states} value to a large number, but
makes the input file prettier.

Only users that know exactly what they're doing should use this option alongside
an ill-conditioned basis set. Strange output may occur in this case, which may
either be spurious or a symptom of a deeper problem. (This is true of
ill-conditioning in general; this keyword only exposes it more openly.)

\keydefinition{fermi\_acc}{control.in}
{
  \noindent
  Usage: \keyword{fermi\_acc} \option{tolerance} \\[1.0ex]
  Purpose: The precision with which the Fermi level for occupation numbers will
    be determined. \\[1.0ex]
  \option{tolerance} : Tolerance parameter for the zero point search of the
    equation $\sum_i f_\text{occ}[\epsilon_F](\epsilon_i) - n_\text{el} = 0$.
    Default: 10$^{-20}$. \\
}
Usually, this tag need not be modified from the default. Within the (standard)
Brent's method search for the Fermi level, \option{tolerance} has more than one
function. Leave untouched unless problems arise.

\keydefinition{initial\_ev\_solutions}{control.in}
{
  \noindent
  Usage: \keyword{initial\_ev\_solutions} \option{number} \\[1.0ex]
  Purpose: \emph{Experimental!} Number of initial eigenvalue solutions using
    direct methods before switching on the lopcg-solver. Applies for both
    LAPACK and ScaLAPACK variants of the solver. \\[1.0ex]
  \option{number} is a positive integer. Default: 5. \\
}

\keydefinition{KS\_method}{control.in}
{
  \noindent
  Usage: \keyword{KS\_method} \option{KS\_type} \\[1.0ex]
  Purpose: Algorithm used to solve the generalized eigenvalue problem
    Eq. (\ref{Eq:EVP}). \\[1.0ex]
  \option{KS\_type} is a keyword (string) which specifies the update method used
    for the Kohn-Sham eigenvectors or density matrix in each s.c.f. iteration.
    Default: \texttt{serial} or \texttt{parallel}, depending on available number
    of CPUs. \\
}
\emph{Important change} : The naming scheme of the supported options has
changed. Notably, the \texttt{lapack} and \texttt{scalapack} keywords are
superseded by \texttt{serial} and \texttt{parallel}, respectively. The reason is
that \texttt{lapack} and \texttt{scalapack} have indicated other linear algebra
than custom (Sca)LAPACK in FHI-aims for a long time. For example, the
\texttt{scalapack} option actually employed the ELPA eigenvalue solver,
currently called through the ELSI infrastructure. The new naming scheme
indicates linear algebra default options that may develop further over time, but
these default options will not necessarily be tied to one and the same library
going forward.

Available options for the eigensolver, \option{KS\_type}, are:
\begin{itemize}
  \item \texttt{serial} : \textbf{Default serial eigensolver implementation}.
    Currently the serial eigensolver in ELSI, based on LAPACK and ELPA, will be
    employed.
  \item \texttt{lapack\_fast} : Synonymous with \texttt{serial}.
  \item \texttt{lapack\_2010} : LAPACK-based, and similar to the divide\&conquer
    based standard solver provided by LAPACK itself.
  \item \texttt{lapack\_old} : Expert solver provided by standard LAPACK. This
    is stable and not a bottleneck in most standard situations.
  \item \texttt{lapack} : Disabled now.
  \item \texttt{parallel} : \textbf{Default parallel eigensolver
    implementation}. Currently the ELPA eigensolver will be called through the
    ELSI interface.
  \item \texttt{elsi} : Synonymous with \texttt{parallel}.
  \item \texttt{elpa} : Synonymous with \texttt{parallel}.
  \item \texttt{elpa\_2013} : Same functionality as \texttt{scalapack\_old},
    however, substantially rewritten for an overall speedup and much improved
    scalability. See the \keyword{elpa\_settings} keyword for some ELPA
    internals (usually determined automatically, but who knows). \\
    Note that you must set the shell variable \texttt{OMP\_NUM\_THREADS}=1
    prior to running FHI-aims on some platforms. (see Appendix
    \ref{appendix_trouble_shooting})
  \item \texttt{scalapack\_fast} : Synonymous with \texttt{parallel}.
  \item \texttt{scalapack\_old} : Fully memory-parallel implementation of the
    eigenvalue solver based on ScaLAPACK itself, scales much worse than our own
    \texttt{scalapack\_fast}.
  \item \texttt{scalapack} : Disabled now.
  \item \texttt{svd} : Effectively the same as \texttt{lapack\_old}.
  \item \texttt{lopcg} : \emph{Experimental -- under development} Iterative,
    locally optimal preconditioned conjugate gradient eigensolver. Potentially
    useful for \emph{very} large systems where \texttt{lapack} becomes a
    bottleneck. However, implementation without any serious testing---contact
    us if interested.
  \item \texttt{scalapack+lopcg} : \emph{Experimental -- under development}.
    Same as \texttt{lopcg}, but parallel with ScaLAPACK-type memory distribution.
\end{itemize}

The \texttt{parallel} eigensolvers are only available if ScaLAPACK support has
been compiled into the FHI-aims binary---see the Makefile for more information.

In fact, the default \texttt{parallel} eigensolver in FHI-aims is the ``ELPA''
solver through the ELSI interface, which uses some ScaLAPACK infrastructure but
has been rewritten from the ground up for much improved parallel scalability.

Note that a (separate) parallelization over $k$-points will be performed in
periodic systems in any case.

\keyword{KS\_method} \texttt{parallel} also allows calculations \emph{without}
explicitly collecting the resulting eigenvectors to each thread after the
eigensolution is complete. This improves the memory efficiency especially in
large-scale / massively parallel situations. For details, see keyword
\keyword{collect\_eigenvectors}.

Prior to the solution of Eq. (\ref{Eq:EVP}) using the \texttt{serial} or
\texttt{parallel} solvers, the overlap matrix $s_{ij}$ is checked for
ill-conditioning (see \keyword{basis\_threshold} keyword). For very large basis
sets or periodic calculations with many $k$-points, this criterion may trigger.
In that case, the Hamiltonian matrix is transformed to the ``safe'' set of
eigenvectors of $s_{ij}$, and the transformed eigenvalue problem is solved. If
you suspect ill-conditioning to be a problem, it may sometimes be helpful to
increase the density of the 3D integration grids in order to minimize any
numerical noise in $s_{ij}$ and $h_{ij}$. That said: In our experience,
ill-conditioning is not a problem with accurate basis sets in standard
calculations; see Appendix \ref{appendix_trouble_shooting} for some additional
comments.

\keydefinition{elsi\_method}{control.in}
{
  \noindent
  Usage: \keyword{elsi\_method} \option{method} \\[1.0ex]
  Purpose: Determines the usage of eigensolvers or density matrix solvers in
    ELSI. Must be compatible with the \keyword{density\_update\_method} keyword
    (see Sec. \ref{Sec:density_update}). \\[1.0ex]
  \option{method} is a keyword (string). Default: \texttt{ev}. \\
}
Available options for \option{method} are:
\begin{itemize}
  \item \texttt{ev} : Use eigensolvers to solve the wave functions explicitly
    through ELSI. Supported serial solver is LAPACK. Supported parallel solvers
    are ELPA and SLEPc-SIPs (if compiled in). Compatible with all options of
    \keyword{density\_update\_method}.
  \item \texttt{dm} : Use density matrix solvers to directly compute the density
    matrix without explicitly solving the eigenproblem in Eq. (\ref{Eq:EVP}).
    Note that this will not work with any post-processing that requires the wave
    functions. Supported solvers are ELPA, libOMM, PEXSI, SLEPc-SIPs, and
    NTPoly. Only compatible with \keyword{density\_update\_method}
    \texttt{density\_matrix}.
\end{itemize}

\keydefinition{elsi\_solver}{control.in}
{
  \noindent
  Usage: \keyword{elsi\_solver} \option{solver} \\[1.0ex]
  Purpose: Specifies the eigensolver or density matrix solver to use. \\[1.0ex]
  \option{solver} is a keyword (string). Default: \texttt{elpa}. \\
}

Available options for \option{solver} are:
\begin{itemize}
  \item \texttt{elpa} : Direct, dense eigensolver ELPA (EigensoLvers for
    Petaflop Applications). Scales as $O(N^3)$ with respect to system size.
    Fast for systems of small and medium sizes (up to hundreds of atoms).
  \item \texttt{omm} : Density matrix solver libOMM (the Orbital Minimization
    Method). Scales as $O(N^3)$. No support for metallic systems. For now, not
    recommended in most cases.
  \item \texttt{pexsi} : Density matrix solver PEXSI (the Pole EXpansion and
    Selected Inversion method). Scales as $O(N^2)$ for 3D systems, $O(N^{1.5})$
    for 2D systems, and $O(N)$ for 1D systems. Fast when solving a large system
    with sufficiently many MPI tasks (a thousand or more). PEXSI is not compiled
    with FHI-aims by default. To use it, either enable the compilation of PEXSI
    when building FHI-aims wiht CMake, or link FHI-aims against a precompiled
    ELSI library with PEXSI support.
  \item \texttt{sips} : \emph{Experimental} Sparse eigensolver SLEPc-SIPs (the
    Shift-and-Invert Parallel spectral transformation method). An externally
    compiled ELSI with SLEPc-SIPs support is required. For now, not recommended
    in most cases.
  \item \texttt{ntpoly} : Density matrix purification algorithms implemented in
    the NTPoly library. For sufficiently large systems, scales as $O(N)$. Only
    competitive for thousands of atoms.
\end{itemize}

\keydefinition{elsi\_output}{control.in}
{
  \noindent
  Usage: \keyword{elsi\_output} \option{verbosity} \\[1.0ex]
  Purpose: Controls the output level of ELSI. \\[1.0ex]
  \option{verbosity} is a keyword (string). Default: \texttt{detail}. \\
}
Available options for \option{verbosity} are:
\begin{itemize}
  \item \texttt{none} : No output from ELSI. This is the default if the overall
    output level of FHI-aims is \texttt{MD\_light}.
  \item \texttt{light} : Enables output from the ELSI interface, but no output
    from the solvers.
  \item \texttt{detail} : Enables output from the ELSI interface as well as the
    solvers. When using libOMM or PEXSI, additional output will be written to an
    separate log file.
  \item \texttt{debug} : Enables the same output as does the \texttt{detail}
    option, with additional memory usage information. Creates large output files
    and thus should not be chosen in production runs.
  \item \texttt{json} : Enables the output of the runtime parameters used in
    ELSI in a separate JSON file, powered by the FortJSON library. May be used
    on top of the above options.
\end{itemize}

\keydefinition{elsi\_elpa\_n\_single}{control.in}
{
  \noindent
  Usage: \keyword{elsi\_elpa\_n\_single} \option{n\_single} \\[1.0ex]
  Purpose: \emph{Experimental} Specifies the number of s.c.f. steps in which the
    eigenproblems Eq. (\ref{Eq:EVP}) are solved using the single precision
    version of ELPA. Requires an externally compiled ELSI or ELPA library with
    single precision enabled in ELPA. \\[1.0ex]
  \option{n\_single} is an integer. Default: 0. \\
}

\keydefinition{elsi\_omm\_n\_elpa}{control.in}
{
  \noindent
  Usage: \keyword{elsi\_omm\_n\_elpa} \option{n\_elpa} \\[1.0ex]
  Purpose: When using libOMM, specifies the number of s.c.f. steps in which the
    eigenproblems Eq. (\ref{Eq:EVP}) are solved explicitly using ELPA. As an
    iterative solver, libOMM's performance heavily depends on the quality of the
    initial guess. By default, random numbers are used as inital guess. The
    eigensolution computed by ELPA proves to be a better choice. \\[1.0ex]
  \option{n\_elpa} is an integer. Default: 6. \\
}

\keydefinition{elsi\_omm\_flavor}{control.in}
{
  \noindent
  Usage: \keyword{elsi\_omm\_flavor} \option{flavor} \\[1.0ex]
  Purpose: Specifies the flavor of libOMM to be used. \\[1.0ex]
  \option{flavor} is an integer. Default: 0. \\
}
Available options for \option{flavor} are:
\begin{itemize}
  \item 0 : Directly minimizes the OMM energy functional without transforming
    the generalized eigenproblem to the standard form before minimization. This
    is usually faster than flaver 2 if using several ELPA steps before switching
    to libOMM.
  \item 2 : Before OMM minimization, first transforms the generalized
    eigenproblem to the standard form using the Cholesky decomposition of the
    overlap matrix.
\end{itemize}

\keydefinition{elsi\_omm\_tol}{control.in}
{
  \noindent
  Usage: \keyword{elsi\_omm\_tol} \option{tolerance} \\[1.0ex]
  Purpose: Specifies the convergence criterion of the OMM energy functional
    minimization. \\[1.0ex]
  \option{tolerance} is a small positive real number. Default: 10$^{-12}$. \\
}

\keydefinition{elsi\_pexsi\_np\_symbo}{control.in}
{
  \noindent
  Usage: \keyword{elsi\_pexsi\_np\_symbo} \option{np\_symbo} \\[1.0ex]
  Purpose: Specifies the number of MPI tasks assigned for the symbolic
    factorization step in PEXSI. \\[1.0ex]
  \option{np\_symbo} is a positive integer. Default: 1. \\
}
Parallel symbolic factorization with more than 1 MPI task is not always stable,
hence the default. Increasing \option{np\_symbo} might accelerate the symbolic
factorization, however might also cause a segfault. Note that the symbolic
factorization step needs to be performed only once per s.c.f. cycle. Unless
facing a memory bottleneck, using the default value is recommended.

\keydefinition{elsi\_sips\_slice\_type}{control.in}
{
  \noindent
  Usage: \keyword{elsi\_sips\_slice\_type} \option{type} \\[1.0ex]
  Purpose: Specifies the slicing method used in SLEPc-SIPs. \\[1.0ex]
  \option{type} is an integer. Default: 4. \\
}
Available options for \option{type} are:
\begin{itemize}
  \item 0 : Equally spaced slices.
  \item 2 : Equally populated slices.
  \item 4 : Equally populated slices (with energy gaps taken into account).
\end{itemize}

\keydefinition{elsi\_sips\_n\_slice}{control.in}
{
  \noindent
  Usage: \keyword{elsi\_sips\_n\_slice} \option{n\_slice} \\[1.0ex]
  Purpose: Specifies the number of slices used in SLEPc-SIPs. Note that the
    total number of MPI tasks must be a multiple of the number of slices. In
    practice, setting \option{n\_slice} to be equal to the number of nodes seems
    to work well. The default value should always work, but by no means leads to
    the best performance. \\[1.0ex]
  \option{type} is a positive integer. Default: 1. \\
}

\keydefinition{elsi\_sips\_n\_elpa}{control.in}
{
  \noindent
  Usage: \keyword{elsi\_sips\_n\_elpa} \option{n\_elpa} \\[1.0ex]
  Purpose: Specifies the number of s.c.f. steps to be solved with ELPA. The
    performance of SIPs relies on a decent knowledge on the eigenvalue
    distribution, which is key to an efficient spectrum slicing. This can be
    calculated by ELPA in the first \option{n\_elpa} s.c.f. steps. \\[1.0ex]
  \option{type} is an integer. Default: 0. \\
}

\keydefinition{elsi\_ntpoly\_method}{control.in}
{
  \noindent
  Usage: \keyword{elsi\_ntpoly\_method} \option{method} \\[1.0ex]
  Purpose: Specifies the purification algorithm used in NTPoly. \\[1.0ex]
  \option{method} is an integer. Default: 2. \\
}
Available options for \option{method} are:
\begin{itemize}
  \item 0 : Canonical purification.
  \item 1 : Trace-correcting purification.
  \item 2 : 4th order trace-resetting purification.
  \item 3 : Generalized canonical purification.
\end{itemize}

\keydefinition{elsi\_ntpoly\_tol}{control.in}
{
  \noindent
  Usage: \keyword{elsi\_ntpoly\_tol} \option{tolerance} \\[1.0ex]
  Purpose: Specifies the convergence criterion of the density matrix
    purification. \\[1.0ex]
  \option{tolerance} is a small positive real number. Default: 10$^{-4}$. \\
}

\keydefinition{elsi\_ntpoly\_filter}{control.in}
{
  \noindent
  Usage: \keyword{elsi\_ntpoly\_filter} \option{threshold} \\[1.0ex]
  Purpose: Specifies the threshold smaller than which the matrix elements will
    be discarded in the process of density matrix purification. \\[1.0ex]
  \option{tolerance} is a small positive real number. Default: 10$^{-8}$. \\
}

\keydefinition{frozen\_core\_scf}{control.in}
{
  \noindent
  Usage: \keyword{frozen\_core\_scf} \option{boolean} \\[1.0ex]
  Purpose: Enables the frozen core approximation to reduce the dimension of the
    Kohn-Sham eigenproblem. Atomic basis functions whose eigenvalue is lower
    than \keyword{frozen\_core\_scf\_cutoff} will be treated as core states.
    Useful for systems consisting of heavy elements. This keyword applies only
    to the solution of the Kohn-Sham eigenproblem. It does not imply a frozen
    core treatment anywhere else. See also \keyword{frozen\_core} and
    \keyword{frozen\_core\_postscf}, which control the use of frozen core in
    other parts of the code. \\[1.0ex]
  \option{boolean} is either \texttt{.true.} or \texttt{.false.}. Default:
    \texttt{.false.} \\
}

\keydefinition{frozen\_core\_scf\_cutoff}{control.in}
{
  \noindent
  Usage: \keyword{frozen\_core\_scf\_cutoff} \option{cutoff} \\[1.0ex]
  Purpose: Determines the number of core states when the frozen core
    approximation is enabled by \keyword{frozen\_core\_scf}. Atomic basis
    functions whose eigenvalue is lower than \keyword{cutoff} (eV) will be
    treated as core states. \\[1.0ex]
  \option{cutoff} is a negative number. Default: \texttt{-13605.5} (eV, which is
    about -500 Ha) \\
}

\keydefinition{lopcg\_adaptive\_tolerance}{control.in}
{
  \noindent
  Usage: \keyword{lopcg\_adaptive\_tolerance} \option{flag} \\[1.0ex]
  Purpose: \emph{Experimental!} Allows the lopcg-algorithm to dynamically
    adjusts its convergence tolerance as
    $\max{\{0.01 \, |\delta n|, \hbox{\keyword{lopcg\_tolerance}} \}}$ where
    $\delta n$ is the change in the electron density as recorded in the
    self-consistency cycles. \\[1.0ex]
  \option{flag} is a logical expression. Default: \texttt{.false.} \\
}

\keydefinition{lopcg\_block\_size}{control.in}
{
  \noindent
  Usage: \keyword{lopcg\_block\_size} \option{number} \\[1.0ex]
  Purpose: \emph{Experimental!} The maximal size of a block in lopcg-iteration.
    \\[1.0ex]
  \option{number} is a positive integer. Default: 1. \\
}

\keydefinition{lopcg\_auto\_blocksize}{control.in}
{
  \noindent
  Usage: \keyword{lopcg\_auto\_blocksize} \option{flag} \\[1.0ex]
  Purpose: \emph{Experimental!} Selects if the lopcg algorithm tries to find
    automatically a better blocksize than the maximal one by grouping close
    eigenvalues together. \\[1.0ex]
  \option{flag} is a logical expression. Default: \texttt{.false.} \\
}

\keydefinition{lopcg\_preconditioner}{control.in}
{
  \noindent
  Usage: \keyword{lopcg\_preconditioner} \option{type} \\[1.0ex]
  Purpose: \emph{Experimental!} For \keyword{KS\_method} \texttt{lopcg},
    specifies the preconditioner used. \\[1.0ex]
  \option{type} is a string, either \texttt{diagonal} (diagonal preconditioning
    matrix) or \texttt{ovlp\_inverse} (use inverse of the overlap matrix for
    preconditioning). \\
}

\keydefinition{lopcg\_start\_tolerance}{control.in}
{
  \noindent
  Usage: \keyword{lopcg\_start\_tolerance} \option{tolerance} \\[1.0ex]
  Purpose: \emph{Experimental!} Sets the tolerance for starting the lopcg-solver
    using the change in the sum of eigenvalues as a criterion. The lopcg-solver
    is activated as set in \keyword{initial\_ev\_solutions} latest, but
    \keyword{lopcg\_start\_tolerance} may trigger it earlier. \\[1.0ex]
  \option{tolerance} is a double precision real. Default: 0.0 \\
}

\keydefinition{lopcg\_tolerance}{control.in}
{
  \noindent
  Usage: \keyword{lopcg\_tolerance} \option{tolerance} \\[1.0ex]
  Purpose: \emph{Experimental!} Sets the convergence tolerance for the
    lopcg-solver.\\[1.0ex]
  \option{tolerance} is a double precision real. Default: 10$^{-6}$. \\
}

\keydefinition{max\_lopcg\_iterations}{control.in}
{
  \noindent
  Usage: \keyword{lopcg\_tolerance} \option{number} \\[1.0ex]
  Purpose: \emph{Experimental!} Sets the maximal number of iterations for one
    block in the the lopcg-solver.\\[1.0ex]
  \option{number} is an integer. Default: 100. \\
}

\keydefinition{mu\_determination\_method}{control.in}
{
  \noindent
  Usage: \keyword{mu\_determination\_method} \option{type} \\[1.0ex]
  Purpose: Specifies the algorithm used to search for the Fermi level. \\[1.0ex]
  \option{type} is a descriptor (string) which specifies the desired algorithm
    to determine the Fermi level.  Default: \texttt{bisection} \\
}
Available options are:
\begin{itemize}
  \item \texttt{bisection} : Standard bisection algorithm. Usually robust to
    reach an accuracy of 10$^{-13}$ in terms of electron count. If a desired
    accuracy cannot be reached by the bisection iteration, e.g., due to the
    limit of the machine precision, the remaining error (very small) will be
    arbitrarily cancelled out. Not compatible with the \texttt{integer}
    \keyword{occupation\_type}.
  \item \texttt{zeroin} : Standard Brent's method. Not compatible with the
    \texttt{cubic} or the \texttt{cold} \keyword{occupation\_type}.
\end{itemize}

\keydefinition{max\_zeroin}{control.in}
{
  \noindent
  Usage: \keyword{max\_zeroin} \option{number} \\[1.0ex]
  Purpose: Number of iterations allowed in Brent's method to find the Fermi
    level. \\[1.0ex]
  \option{number} is an integer number. Default: 200. \\
}
Usually, this tag need not be modified from the default. This limits the number
of allowed iterations for the (standard) Brent's method search for the Femi
level. Leave untouched unless problems arise. Note that changing the values
given for \keyword{occupation\_type} or \keyword{empty\_states} may be the true
fixes if the search for a Fermi level really ever fails.

\keydefinition{occupation\_acc}{control.in}
{
  \noindent
  Usage: \keyword{occupation\_acc} \option{tolerance} \\[1.0ex]
  Purpose: Accuracy with which the sum of calculated occupation numbers for a
    given Fermi level reproduces the actual number of electrons in the system.
    \\[1.0ex]
  \option{tolerance} is a small positive real number. Default: 10$^{-13}$. \\
}
Usually, this tag need not be modified from the default. Determines the target
accuracy for the Fermi level (calculated vs. actual number of electrons in the
system). Note that changing the values given for \keyword{occupation\_type} or
\keyword{empty\_states} may be the true fixes if the search for a Fermi level
really ever fails.

\keydefinition{occupation\_type}{control.in}
{
  \noindent
  Usage: \keyword{occupation\_type} \option{type} \option{width}
    [\option{order}] \\[1.0ex]
  Purpose: Determines the broadening scheme used to find the Fermi level and
    occupy the Kohn-Sham eigenstates. \\[1.0ex]
  \option{type} is a string which determines the desired broadening function.
    Default: \texttt{gaussian} \\
  \option{width} specifies the width of the broadening function [in eV].
    Default: 0.01 eV. \\
  \option{order} is an integer, and only required to specify the order of
    \option{type} \texttt{methfessel-paxton}. \\
}
Based on the eigenvalues $\epsilon_l$ of each s.c.f. iteration, the selected
\keyword{occupation\_type} determines the Fermi level $\epsilon_F$ and occupies
all Kohn-Sham states with fractional occupation numbers $f_l(\epsilon_F)$ for
the following electron density update. Detailed discussions can be found in
Ref. \cite{Blum08} or other standard literature \cite{Kre96}. We only briefly
list the available options for the occupation \option{type} here:
\begin{itemize}
  \item \texttt{gaussian} : Gaussian broadening function \cite{Fu83}
    \[
       f_l = 0.5 \cdot [1 -
       \erf(\frac{\epsilon_l-\epsilon_F}{\mathtt{width}})]
    \]
  \item \texttt{methfessel-paxton} : Generalized Gaussian-type distribution
    functions of Methfessel and Paxton (see Ref. \cite{Met89} for details). In
    practice, any \option{order} beyond 1 is not recommended, and is not
    supported if \texttt{bisection} is chosen for the keyword
    \keyword{mu\_determination\_method}.
  \item \texttt{fermi} : Formally correct finite-temperature broadening
    scheme \cite{Mermin65}
    \[
       f_l = \frac{1}{1+\exp[(\epsilon_l-\epsilon_F)/\mathtt{width}]}
    \]
    However, to be useful in practice, \option{width} must take on values
    significantly greater than $kT$ at room temperature, and therefore mostly
    loses its physical meaning. In practice, \texttt{fermi} broadening seems to
    lead to faster-increasing total energy inaccuracies than \texttt{gaussian}
    broadening, which is why the latter is preferred in FHI-aims.
  \item \texttt{integer} : Forces the occupation numbers to be integers.
  \item \texttt{cubic} : \emph{Experimental} Cubic polynomial broadening.
  \item \texttt{cold} : Cold smearing technique proposed by Marzari and
    Vanderbilt.
\end{itemize}
\emph{For metallic systems / systems with small HOMO-LUMO gap}, the availability
of an occupation scheme with finite width (e.g., 0.1~eV) is critical to
guarantee the stable convergence of the s.c.f. cycle. Especially for metallic
systems, FHI-aims outputs an extrapolated total energy, which estimates the
total energy for zero broadening based on the entropy of the electron gas
\cite{Kre96,Gillan89,Wagner98}. This extrapolated total energy must only be used
for metallic systems, not, e.g., for atoms with a decidedly discrete density of
states.

\emph{For non-metallic systems / systems with appreciable HOMO-LUMO gap}, the
broadening width must be finite in order to guarantee the existence of a formal
Fermi level, but not so large as to lead to any actual fractional occupation
numbers. In our experience, the default width of 0.01~eV performs well for this
purpose.

\keydefinition{override\_illconditioning}{control.in}
{
  \noindent
  Usage: \keyword{override\_illconditioning} \option{flag} \\[1.0ex]
  Purpose: Allows to override a built-in stop and run with a nearly singular
    overlap matrix. \\[1.0ex]
  \option{flag} is a logical flag, either \texttt{.true.} or \texttt{.false.}
    Default: \texttt{.true.} \\
}
If the overlap matrix $s_{ij}$ has an eigenvalue below
\keyword{basis\_threshold} or below 10$^{-5}$ (whichever is larger), FHI-aims
will stop and warn the user of a potentially ill-conditioned basis set. Usually
this situation can still be resolved by setting an appropriate value of
\texttt{basis\_threshold}, but anyone relying on this functionality should first
check whether their ``ill-conditioning'' condition is not also due to another,
inadvertent choice, such as an insufficient integration grid for very extended
functions, or an excessively large cutoff radius in dense periodic systems (is
it really necessary?).

In other words: By all means, override if you wish, but check first whether all
computational settings are actually intentional and appropriate.
