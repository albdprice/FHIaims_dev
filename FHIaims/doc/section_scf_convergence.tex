\section{SCF Cycle: Initialization, density mixing, preconditioning, convergence}
\label{Sec:scf}

The preceding tasks (charge density update, Hartree potential,
Hamiltonian and eigenvalue solver) are all methodologically simple,
with well-defined standard choices, since they all relate to the
densities and potentials within a single s.c.f. iteration of the
Kohn-Sham equations only.

However, in order to run a complete, self-consistent Kohn-Sham or
generalized ground state calculation, many such cycles must be
performed. Beginning with well-defined initial criteria,
self-consistency of the charge density and orbitals must be reached,
and must be reached within a rather finite number of iterations. This
is a non-linear optimization problem and not always trivial.

The most important keywords related to this problem in FHI-aims are
\keyword{adjust\_scf}, which is set by default and automates the
process with a choice of s.c.f. settings that is often safe. The key
parameters that can be manually adjusted are
\keyword{charge\_mix\_param} and \keyword{occupation\_type}. Many more
keywords are described below, but usually, these are the relevant
choices.

For many standard problems in electronic structure theory--especially
systems with a large, well-defined HOMO-LUMO or band gap---reaching
self-consistency today presents essentially no problem, and is
achieved to great accuracy already within $\approx$10 or so
iterations.

However, in cases where the band structure is metallic, different
charge or spin states are close to one another or in competition,
there may be several self-consistent solutions, depending on the
exact chosen initialization. Even worse, in such cases reaching even a
\emph{single} one of potentially several different self-consistent
solutions can be problematic.

It is very important to remember that different stationary densities
for the exact same atomic geometry and for the exact same density
functional are a real and not always unrealistic possibility in DFT.
A simple example are
antiferromagnetic vs. ferromagnetic spin states in some systems. In
such cases, the true ground state in a DFT sense is the stationary density
that yields the lowest energy. It can be found by way of a global search for
different stationary densities, usually by varying the initial density guess.

The present section summarizes all available options in FHI-aims to
facilitate the self-consistent solution of any given problem in
FHI-aims in as few iterations as possible, including:
\begin{itemize}
 \item Initialization of the s.c.f. cycle
 \item Criteria for the convergence of the self-consistency solution
 \item Electron density mixing
 \item Electron density \emph{preconditioning}
\end{itemize}
Please refer to Ref. \cite{Blum08} for a more exhaustive discussion of
the physics / mathematics behind the individual choices laid out
below.

\emph{Important note:} The following settings are made or required by
default.
\begin{itemize}
  \item \emph{The initial spin density must be specified in a spin-polarized
    calculation.} In spin-polarized systems, the choice of a good initial
    spin density can be critical for good convergence. For example, for a free
    atom, you might wish for a high-spin initial density according to Hund's
    rules. In a ferromagnetic Fe crystal, you might want to use a
    \keyword{default\_initial\_moment} of 2 (far lower than the Hund's rule
    value) to obtain fast convergence. In an antiferromagnetic Cr crystal, a
    ferromagnetic default initialization might do no good at all. And in a
    molecule with a single magnetic atom enclosed, you might want a
    spin-polarized initial moment only for that atom, but not for the
    surrounding molecule. In short, FHI-aims can not and should not guess the
    spin initialization for you. The program will stop if no initial
    moments of any kind are provided by the user. Setting either an overall
    \keyword{default\_initial\_moment} (in \texttt{control.in}), or (better!)
    at least one individual \keyword{initial\_moment} tag in \texttt{geometry.in}, or
    both will allow you to run.
  \item \emph{Use of the Kerker preconditioner for periodic systems.} This
    option can greatly improve the s.c.f. convergence especially of large
    periodic systems (see \keyword{preconditioner} for more details). At the
    very least, it does not appear to do much harm, and is therefore now used
    by default in any periodic calculation. {\it
    However, for very large systems the Kerker preconditioner can cost significant
    amounts of time -- see the detailed timing output that is written by the
    code at the end of each s.c.f. iteration. You may try to switch it off.}
    Should you encounter any
    difficulties, either turn the \keyword{preconditioner} off by hand, or
    play with associated screening momentum, $q_0$ (default: $q_0$=2.0
    bohr$^{-1}$).
  \item The keyword \keyword{sc\_init\_iter} sets the number of s.c.f. iterations
    after which the Pulay mixer resets itself from scratch. This can significantly
    help in cases of bad convergence. If you have real mixer trouble, please
    consider this keyword.
\end{itemize}

Regarding options to converge the self-consistency cycle, note that
one further important parameter is not covered here but instead in
Sec. \ref{Sec:EV}: The ``broadening'' of (fractional) occupation numbers
around the Fermi level. Especially in metallic systems, this
broadening must be large enough to prevent oscillations around the
Fermi level, independent of the methods laid out below.

For further suggestions to improve s.c.f. convergence, see Sec. \ref{sec:trouble-scf}.

\subsection{Visualizing the convergence of the s.c.f. cycle}
\label{sub:vis-scf}

There is a simple tool that can be used to visualize the s.c.f. convergence
behavior of FHI-aims graphically for a given run. Preferably do this
analysis on your desktop computer (i.e., copy over the necessary
files). This is what you need to do:
\begin{itemize}
  \item Install the Grace 2D visualization program
    (http://plasma-gate.weizmann.ac.il/Grace/) on your computer.
  \item Go to the directory with the FHI-aims output file you wish to analyze.
  \item From the FHI-aims \emph{utilities} directory, copy over the
    file \texttt{scf\_convergence\_template.agr} .
  \item At the commandline, call the FHI-aims utility \\
    \texttt{plot\_scf\_convergence.pl} [FHI-aims output file] \\
    to visualize the s.c.f. convergence behavior of the FHI-aims
    output file. \texttt{plot\_scf\_convergence.pl} can be found in
    the FHI-aims \emph{utilities} directory and must (of course) be
    called with the correct directory path preceding the file name.
\end{itemize}
If successful, this procedure will assemble and open a graphical
representation of the s.c.f convergence of FHI-aims.

\newpage

\subsection*{Tags for \texttt{geometry.in}:}

\keydefinition{initial\_charge}{geometry.in}
{
  \noindent
  Usage: \keyword{initial\_charge} \option{charge} \\[1.0ex]
  Purpose: Allows to place an initial charge on an \keyword{atom} in
    file \texttt{geometry.in}. \\[1.0ex]
  \option{charge} is a real number. Default: 0. \\
}
The \keyword{initial\_charge} keyword always applies to the last
\keyword{atom} previously specified in input file
\texttt{geometry.in}. The charge is introduced by using an ionic
instead of neutral spherical free-atom density on that site in the
initial superposition-of-free-atoms density. Note that initial charge
densities are generated by the functional specified with \keyword{xc}
for DFT-LDA/GGA, but refer to \texttt{pw-lda} densities for all other
functionals (hybrid functionals, Hartree-Fock, ...).

\keydefinition{initial\_moment}{geometry.in}
{
  \noindent
  Usage: \keyword{initial\_moment} \option{moment} \\[1.0ex]
  Purpose: Allows to place an initial spin moment on an \keyword{atom} in
    file \texttt{geometry.in}. \\[1.0ex]
  \option{moment} is a real number, referring to the electron
  difference $N^\uparrow - N^\downarrow$ on that site. Default: Zero,
  unless \keyword{default\_initial\_moment} is set explicitly. \\
}
The \keyword{initial\_moment} keyword always applies to the last
\keyword{atom} previously specified in input file
\texttt{geometry.in}. The moment is introduced by using a
spin-polarized instead of an unpolarized spherical free-atom density
on that site in the initial superposition-of-free-atoms density. Note
that initial charge densities are generated by the functional
  specified with \keyword{xc}
for DFT-LDA/GGA, but refer to \texttt{pw-lda} densities for all other
functionals (hybrid functionals, Hartree-Fock, ...).

\newpage

\subsection*{Tags for general section of \texttt{control.in}:}

\keydefinition{adjust\_scf}{control.in}
{
  \noindent
  Usage: \keyword{adjust\_scf} \option{frequency} \option{number} \\[1.0ex]
  Purpose: Adjusts key parameters that govern the s.c.f. cycle based on
    a rough estimate of the system's band gap. \\[1.0ex]
  \option{frequency} is a keyword, \texttt{once}, \texttt{never}, or \texttt{always}.
      Default: \texttt{once}. \\[1.0ex]
  \option{number} is an integer number (zero or greater). Default: 2. \\
}
This keyword decides whether key s.c.f. convergence parameters will be adjusted
automatically during the s.c.f. cycle, based on a simple estimate of the system
character according to its approximate HOMO-LUMO gap or band gap.

The \option{number} keyword determines in which iteration of the s.c.f. cycle the
adjustment will be attempted. A zero value corresponds to the initial s.c.f. cycle;
values of 1, 2, etc. correspond to an update after the first, second, etc. s.c.f.
iterations are almost complete and their eigenvalue spectra known.

The \option{frequency} keyword determines for which full s.c.f. cycle
(i.e., for which geometry step) an adjustment will be made:
\begin{itemize}
\item \texttt{once} means that an adjustment of the s.c.f. parameters
  will only be made in the first geometry in a geometry relaxation or
  MD run.
\item \texttt{always}  indicates that an adjustment will be made for
  every new s.c.f. cycle, i.e., for every new geometry in a run.
\item \texttt{never} indicates that no adjustment will be attempted.
\end{itemize}

Parameters will only be adjusted if
they are not explicitly set by a keyword in \texttt{control.in}. Any
parameter that is included in \texttt{control.in} will not be modified
by the \keyword{adjust\_scf} keyword. For \option{frequency}
\texttt{once} or \texttt{always}, the following parameters may be adjusted:
\begin{itemize}
  \item The initial default value of \keyword{charge\_mix\_param} for
    \keyword{mixer} \texttt{pulay} will be set to 0.05 (i.e., an overall
    cautious value). For \option{frequency} {never}, the default value
    of the \keyword{charge\_mix\_param} keyword remains at its usual
    default value 0.2 (for many metallic or spin-polarized systems,
    this is a fairly aggressive value).
  \item In s.c.f. iteration \texttt{number}, the current estimated
    value of the HOMO-LUMO gap (for solids, the band gap) is
    checked. If the system shows fractional occupation numbers or if
    the estimated gap has a value of less than 0.2~eV at this point,
    the system is likely near-degenerate or metallic and the
    s.c.f. cycle could be diffcult to converge. In this case, the
    \keyword{charge\_mix\_param} is set to 0.02 -- a cautious value
    but, in conjunction with the Pulay \keyword{mixer}, still
    surprisingly effective. The broadening of the occupation numbers
    near the Fermi level is increased to \keyword{occupation\_type}
    \texttt{option} 0.05 [eV].
  \item If, instead, the gap is found to be equal or greater than
    0.2~eV in s.c.f. iteration \texttt{number}, the rather aggressive
    default \keyword{charge\_mix\_param} 0.2 is kept for the Pulay
    mixer, and the default broadening value (suitable for non-metallic
    systems) \keyword{occupation\_type} \texttt{option} 0.05 [eV] is
    also kept.
\end{itemize}

\keydefinition{charge\_mix\_param}{control.in}
{
  \noindent
  Usage: \keyword{charge\_mix\_param} \option{value} \\[1.0ex]
  Purpose: Parameter for simple linear mixing of electron densities of
    previous and present s.c.f. iterations \\[1.0ex]
  \option{value} is a real number between 0. and 1. Default: Depends
    on chosen \keyword{mixer} algorithm. Now set by the
    \keyword{adjust\_scf} keyword. \\
}
See Ref. \cite{Blum08} for details regarding the available density
mixing algorithms. In the simplest case of a \texttt{linear}
\keyword{mixer}, \option{value} specifies a constant value $\hat{G}^1$
to mix the output density of the Kohn-Sham Equations in iteration
number $\mu$, $n_\text{KS}^{(\mu)}$, with the (already mixed)
\emph{input} density that defined those equations, $n^{(\mu-1)}$:
\begin{equation}\label{Eq:dmp}
  n^{(\mu)}_\text{dmp} = n^{(\mu-1)} + \hat{G}^1
  (n_\text{KS}^{(\mu)}- n^{(\mu-1)}) \, .
\end{equation}
If a \keyword{preconditioner} is specified,
\keyword{charge\_mix\_param} defines an additional linear factor to
that preconditioner. In case of a \texttt{pulay} \keyword{mixer}, all
density residuals are mixed with this factor.

In general, the best choice for \option{value} is system-dependent,
and also depends on the chosen \keyword{mixer} algorithm. In general,
please also see Appendix \ref{sec:trouble-scf}, since several keywords
can be used to alleviate s.c.f. mixing instabilities.
\begin{itemize}
  \item In principle, a \texttt{linear} \keyword{mixer} will always
    converge with a sufficiently small \option{value}. In easy cases,
    we recommend \option{value}=0.1-0.2, but in difficult cases,
    ``sufficiently small'' can mean one to three orders of magnitude(!)
    lower, i.e., the process can be apallingly slow.
  \item For a straight \texttt{pulay} \keyword{mixer}, our default
    \option{value} is adjusted according to the \keyword{adjust\_scf}
    keyword, depending on the estimated band gap / HOMO-LUMO gap. For
    non-metallic systems, we choose a conservative value of 0.2. In
    metallic systems or systems that are otherwise problematic, the
    default value set by \keyword{adjust\_scf} is
    \option{value}=0.02. Note that
    this small value does not necessarily correspond to slow mixing
    since the Pulay mixer will learn over time and accelerate
    the mixing process.
  \item In metallic systems, density oscillations can occur from one
    iteration to the next (charge sloshing). This can be alleviated by
    a \keyword{preconditioner}. With a preconditioner and
    \texttt{pulay} \keyword{mixer} specified together, \option{value}
    is still important and may be chosen around 0.05 .
\end{itemize}
See also the \keyword{mixer} and \keyword{preconditioner}
keywords.

\keydefinition{relative\_fp\_charge\_mix}{control.in}
{
  \noindent
  Usage: \keyword{relative\_fp\_charge\_mix} \option{value} \\[1.0ex]
  Purpose: Parameter for under-relaxation of the fixed point part of s.c.f. cycle with the \option{broyden} \keyword{mixer} \\[1.0ex]
  \option{value} is a real number between 0. and 1. Default: 0.05\\
}
\keyword{relative\_fp\_charge\_mix} determines the under-relaxation of the fixed point part of the Broyden mixer s.c.f. cycle together with \keyword{charge\_mix\_param}.
\keyword{relative\_fp\_charge\_mix} and \keyword{charge\_mix\_param} are multiplicative, and if no history is included the effective under-relaxation is \keyword{relative\_fp\_charge\_mix} times \keyword{charge\_mix\_param}.

\keydefinition{default\_initial\_moment}{control.in}
{
  \noindent
  Usage: \keyword{default\_initial\_moment} \option{moment} \\[1.0ex]
  Purpose: For spin-polarized calculations, sets the default initial
    moment of the spin-polarized atoms that make up the initial
    electron density. \\[1.0ex]
  \option{moment} is either a string or a number that defines the
    desired initial number of electrons, $N^\uparrow -
    N^\downarrow$. Default: \texttt{hund} for isolated atoms. Zero
    otherwise. \\
}
Sets the default initial spin moment for all atoms whose
\keyword{initial\_moment}s are not specified explicitly in
\texttt{geometry.in}.

If there is at least a single \keyword{initial\_moment} keyword specified in
\texttt{geometry.in}, the \keyword{default\_initial\_moment} will be zero for
all other atoms, for which no \keyword{initial\_moment} is specified
explicitly.

If no \keyword{initial\_moment} is included in \texttt{geometry.in} at all, the
\keyword{default\_initial\_moment} must be specified explicitly by the user
for the code to run at all.

For most (bonded) systems, it is advisable to set the
\keyword{default\_initial\_moment} to a numerical value close to what most
atoms in the structure will do. For example, ferromagnetic Fe would be close to
2, whereas a large non-magnetic molecule would be better served with something
close to zero.

For isolated free-atom calculations, \keyword{default\_initial\_moment}
\texttt{hund} can be used. This will result in the usual high-spin atom
initialization characteristic of free atoms.

Note that at least one moment in the system must be set to a non-zero value in
order to reach any spin-polarized state at all. If the initial spin
polarization is zero, the final s.c.f. result will also not be spin-polarized,
no matter how magnetic the system is in reality.

\emph{Warning.} It is not advisable to set a blanket
\keyword{default\_initial\_moment} \texttt{hund} for any structures
other than free atoms. The result can be enormous convergence
difficulties because the calculation begins from a bad starting point
-- a high-spin state that is completely unrealistic for most bonded
structures. A calculation will converge \emph{much} better if the
initial spin moment is realistic for \emph{each} individual atom in
the structure.

\emph{One more warning:} It is \emph{not} advisable to set a
\keyword{default\_initial\_moment} other than zero in structures in
which only a few atoms actually carry spin (such as a large molecule
with a few transition metal atoms). Rather, a good choice would be to
set a zero \keyword{default\_initial\_moment} and
then set finite \keyword{initial\_moment} values for individual atoms in
\texttt{geometry.in}.

\emph{And again:} Setting
\keyword{default\_initial\_moment} to zero and \emph{not} specifying
any \keyword{initial\_moment} values in \texttt{geometry.in} will lead
to a zero spin state in the final result, simply because the system is
never given any indication which way to break its symmetry. So one does
need to set a finite moment somewhere if a finite-spin converged
solution is expected in the calculation.

The upshot is: It pays to think about the right spin
initialization. There may be multiple different
self-consistent solutions, and in spin-polarized systems, this can
happen for very natural reasons  (e.g., ferromagnetic
vs. antiferromagnetic states). Similar to geometry optimization,
starting from a very bad initial guess can cause
problems. Conversely, a good starting point may greatly simplify a
calculation.

\keydefinition{force\_potential}{control.in}
{
  \noindent
  Usage: \keyword{force\_potential} \option{type} \\[1.0ex]
  Purpose: Determines how far / with which potential the Kohn-Sham
    equations are solved. \\[1.0ex]
  \option{type} is a string that determines the potential
    used. Default: \texttt{sc} \\
}
This option is not required under normal circumstances. It is mainly
useful to produce / test a fast, non-self-consistent solution for a
superposition-of-atoms potential that yields only the sum of
eigenvalues as a result. If a non-self-consistent total energy is
needed (correct only for the non-spinpolarized free-atom density!),
running a normal calculation with \keyword{sc\_iter\_limit}=0 is the
better way.

Options for \option{type} are:
\begin{itemize}
  \item \option{sc}: Self-consistent Kohn-Sham potential in each
    s.c.f. iteration
  \item \option{superpos\_pot}: Superposition of free-atom potentials,
    evaluation only once (no self-consistency cycle). \emph{Restriction: This
    method works only for self-adapting} \subkeyword{species}{angular\_grids}
    \emph{(i.e.,} \subkeyword{species}{angular\_acc} \emph{not equal zero for at least one
    \keyword{species}). This also means that the option will not perform well
    in periodic boundary conditions. A fix is simple, contact us if needed.}
  \item \option{superpos\_rho}: Superposition potential created from sum of
    free-atom densities; evaluation only once (no self-consistency cycle)
  \item \option{non-self-consistent}: Same as \option{superpos\_rho}.
\end{itemize}

\keydefinition{ini\_linear\_mixing}{control.in}
{
  \noindent
  Usage: \keyword{ini\_linear\_mixing} \option{number} \\[1.0ex]
  Purpose: If \keyword{mixer} is \texttt{pulay}, specifies simple
    linear mixing for a \option{number} of initial iterations. \\[1.0ex]
  \option{number} is the integer number of iterations for which linear
    mixing is done. Default: 0 . \\
}
Try only if the standard / preconditioned \texttt{pulay}
\keyword{mixer} definitely fails. Keywords
\keyword{ini\_linear\_mix\_param}, \keyword{ini\_spin\_mix\_param}
can be used to specify separate mixing parameters for the initial
linear mixing.

\keydefinition{ini\_linear\_mix\_param}{control.in}
{
  \noindent
  Usage: \keyword{ini\_linear\_mix\_param} \option{value} \\[1.0ex]
  Purpose: Separate parameter for simple linear mixing of electron
  densities for \keyword{ini\_linear\_mixing}. \\[1.0ex]
  \option{value} is a real number between 0. and 1. Default: same as
  \keyword{charge\_mix\_param}. \\
}
\keyword{ini\_linear\_mixing} should only be tried if the standard
algorithms provably fail. In that case, \option{value} should be
relatively small.

\keydefinition{ini\_spin\_mix\_param}{control.in}
{
  \noindent
  Usage: \keyword{ini\_spin\_mix\_param} \option{value} \\[1.0ex]
  Purpose: For spin-polarized calculations, separate parameter to mix
    the \emph{spin} density during
  \keyword{ini\_linear\_mixing}. \\[1.0ex]
  \option{value} is a real number between 0. and 1. Default: same as
  \keyword{spin\_mix\_param} \\
}
\keyword{ini\_spin\_mix\_param} should only be tried if the standard
algorithms provably fail. In that case, \option{value} should be
relatively small.

\keydefinition{mixer}{control.in}
{
  \noindent
  Usage: \keyword{mixer} \option{type} \\[1.0ex]
  Purpose: Specifies the electron density mixing algorithm used to
    achieve fast and stable convergence towards the self-consistent
    solution. \\[1.0ex]
  \option{type} specifies the density mixing algorithm used and can
  be set to either \option{linear}, \option{pulay}, or \option{broyden}. Default: \texttt{pulay}. \\
}
FHI-aims provides three mainstream density mixing algorithms across the
s.c.f. cycle, \option{type} \texttt{linear}, \texttt{pulay}, and \option{broyden}. We
here only give a brief summary of options, please see
Ref. \cite{Blum08} for further references and for the exact
mathematical details.

For most practical purposes (non-pathological systems), Pulay's DIIS
mixing algorithm \cite{Pulay80} is robust and fast, and should be the
algorithm of choice. For this algorithm, \keyword{n\_max\_pulay} $n$
determines the number of past iterations $\mu-k$ ($k$=1,...,$n$) to be
mixed with the Kohn-Sham output density of iteration
$\mu$. \keyword{charge\_mix\_param} determines an additional
(system-dependent) linear factor that is multiplied with the output
density change of the Pulay mixer. Normally, this (and perhaps a
\keyword{preconditioner}) is all you need to do to ensure convergence.

In some pathological cases, reaching self-consistency is a more tricky
problem. Broadly speaking, these are systems with a small HOMO-LUMO
gap (band gap) and/or several competing possibilities for a
self-consistent solution. Specifically, these difficult cases include:
\begin{itemize}
  \item Large metallic systems (e.g., slabs), where charge may
    ``slosh'' from one end of the system to another before reaching
    self-consistency. In that case, the \texttt{pulay} \keyword{mixer}
    may be used together with a large \keyword{charge\_mix\_param}
    \emph{and} a \keyword{preconditioner} (see that keyword) to dampen
    the resulting oscillations. Also make sure that
    \keyword{occupation\_type} is set to a sufficiently large
    broadening of occupation numbers near the Fermi level in metallic
    systems.
  \item Spin-polarized systems with competing spin states. A
    classic. If problems arise, playing with
    \keyword{ini\_linear\_mixing}, the \keyword{charge\_mix\_param}
    and \keyword{spin\_mix\_param} and further options listed in this
    section may help. Likewise, setting a specific
    \keyword{fixed\_spin\_moment} may be helpful. Finally, different
    \keyword{initial\_moment} settings may easily switch between
    different metastable self-consistent spin states (e.g.,
    ferromagnetic vs. antiferromagnetic), and should be tested
    separately if different competing spin states are suspected.
  \item Systems near a level crossing (even dimers, if two or more
    Kohn-Sham levels of different symmetry come close for a given
    binding distance). Apart from the usual mixing mechanisms, keyword
    \keyword{occupation\_type} with a larger broadening near the Fermi
    level may help alleviate this situation.
  \item Spin-polarized free atoms. The simplest conceivable systems
    may exhibit unexpected problems towards self-consistency, likely
    because the electron density can rotate between several fully
    equivalent spin states. Here, demanding a specific orbital
    occupation using the
    \keyword{force\_occupation\_basis} keyword may be useful.
\end{itemize}
\emph{In principle}, even in the toughest cases
a \texttt{linear} \keyword{mixer} will always converge with a
sufficiently small \keyword{charge\_mix\_param}. Unfortunately,
``sufficiently small'' can mean a \keyword{charge\_mix\_param} of
10$^{-2}$-10$^{-4}$, i.e., the process can be apallingly slow. Playing
with the \texttt{pulay} \keyword{mixer} settings is usually the better
strategy, unless a proof of principle is sought.

The \option{broyden} \keyword{mixer} is an improvement on the \option{linear} \keyword{mixer}. The \option{broyden} \keyword{mixer} works by effectively dividing the s.c.f. cycle into two separte parts: the space in which we have local information gained by previous evaluations of the s.c.f. cycle, and the remaining space in which we do not have information. \keyword{charge\_mix\_param} determines the linear factor which under-relaxes the density predicted by the \option{broyden} \keyword{mixer}, \keyword{n\_max\_broyden} \emph{n} controls the number of past iterations used to construct the next estimate, and \keyword{relative\_fp\_charge\_mix} is the additional (multiplicative) under-relaxation affecting only the step length in the space where we have no further information.

Note that a modification is needed when going beyond DFT-LDA/GGA
(Hartree-Fock, hybrid functionals, ...). In
that case, the density implicitly enters the two-electron exchange
operator (via the density matrix,
$\hat{n}_{ij}=\sum_l f_l c_{il} c_{jl}$, where
$i$ and $j$ label basis functions, and $l$ label the Kohn-Sham
states), and should also be mixed.

By default, for \texttt{linear}
mixing, we do not mix the exchange operator, unless keyword
\keyword{use\_density\_matrix\_hf} is enabled. The latter is the
default if the \texttt{pulay} mixer is selected. Then, the
density \emph{matrix} is submitted to the same Pulay mixing factors as
the normal charge density $n(r)$ before constructing the exchange
operator. Note that we do not have a formal density matrix
available that corresponds to the \emph{initial} superposition of
free-atom densities, making this form of mixing slightly less
efficient than for normal Kohn-Sham DFT-LDA/GGA.

\keydefinition{mixer\_threshold}{control.in}
{
  \noindent
  Usage: \keyword{mixer\_threshold} \option{keyword}
    \option{threshold} \\[1.0ex]
  Purpose: Allows to cap the density step between two iterations
    rigorously by setting an explicit threshold. \\[1.0ex]
  \option{keyword} is a string, indicating whether the following is
    the \texttt{charge} or \texttt{spin} density threshold. \\
  \option{threshold} is a real number, the maximum allowed change in
    the density norm (in electrons). Default: no thresholds. \\
}
This option is perhaps useful when there are definite convergence
problems with the standard mixing algorithms, but can otherwise safely
be ignored.

\keydefinition{n\_max\_pulay}{control.in}
{
  \noindent
  Usage: \keyword{n\_max\_pulay} \option{number} \\[1.0ex]
  Purpose: The number of past iterations that the \texttt{pulay}
    \keyword{mixer} uses for density mixing. \\[1.0ex]
  \option{number} is the number of stored iterations used by the
    mixer. Default: 8 \\
}
A larger \option{number} of stored iterations can sometimes lead to a
stabilization of the mixing process. Choosing \option{number} too
large (e.g., 20 and above), though, may destabilize the Pulay matrix,
which can become near-singular.

Note that the storage effort associated with Pulay mixing is
significant on systems with few CPUs / low memory. Each additional
stored iteration requires the storage of two charge density
residuals, and two times three charge density gradient residual
components. For large systems, low memory, and overall stable mixing,
reducing \keyword{n\_max\_pulay} may be a way to get a given
calculation below the most difficult memory barriers.

\keydefinition{n\_max\_broyden}{control.in}
{
  \noindent
  Usage: \keyword{n\_max\_broyden} \option{number} \\[1.0ex]
  Purpose: The number of past iterations that the \texttt{broyden}
    \keyword{mixer} uses for density mixing. \\[1.0ex]
  \option{number} is the number of stored iterations used by the
    mixer. Default: 8 \\
}
A larger \option{number} of stored iterations can sometimes lead to a
stabilization of the mixing process. Choosing \option{number} too
large (e.g., 20 and above), though, may destabilize the Broyden matrix,
which can become near-singular.

Note that the storage effort associated with Broyden mixing is
significant on systems with few CPUs / low memory. Each additional
stored iteration requires the storage of two charge density
residuals, and two times three charge density gradient residual
components. For large systems, low memory, and overall stable mixing,
reducing \keyword{n\_max\_broyden} may be a way to get a given
calculation below the most difficult memory barriers.

\keydefinition{postprocess\_anyway}{control.in}
{
  \noindent
  Usage: \keyword{postprocess\_anyway} \option{boolean} \\[1.0ex]
  Purpose: By default, FHI-aims simply stops if the SCF procedure does not
  converge.  In particular, all desired postprocessing steps are skipped.  If
  you do want postprocessing to be done anyway, set \option{boolean} to
  \texttt{.true.}. \\[1.0ex]
  \option{boolean} is either \texttt{.true.} or \texttt{.false.}. Default:
  \texttt{.false.}.
  \\
}

\keydefinition{prec\_mix\_param}{control.in}
{
  \noindent
  Usage: \keyword{prec\_mix\_param} \option{value} \\[1.0ex]
  Purpose: Possible separate mixing parameter while the preconditioner
  is on. \\[1.0ex]
  \option{value} is a real number between 0. and 1. Default: same as
  \keyword{charge\_mix\_param}. \\
}
It's our tentative observation that a larger mixing parameter (0.5-0.8)
is sometimes helpful with the \keyword{preconditioner}, but after the
\keyword{preconditioner} is switched off, a smaller mixing parameter
(as set by \keyword{charge\_mix\_param}) may be
desirable. \keyword{prec\_mix\_param} can provide the needed separate
setting, if desirable.

\keydefinition{preconditioner}{control.in}
{
  \noindent
  Usage: \keyword{preconditioner} \option{keyword} [\option{type}]
    [\option{value}] \\[1.0ex]
  Purpose: ``Master keyword'' that precedes any information related to
    the preconditioner. May appear multiple times in
    \texttt{control.in}, in different contexts. \\[1.0ex]
  Restriction: Because it cannot simply be written in a density matrix
    formulation, the preconditioner has no effect on the density matrix
    entering the exchange operator for Hartree-Fock, hybrid
    functionals, etc. \\[1.0ex]
  \option{keyword} : A string, indicating the type of information
    following. \\
  \option{type} : If required by \option{keyword}, a string with more
    details. \\
  \option{value} : If required by \option{keyword}, a numerical value. \\
}
Default values:
\begin{itemize}
  \item Non-periodic systems: \keyword{preconditioner} \texttt{kerker}
    \texttt{off} (no preconditioner).
  \item Periodic systems: \keyword{preconditioner} \texttt{kerker}
    \texttt{2.0} (Preconditioner with a momentum of $q_0$=2.0
     bohr$^{-1}$).
\end{itemize}
See Ref. \cite{Blum08} regarding the mathematical
definition of the Kerker-type \cite{Manninen75,Nieminen77,Kerker81}
preconditioner, which is the currently implemented form.

See keyword \keyword{precondition\_max\_l} for the angular momentum
cutoff specified for the \texttt{kerker} \keyword{preconditioner}.

\option{keyword} can have the following forms, controlling various
aspects of the preconditioner:
\begin{itemize}
  \item \texttt{kerker} :
    \begin{itemize}
      \item if followed by \texttt{off} : No preconditioner used.
      \item if followed by \option{value} : \option{value} is a real
        positive number, indicating the characteristic momentum $q_0$
        associated with the Thomas-Fermi type screening assumed in the
        preconditioner (in bohr$^{-1}$). Typical values in the
        literature range around 1.0-2.0 bohr$^{-1}$, but larger values
        may be useful in small clusters.
    \end{itemize}
  \item \texttt{turnoff} : To avoid any residual influence on the
    s.c.f. cycle, the preconditioner can be switched off at a given
    level of s.c.f. convergence, leaving the remaining convergence to
    a pure \texttt{pulay} \keyword{mixer}. Possible \option{type}s of
    turnoff criteria are:
    \begin{itemize}
      \item \texttt{charge} : \option{value} refers to the
        root-mean-square deviation between $n^{(\mu-1)}$ (the mixed
        and preconditioned \emph{input} density to the Kohn-Sham
        Equations) and $n_\text{KS}^{(\mu)}$ (the unmixed
        \emph{output} density from the Kohn-Sham Equations). Default:
        \keyword{sc\_accuracy\_rho}.
      \item \texttt{energy} : \option{value} refers to the total
        energy difference between two successive iterations [in eV].
      \item \texttt{sum\_ev} : \option{value} refers to the
        difference in the eigenvalue sums between two successive
        iterations [in eV].
    \end{itemize}
    All requested convergence criteria for the preconditioner must be
    fulfilled. If \emph{no} explicit turnoff criterion is set, the
    \keyword{preconditioner} \texttt{turnoff} \texttt{charge},
    \texttt{energy} and \texttt{sum\_ev} defaults
    to the same values as \keyword{sc\_accuracy\_rho},
    \keyword{sc\_accuracy\_etot}, and \keyword{sc\_accuracy\_eev},
    respectively.
  \item \texttt{dielectric} : The inverse of the microscopic dielectric function,
    $\epsilon^{-1}(\mathbf{r},\mathbf{r}' ; 0)$ is used to precondition the density. This
    option is experimental and computationally expensive but the the dielectric function
    can be theoretically justified to be a good preconditioner. Works only for non-periodic systems.
  \item \texttt{none} or \texttt{off} : No preconditioner used.
\end{itemize}

\keydefinition{precondition\_max\_l}{control.in}
{
  \noindent
  Usage: \keyword{precondition\_max\_l} \option{value} \\[1.0ex]
  Purpose: Angular momentum cutoff used in the partitioned
    atom-centered real-space form of the \texttt{kerker}
    \keyword{preconditioner}. \\[1.0ex]
  \option{value} is an integer number, specifying an angular
    momentum. Default: 0. \\
}
We use a partitioned atom-centered multipole rewrite of the Kerker
preconditioner in angular momentum space, similar in spirit to the
Hartree potential (see Ref. \cite{Blum08} for details). In principle,
\keyword{precondition\_max\_l} is thus the equivalent of the
\subkeyword{species}{l\_hartree} angular momentum cutoff for the expansion.
However, since
we here precondition a density \emph{difference} (which reduces to
zero as we approach self-consistency), and since we are combatting
charge sloshing across potentially faraway parts of the systems,
preconditioning the atom-centered \emph{monopole} component of the
density difference (\option{value}=0) is often all that is needed
for the preconditioner to work. This is also the numerically most
efficient way of running the preconditioner.

\keydefinition{kerker\_factor}{control.in}
{
  \noindent
  Usage: \keyword{kerker\_factor} \option{value} \\[1.0ex]
  Purpose: \option{value} is a real
        positive number. Additional empirical factor for the last
        step of the kerker precondioner. Best results were achieved by
        choosing \option{value} as the characteristic momentum $(q_0-0.5)$
        associated with the Thomas-Fermi type screening assumed in the
        kerker preconditioner (\texttt{kerker}
    \keyword{preconditioner}). \\[1.0ex]
}
This keyword is experimental and derived fully empirically. For notorious cases
the keyword may help to achieve convergence (faster). For $q_0=1$ the method is
identical to the standard kerker preconditioner.

\keydefinition{restart}{control.in}
{
  \noindent
  Usage: \keyword{restart} \option{file} \\[1.0ex]
  Purpose: Saves and reads the final wave function of each scf-cycle
    to/from \option{file}.\\[1.0ex]
  \option{file} is a string, corresponding to the desired restart filename. \\
}
If \option{file} is not yet present, the calculation simply writes
that file during the run. If \option{file} is already present, it is
read and the wave function contained therein is used to restart the
calculation, instead of a fresh superposition of free atoms
initialization. Mind that \keyword{restart} is currently not supported for
keywords \keyword{load\_balancing} and \keyword{use\_local\_index}
so that \option{file} will not be written or read when either keyword is
set.

Note that there is now a potentially better restart option than the
\keyword{restart} keyword. \keyword{restart} comes in many internal
variants and, for example, restarting a calculation with more MPI
tasks than a previous run is not possible. The alternative
\keyword{elsi\_restart} keyword (which uses a different storage format
than the \keyword{restart} keyword) is much more flexible and may be
the better choice.

It is important to note that the \keyword{restart} infrastructure
corresponds to a restart from the last Kohn-Sham orbitals, not from
the last density. In practice, this means that the code will restart
from the last \emph{unmixed} Kohn-Sham density, not from the last
\emph{mixed} density. When restarting from a non-self-consistent
starting point, this can lead to unexpected jumps in the calculated
non-self-consistent total energy. Only the
self-consistent total energy is truly meaningful and this (the
self-consistent) total energy should be the same for the same
stationary density. (Note also that some systems may exhibit several
different self-consistent stationary densities -- a simple example are
antiferromagnetic vs. ferromagnetic spin states in some systems. In
such cases, the true ground state in a DFT sense is the one with the
lowest energy and must be found by varying the initial density guess.)

In parallel runs, there is one file for each process, numbered
as \option{fileXXX}. See also \keyword{restart\_read\_only}
and \keyword{restart\_write\_only}. There are limited checks on
whether or not the restart file provided is actually from the same
system, but ensuring that a given restart file works is mainly the
user's responsibility. See also \keyword{restart\_relaxations}
and \keyword{MD\_restart} for more information on the separate restarting process for
relaxations and molecular dynamics respectively.

A much more complete overview of the restart infrastructure in FHI-aims
can be found in the dedicated Section \ref{Sec:restarts}.

\keydefinition{restart\_read\_only}{control.in}
{
  \noindent
  Usage: \keyword{restart\_read\_only} \option{file} \\[1.0ex]
  Purpose: reads the final wave function of the last scf-cycle in a
    preceding calculation from \option{file}.\\[1.0ex]
  \option{file} is a string, corresponding to the desired restart
  filename. \\
}
Similar to keyword \keyword{restart}, but does not overwrite
\option{file} at any time (this may facilitate another restart from
the same file later on).

\keydefinition{restart\_write\_only}{control.in}
{
  \noindent
  Usage: \keyword{restart\_write\_only} \option{file} \\[1.0ex]
  Purpose: writes the final wave function of the last scf-cycle to
  \option{file} for a later restart. \\[1.0ex]
  \option{file} is a string, corresponding to the desired restart
  filename. \\
}
Similar to keyword \keyword{restart}, but does not read the restart
  \option{file} in case it already exists.

\keydefinition{restart\_save\_iterations}{control.in}
{
  \noindent
  Usage: \keyword{restart\_save\_iterations} \option{number} \\[1.0ex]
  Purpose: writes restart information every \option{number}
    scf-iterations or at the end of each cycle. \\[1.0ex]
  \option{number} is the integer number of s.c.f. iterations after
  which the restart information is rewritten. \\
}
See also \keyword{restart}.

\keydefinition{force\_single\_restartfile}{control.in}
{
  \noindent
  Usage: \keyword{force\_single\_restartfile} \option{.true.} \\[1.0ex]
  Purpose: Forces FHI-aims to always write a single, wavefunction based restart
  file if possible.\\[1.0ex]
}
For technical and efficiency reasons FHI-aims uses different types of
restartfiles depending on the eigenproblem solver (see
Section~\ref{Sec:restarts} for details). In cases where the wavefunction is
needed (e.g. for external post-processing, \ldots), the default \keyword{restart}
handling might make it difficult to do so.

This option only works for cluster (non-periodic) and periodic
$\Gamma$-only calculations.

\keydefinition{elsi\_restart}{control.in}
{
  \noindent
  Usage: \keyword{elsi\_restart} \option{task} \option{freq} \\[1.0ex]
  Purpose: Controls the density-matrix-based restart using parallel matrix I/O
    routines provided by the ELSI software. Although they are not known to
    interfere with each other, the two restart keywords, \keyword{restart} and
    \keyword{elsi\_restart}, should not be requested in the same calculation. \\[1.0ex]
  \option{task} is a string, specifying the desired restart task. \\[1.0ex]
  \option{freq} is a positive integer, specifying the frequency of outputting
    restart info. \\
}
Available options for \option{task} are:
\begin{itemize}
  \item \texttt{write} : Writes restart info (in ELSI format) to files. The info
    will be written to files in every \option{freq} s.c.f. iterations, and will
    always be written out after a converged cycle. The number of files depends
    on the choice of \keyword{elsi\_restart\_use\_overlap}.
  \item \texttt{read} : Reads restart info from files and uses it (instead of
    free atom superpositions) as the initial guess of an s.c.f. cycle. The code
    will search for files written by the \texttt{write} option. If relevant
    files do not exist, the code will abort. \option{freq} is not used for this
    option. The restarted run can use any number of MPI tasks, i.e., not
    necessarily the same number as used in the original run.
  \item \texttt{read\_and\_write} : Performs what the \texttt{write} and
    \texttt{read} options do. Note that if the restart files do not exist, the
    code will still proceed normally.
\end{itemize}

Restarting hybrid functional calculations from the density matrix obtained using
a different (i.e. less expensive) functional may reduce the number of SCF
steps taken by the hybrid calculation. However, hybrid functional calculations
currently require that the \keyword{symmetry\_reduced\_k\_grid} be set to
\texttt{.false.}. If you are planning to restart a calculation with a hybrid
functional from a less-expensive functional, first use \keyword{elsi\_restart}
combined with \keyword{symmetry\_reduced\_k\_grid} \texttt{.false.} for initial
SCF convergence with the less expensive functional, then restart your
calculation with the hybrid functional.

\keydefinition{elsi\_restart\_use\_overlap}{control.in}
{
  \noindent
  Usage: \keyword{elsi\_restart\_use\_overlap} \option{boolean} \\[1.0ex]
  Purpose: By default, \keyword{elsi\_restart} uses density matrices as its
    restart info. The number of density matrix files is equal to the number of
    \texttt{k}-points times the number of spin channels. The density matrix
    files can be used to restart a calculation of the same geometry.
    \keyword{elsi\_restart\_use\_overlap} should be used to initialize a
    calculation with the density matrices obtained from a different structure.
    When this keyword is set to \texttt{.true.}, the code writes overlap matrix
    files in addition to density matrix files. The number of overlap files is
    equal to the number of \texttt{k}-points. The stored overlap matrices are
    used to extrapolate the stored density matrices. \\[1.0ex]
  \option{boolean} is either \texttt{.true.} or \texttt{.false.}. Default:
  \texttt{.false.}. \\
}

\keydefinition{calc\_dens\_superpos}{control.in}
{
  \noindent
  Usage: \keyword{calc\_dens\_superpos} .true. \\[1.0ex]
  Purpose: When reinitializing the SCF cycle, fall back
           to the superposition of free atoms density for
           the initial guess (instead of using the guess from
           the previous converged cycle). \\[1.0ex]
}

\keydefinition{sc\_abandon\_etot}{control.in}
{
  \noindent
  Usage: \keyword{sc\_abandon\_etot} \option{iter} \option{threshold} \\[1.0ex]
  Purpose: If the s.c.f. cycle diverges, abort the s.c.f. cycle after a
  specified number of iterations between which the total energy changed
  by more than a given threshold.\\[1.0ex]
  \option{iter} is an integer number of iterations after which the
  calculation is aborted. Default: 5 iterations. \\[1.0ex]
  \option{threshold} is a positive real number - if the total energy
  keeps changing by more than this number [in eV], the abort will be
  triggered. Default: 1000~eV.\\
}
This keyword allows to catch obviously ludicrous runs. If it triggers,
something went seriously wrong during the mixing procedure and the
settings for mixers, occupation broadening, preconditioner etc. should
all be revisited very carefully.

The alternative setting \keyword{sc\_abandon\_etot} \texttt{never}
switches the abort off, restoring the previous behaviour.

\keydefinition{sc\_accuracy\_eev}{control.in}
{
  \noindent
  Usage: \keyword{sc\_accuracy\_eev} \option{value} \\[1.0ex]
  Purpose: Convergence criterion for the self-consistency cycle, based
    on the sum of eigenvalues. \\[1.0ex]
  \option{value} is a small positive real number [in eV], against
  which the difference of the eigenvalue sum between the present and
  previous s.c.f. iteration is checked. \\
}
Very sensitive criterion for s.c.f. convergence. Usually,
\option{value}=10$^{-3}$~eV is enough to indicate a
reliable total-energy and force convergence. If value is set to zero or
not given, the sum of eigenvalues will not be used as a convergence
criterion.

\keydefinition{sc\_accuracy\_etot}{control.in}
{
  \noindent
  Usage: \keyword{sc\_accuracy\_etot} \option{value} \\[1.0ex]
  Purpose: Convergence criterion for the self-consistency cycle, based
    on the total energy. \\[1.0ex]
  \option{value} is a small positive real number [in eV], against
  which the difference of the total energy between the present and
  previous s.c.f. iteration is checked. \\
}
The Harris-Foulkes form of the functional is used as the total energy
in FHI-aims (see Ref. \cite{Blum08} for a brief discussion). A typical tight
convergence criterion is \option{value}=10$^{-6}$~eV. If value is set
to zero or not given, the total energy will not be used as a
convergence criterion.

\keydefinition{sc\_accuracy\_forces}{control.in}
{
  \noindent
  Usage: \keyword{sc\_accuracy\_forces} \option{value} \\[1.0ex]
  Purpose: Convergence criterion for the self-consistency cycle, based
    on energy derivatives (``forces''). \\[1.0ex]
  \option{value} is EITHER a small positive real number [in eV/{\AA}],
  against which the maximum difference of atomic forces between the
  present and previous s.c.f. iteration is checked, OR a string,
  '\texttt{not\_checked}'. \\[1.0ex]
  Default: \texttt{not\_checked} \\
}
\emph{Attention:} If keyword \keyword{sc\_accuracy\_forces} is set in
\texttt{control.in}, forces are by default computed, regardless of
whether or not they are later needed. The rationale is that the only
way to check a requested force convergence criterion is to compute the
necessary forces, despite the added numerical effort. For single-point
calculations (no relaxation required), \keyword{sc\_accuracy\_forces}
should therefore not be set unless explicitly needed for some reason.

One can explicitly set the keyword value to the string
'\texttt{not\_checked}'. In this case, the forces are computed in only
a single shot, their s.c.f. convergence will not be checked.

In this case, the code now relies on the default convergence criterion for
the electron density itself (\keyword{sc\_accuracy\_rho}) to be sufficiently
tight to guarantee good forces. This avoids excessive numbers of force
evaluations in production runs.

Calculating forces is relatively expensive, so this convergence
criterion is either not checked (default behavior), or it
is checked \emph{after} the purely electronic /
energetic criteria, \keyword{sc\_accuracy\_eev},
\keyword{sc\_accuracy\_etot}, \keyword{sc\_accuracy\_rho}, are all
fulfilled. To avoid too many iterations with force computations before
the forces are converged, it is important to set the other criteria
(especially \keyword{sc\_accuracy\_eev}, which checks a
non-variational quantity) to tight convergence, as indicated
above. For \keyword{sc\_accuracy\_forces} itself, e.g.,
\option{value}=10$^{-4}$ eV/{\AA} is a reliable and robust criterion
to avoid noise in geometry relaxations.

For simple structure relaxation, \texttt{not\_checked} will often be
good enough if the electronic criteria (especially
\keyword{sc\_accuracy\_rho}) are set reasonably tightly.

For molecular
dynamics, however, even slightly underconverged forces may lead to
long-term energy drifts in (nominally) constant-energy runs. Here, it
is a good idea to try out in explicit tests how tightly
\keyword{sc\_accuracy\_eev} must be set to guarantee drift-free
trajectories.

\keydefinition{sc\_accuracy\_rho}{control.in}
{
  \noindent
  Usage: \keyword{sc\_accuracy\_rho} \option{value} \\[1.0ex]
  Purpose: Convergence criterion for the self-consistency cycle, based
    on the charge density. \\[1.0ex]
  \option{value} is a small positive real number [in electrons],
  against which the volume-integrated root-mean square change of the
  charge density between the present and previous s.c.f. iteration is
  checked. \\[1.0ex]
  Default: Between 1d-6 and 1d-3 eV/AA, depending on number of atoms (see below). \\
}
By default, FHI-aims checks separately the convergence of the charge
density and the spin density, using the same criterion. Specifically,
the \emph{unmixed} output density of the Kohn-Sham Equations,
$n_\text{KS}^{(\mu)}$, is checked against the input density
$n^{(\mu-1)}$ to the same equations. A typical tight convergence
criterion is \option{value}=10$^{-5}$.

\keyword{sc\_accuracy\_rho}
is the most important s.c.f. convergence criterion and must be
set. If the keyword is not set in \texttt{control.in}, the code chooses
its default as follows:
\begin{itemize}
  \item Up to 6 atoms in \texttt{geometry.in}: \option{value}=10$^{-6}$
  \item Between 6 and 60 atoms in \texttt{geometry.in}: \option{value}=10$^{-6}\cdot$\texttt{n\_atoms}/6
  \item Above 6000 atoms in \texttt{geometry.in}: \option{value}=10$^{-3}$
\end{itemize}

\keydefinition{sc\_accuracy\_stress}{control.in}
{
  \noindent
  Usage: \keyword{sc\_accuracy\_stress} \option{value} \\[1.0ex]
  Purpose: Convergence criterion for analytical stress. \\[1.0ex]
  \option{value} is EITHER a small positive real number [in
    eV/{\AA$^3$}], against which the maximum difference of the
  analytical stress components between the present and previous
  s.c.f. iteration is checked. A negative number, or simply the string
  '\texttt{not\_checked}' results in no convergence check. \\[1.0ex]
  Default: \texttt{not\_checked} .\\
}
\textbf{Warning. Setting \keyword{sc\_accuracy\_stress} to a finite
  non-negative value will result in a disproportionately large
  computational cost.} In this case, the number of stress evaluations
  per relaxation step is at least doubled. This is normally by far the
  most expensive part of the calculation. Only set this to a finite
  value if you have a good reason to do so.

The default for \option{value} is \texttt{not\_checked}, i.e. the convergence of
the analytical stress will not be checked. \emph{However}, you have to ensure
that other convergence criteria (especially \keyword{sc\_accuracy\_eev},
which checks a non-variational quantity) are set to tight values, as
indicated above.

Calculating the analytical stress is relatively expensive, so this convergence
criterion is only checked \emph{after} the purely electronic / energetic
criteria, \keyword{sc\_accuracy\_eev}, \keyword{sc\_accuracy\_etot},
\keyword{sc\_accuracy\_rho}, are all fulfilled. To avoid too many iterations
with analytical stress computations before the analytical stress is converged,
it is important to set the energetic criteria to tight convergence.

\keydefinition{sc\_accuracy\_potjump}{control.in}
{
  \noindent
  Usage: \keyword{sc\_accuracy\_potjump} \option{value} \\[1.0ex]
  Purpose: Convergence criterion for the self-consistency cycle, based
    on the vacuum level potential shift. \\[1.0ex]
  \option{value} is a small positive real number [in eV], against
  which the difference of the dipole correction potential jump between the present and
  previous s.c.f. iteration is checked. \\
}

This keyword only makes sense (and is only accepted) for periodic slab calculations
with the option \keyword{use\_dipole\_correction} set. If you are interested in the work function or
vacuum level shifts explicitly, it is recommended to use this flag. A typical tight convergence
criterion is \option{value}=10$^{-4}$.

\keydefinition{sc\_init\_factor}{control.in}
{
  \noindent
  Usage: \keyword{sc\_init\_factor} \option{number} \\[1.0ex]
  Purpose: The \keyword{sc\_init\_iter} keyword will not trigger
    if the density convergence criteria are already within a factor
    \keyword{sc\_init\_factor} of the density convergence criterion,
    \keyword{sc\_accuracy\_rho}. \\[1.0ex]
  \option{number} is an real (double precision) number. Default: 1.d0 \\
}
See the \keyword{sc\_init\_iter} keyword for a more complete description of this
behavior.

\keydefinition{sc\_init\_iter}{control.in}
{
  \noindent
  Usage: \keyword{sc\_init\_iter} \option{number} \\[1.0ex]
  Purpose: If the s.c.f. cycles for the initial geometry of a run
    fails to converge within (number) iterations, then FHI-aims
    will end this s.c.f. cycle and begin a new one with the exact
    last wave function as its starting guess. \\[1.0ex]
  \option{number} is an integer number. Default: 1001 \\
}
\keyword{sc\_init\_iter} ends only the s.c.f. cycle for the first
geometry of a run. The idea is to do a step that looks exactly like
a new geometry step, but to not alter the geometry. Rather, reinitializing
from the exact orbitals reached after (number) iterations ensures that
the mixer and other parts of the calculation will not drag along some
misguided information of the original superposition-of-free-atoms
initialization. In some cases, such a clean start can help converge
the s.c.f. cycle of a calculation that otherwise has difficulties to
converge at all.

See also the \keyword{sc\_init\_factor} keyword, which can be used to tune
this behavior further.

\keydefinition{sc\_iter\_limit}{control.in}
{
  \noindent
  Usage: \keyword{sc\_iter\_limit} \option{number} \\[1.0ex]
  Purpose: Maximum number of s.c.f. cycles before a calculation is
    considered and abandoned. \\[1.0ex]
  \option{number} is an integer number. Default: 1000 \\
}
\keyword{sc\_iter\_limit} is a keyword that should be set in every
run. Note: \textbf{You must ensure for every run that the
self-consistency cycle was actually converged.} If this is not the
case, a loud warning is issued in the standard output of FHI-aims at
the end of the s.c.f. cycle, and relaxations, molecular dynamics,
and postprocessing may continue anyway depending on the
\keyword{postprocess\_anyway} setting.

If the end of the s.c.f. cycle is reached in this way, forces are
computed regardless of whether the electronic convergence was reached.

\keydefinition{spin\_mix\_param}{control.in}
{
  \noindent
  Usage: \keyword{spin\_mix\_param} \option{value} \\[1.0ex]
  Purpose: Separate parameter to mix the spin density between
    different s.c.f. iterations. \\[1.0ex]
  \option{value} is a real number between 0. and 1. Default: same as
  \keyword{charge\_mix\_param}. \\
}
\keyword{spin\_mix\_param} may be different from
\keyword{charge\_mix\_param}, but there is not usually a clear recipe
\emph{how} it should be different. This option is thus only needed if
the standard algorithms provably fail.

\keydefinition{switch\_external\_pert}{control.in}
{
  \noindent
  Usage: \keyword{switch\_external\_pert} \option{number} \option{type} \\[1.0ex]
  Purpose: May be used as a combined parameter to switch on
    an artificial perturbing \keyword{homogeneous\_field} only for a given
    number of iterations. \\[1.0ex]
  \option{number} is the integer number of s.c.f. iterations before
    the external perturbation is switched off. \\
  \option{type} is a string. If set to \texttt{safe}, specific
    settings for the \keyword{occupation\_type} and for the
    \keyword{homogeneous\_field} are enforced (see below). \\
}
This is an experimental keyword that allows to switch on an initial
\keyword{homogeneous\_field}, e.g., to lock in the symmetry of a free
atom in order to enforce smoother s.c.f. convergence. The field is
switched off after \option{number} iterations, before self-consistency
is reached.

If \option{type} is not \texttt{safe}, the actual value of
\keyword{homogeneous\_field} and \keyword{occupation\_type} should be
set explicitly in \texttt{geometry.in} and \texttt{control.in},
respectively. The default homogeneous field is 10$^{-3}$ eV/{\AA}.

If \option{type} is set to safe, a homogeneous field of 10$^{-3}$
eV/{\AA} and a Gaussian occupation with very small (10$^{-5}$ eV)
broadening are automatically enforced.

This flag is mainly used to artificially break the symmetry of
spin-polarized free atoms with open $d$ and $f$ shells, which are
sometimes very hard to converge otherwise (see special cases listed
for \keyword{mixer}). Remember that FHI-aims does not allow to enforce
a given symmetry automatically.


\keydefinition{use\_density\_matrix\_hf}{control.in}
{
  \noindent
  Usage: \keyword{use\_density\_matrix\_hf} \\[1.0ex]
  Purpose: Technical keyword that states that the density matrix is
    mixed prior to constructing the exchange matrix in hybrid
    functionals, Hartree-Fock, \emph{et al.} \\[1.0ex]
  Default: When possible, \keyword{use\_density\_matrix\_hf} is
    assumed.
}


\keydefinition{apply\_boys}{control.in}
{
  \noindent
  Usage: \keyword{apply\_boys} \option{KSmin$_{\alpha}$}
  \option{KSmax$_{\alpha}$} \option{KSmin$_{\beta}$} \option{KSmax$_{\beta}$} \option{integer} \\[1.0ex]
  Purpose: In the cluster case is used to switch on a subspace localization
  using the Boys localization algorithm. \\[1.0ex]
  \option{KSmin$_{\alpha}$} and \option{KSmax$_{\alpha}$} are the integer
  KS-state indexes which define the lower and upper boundary of the subspace which should
  be included in the transformation. In calculations without spin, the $\beta$
  parameters are omitted.\\
  \option{integer} is an integer and can either be 0, 1, or 2. If set to
  0, the localization is performed at the beginning of the SCF cycle. If set to
  1, it is performed throughout the cycle and if set to 2, it is performed at
  the end (before writing the restart information).\\
}
This is an experimental keyword that allows to perform a Boys localization on
one or more subspaces of the KS eigenvector. Boys localization is only
applicable in non-periodic calculations. Since each localization requires
calculation of the transition dipole matrix, this adds a considerable overhead
in computation time if it is performed in each SCF cycle. The currently
recommended setting is either 0 or 2.


\keydefinition{xc\_pre}{control.in}
{
 \noindent
 Usage: \keyword{xc\_pre} \option{xc-type} \option{steps} \\[1.0ex]
 Purpose: Specifies the exchange-correlation approach used during the initial steps of a self-consistent DFT / Hartree-Fock. \\[1.0ex]
 \option{xc-type} is a keyword (string) which specifies the chosen
    exchange-correlation functional. \\
 \option{steps} is an integer that determines the number of SCF steps performed with this functional before switching to the functional defined by \keyword{xc}.
}
