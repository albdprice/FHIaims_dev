\chapter[Debug Manager]{Debug Manager - a centralized debugging facility for developers}
\label{appendix_debugmanager}

FHI-aims features a centralized utility for controlling debug output. This utility keeps track of all registered
modules (in a physical sense, e.g. the DFPT features are a module in this context) and can enable their debug output
selectively via the \emph{control.in} file.

The main idea behind this utility is to provide a simple tool to generate debug-flags that can be set in the
\emph{control.in} without having to register them manually in the input parsing routines. Furthermore, it also
provides a central debug interface to the user, instead of having to use different flags, flags that can
only be toggled in the sourcecode or even no toggles at all.

You can enable debugging for any registered module in the input file by adding a \texttt{debug\_module my\_module}.
The list of currently supported modules can be found in the file \emph{init\_debug.f90}.

To register your module, you only need to add a single line to the \emph{init\_debug.f90}:
\begin{verbatim}
    register_debugmodule("your_tag")
\end{verbatim}
Afterwards, you can conveniently activate debugging in the \emph{control.in} by using the flag:
\begin{verbatim}
    debug_module your_tag
\end{verbatim}
This tells the code to enable debugging for your code. Just call
\begin{verbatim}
    debugprint(message, your_tag)
\end{verbatim}
in your code to only print the message if debugging for your module is enabled. For more complicated debug
functionality, there also is
\begin{verbatim}
    module_is_debugged(your_tag)
\end{verbatim}
This function returns a logical value and thus can be used as condition in an enclosing if-block. Both functions are
located inside the \texttt{debugmanager} module, which you can import like any other module with a use-statement.