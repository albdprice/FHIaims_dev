\section{Integration, grids, and partitioning}

The next single most important set of specifications required for
FHI-aims are the settings regarding the numerical grids used in many
contexts. Details regarding the shape and physical motivation behind
these grids are given in Refs. \cite{Blum08,Havu08}, and we do not
repeat them here. 

Notice that the actual required grids may depend on the context of the
calculation, for example whether Hartree-Fock, hybrid functionals, and
or $GW$ calculations are required. In these cases, some specific
settings may require tightening, and some defaults may automatically
be chosen differently depending on whether or not those techniques are
used. 

Specifically, the present section deals with the following topics:
\begin{itemize}
  \item the 1D logarithmic grid infrastructure required for atomic /
    free-atom like calculation
  \item radial and angular grids for all three-dimensional integrals
  \item shaping the partition functions used to split the full
    three-dimensional integrals into effective atom-per-atom pieces
  \item Splitting the grids into different batches for localization /
    parallelization efficiency
\end{itemize}

While many of the settings below take safe defaults for standard
FHI-aims calculations and need not be modified, it is particularly
important to verify the accuracy and efficiency of 
all three-dimensional integration grids (\subkeyword{species}{radial\_base},
\subkeyword{species}{angular\_grids}, and associated tags), since
these determine the performance of the code. In the
\texttt{species\_defaults} files, (very) safe settings for DFT-LDA/GGA
are provided, but for many tasks, may be reduced at very little
accuracy loss. 

\newpage

\subsection*{Tags for general section of \texttt{control.in}:}

\keydefinition{batch\_size\_limit}{control.in}
{
  \noindent
  Usage: \keyword{batch\_size\_limit} \option{value} \\[1.0ex]
  Purpose: Hard upper bound to the number of points in an integration
  batch. \\[1.0ex]
  Restriction: Applies to the \texttt{maxmin} and \texttt{octree}
  \keyword{grid\_partitioning\_method}. \\[1.0ex] 
  \option{value} is an integer number. Default: 200. \\
}
See \keyword{grid\_partitioning\_method} and Ref. \cite{Havu08} for
details regarding integration batches. 

\keydefinition{force\_lebedev}{control.in}
{
  \noindent
  Usage: \keyword{force\_lebedev} \option{type} \\[1.0ex]
  Purpose: Allows to switch between Delley's \cite{Delley95} angular grids
    (17 digits) and the original angular grids tabulated by Lebedev
    and Laikov \cite{Lebedev75,Lebedev76,Lebedev99} (12 digits) \\[1.0ex] 
  \option{type} is a keyword (string), either \texttt{original} or
    \texttt{Delley}. Default: \texttt{Delley}. \\
}
This option need not be changed (or invoked) in any  normal runs,
since there is no quantitative difference between integrals with
Delley's and Lebedev's tabulated grids to our knowledge.

Lebedev's grids may be explicitly invoked when denser angular grids
than 1202 points (already very dense!) per radial integration shell
around each species are required. In detail, grids with the following
numbers of grid points are provided:
\begin{itemize}
  \item \texttt{Delley} : 6, 14, 26, 50, 110, 194, 302, 434, 590, 770, 
        974, 1202
  \item \texttt{Lebedev} : 6, 14, 26, 38, 50, 86, 110, 146, 170, 194,
        302, 350, 434, 590, 770, 974, 1202, 1454, 1730, 2030, 2354, 2702,
        3074, 3470, 3890, 4334, 4802, 5294, 5810
\end{itemize}
These numbers of grid points can be invoked in the
subtags of the \subkeyword{species}{angular\_grids} \texttt{specified}
description for fixed angular grids (the default in the preconstructed
\texttt{species\_defaults} files), and in further tags such as
\subkeyword{species}{angular} or \subkeyword{species}{angular\_min}. 


\keydefinition{grid\_partitioning\_method}{control.in}
{
  \noindent
  Usage: \keyword{grid\_partitioning\_method} \option{method}
    \\[1.0ex]
  Purpose: Allows to switch between different methods to partition the
    full (3D) integration grids into batches for individual
    operations. \\[1.0ex]
  \option{method} is a string, charactrizing one of the different
    methods outlined below. Default: serial--\texttt{maxmin},
    parallel--\texttt{parallel\_hash+maxmin} \\ 
}
Partitioning the integration grid properly can be performance-critical
for the expensive grid-based Hamiltonian integration and charge
density update steps. Details on these methods are given in
Ref. \cite{Havu08}. In particular, we support:
\begin{itemize}
  \item \texttt{maxmin} :
    The default for serial computations: the ``grid-adapted cut-plane'' method
    described in Ref. \cite{Havu08}
  \item \texttt{parallel\_hash+maxmin} :
    The default for all parallel runs. We first hash the grid points to tasks
    by the geometric location and then run a maxmin algorithm locally in each
    task. 
  \item \texttt{parallel\_maxmin} :
    Memory-parallel implementation of the \texttt{maxmin} method. However, the
    exact implementations requires rather much communication, and has been
    superseded by \texttt{parallel\_hash+maxmin}. 
  \item \texttt{octree} :
    The ``octree'' method described in Ref. \cite{Havu08}
  \item \texttt{parallel\_octree}
    Parallel version of the ``octree'' method described in
    Ref. \cite{Havu08}. Only useful for research purposes---superseded by
    \texttt{parallel\_hash+maxmin} for all practical applications. 
  \item \texttt{octant}
    Simple partitioning of the grid into ``octants'' of each radial
    integration shell arround each atom.
\end{itemize}
Note that additional parameters may be invoked to specify details for
these methods, most importantly \keyword{batch\_size\_limit} and
\keyword{points\_in\_batch}. 

We mention for completeness that FHI-aims supports further,
experimental grid batching methods, including the possibility to link
to external libraries. The associated \option{method} strings are
\texttt{tetgen+metis}, \texttt{qhull+metis}, \texttt{nearest+metis},
and \texttt{group}. As discussed in Ref. \cite{Havu08}, the
conceptually simpler \texttt{maxmin} method performs as well or even
better than these ``bottom-up'' type approaches, and should be
preferred. 

\keydefinition{min\_batch\_size}{control.in}
{
  \noindent
  Usage: \keyword{min\_batch\_size} \option{value} \\[1.0ex]
  Purpose: Sets the minimum number of points allowed in an integration
  batch. \\[1.0ex]
  Restriction: Affects the \texttt{octree}
  \keyword{grid\_partitioning\_method} method only. \\[1.0ex] 
  \option{value} is an integer number. Default: 1. \\
}
No need to tweak for standard production calculations. See
\keyword{grid\_partitioning\_method} for details regarding integration
batches. 

\keydefinition{partition\_acc}{control.in}
{
  \noindent
  Usage: \keyword{partition\_acc} \option{threshold} \\[1.0ex]
  Purpose: If the partition function norm for 3D integrals and the
  Hartree potential is below \option{threshold}, that integration
  point is ignored. \\[1.0ex]
  \option{threshold} is a small positive real number. Default:
    10$^{-15}$. \\
}
See Ref. \cite{Blum08} for details regarding ``partitioning of unity''
of the charge density in integrations and the Hartree potential. 
The partition functions $p_\text{at}(\boldr)$ are only
calculated if their denominator (the norm; e.g. $\sum_\text{at}
\rho_\text{at}/|\boldr-\boldR_\text{at}|^2$ is greater than \option{value},
else that integration point is ignored.

Notice that this type of partitioning is strictly
rigorous for integrands that extend no further than the free-atom like
densities used to define our partition functions. This is always true
for DFT-LDA/GGA, with NAO's but if you suspect (e.g., with very diffuse
Gaussian basis functions) some kind of integration noise, reducing
\option{threshold} may be a good first test. 

\keydefinition{partition\_type}{control.in}
{
  \noindent
  Usage: \keyword{partition\_type}   \option{type} \\[1.0ex]
  Purpose: Specifies which kind of partition table is used for all
    three-dimensional integrations. \\[1.0ex] 
  \option{type} : A string that specifies which kind of partition
    table is used. Default: \texttt{stratmann\_sparse} \\ 
}
Usually, this tag need not be modified from the default. See the
Computer Physics Communications description of FHI-aims for a 
description of the numerical integration technique used in FHI-aims. 

In brief, each extended three-dimensional integrand is broken down
into atom-centered pieces, using a set of localized, atom-centered
partition functions:
\begin{equation}
  \label{Eq:part}
  p_\text{at}(\boldr) =
  \frac{g_\text{at}(\boldr)}{\sum_{\text{at}^\prime}
  g_{\text{at}^\prime}(\boldr)} \, .
\end{equation}
where $g_\text{at}(\boldr)$ is an atom-centered weight function. The
following options for \option{type} are available:
\begin{itemize}
  \item \texttt{rho\_r2} : $g_\text{at}(\boldr) =
    n_\text{at}^\text{free}(r)/r^2$ \\ (first suggested
    by Delley \cite{Delley90}).
  \item \texttt{rho\_r} : $g_\text{at}(\boldr) = n_\text{at}^\text{free}(r)/r$
  \item \texttt{rho} : $g_\text{at}(\boldr) = n_\text{at}^\text{free}(r)$
  \item \texttt{fermi} : \emph{Deprecated---do not use.} A
    Fermi-function like approach, requires two additional parameters.
  \item \texttt{stratmann}: The shape suggested by Stratmann \emph{et
    al.}, Ref. \cite{Stratmann96}. This saves $\approx$10-20~\% of the
    numerical effort compared to \texttt{rho\_r2}. More importantly, however,
    our recent testing shows that \texttt{stratmann} is also significantly
    accurate in some corner cases where the effects of integration accuracy
    even make a difference. \emph{Note the properly bounded ``stratmann\_smoother'' default
    function below. Straight ``stratmann'' should not be used.}
  \item \texttt{stratmann\_smooth}: Partial update to guarantee a smooth edge 
    at the ``outer radius'' or atoms. \emph 
  \item \texttt{stratmann\_smoother}: Corrected version of the \texttt{stratmann} partition table.
    The following explanation refers to the prescription given in Eqs. (8), (11), 
    and (14) of Ref. \cite{Stratmann96}. The actual (normalized) partition function is given by Eq. (10).
    At each grid point, it depends on a product of cell functions Eq. (9) over potentially all atoms
    in the system---unless its cell function is equal to one, a faraway atom may contribute to the
    partition function at a given grid point. Through the definition of $\mu_{ik}$ in Eq. (4) of 
    Ref. \cite{Stratmann96} and the limitation of the cell function to unity for $\mu_{ik} < a = 0.64$
    through Eqs. (11) / (14), the distance from which an atom can contribute is restricted, but
    potentially to a very large radius indeed. This becomes a problem for periodic systems (in our case,
    a theoretical radius of 25 {\AA} would have resulted even for light settings and the farthest grid
    points from each atom, set to 5~{\AA} for light settings). \\
    To avoid an overly large volume of contributing atoms, we restrict the list of contributing atoms
    to only those whose free-atom charge density would not be zero at the integration point in question.
    To that end, Eq. (8) of Ref. \cite{Stratmann96} is additionally multiplied with a function that 
    smoothly interpolates between the original $s$ of Stratmann and coworkers and unity. The interpolation
    is done only for atom distances between 0.8 and 1.0 times their free-atom radius, and uses a 
    $[1-\cos(2x)]$-like interpolating function. The bottom line is that we get the benefits of both
    the Stratmann table and a restricted atom list without any discontinuities or wiggles as a 
    function of atomic positions or unit cell vectors --- which is as it should be.
  \item \texttt{stratmann\_sparse}: This version of the \texttt{Stratmann} partition table is the same as
    \texttt{stratmann\_smoother}, but it stores the relevant interatomic distances in a memory saving form.
\end{itemize}
Note that the free atom electron density $n_\text{at}^\text{free}(r)$
still determines the extent of many partition function types. This is
controlled by the \subkeyword{species}{cut\_free\_atom} keyword. See
also the \keyword{hartree\_partition\_type}
keyword, which presently must have the same setting as the
\keyword{partition\_type} keyword.

\keydefinition{points\_in\_batch}{control.in}
{
  \noindent
  Usage: \keyword{points\_in\_batch} \option{value} \\[1.0ex]
  Purpose: Target number of grid points per integration batch. \\[1.0ex]
  Restriction: Applies to the \texttt{maxmin} and \texttt{octree}
    \keyword{grid\_partitioning\_method}. \\[1.0ex] 
  \option{value} is an integer number. Default: 100 for most calculations.  When GPU acceleration is used for tasks involving the batch integration scheme, this value is raised to 200. \\
}
See \keyword{grid\_partitioning\_method} and Ref. \cite{Havu08} for
details regarding integration batches. 


\newpage

\subsection*{Subtags for \emph{species} tag in \texttt{control.in}:}

\subkeydefinition{species}{angular}{control.in}
{
  \noindent
  Usage: \subkeyword{species}{angular} \option{limit} \\[1.0ex]
  Purpose: For \emph{self-adapting} angular integration grids, the
    maximum allowed number of points per radial shell. \\[1.0ex]
  Restriction: This flag has no effect for species where
    \subkeyword{species}{angular\_grids} is explicitly specified (the default in
    our \texttt{species\_default} files). \\[1.0ex]
  \option{limit} is the maximum allowed number of integration points
    per radial shell. \\
}
This option is only meaningful for self-adapting angular grids, which
are \emph{not} the recommended default for production calculations
with FHI-aims -- (i) because these grids are often rather dense, and
(ii) because they are meaningful only for cluster-type geometries. In 
order to specify self-adapting angular grids anyway, you must also set
the keywords \subkeyword{species}{angular\_min} and
\subkeyword{species}{angular\_acc}.  

The available values of integration points in given angular grids are
listed with the keyword \keyword{force\_lebedev}. 

\subkeydefinition{species}{angular\_acc}{control.in}
{
  \noindent 
  Usage: \subkeyword{species}{angular\_acc} \option{threshold}
    \\[1.0ex]
  Purpose: For \emph{self-adapting} angular integration grids,
    specifies the desired integration accuracy for the initial 
    Hamiltonian and overlap matrix elements. \\[1.0ex]
  Restriction: Use only for cluster-type geometries. \\[1.0ex]
  \option{threshold} is a small positive real number; if 0., no
    adaptation is performed. Default: 0. \\
}
If \option{threshold} is not zero, this option invokes the
self-adaptation of all angular integration grids, within the limits
given by \subkeyword{species}{angular\_min} and
\subkeyword{species}{angular}. The adaption criteria are the initial 
Hamiltonian / overlap matrix integrals. 

In all
preconstructed \texttt{species\_default} files, we specify reliable
angular integration grids for all elements for DFT. No adaptation
is required. For the curious, our own grids are adapted for symmetric
dimers at a tight bond distance, using \option{threshold} = $10^{-8}$.

\subkeydefinition{species}{angular\_grids}{control.in}
{
  \noindent
  Usage: \subkeyword{species}{angular\_grids} \option{method}
    \\[1.0ex]
  Purpose: Indicates how the angular integration grids (in each radial
    integration shell) for this \keyword{species} are
    determined. \\[1.0ex]
  \option{method} is a string, either \texttt{auto} or
    \texttt{specified}. \\
}
The standard \texttt{species\_default} files provided with FHI-aims
provide \texttt{specified} angular grids (on the safe side, i.e.,
rather dense) for each \subkeyword{species}{radial\_base} integration shell
around an atom. The line: \\[1.0ex]
\subkeyword{species}{angular\_grids} \texttt{specified} \\[1.0ex]
must be immediately followed by a series of lines with \\[1.0ex]
    \subkeyword{species}{division} [...] \\
    \subkeyword{species}{outer\_grid} [...] \\[1.0ex]
tag(s). These contain the actual grid specification. 

If method \texttt{auto} is given, appropriate specifications for
self-adapting grids should be included in \texttt{control.in} (keywords 
\subkeyword{species}{angular}, \subkeyword{species}{angular\_min}, 
\subkeyword{species}{angular\_acc}).

\subkeydefinition{species}{angular\_min}{control.in}
{
  \noindent
  Usage: \subkeyword{species}{angular\_min} \option{value} \\[1.0ex]
  Purpose: specifies the minimum number of angular grid points per
    radial integration shell \\[1.0ex]
  \option{value} is the minimum number of grid points per shell. \\
}
For \texttt{specified} \subkeyword{species}{angular\_grids}, acts as a
lower bound for the number of points per radial shell (specified grids
will be increased accordingly). 

For self-adapting angular grids, use together with the
\subkeyword{species}{angular} and \subkeyword{species}{angular\_acc}
keywords. 

In practice, \option{value} will be reduced to the next-highest
available Lebedev integration grid (see
\keyword{force\_lebedev} tag for possible values). 

\subkeydefinition{species}{cut\_free\_atom}{control.in}
{
  \noindent
  Usage: \subkeyword{species}{cut\_free\_atom} \option{type} [\option{radius}]
    \\[1.0ex]
  Purpose: Adds a cutoff potential to the initial, non-spinpolarized
    free-atom calculation that yields free-atom densities and
    potentials for many basic tasks.\\[1.0ex]
  \option{type} : A string, either \texttt{finite} or
    \texttt{infinite}. Default: \texttt{finite} for DFT-LDA/GGA and
      for \keyword{RI\_method} \texttt{LVL};
    \texttt{infinite} for Hartree-Fock, hybrid functionals, $GW$, etc. 
      if \keyword{RI\_method} \texttt{V} is used.\\
  \option{radius} : A real number, in {\AA}: Onset radius for the
    cutoff potential, as defined in the \subkeyword{species}{cut\_pot}
    tag. Default: For DFT-LDA/GGA, $r_\text{onset}$ as given by the
    \texttt{onset} parameter in \subkeyword{species}{cut\_pot} .\\
}
Although this is a technical parameter (ideally, no influence on
self-consistent, converged results), it has important implications for
a variety of numerical tasks in the code:
\begin{itemize}
  \item It influences (slightly) the basis-defining potential for the
    minimal basis, and for \subkeyword{species}{confined} basis
    functions.
  \item It limits the radius of the free-atom density, which in turn
    limits the extent of the default integration partition table. For
    DFT-LDA/GGA, this extent need must not be smaller than the radius
    of the most extended basis function, but it also need not be
    larger, since all integrands are zero outside anyway. This is
    \emph{not} the case for the two-electron Coulomb operator, which
    is needed for Hartree-Fock, hybrid functionals, $GW$, etc, in
    which case the default is currently \texttt{infinite} (no cutoff
    potential applied).
  \item It also limits the extent of the partition table used for the
    Hartree potential.Especially in periodic calculations, it is vital
    that the real-space part of the Hartree potential is kept
    small. In that case, it is thus critical to keep \option{radius}
    as small as possible.
\end{itemize}
Usually, the default specified in the code should be accurate for all
requirements. If, however, you suspect some kind of integration noise
which is not related to the grid, increasing the
\subkeyword{species}{cut\_free\_atom} value may be a good test. 

\subkeydefinition{species}{division}{control.in}
{
  \noindent
  Usage: \subkeyword{species}{division} \option{radius}
    \option{points} \\[1.0ex]
  Purpose: For \texttt{specified}
    \subkeyword{species}{angular\_grids}, the number of angular points 
    on all radial shells that are within \option{radius}, but not
    within another, \emph{smaller} division. \\[1.0ex]
  Restrictions: Meaningful only in a block immediately following an
    \subkeyword{species}{angular\_grids} \texttt{specified}
    line. \\[1.0ex]
  \option{radius} : Outer radius (in {\AA}) of this division. \\
  \option{points} : Integer number of angular points requested in this
    division (see \keyword{force\_lebedev} tag for
    possible values). \\
}
Use the \subkeyword{species}{outer\_grid} tag to specify the number of
angular grid points used outside the outermost division \option{radius}.

\subkeydefinition{species}{innermost\_max}{control.in}
{
  \noindent 
  Usage: \subkeyword{species}{innermost\_max} \option{number} \\[1.0ex]
  Purpose: Monitors the quality of the radial integration
    grid. \\[1.0ex]
  \option{number} is an integer number, corresponding to a radial grid 
  shell. Default: 4. \\
}
If, after on-site orthonormalization, a radial function's innermost
extremum is inside the radial grid shell \option{number}, counting from the
nucleus, that radial function is rejected in order to prevent
inaccurate integrations.

\subkeydefinition{species}{logarithmic}{control.in}
{
  \noindent
  Usage: \subkeyword{species}{logarithmic} \option{r\_min}
    \option{r\_max} \option{increment} \\ [1.0ex]
  Purpose: Defines the dense one-dimensional ``logarithmic'' grid for
    the direct solution of all radial equations (free atom quantities,
    Hartree potential). \\[1.0ex]
  \option{r\_min} is a real number (in bohr); the innermost point 
    of the logarithmic grid is defined as $r(1)$=\option{r\_min}/$Z$,
    where $Z$ is the atomic number of the
    \subkeyword{species}{nucleus} of the \keyword{species}. Default:
    0.0001 bohr. \\ 
  \option{r\_max} is a real number (in bohr), the outermost point of
    the logarithmic grid, $r(N)$. Default: 100 bohr. \\
  \option{increment} is a real number, the increment factor $\alpha$
    between successive grid points, $r(i) = \alpha \cdot
    r(i-1)$. Default: 1.0123. \\
}
The number of logarithmic grid shells, $N$, is uniquely determined by
\option{r\_min}, \option{r\_max}, and \option{increment}. Specifying a
dense logarithmic grid is not performance-critical.

\subkeydefinition{species}{outer\_grid}{control.in}
{
  \noindent
  Usage: \subkeyword{species}{outer\_grid} 
    \option{points} \\[1.0ex]
  Purpose: For \texttt{specified}
    \subkeyword{species}{angular\_grids}, the number of angular points 
    on all radial shells outside the largest
    \subkeyword{species}{division}. \\[1.0ex] 
    Restrictions: Meaningful only in a block immediately following an
    \subkeyword{species}{angular\_grids} \texttt{specified}
    line. \\[1.0ex]
  \option{points} : Integer number of angular points (see
    \keyword{force\_lebedev} tag for possible values). \\ 
}
Use the \subkeyword{species}{division} tag to specify the number of 
angular grid points used for radial shells within specified radii.

\subkeydefinition{species}{radial\_base}{control.in}
{
  \noindent
  Usage: \subkeyword{species}{radial\_base} \option{number}
    \option{radius} \\[1.0ex]
  Purpose: Defines the basic grid of radial integration shells
    according to Ref. \cite{Baker94} \\[1.0ex]
  \option{number} is an integer number (the total number of grid points,
    $N$). \\
  \option{radius} is a positive real number which specifies the outermost
    shell of the basic grid, $r_\text{outer}$, in~\AA. \\
}
The location of the \option{number} radial shells is given by
\begin{equation}
  r(i) = r_\text{outer} \cdot \frac{\log\{1-[i/(N+1)]^2\}}{\log\{1-[N/(N+1)]^2\}}
\end{equation}
With this prescription, shell $i$=0 would be located exactly at
$r(i)=0$, and shell $i$=$N+1$ would be located exactly at
$r(i)=\infty$, i.e., this provides an exact mapping of the interval
[0,$\infty$]. 

The FHI-aims \texttt{species\_default} files provide values for
\option{number} according to the formula
$N$=1.2$\times$14(\subkeyword{species}{nucleus}+2)$^{1/3}$, as
determined empirically in Ref. \cite{Baker94}. These ``basic'' grids are
can then be augmented by adding uniform subdivisions, using the
\subkeyword{species}{radial\_multiplier} keyword described below. 


\subkeydefinition{species}{radial\_multiplier}{control.in}
{
  Usage: \subkeyword{species}{radial\_multiplier} \option{number} \\[1.0ex]
  Purpose: Systematically increases the radial integration grid
  density. \\[1.0ex]
  \option{value} is an integer, number specifying the number of
    added subdivisions per basic grid spacing. Default: \option{value}
    = 2 \\
}
The basic grid of $N$ radial shells (see
\subkeyword{species}{radial\_base} definition) is evenly subdivided
\option{number}-1 times, to yield \option{number}$\cdot (N+1) - 1$
shells in the \emph{actually used} integration grid. Thus, the
\subkeyword{species}{radial\_multiplier} tag allows to systematically
increase the number of radial shells (by factors). For example,
\option{number}=2 (the default) would yield $2N+1$ shells total.

Note that some all-electron Gaussian basis sets contain either very
high or very low exponents. If such basis sets are used for test
purposes, it may be necessary to test the convergence of the radial
integration grid directly by increasing the
\subkeyword{species}{radial\_multiplier}. 

The effect of the\subkeyword{species}{radial\_multiplier} is explained
in Ref. \cite{Zhang2013} 
(open access at \url{http://iopscience.iop.org/1367-2630/15/12/123033/article})
Look at Figure A.1 and the accompanying explanation in that reference.
