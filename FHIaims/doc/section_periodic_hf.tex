\section{Hartree-Fock and hybrid functionals, including periodic systems}
\label{Sec:periodic_hf}

Periodic versions of the Hartree-Fock method and of and hybrid density
functionals are implemented in FHI-aims. Generally, our experiences are
very good. The implementation is stable and seems to scale well
towards large systems. However, we still ask you to exercise some
care. If you encounter any unexpected difficulties, 
consult the developers.

\emph{Periodic versions of MP2, RPA and $GW$ are not yet part of the 
main FHI-aims 
distribution, as they are in various stages of development. Please feel
free to ask at aimsclub regarding these methods for periodic systems.
They are very important to us, and we will be happy to share them as
they become ready and more usable.}

Complete descriptions of the material described below, as well as
extensive benchmarks, are summarized in Ref. \cite{Levchenko2015}, as
well as Ref. \cite{Ihrig2015} (for the non-periodic implementation of
the ``LVL'' approach). Forces and the stress tensor are also
implemented, with a clear description of the stress tensor given in
Ref. \cite{Knuth2015}. 

There are two specific issues from a usability point of view which
should be considered:
\begin{itemize}
\item The \keyword{RI\_method} ``LVL'' described below is
    implemented in a linear scaling and is therefore the only reasonable
    pathway for large and/or periodic systems. It is, however, slightly less
    accurate (for formal reasons) than the non-linear-scaling version for non-periodic
    systems, RI-V. Please bear this in mind. For standard solids and
    hybrid functionals, the effects seem very small. See
    reference \cite{Ihrig2015} for quantitative tests. In general,
    RI-LVL has been extremely reliable for us.
  \item For band structure output, the
    \keyword{exx\_band\_structure\_version} keyword  allows to toggle
    between a faster real-space version that works only when a relatively dense
    $k$-space grid has been used during the regular s.c.f. cycle and
    a slow(!) fallback version that is calculated in reciprocal space.
    The underlying reason is that the real-space Born-von Karman cell of
    the regular s.c.f. cycle may become too small to accommodate some
    $k$-vectors that are not exact reciprocal lattice vectors of the
    Born-von Karman cell. The slow fallback version should not simply be
    used by default since it can easily become the computational bottleneck --
    both regarding time and memory.
\end{itemize}

In periodic Hartree-Fock (and hybrid functional) implementations, the key
quantity that needs to be evaluated is the exact-exchange matrix, 
%
 \begin{equation}
   K_{ij}(\bfk) = \sum_{kl,\bfq} D_{kl}(\bfq) \iint d\bfr d\bfrp 
  \frac{\phi_{i\bfk}^\ast(\bfr)\phi_{k\bfq}(\bfr)\phi_{l\bfq}^\ast(\bfrp)\phi_{j\bfk}(\bfrp)} 
       {|\bfr - \bfrp|} \,
  \label{eq:exact_exchange_kspace}
 \end{equation}
%
where $\bfk,\bfq$ are the Bloch vectors,  $\phi_{i\bfk}(\bfr)$ is the Bloch
summation of the $i$-th atomic orbital $\phi_i(\bfr -\bfR)$ living in the unit cell
$\bfR$, and $D_{kl}(\bfq)$ is the density matrix. $K_{ij}(\bfk)$ can be obtained
from its Fourier transform,
 \begin{equation}
  K_{ij}(\bfk) = \sum_{\bfR} e^{i\bfk \bfR} X_{ij}(\bfR)\, ,
 \end{equation}
where 
%
 \begin{equation}
  X_{ij}(\bfR) =\sum_{kl} \sum_{\bfRp} D_{kl}(\bfRp) \sum_{\bfRpp} \iint d\bfr d\bfrp
   \frac{\phi_i(\bfr)\phi_k(\bfr+\bfRpp)\phi_j(\bfrp+\bfR)\phi_l(\bfrp+\bfRp+\bfRpp)}
        {|\bfr - \bfrp|}
  \label{eq:exact_exchange_realspace}
 \end{equation}
%

In FHI-aims, periodic Hartree-Fock and hybrid density functionals are implemented in two
different ways. One implementation is based on the ``k-space'' formulation, where one 
computes $K_{ij}(\bfk)$ directly from Eq.~(\ref{eq:exact_exchange_kspace}). An 
alternative, and more efficient implementation is based on the``real-space'' formulation, 
where one first computes $X_{ij}(\bfR)$ from Eq.~(\ref{eq:exact_exchange_realspace}), 
and then Fourier transform it to $K_{ij}(\bfk)$.  The ``real-space'' implementation is used 
in the code by default, and the ``k-space'' implementation is only used for crossing-check
purposes.

Both implementations are based on a localized resolution-of-identity approximation,
which we termed as ``RI-LVL'', in analogy to ``RI-SVS'' and ``RI-V'' introduced in
Sec.~\ref{Sec:auxil}. Under ``RI-LVL'', the products of two normal basis functions 
$(i,j)$ centering at atoms $A_i$ and $A_j$ are expanded only in terms of auxiliary 
functions centering on these two atoms. Possible contributions of auxiliary functions 
from a third center are excluded in this approximation, in contrast to ``RI-V''. 
Specifically, one has
  \begin{equation}
    \phi_i(\bfr - A_i) \phi_j(\bfr - A_j) =
    \sum_\mu C_{i(A_i),j(A_j)}^{\mu(A_i)} P_\mu(\bfr-A_i) + 
    \sum_\nu C_{i(A_i),j(A_j)}^{\nu(A_j)} P_\nu(\bfr-A_j) \, ,
  \end{equation}
where $\mu$ and $\nu$ enumerate the auxiliary basis functions centering 
on atom $A_i$ and $A_j$ respectively. This approximation has been extensively benchmarked
with respect to the more accurate ``RI-V'' approximation for finite systems, and with
respect to other independent implementations for molecular systems. The achieved accuracy 
is remarkable and should be sufficiently good for production calculations.

Periodic Hartree-Fock and hybrid-functional calculations can be run
in the same manner as the periodic LDA and GGA cases, by setting the keyword 
\keyword{xc} to \option{hf} or desired hybrid functionals, and setting the 
\keyword{k\_grid} mesh to appropriate values. As mentioned above, by default
the ``real-space'' periodic Hartree-Fock implementation will be invoked. There are two
thresholding parameters (detailed below) which control the balance between the 
computational load and accuracy in the calculation.  One may also switch to the 
``k-space'' implementation of periodic Hartree-Fock and hybrid functionals for testing or 
comparison purposes by setting the keyword \texttt{use\_hf\_kspace} to be true 
(see below).  The thresholding parameters do not apply to the ``k-space'' implementation,
however.

\newpage

\subsection*{Tags for general section of \texttt{control.in}:}

\keydefinition{coulomb\_threshold}{control.in}
{ \noindent
  Usage: \texttt{periodic\_hf} \keyword{coulomb\_threshold} \option{value} \\[1.0ex]
  Purpose: This sets a threshold \option{value} for a key ingredient in the construction
  of the exact-exchange matrix -- the Coulomb matrix. 
  The Coulomb matrix elements below the specified threshold
  \option{value} are discarded in the calculation. Suggested values
  are between $10^{-6}$ and 0. The default value is $10^{-10}$.
}

\keydefinition{exx\_band\_structure\_version}{control.in}
{ \noindent
  Usage: \texttt{exx\_band\_structure\_version} \option{value}
  \\[1.0ex]
  Purpose: A periodic band structure calculation can be performed
    either using a real-space version (\texttt{value}=1) or a
    reciprocal-space version (\texttt{value}=2).  No default -- user 
    must decide.  \\[1.0ex]
  \option{value} is an integer, either 1 or 2. \keyword{exx\_band\_structure\_version}
    \texttt{1} is \underline{preferred} (but see below).\\
}

The distinction between real-space and reciprocal-space pertains to the method used to 
calculate the Fock matrix; in both cases, the coordinate system used when specifying the 
k-path via the \keyword{output} \subkeyword{output}{band} keyword is expressed in terms 
of reciprocal coordinates.

If \keyword{output} \subkeyword{output}{band} is requested for a periodic Hartree-Fock
or hybrid functional calculation, adhere to the following rules:
\begin{itemize}
  \item Do not use excessively many $k$ points in each band segment,
    for instance no more than 11. We also note that 21 is a reasonable value 
    to sample the fine features of a band structure.
  \item \keyword{exx\_band\_structure\_version}
    \texttt{1} is \underline{preferred}. The real-space band structure
    version \texttt{value}=1 has low overhead and is
    accurate IF a reasonably dense \keyword{k\_grid} is used
    during the preceding
    s.c.f. calculation. \keyword{exx\_band\_structure\_version} 
    \texttt{1} is therefore the \underline{recommended} approach. For
    very sparse s.c.f. \keyword{k\_grid} settings, it can, however,
    fail. In that case, the failure is so obvious that one cannot miss
    it. For better results, please avoid particularly
    \keyword{k\_grid} dimensions of 1 (one) in the s.c.f. part of the
    calculation. We apologize for this inconvenience. On the bright
    side, if you use \keyword{exx\_band\_structure\_version}
    \texttt{1} correctly, it will give reliable results without much
    overhead compared to the underlying s.c.f. calculation.
    We recommend to test to be sure.
  \item \keyword{exx\_band\_structure\_version} 2 is a
    \underline{fallback} method that will always work but comes with
    significant time and memory overhead. 
    If the plotted band structure from the real-space version
    \keyword{exx\_band\_structure\_version} \texttt{value}=1 has
    obvious numerical problems, please switch to 
    a denser \keyword{k\_grid} during s.c.f. Only if this
    approach is not successful or possible, consider
    \keyword{exx\_band\_structure\_version} 2. The latter will always
    work, as the critical part of the work is handled in reciprocal
    space. As a consequence, though, sparsity in real space can no
    longer be exploited, and the band structure calculation becomes
    much slower than the real-space version.
  \item In case of doubt, the band structure ONLY at $k$-points used
    during the s.c.f. cycle itself can also be printed along certain
    directions by using the \keyword{output} 
    \subkeyword{output}{band\_during\_scf} keyword, which ensures that
    only the information that went into the s.c.f. cycle is actually
    used. This is mainly useful for debugging purposes.
\end{itemize}
We apologize that this decision process is a bit rough around
the edges and leaves an essential decision up to the user (because we
want you to know). However, consider this: The above
procedure provides a simple way to make band structure output work,
and works safely. It is now generally not a problem to produce band 
structures with hybrid functionals in FHI-aims and this functionality
has produced much successful science. 

\keydefinition{screening\_threshold}{control.in}
{ \noindent
  Usage: \texttt{periodic\_hf} \keyword{screening\_threshold} \option{value} \\[1.0ex]
  Purpose: This sets a screening parameter in a periodic Hartree-Fock (or hybrid 
  functional) calculation. The real-space exact-exchange matrix elements below the specified
  threshold \option{value} are neglected in the calculation. Suggested values are between
  $10^{-6}$ and 0. Smaller values mean better accuracy but heavier computational loads.
  The default value is $10^{-7}$.
}

\keydefinition{use\_hf\_kspace}{control.in}
{ \noindent
  Usage: \keyword{use\_hf\_kspace} \option{flag} \\[1.0ex]
  Purpose: The ``k-space'' periodic HF implementation can be invoked
  by setting \option{flag} to be \texttt{.true.} This is, however,
  very expensive.
}


\keydefinition{split\_atoms}{control.in}
{ \noindent
  Usage: \keyword{split\_atoms} \option{flag} \\[1.0ex]
  Purpose: The ``split\_atoms'' periodic HF implementation can be switched off
  by setting \option{flag} to be \texttt{.false.}
}


The keyword \texttt{split\_atoms} helps to reduce the peak memory and increases
performance for systems with heavy elements or systems containing atoms of
different size (that is the number of basis functions). Internally, the number
of basis functions per atoms are split into smaller batches. The batch size is
 calculated according to \texttt{basis\_functions\_per\_atom/split\_batch}, where the
default value of \texttt{split\_batch} is 14. Its value can be changed in
\texttt{control.in} with \texttt{split\_batch value}, where \texttt{value} is a positive
integer number.

The implementation seemed to improve the performance of systems containing heavy
 elements. If you encounter performance problems, especially systems with many
atoms, it might be worth to switch-off the \texttt{split\_atoms} feature and test the
 old routine. However, we could not find systems where the current implementation
 significantly harms the performance. In the case you do, please report the issue.
