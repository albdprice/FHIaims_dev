\section{Bethe-Salpeter equation: BSE}\label{Sec:BSE}
The serial BSE runs both with and without Tamm-Dancoff Approximation (TDA) and print out both results by default. The parallel BSE runs only with TDA at the moment.

The BSE eigenvalue equation in matrix form is:
        $$\begin{bmatrix}
          A       & B \\
          -B       & -A \\
        \end{bmatrix}
        \begin{bmatrix}
          X_1 \\
          X_2 \\
        \end{bmatrix} = 
        \Lambda
        \begin{bmatrix}
          X_1 \\
          X_2 \\
        \end{bmatrix}$$
where $\Lambda$: Excitation energies; $X_1, X_2$: Excitation eigenvectors;
matrix element A and B are calculated by quasiparticle energies, Coulomb and screened Coulomb integrals; $$A_{ia}^{jb} = -\alpha^{S/T}<ia|V|jb>+<ij|W(\omega = 0)|ab> + (E_a^{QP} - E_i^{QP})\delta_{ij}\delta_{ab}$$
$$B_{ia}^{jb}=-\alpha^{S/T}<ia|V|bj>+<ib|W(\omega = 0)|aj>$$
where $i, j$: occupied states; $a, b$: unoccupied states; $\alpha^{S/T}$ = 2 for singlet states; 0 for triplet states. The TDA considers only block A, which is symmetric and easy to solve.
\subsection*{Tags for general section of \texttt{control.in}}

\keydefinition{neutral\_excitation}{control.in} {
\noindent
Usage: \keyword{neutral\_excitation} \option{type}\\
Purpose: Triggers the calculation of neutral excitations.\\
\option{type}: String that defines the type of calculation to be performed.
\begin{itemize}
\item bse: Full BSE calculation without TDA for serial run; TDA BSE for parallel run.
\end{itemize}
}
Also the keyword \keyword{empty\_states} should be set to a large number (e.g., 1000), or the keyword \keyword {calculate\_all\_eigenstates} should be used, to make sure the code generates all
possible empty states provided from the basis set. This number will also be reduced automatically by the code to the maximum number that can be generated from the basis set.

\keydefinition{read\_write\_qpe}{control.in} {
\noindent
Usage: \keyword{read\_write\_qpe} \option{type}\\
Purpose: Specify write quasiparticle energies (qpe) in GW calculation or read qpe in BSE calculation \\
\option{type}: String that specify read or write qpe, or both.
\begin{itemize}
\item w: write qpe to a file, used together with \keyword{qpe\_calc}
\item r: read qpe from a file, if this value('r') is set, a file "energy\_qp" should be provided by the user. The energy in the file "energy\_qp" should be in hartree unit.
\item wr or rw: write qpe in GW and read qpe in BSE
\end{itemize}
}

\keydefinition{bse\_s\_t}{control.in} {
\noindent
Usage: \keyword{bse\_s\_t} \option{type}\\
Purpose: Specify whether singlet or triplet excitation energies should be calculated in BSE calculation. \\
\option{type}: String that specify wheter singlet or triplet, can not do both at the moment.
\begin{itemize}
\item singlet: Singlet states calculated in BSE.
\item triplet: Triplet states calculated in BSE.
\end{itemize}
}
