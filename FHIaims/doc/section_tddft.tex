\section{TDDFT - linear response}\label{Sec:TDDFT}

\emph{These routines are not completed yet. For now it is only possible do use $f_{xc}$ kernels from LDA.
The development goes on and more funcionality will be added.}
When publishing results obtained from this routine, please do cite me, Jan Kloppenburg as the author, as well as of course the
usual people in the aims references. When problems, questions or suggestions arise, feel free to contact me at {\tt kloppenburg@fhi-berlin.mpg.de}.
Only use there routines if you know what you are doing!

\subsection*{Theory}
The goal is to calculate excitation energies $\omega_I$ and corresponding oscillator strengths $f_I$ from
\begin{align}
{\mathbf\Omega}\mathbf{F}_I=\omega^2_I\mathbf{F}_I\;.
\end{align}
Linear response theory (see \cite{casida96}) is the basis for this calculation. We construct
\begin{align}
{\mathbf\Omega}_{ias,jbt} = \delta_{i,j}\delta_{a,b}\delta_{s,t}(\epsilon_a-\epsilon_i)^2 + 2\sqrt{\epsilon_{as}-\epsilon_{is}}\mathbf{K}_{ias,jbt}\sqrt{\epsilon_{bt}-\epsilon_{jt}}
\end{align}
with the coupling kernel
\begin{align}\label{casidakernel}
\mathbf{K}_{ias,jbt} = \int\int\varphi^*_{i}(\mathbf{r})\varphi_{a}(\mathbf{r})
\left[\frac{1}{\left|\mathbf{r}-\mathbf{r}'\right|}+f_{xc}(\mathbf{r},\mathbf{r}')\right]
\varphi^*_{j}(\mathbf{r}')\varphi_{b}(\mathbf{r}')d\mathbf{r}d\mathbf{r}'\;.
\end{align}
In this notation I refer with the indices $i,j$ to occupied and with $a,b$ to virtual orbitals, while $s$ and $t$ denote the spin.
The input energies $\epsilon$ are obtained from either ground state {\em Hartree-Fock} or DFT calculations. Going by this rule we contruct the matrix
$\mathbf{\Omega}$ which then is solved for eigenvalues and eigenvectors. The excitation energies $\omega_I$ then follow from
From the eigenvectors $\mathbf{F}_I$ the oscillator strengths $f_I$ can be obtained from
\begin{align}
f_I=\frac{2}{3}\omega_I\Big[\big|\left<\Psi_0\big|\hat{\bf{X}}\big|\Psi_I\right>\big|^2+
\big|\left<\Psi_0\big|\hat{\bf{Y}}\big|\Psi_I\right>\big|^2+\big|\left<\Psi_0\big|\hat{\bf{Z}}\big|\Psi_I\right>\big|^2\Big]\;,
\end{align}
with the $\hat{\bf{X}}$ being the spatial operator for the X direction and the others respectively with $\Psi_0$ being the all electron ground state wave function
and $\Psi_I$ being the all electron excited state wave function for the state $I$ with excitation energy $\omega_I$.

For the TDHF calculation mode, the kernel $\mathbf{K}_{ias,jbt}$ is modified to become
\begin{align}\label{tdhfcalcmode}
\mathbf{K}_{ias,jbt} = (ias|jbt) + \delta_{s,t}(ij|ab)
\end{align}
that has only the bare {\em Coulomb} part $(ia|jb)$ and the exact exchange part $(ij|ab)$ from {\em Hartree-Fock} theory. This {\em Hartree-Fock} Kernel 
creates a non-Hermitian matrix $\mathbf{\Omega}$. Please not that this calculation mode
is as yet only available for single processor runs due to the lack of a non-Hermitian parallel eigenvalue solver.

\subsection*{Available Kernels and \textit{libxc}}
As of now, there is only the \texttt{pw-lda} $f_{xc}$ kernel available for the TDDFT calculation in aims. If the user wants to make use of
additional $f_{xc}$ kernels he is requested to install \textit{libxc}. \textit{libxc} is a library of exchange-correlation functionals for 
density-functional theory (see \cite{libxcpaper12}) available free under the LGPL license v3 from the internet.
\footnote{\url{http://www.tddft.org/programs/octopus/wiki/index.php/Libxc}} 
Additionally, for the full TDDFT calculation it is possible to choose functionals at will that are available from this {\em libxc}. It is not required to
have the DFT level calculation that generates the input energies $\epsilon$ for \ref{casidakernel} using the same XC functional as the TDDFT calculation.
You should really know what you are doing when you choose to experiment with different functionals and always keep in mind that the results might be
unpredictable and not necessarily have any physical meaning at all. Nevertheless it might come in handy to be able to mix different hybrid functionals
or do a DFT ground state calculation with a functional that does not provide and $f_{xc}$ and still be able to do a TDDFT calculation on top of that
when switching to functionals that do provide an $f_{xc}$.

\subsection*{Tags for general section of \texttt{control.in}}

\keydefinition{neutral\_excitation}{control.in} {
\noindent
Usage: \keyword{neutral\_excitation} \option{type}\\
Purpose: Triggers the calculation of neutral excitations.\\
\option{type}: String that defines the type of calculation to be performed.
\begin{itemize}
\item tddft: Full TDDFT calculation
\item tdhf: Full TDHF calculation (Kernel from \ref{tdhfcalcmode}, serial CPU only)
\item rpa: random phase approximation only (set $f_{xc}=0$ in \ref{casidakernel})\\
\end{itemize}
}
With the keyword \keyword{neutral\_excitation} the user can specify the calculation mode for the linear response theory.\\
Also the keyword \keyword{empty\_states} should be set to 1000, or the keyword \keyword {calculate\_all\_eigenstates} should be used, to make sure the code generates all
possible empty states provided from the basis set. This number will also be reduced automatically by the code to the maximum number that can be generated from the basis set.

\keydefinition{tddft\_kernel}{control.in} {
\noindent
Usage: \keyword{tddft\_kernel} \option{string}\\
Purpose: Specify the origin of the TDDFT kernel for \ref{casidakernel}\\
\option{string}: pw-lda/pz-lda or libxc

Both pw-lda or pz-lda are built-in options in FHI-aims. They are equivalent to the keywords defined in \keyword{xc}.\\
When using libxc, one must specify the desired kernels through keywords \keyword{tddft\_x} and \keyword{tddft\_c}. Note that libxc is only possible when the user has compiled aims with libxc binding.
}

\keydefinition{tddft\_x}{control.in} {
\noindent
Usage: \keyword{tddft\_x} \option{string}\\
Purpose: Set the desired exchange kernel to use from libxc. The definition is from libxc's manual and can be found at the 
libxc\footnote{\url{http://www.tddft.org/programs/octopus/wiki/index.php/Libxc:manual}} website.\\
\option{string}: The name of the selected exchange functional, i.e. XC\_LDA\_X
}

\keydefinition{tddft\_c}{control.in} {
\noindent
Usage: \keyword{tddft\_c} \option{string}\\
Purpose: Set the desired correlation kernel to use from libxc. The definition is from libxc's manual and can be found at the 
libxc website as well.\\
\option{string}: The name of the selected correlation functional, i.e. XC\_LDA\_C\_PW
}

\keydefinition{excited\_mode}{control.in} {
\noindent
Usage: \keyword{excited\_mode} \option{string}\\
Purpose: Select which excitations will be calculated.\\
\option{string}: one of \{singlet|triplet|both\}. To calculate both singlets and triplets is set as the default when this keyword is omitted.
}

\keydefinition{excited\_states}{control.in} {
\noindent
Usage: \keyword{excited\_states} \option{n}\\
Purpose: Specify the number of excited state energies and oscillator strengths to be printed.\\
\option{n}: Integer number $\mathtt{n}\in\mathbb{N},\;\mathtt{n}\ge 0$
}

With the keyword \keyword{excited\_states} the user can specify the number of excited states that
will be printed in the output. The default for this is 50 if there are that many. Normal production runs will generally have many more (depending on the
basis set and the number of electrons involved) that can easily reach beyond 10000. So to avoid a really huge output from this routine this defaut is
set rather low. Feel free to choose any number of your liking, if it should be too large it will automatically be defaulting to all excited states. At the end
of a calculation the file TDDFT\_LR\_Spectrum\_(singlet/triplet).dat will be written into the directory the FHI-aims program was run in.
It contains all computed excitation energies and the corresponding oscillator strengths.

\keydefinition{casida\_reduce\_matrix}{control.in} {
\noindent
Usage: \keyword{casida\_reduce\_matrix} \option{boolean}\\
Purpose: Set to .true. if you want to reduce the energy range for the Kohn-Sham eigenvalues to be included in the computation.\\
\option{boolean}: .true. or .false.\\
}

\keydefinition{casida\_reduce\_occ}{control.in} {
\noindent
Usage: \keyword{casida\_reduce\_occ} \option{x}\\
Purpose: Specify the energy in Hartree below which the occupied states are cut off.\\
\option{x}: Cutoff energy in Hartree
}

\keydefinition{casida\_reduce\_unocc}{control.in} {
\noindent
Usage: \keyword{casida\_reduce\_unocc} \option{x}\\
Purpose: Specify the energy in Hartree above which the virtual states are cut off.\\
\option{x}: Cutoff energy in Hartree
}
