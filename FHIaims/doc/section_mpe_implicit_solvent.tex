\section{Continuum Solvation Methods}
\label{Sec:ContSolvMethods}

Continuum or implicit solvation methods provide a fast way the influence of solvents and electrolytes on chemical reactions. Currently, FHI-aims supports two models which have different strengths and capabilities which are summarized in Table \ref{tab:ImpSolvMethods}. Both models place a dielectric continuum outside the charge distribution modeling the polarizibility of the solvent. Differences arise in the solvation cavity definition. The Multipole Expansion (MPE) implicit solvation method separates the FHI-aims grid into two domains and couples them via electrostatic boundary conditions. It therefore in fact solves two coupled Poisson equations with different dielectric permittivities. The Finite ion-size and Stern layer modified Poisson-Boltzmann (SMPB) method solves a single Poisson equation on the full FHI-aims integration grid by defining a smooth dielectric permittivity function.  In general, the accuracy of the evaluation of solvation energies is expected to be similar. In fact, both methods  merge into each other if the dielectric transition in the SMPB model is turned into a sharp step function. On top of this ion-free implicit solvation model, the SMPB approach also supports the modeling of finite ionic strengths in the solution.

The MPE solvation model is the faster one of both approaches with only a small overhead with respect to vacuum calculations. The overhead of both implicit solvation methods is reduced, when performing expensive hybrid calculations, since the actual time for the implicit solvation calculations does not vary with the functional.

\begin{table}[htb]
\centering
 \begin{tabular}{ l | c | c }
   & \textbf{MPE} & \textbf{SMPB}\\
  \hline		
  Solvent Parametrizations  & H$_2$O (N,C) & H$_2$O (N,C), CH$_3$OH (N,C)\\
    & & (more in work)\\
  Dissolved ions/salt & no & yes (SMPB/LPB)\\
  Salt Parametrizations & -- & Aq. Monoval. Salt Solutions\\
  CPU speed  & fast & moderate \\
  Forces & no & yes\\
  PBCs & no & no (in work)\\
  Developers	 & \href{markus.sinstein@mytum.de}{\underline{Markus Sinstein}} & \href{mailto:sringe@stanford.edu}{\underline{Stefan Ringe}}\\ 
  & & \small{Christoph Muschielok, Marvin H. Lechner} \\
  \hline  
\end{tabular}
 \caption{Comparison of the two implicit solvation methods in FHI-aims. Parameter sets for the MPE method are available in ref. \cite{Sinstein2017_MPE}, for the SMPB method in ref. \cite{Andreussi2012} and \cite{Dupont2013-bc} (parameters for methanol as more solvents in current work). N and C indicate parameter sets fitted for neutral and charged solutes, respectively.}
 \label{tab:ImpSolvMethods}
\end{table}

In the following, both models are summarized and the key input parameters presented.

\subsection{MPE Implicit Solvent Model}
\label{Sec:MultiPoleExpansion}

\emph{This is an experimental feature which is still under development. 
Do not rely on properties calculated by this method! 
Please contact the authors for further details.}

\emph{This functionality is not yet available for periodic systems.}

\emph{The current implementation does not have analytical forces yet.}

\emph{Generally, when combining MPE with other functionality, you should know what you are doing. No specific interactions with other methods beyond single point DFT are implemented, so only methods which do not interfere with MPE are safe to use.}

The simulation of a solvent in a quantum mechanical calculation can, in principle, be done in two ways. One way is to include explicit solvent molecules in the calculation. This straightforward approach usually requires molecular dynamics (MD) simulations in order to yield thermodynamically meaningful observables as e.g.~solvation free energies.

The second way is to average the effect of the solvent and treat it as a continuum which responds to the electrostatic potential created by the solute, i.e.~the entity that is to be solvated. There are several flavors to this comparatively inexpensive approximation, e.g.~
the polarizable continuum model (PCM) \cite{Mennucci02}, 
the conductor like screening model (COSMO) \cite{Klamt93}, 
the self-consistent continuum solvation (SCCS) model \cite{Andreussi2012}, 
the ``SMx'' models \cite{Cramer2008_SM8,Marenich2009_SMD,Marenich2013_SM12}, 
or CMIRSv1.1 \cite{You2016_CMIRS11} 
to name some of the more popular ones. 
Statistical sampling then only needs to be performed for the degrees of freedom of the solute which obviously makes it computationally much cheaper. 

In general, the necessary integration of the solvent's degrees of freedom beforehand leads to a problem where one now needs to solve a generalized Poisson's equation, 
\begin{align}
\Del \left( \varepsilon_0\varepsilon(\Vect{r}) \Del \Phi(\Vect{r}) \right) = - 4\pi \varrho(\Vect{r}) \label{eq:mpe_gen_poisson}, 
\end{align}
to obtain the electrostatic potential $\Phi(\Vect{r})$ created by the total charge density $\varrho(\Vect{r})$ which now accounts for the electrostatic polarization potential of the solvent (often called ``reaction field''). 
Eq.~\ref{eq:mpe_gen_poisson} contains a spatially dependent dielectric permittivity function $\varepsilon(\Vect{r})$ in contrast to the regular Poisson's equation,
\begin{align}
\Del \left( \varepsilon_0 \Del \Phi_\mathrm{H}(\Vect{r}) \right) = - 4\pi \varrho(\Vect{r}) \label{eq:mpe_poisson}, 
\end{align}
which is solved in a regular DFT calculation in every step of the SCF cycle to get the Hartree potential $\Phi_\mathrm{H}$. 

As outlined in more detail in Ref.~\cite{Sinstein2017_MPE}, the multipole expansion (MPE) implicit solvent model offers an efficient way of solving Eq.~\ref{eq:mpe_gen_poisson} based on the knowledge of the Hartree potential $\Phi_\mathrm{H}$ readily available from a splined representation in FHI-aims (cf.~Sec.~\ref{Sec:Hartree}) via least-squares fitting instead of integration. 
The dielectric function $\varepsilon(\Vect{r})$ here needs to be a step-function in 3D-space where the following boundary conditions apply at the step ($\Vect{n}$ denotes the normal direction to the interface): 
\begin{subequations}
\begin{align}
\Phi_+ = \Phi_- \label{eq:mpe_bc1_gen}
\end{align}
\begin{align}
\Vect{n} \cdot \varepsilon_+ \Del \Phi_+ = \Vect{n} \cdot \varepsilon_- \Del \Phi_- \label{eq:mpe_bc2_gen}
\end{align}
\end{subequations}

Then, the above equations are discretized in two ways: 
\begin{itemize}
\item The potentials $\Phi_+$ and $\Phi_-$ are expressed in a truncated multipole series with expansion orders $l_\mathrm{max,R}$ and $l_\mathrm{max,O}$, and
\item equations~\ref{eq:mpe_bc1_gen} and \ref{eq:mpe_bc2_gen} are evaluated at $N$ points on the interface manifold. 
\end{itemize}
Thereby, $N$ is chosen such that the resulting system of linear equations (SLE) is overdetermined (typically by a factor of two to three). 


%\newpage


\subsubsection*{Tags for general subsection of \texttt{control.in}:}

The keywords controlling the MPE module are divided into four categories:
\begin{itemize}
\item[elementary] These are the most important keywords---some are even mandatory---which likely need to be specified for every calculation. 
\item[convergence] Here, the most important convergence parameters are collected which should be checked before doing (large scale) production runs. 
\item[expert] These settings should only be modified by an experienced user as they allow quite profound modifications. 
\item[debug] Debug settings are intended to give valuable insight for developers into intermediate results. 
\end{itemize}

The authors strongly encourage new users to try out ``elementary'' and ``convergence'' settings first in order to gather some experience with the MPE implementation before any modifications of other settings are made.

\paragraph{elementary}

\keydefinition{solvent}{control.in}
{
  \noindent
  Usage: \keyword{solvent} \option{method} \\[1.0ex] 
  Purpose: Specifies the desired implicit solvent model. \\[1.0ex]
  \option{method} is a string which specifies the implicit solvent method; 
    currently, \option{mpe} (the method presented above) and 
    \option{mpb} (cf.~Sec.~\ref{Sec:ModifiedPoissonBoltzmann}) 
    are supported. \\
}


\keydefinition{mpe\_solvent\_permittivity}{control.in}
{
  \noindent
  Usage: \keyword{mpe\_solvent\_permittivity} \option{epsilon} \\[1.0ex] 
  Purpose: Specifies the dielectric constant of the bulk solvent. \\[1.0ex]
  \option{epsilon} is a positive real number equal to the macroscopic 
    dielectric constant of the solvent. Default: \option{1.0} \\
}


\keydefinition{isc\_cavity\_type}{control.in}
{
  \noindent
  Usage: \keyword{isc\_cavity\_type} \option{type} \\[1.0ex] 
  Purpose: This keyword controls the model used to sample 
    the implicit solvent cavity for the MPE method. 
    Depending on \option{type}, further flags (or even lines) might be 
    necessary. Those are explained below. \\[1.0ex]
  Options: Currently supported options are 
    \option{overlapping\_spheres}, 
    \option{rho\_free}, 
    \option{rho\_multipole\_static}, 
    and \option{rho\_multipole\_dynamic}. \\
}

\subkeydefinition{isc\_cavity\_type}{rho\_free}{control.in}
{
  \noindent
  Usage: \keyword{isc\_cavity\_type} 
    \subkeyword{isc\_cavity\_type}{rho\_free} 
    \option{rho\_iso} \\[1.0ex] 
  Purpose: Constructs the cavity as an iso-density surface of the 
    superposed electron density of the neutral, free atoms in
    the solute. \\[1.0ex]
  \option{rho\_iso} is a positive real number specifying the 
    desired iso-density value in units of 
    \si{\elementarycharge\per\cubic\angstrom} .
}

\subkeydefinition{isc\_cavity\_type}{rho\_multipole\_static}{control.in}
{
  \noindent
  Usage: \keyword{isc\_cavity\_type} 
    \subkeyword{isc\_cavity\_type}{rho\_multipole\_static} 
    \option{rho\_iso} \\[1.0ex] 
  Purpose: Constructs the cavity as an iso-density surface of the 
    \emph{initial} (multipole-expanded) electron density of the solute. \\[1.0ex]
  \option{rho\_iso} is a positive real number specifying the 
    desired iso-density value in units of 
    \si{\elementarycharge\per\cubic\angstrom} .
}
Note, that \emph{initial} electron density in this context means 
the electron density at the time of the first call to the cavity 
generation routine is used. Thus, the shape of the cavity depends 
on several other parameters of the calculation, such as after how 
many SCF steps the cavity is built (mainly influenced by 
\option{mpe\_skip\_first\_n\_scf\_steps}) or which restart 
information was used---if any. 
In any case, the cavity shape then stays constant throughout 
the whole calculation. 
Although care should be taken due to the various influences on the 
cavity definition, this cavity type can be used as a cheaper 
alternative to the fully self-consistent cavity when the 
initialization is based on the converged density of the vacuum 
calculation, i.e.~restarting from it, as the polarization 
potential often (not always!) has a minor influence on the 
electron density. 

\subkeydefinition{isc\_cavity\_type}{rho\_multipole\_dynamic}{control.in}
{
  \noindent
  Usage: \keyword{isc\_cavity\_type} 
    \subkeyword{isc\_cavity\_type}{rho\_multipole\_dynamic} 
    \option{rho\_iso} \\[1.0ex] 
  Purpose: Constructs the cavity as an iso-density surface of the 
    \emph{self-consistent} (multipole-expanded) electron density 
    of the solute. \\[1.0ex]
  \option{rho\_iso} is a positive real number specifying the 
    desired iso-density value in units of 
    \si{\elementarycharge\per\cubic\angstrom} .
}
With this method, the cavity is updated to the current electron 
density in every SCF step. This also means that the MPE equations 
have to be solved in every SCF step making it computationally 
more expensive. 

\subkeydefinition{isc\_cavity\_type}{overlapping\_spheres}{control.in}
{
  \noindent
  Usage: \keyword{isc\_cavity\_type} 
    \subkeyword{isc\_cavity\_type}{overlapping\_spheres} 
    \option{type} \option{value} \\[1.0ex] 
  Purpose: Constructs the cavity as a superposition of overlapping 
    spheres around all atoms. \\[1.0ex]
  \option{type} specifies how the radii of the atomic spheres 
    are determined.
    \begin{itemize}
    \item \option{radius}: all spheres have the same radius
      given by \option{value} in units of \si{\angstrom};
    \item \option{rho}: the atomic spheres are iso-density surfaces 
      based on the electron density of the isolated, neutral atom 
      with an iso-value of \option{value} in units of 
      \si{\elementarycharge\per\cubic\angstrom} .
    \end{itemize} %\\
  \option{value} is a real number whose meaning and units
    depend on the choice of \option{radius} (see above). \\
}
\emph{WARNING}: The usage of this cavity type is strongly discouraged! 
It has been helpful in the development to analyze the cavity sampling 
process itself. The resulting cavities, however, are almost certainly 
not smooth and were never intended to be used in production calculations.
When used with the MPE model, the whole calculation is prone to numerical 
problems and the results are very often unphysical. 
Instead, use the \option{rho\_free} type that builds the cavity based on the
superposition of atomic densities (which is again smooth) or use other 
types based on the (self-consistent) electron density of the solute. 


\keydefinition{mpe\_nonelectrostatic\_model}{control.in}
{
  \noindent
  Usage: \keyword{mpe\_nonelectrostatic\_model} \option{model} \\[1.0ex] 
  Purpose: This keyword controls any additional, ``non-electrostatic'' terms 
    not included in the purely electrostatic treatment of the solvent. 
    Depending on \option{model}, further flags (or even lines) might be 
    necessary. Those are explained below. \\[1.0ex]
  Options: Currently, only \option{linear\_OV} is supported. \\
}

\subkeydefinition{mpe\_nonelectrostatic\_model}{linear\_OV}{control.in}
{
  \noindent
  Usage: \keyword{mpe\_nonelectrostatic\_model} 
    \subkeyword{mpe\_nonelectrostatic\_model}{linear\_OV} 
    \option{$\alpha$} \option{$\beta$} \\[1.0ex] 
  Purpose: Corrects the total energy term by $\alpha O + \beta V$ 
    where $O$ is the surface area of the cavity and
    $V$ its volume. \\[1.0ex]
  \option{$\alpha$} is a real number in units of 
    \si{\electronvolt\per\angstrom\squared}. 
    Default: \option{0.0} \\
  \option{$\beta$} is a real number in units of 
    \si{\electronvolt\per\cubic\angstrom}. 
    Default: \option{0.0} \\
}
This non-electrostatic model is in principle identical to the one 
proposed by Andreussi~\emph{et al.} \cite{Andreussi2012}. 
Note, however, that the surface tension of the solvent is here 
included in the parameter $\alpha$. 


\paragraph{convergence}

\keydefinition{mpe\_lmax\_rf}{control.in}
{
  \noindent
  Usage: \keyword{mpe\_lmax\_rf} \option{lmax} \\[1.0ex] 
  Purpose: Specifies the expansion order of the 
    polarization potential aka reaction field 
    inside the cavity. \\[1.0ex]
  \option{lmax} is a non-negative integer number. 
    Default: \option{8} \\
}
This is a critical convergence parameter of 
the MPE model. You should never forget to test 
convergence with respect to this parameter before 
doing production runs. 
For small organic molecules, the largest and 
successfully tested expansion order so far has been 14. 
Note, however, that numerical problems might arise when 
choosing even larger values for \option{lmax} or 
when going to larger systems. 

\keydefinition{mpe\_lmax\_ep}{control.in}
{
  \noindent
  Usage: \keyword{mpe\_lmax\_ep} \option{lmax} \\[1.0ex] 
  Purpose: Specifies the expansion order of the 
    polarization potential aka reaction field 
    outside of the cavity. \\[1.0ex]
  \option{lmax} is a non-negative integer number. 
    Default: maximum value of 
      \subkeyword{species}{l\_hartree} for all
      species. \\
}
This parameter is similar to but usually less critical 
than \keyword{mpe\_lmax\_rf}. However, careful convergence 
tests with respect to this parameter before 
doing production runs is advisable since this parameter 
dictates the size of the MPE matrix equation. 
Choosing larger values than the default should 
usually have little to no impact on the results. 


\keydefinition{mpe\_degree\_of\_determination}{control.in}
{
  \noindent
  Usage: \keyword{mpe\_degree\_of\_determination} 
    \option{dod} \\[1.0ex] 
  Purpose: Defines the desired ratio of number of 
    rows to columns in left-hand side matrix of the 
    MPE equation. \\[1.0ex]
  \option{dod} is a real number $\ge 1.0$. 
    Default: \option{5.0} \\
}
For the very limited (!) number of applications of the MPE 
method so far, the default value of 5.0 has been a save choice. 
However, you should never forget to test convergence with 
respect to this parameter before doing production runs. 
Note, that the requested degree of determination can only 
very approximately be reached. This can lead to an under-
determination of the MPE equations and a subsequent 
termination of the program when values for \option{dod} 
very close to 1 are chosen.


\keydefinition{mpe\_tol\_adjR2}{control.in}
{
  \noindent
  Usage: \keyword{mpe\_tol\_adjR2}
    \option{tol} \\[1.0ex]
  Purpose: Defines the tolerance for the adjusted coefficient
    of determination $\bar{R}^2$ of the solved MPE equations.
    Will abort if $\bar{R}^2 < 1 - tol$. \\[1.0ex]
  \option{tol} is a real number between $0.0$ and $1.0$.
    Default: \option{0.075} \\
}
The MPE equations are sometimes not solvable in the regular
solid harmonic basis used for the reaction field. This is the case
especially for large molecules. A low $\bar{R}^2$ indicates such
a bad solution. In some cases increasing \keyword{mpe\_lmax\_rf}
helps, but there are pathologic cases where increasing
\keyword{mpe\_lmax\_rf} leads to a perpetual decrease in
$\Delta_\text{solv}^\text{el}G$ without ever converging.

For $\bar{R}^2 < 0.925$ it is likely that less than 90\%
of $\Delta_\text{solv}^\text{el}G$ are captured. This is however
based on experience from a limited number of cases. Feedback to
the developers (\href{mailto:jakob.filser@tum.de}{\underline{jakob.filser@tum.de}}) will be appreciated!

\keydefinition{mpe\_tol\_adjR2\_wait\_scf}{control.in}
{
  \noindent
  Usage: \keyword{mpe\_tol\_adjR2\_wait\_scf}
    \option{bool} \\[1.0ex]
  Purpose: If \option{.true.}, will wait until the SCF cycle is
  converged before it is checked whether $\bar{R}^2 < 1 - tol$. \\[1.0ex]
    Default: \option{.false.} \\
}
Although MPE does not actively try to converge, $\bar{R}^2$
tends to improve during the SCF procedure. Setting
\keyword{mpe\_tol\_adjR2\_wait\_scf} can thus help borderline cases
converge, at the cost of spending the full computation time
of the SCF procedure on a calculation that might ultimately fail.

\paragraph{expert}

\keydefinition{mpe\_factorization\_type}{control.in}
{
  \noindent
  Usage: \keyword{mpe\_factorization\_type} \option{type} \\[1.0ex] 
  Purpose: Defines the numerical method used to factorize the 
    left-hand side of the MPE equations as the first step to 
    the numerical solution. \\[1.0ex]
  \option{type} can be chosen from: \option{qr}, \option{qr+svd}, 
    and \option{svd}. Default: \option{qr+svd} \\
}
\emph{The option} \option{qr} \emph{is temporarily disabled until
$\bar{R}^2$ is implemented for this case!}

The default behavior is to perform a QR factorization with a 
singular value decomposition (SVD) on top. This allows to 
robustly solve the MPE equation via the pseudo-inverse of 
the left-hand side. 

\emph{Be careful!} Using the (non rank-revealing) QR 
factorization alone can fail when the left-hand side is 
rank deficient which can easily happen---especially 
for large expansion orders 
\keyword{mpe\_lmax\_rf} and/or \keyword{mpe\_lmax\_ep}! 
On the other hand, \option{svd} does not necessarily mean that 
no QR factorization is performed as this (at least for 
the parallel implementation) depends on the (Sca)LAPACK 
driver routine used. 


\keydefinition{mpe\_f\_sparsity\_threshold}{control.in}
{
  \noindent
  Usage: \keyword{mpe\_f\_sparsity\_threshold} 
    \option{threshold} \\[1.0ex] 
  Purpose: Can potentially speed up the evaluation of 
    the reaction field on the integration grid by 
    neglecting all its coefficients smaller than
    \option{threshold}. \\[1.0ex]
  \option{threshold} is a non-negative real number. 
    Default: \option{0.0} \\
}
Speed in this case usually comes at the price of 
sacrificing accuracy, i.e. it should always be tested 
if the results are still sufficiently accurate. 
Moreover, a large threshold might cause instabilities 
in the SCF cycle!


\keydefinition{mpe\_skip\_first\_n\_scf\_steps}{control.in}
{
  \noindent
  Usage: \keyword{mpe\_skip\_first\_n\_scf\_steps} \option{n} \\[1.0ex] 
  Purpose: Switch off (or rather, do not switch on) the 
    MPE implicit solvent model in the first 
    \option{n} SCF steps. \\[1.0ex]
  \option{n} is a non-negative integer value. Default: \option{0} \\
}
\emph{Attention:} No check is performed whether the MPE 
has been switched on before the end of the run! For example, 
the implicit solvent will be ignored when the SCF cycle 
happens to converge in \option{n} steps or less. 
Note that the counting of SCF steps can deviate from the 
regular FHI-aims counting as the re-initialization 
from restart is also counted as one SCF step. 

\keydefinition{mpe\_n\_centers\_ep}{control.in}
{
  \noindent
  Usage: \keyword{mpe\_n\_centers\_ep} \option{n} \\[1.0ex] 
  Purpose: Defines the number of centers used 
    for the expansion of the polarization potential 
    outside of the cavity. \\[1.0ex]
  \option{n} is a positive integer number. 
    Default: number of centers for the Hartree 
    potential expansion (cf.~\ref{Sec:Hartree}) \\
}
The first \option{n} centers defined in 
\texttt{geometry.in} are used as expansion centers. 
The default is to use all of them. Only change this 
value if you fully understand what you are doing and 
why you want to do this! 


\keydefinition{mpe\_n\_boundary\_conditions}{control.in}
{
  \noindent
  Usage: \keyword{mpe\_n\_boundary\_conditions} \option{nbc} \\[1.0ex] 
  Purpose: Determines the number of boundary conditions 
    imposed at every point on the cavity interface. \\[1.0ex]
  Valid choices for \option{nbc} are 2 and 4. Default: \option{2} \\
}
As outlined in Ref.~\cite{Sinstein2017_MPE}, there are at least 
two more boundary conditions other than Eqns.~\ref{eq:mpe_bc1_gen} 
and \ref{eq:mpe_bc2_gen} that can be imposed on the electrostatic 
potential / field / flux at a dielectric interface. 
The default is to enforce continuity of the potential and 
continuity of the dielectric flux perpendicular to the interface, 
i.e.~\option{nbc} equals 2. 
Furthermore, continuity of the electric field parallel to the 
interface can be imposed, i.e.~\option{nbc} equals 4. 
However, this should automatically be satisfied by the former 
two boundary conditions and---in the best case---only leads to 
a higher order correction of the fit. 
\emph{Warning:} The non-default has not been tested thoroughly. 
Verify your results carefully when using it! 


\keydefinition{isc\_calculate\_surface\_and\_volume}{control.in}
{
  \noindent
  Usage: \keyword{isc\_calculate\_surface\_and\_volume} \option{bool} \\[1.0ex] 
  Purpose: Determines whether the surface area and volume 
    of the cavity are calculated. \\[1.0ex]
  \option{bool} is of Boolean type. Default: \option{.true.} \\
}
As the only currently implemented \keyword{mpe\_nonelectrostatic\_model} 
\subkeyword{mpe\_nonelectrostatic\_model}{linear\_OV} requires 
the calculated measures, this flag is automatically turned on 
when it has been turned off but is needed. 

\keydefinition{isc\_surface\_curvature\_correction}{control.in}
{
  \noindent
  Usage: \keyword{isc\_surface\_curvature\_correction} \option{bool} \\[1.0ex] 
  Purpose: When this flag is turned on, the calculated surface 
    area (and volume) of the cavity is approximately corrected 
    for the cavity curvature. \\[1.0ex]
  \option{bool} is of Boolean type. Default: \option{.true.} \\
}
The effect of this keyword is usually rather negligible. 
For more details regarding the correction, please consult 
Ref.~\cite{Sinstein2017_MPE}. 


\keydefinition{isc\_rho\_rel\_deviation\_threshold}{control.in}
{
  \noindent
  Usage: \keyword{isc\_rho\_rel\_deviation\_threshold} 
    \option{threshold} \\[1.0ex] 
  Purpose: Defines the convergence criterion of the 
    cavity generation process: The walker dynamics simulation 
    is run until the density values for all walkers deviate 
    from the chosen iso value by at most \option{threshold}. \\[1.0ex]
  \option{threshold} is a small, positive real number. 
    Default: \option{\SI{1e-3}{}} \\
}
This keyword is only applicable for an \keyword{isc\_cavity\_type} 
defined by an iso-density value. 

\keydefinition{isc\_max\_dyn\_steps}{control.in}
{
  \noindent
  Usage: \keyword{isc\_max\_dyn\_steps} \option{num} \\[1.0ex] 
  Purpose: Determines the maximum number of allowed steps 
    to reach convergence of the walker dynamics simulation 
    in the cavity creation process. \\[1.0ex]
  \option{num} is a positive integer number. Default: \option{300} \\
}

\keydefinition{isc\_try\_restore\_convergence}{control.in}
{
  \noindent
  Usage: \keyword{isc\_try\_restore\_convergence} \option{bool} \\[1.0ex] 
  Purpose: When convergence of the cavity creation dynamics run 
    could not be achieved within the number of allowed steps 
    specified by \keyword{isc\_max\_dyn\_steps}, this flag allows 
    to enforce convergence by simply deleting all walkers 
    not satisfying the convergence criterion given by 
    \keyword{isc\_rho\_rel\_deviation\_threshold}. \\[1.0ex]
  \option{bool} is of Boolean type. Default: \option{.false.} \\
}
Although a simple check is done to stop the calculation when 
too many walkers do not satisfy the convergence criterion, one 
should always manually checking the resulting cavity for 
larger holes that might result from the deletion of walkers 
which can lead to a bad estimate of the cavity's surface area 
and volume and maybe also have an impact on the quality of 
the polarization potential. 

\keydefinition{isc\_kill\_ratio}{control.in}
{
  \noindent
  Usage: \keyword{isc\_kill\_ratio} \option{fraction} \\[1.0ex] 
  Purpose: This keyword can be helpful when the walker 
    dynamics run does not converge due to trapped walkers 
    by killing the worst \option{fraction} of walkers 
    at each neighbor list update step (also see 
    \keyword{isc\_update\_nlist\_interval}). \\[1.0ex]
  \option{fraction} is a non-negative real number
    much smaller than 1. Default: \option{0.0} \\
}
As the number of possibly trapped walkers depends a lot 
on the shape of the electron density, it is rather 
difficult to give a recommendation about a sensible 
value for \option{fraction}. In case walkers 
get stuck, we propose to use a rather conservative 
kill ratio of $~\SI{1e-3}{}$ and only increase it 
if necessary. 

\keydefinition{isc\_update\_nlist\_interval}{control.in}
{
  \noindent
  Usage: \keyword{isc\_update\_nlist\_interval} \option{num} \\[1.0ex] 
  Purpose: This keyword triggers a re-evaluation of the 
    neighbor lists in the density walkers dynamics simulation 
    after every \option{num} steps. \\[1.0ex]
  \option{num} is a positive integer number. Default: \option{50} \\
}

\keydefinition{isc\_dynamics\_friction}{control.in}
{
  \noindent
  Usage: \keyword{isc\_dynamics\_friction} \option{fric} \\[1.0ex] 
  Purpose: The value of \option{fric} determines how much 
    ``kinetic energy'' is removed from the walkers in 
    every step of the simulations via a simple  
    velocity scaling. \\[1.0ex]
  \option{fric} is a real number between 0 and 1. 
    Default: \option{0.1} \\
}
A value of 0 for \option{fric} means that no energy is 
removed from the system which may lead to a bad convergence 
behavior. On the other hand, a value of 1 means that all 
kinetic energy is removed at each step which tends to 
slow down the rate of convergence. 

\keydefinition{isc\_dt}{control.in}
{
  \noindent
  Usage: \keyword{isc\_dt} \option{delta} \\[1.0ex] 
  Purpose: Determines the ``time'' step of the 
    walker dynamics simulation. \\[1.0ex]
  \option{delta} is a positive real number. Default: \option{0.1} \\
}
Note: Since this is no actual physical quantity, arbitrary 
time units are used. 

\keydefinition{isc\_rho\_k}{control.in}
{
  \noindent
  Usage: \keyword{isc\_rho\_k} \option{k} \\[1.0ex] 
  Purpose: Determines the force constant \option{k} 
    for the ``density'' force, i.e.~the harmonic force 
    that pulls the walkers along the density gradient 
    to the specified iso-density value. \\[1.0ex]
  \option{k} is a positive real number. Default: \option{1.0} \\
}
Note: Since this is no actual physical quantity, arbitrary 
time units are used. 

\keydefinition{isc\_rep\_k}{control.in}
{
  \noindent
  Usage: \keyword{isc\_rep\_k} \option{k} \\[1.0ex] 
  Purpose: Determines the force constant \option{k} 
    for the repulsive interaction between walkers 
    perpendicular to the density gradient. \\[1.0ex]
  \option{k} is a positive real number. Default: \option{0.01} \\
}
Note: Since this is no actual physical quantity, arbitrary 
time units are used. 

\keydefinition{isc\_g\_k}{control.in}
{
  \noindent
  Usage: \keyword{isc\_g\_k} \option{k} \\[1.0ex] 
  Purpose: Determines the force constant \option{k} 
    for the ``gravitational'' force that drags walkers to the 
    center of gravity of the solute in case the local 
    density gradient is too small. \\[1.0ex]
  \option{k} is a positive real number. Default: \option{2.0} \\
}
Usually this should not happen, but when walkers move too 
far away from the solute, the density gradient becomes 
very small and its direction is unreliable due to numerical 
noise (see \keyword{isc\_gradient\_threshold}).
In this case, the walker is dragged to the center of the 
solute until the density gradient is again large enough. 
Note: Since this is no actual physical quantity, arbitrary 
time units are used. 

\keydefinition{isc\_gradient\_threshold}{control.in}
{
  \noindent
  Usage: \keyword{isc\_gradient\_threshold} 
    \option{thsq} \\[1.0ex] 
  Purpose: When the squared norm of the electron density 
    gradient at the position of a walker is less than 
    \option{thsq}, this gradient is considered unreliable. 
    Instead, a simple ``gravitational'' force towards the 
    center of the solute is applied. \\[1.0ex]
  \option{thsq} is a positive real number. 
    Default: \option{\SI{1e-8}{}} \\
}
The force constant of the ``gravitational'' force is 
determined by \keyword{isc\_g\_k}.


\paragraph{debug}

\keydefinition{mpe\_xml\_logging}{control.in}
{
  \noindent
  Usage: \keyword{mpe\_xml\_logging} \option{filename} 
    \option{level} \\[1.0ex] 
  Purpose: Controls the MPE module's internal XML logging 
    output. \\[1.0ex]
  \option{filename} specifies the name of the log file 
    to be written. Default: \option{mpe\_interface.xml} \\
  \option{level} defines the detail of the output. Supported 
    log levels are: \option{off}, \option{basic}, \option{medium}, and
    \option{detailed}. Default: \option{off} \\
}
This keyword is intended for debugging purposes. 
Note: Depending on the log level, the size of the output 
can become quite large. 


\keydefinition{isc\_cavity\_restart}{control.in}
{
  \noindent
  Usage: \keyword{isc\_cavity\_restart} \option{filename} \\[1.0ex] 
  Purpose: Read the solvation cavity from restart file 
    (if available) and write new cavity to same file. \\[1.0ex]
  \option{filename} is the name of the restart file. \\
}
Specifying this keyword is almost equivalent to specifying 
both \keyword{isc\_cavity\_restart\_read} and 
\keyword{isc\_cavity\_restart\_write} with the same 
\option{filename} option except that with 
\keyword{isc\_cavity\_restart} the program does not abort 
when there is no restart file to read from. \\
Note: This keyword is intended for debugging purposes. 
Do not rely on the current structure of the cavity restart 
file as it might change in the future. 

\keydefinition{isc\_cavity\_restart\_read}{control.in}
{
  \noindent
  Usage: \keyword{isc\_cavity\_restart\_read} \option{filename} \\[1.0ex] 
  Purpose: Read the solvation cavity from restart file 
    instead of constructing a new one. \\[1.0ex]
  \option{filename} is the name of the file (in \texttt{.xyz} format)
  containing the cavity points and normal vectors. Additionally, a file
  \texttt{<filename>.bin} is written which contains the entire cavity
  information in not human-readable form. While the primary purpose of the
  former is visualization, the latter is the actual restart file.\\
}
Note: This keyword is intended for debugging purposes. 
Do not rely on the current structure of the cavity restart 
file as it might change in the future. 

\keydefinition{isc\_cavity\_restart\_write}{control.in}
{
  \noindent
  Usage: \keyword{isc\_cavity\_restart\_write} \option{filename} \\[1.0ex] 
  Purpose: Write the cavity to the specified restart file
    once created. \\[1.0ex]
  \option{filename} is the name of the restart file (in \texttt{.xyz} format).
  If additionally \texttt{<filename>.bin} is present, the cavity is read from
  the latter instead. Note that the \texttt{.xyz} file has to be present in both
  cases. While this file itself is sufficient to create a cavity, only the
  \texttt{.bin} file allows for a fully deterministic restart.\\
}
Note: This keyword is intended for debugging purposes. 
Do not rely on the current structure of the cavity restart 
file as it might change in the future. 


\keydefinition{isc\_record\_cavity\_creation}{control.in}
{
  \noindent
  Usage: \keyword{isc\_record\_cavity\_creation} 
    \option{filename} \option{num} \\[1.0ex] 
  Purpose: Controls the output of snapshots during the 
    cavity generation process. When \option{num} 
    is positive, every \option{num} steps an XYZ snapshot 
    of the cavity is written to file \option{filename}. 
    For other choices of \option{num}, no output will 
    be generated. \\[1.0ex]
  \option{filename} is of type string. \\
  \option{num} is of type integer. Default: \option{0} \\
}
This keyword is intended for debugging purposes. Note that 
the size of the output file can become very large! 

