\section{Electronic Transport}
\label{section electronic transport}

Electronic transport in FHI-aims can be computed either using the
built-in routine described in this chapter or with the help of
aitranss-package (see Chapter \ref{sec:aitranss:source}). The built-in
routine computes the electronic transport through a nanostructure
connected to two, three or four semi-infinite leads using the
Landauer-B\"uttiker formalism. Only calculations within the zero-bias
limit are supported. All transport related actions are initiated with
the keyword \keyword{transport} that is followed by the action
required.

A typical work-flow of a transport calculation is as follows
\begin{enumerate}
\item Information for the semi-infinite leads are created. To this end
  the action \option{lead\_calculation} should be used. Separate
  periodic calculation is required for each lead and the third lattice
  vector must point away from the nanostructure, i.e., into the
  semi-infinite lead. The region used to model the semi-infinite lead
  should be large enough so that the basis functions from the
  nanostructure region do not extend beyond the lead region.
\item Once all the leads are calculated transport through the
  nanostructure is calculated using the action
  \option{transport\_calculation}.  This produces the tunneling
  information through the nanostructure for each pair of the
  leads. The nanostructure should be large enough so that the basis
  functions from different leads do not overlap. In this calculation
  the lead atoms need to be included in the file \texttt{geometry.in}
  and their positions must be exactly the same as in the calculation
  for the lead information. In particular, they should not be relaxed
  when the geometry of the nanostructure is optimized. This
  calculation should also be run as a periodic calculation using a
  sufficient amount of k-points. The transport part of the calculation
  is done using only one k-point but in order to ensure converged
  density and potential of the nanostructure region more k-points are
  usually needed.
\end{enumerate}

\subsection*{Tags for general section of \texttt{control.in}:}

\keydefinition{transport}{control.in}
{
  \noindent
  Usage: \keyword{transport} \option{action} [\option{further
      options}]\\[1.0ex] 
  Purpose: This is the keyword that is used to
  control the built-in transport routines. \\[1.0ex] 
  \option{action} is a string that specifies the kind of requested action in the
  transport calculation; any further needed options depend on
  \option{action}. \\ }

\subsection*{Specific actions \texttt{transport} keyword:}

\subkeydefinition{transport}{lead\_calculation}{control.in}
{
  \noindent
  Usage: \keyword{transport} \subkeyword{transport}{lead\_calculation} \\[1.0ex]
  Purpose: Computes the lead information for the given lead. Only one
  lead can be calculated at at time and each lead needs to be
  calculated separately.  The calculation is periodic and the third
  lattice vector must point into the lead.}

\subkeydefinition{transport}{transport\_calculation}{control.in}
{
  \noindent
  Usage: \keyword{transport} \subkeyword{transport}{transport\_calculation} \\[1.0ex]
  Purpose: Performs the transport calculation. Using this keyword
  requires that the information for all the semi-infinite leads is
  already calculated.  }

\subkeydefinition{transport}{tunneling\_file\_name}{control.in}
{
  \noindent
  Usage: \keyword{transport} \subkeyword{transport}{tunneling\_file\_name} \option{filename} \\[1.0ex]
  Purpose: Specifies the name of the file where the tunneling
  information is written. Each pair of the leads is written to a separate
  column in plain text format.}

\subkeydefinition{transport}{energy\_range}{control.in}
{
  \noindent
  Usage: \keyword{transport} \subkeyword{transport}{energy\_range} \option{$E_\mathrm{min}$} \option{$E_\mathrm{max}$} \option{$n_\mathrm{steps}$} \\[1.0ex]
  Purpose: Sets the energy range for which the tunneling information
  is calculated. The energy range $E_\mathrm{max} - E_\mathrm{min}$ is
  divided into $n_\mathrm{steps}$ steps and the tunneling information
  is calculated for each step. The energy zero is referenced at the
  chemical potential of the nanostructure.}

\subkeydefinition{transport}{lead\_i}{control.in}
{
  \noindent
  Usage: \keyword{transport} \subkeyword{transport}{lead\_i} \option{atom\_index} \option{filename} \\[1.0ex]
  Purpose: Tells the transport calculation where the information on
  the semi-infinite leads is stored. One line for each lead is
  required and the index i is substituted by the lead index (1, 2, 3,
  or 4). The option \option{atom\_index} is set to the index of the
  first atom of the lead in question in the file \texttt{geometry.in}
  and the coordinates of the lead atoms must be the same used when
  calculating the lead information. The option \option{filename}
  provides the name of the file where the lead information is
  stored. Note that for a successful transport calculation information
  for at least two leads is needed. Hence, this keyword must be
  invoked at least twice. The maximum number of leads supported is
  currently four.}

The Green's function for the semi-infinite leads needs to be solved
iteratively for each energy step. The iteration is regularized adding
a small complex part to the energy of the step being computed. At each
iteration the magnitude of the regularizing part is decreased until
either convergence of lower bound of the parameter is reached. The
iteration can be controlled by several keywords.

\subkeydefinition{transport}{number\_of\_boundary\_iterations}{control.in}
{
  \noindent
  Usage: \keyword{transport} \subkeyword{transport}{number\_of\_boundary\_iterations} \option{number} \\[1.0ex]
  Purpose: Sets the maximum number of iterations to solve the Green's
  function for each lead. Default: 10 }

\subkeydefinition{transport}{boundary\_treshold}{control.in}
{
  \noindent
  Usage: \keyword{transport} \subkeyword{transport}{boundary\_treshold} \option{number} \\[1.0ex]
  Purpose: Sets the convergence criterion for the iteration of the
  Green's functions. The metric used is the maximum absolute value
  change in the matrices for the Green's functions of the
  leads. Default: 1.0 }

\subkeydefinition{transport}{boundary\_mix}{control.in}
{
  \noindent
  Usage: \keyword{transport} \subkeyword{transport}{boundary\_mix} \option{number} \\[1.0ex]
  Purpose: Mixing value for the linear mixer in the iteration of the
  Green's function. Values below one correspond to
  under-relaxation. Default: 0.7}

\subkeydefinition{transport}{epsilon\_start}{control.in}
{
  \noindent
  Usage: \keyword{transport} \subkeyword{transport}{epsilon\_start} \option{number} \\[1.0ex]
  Purpose: Starting value for the regularizing complex parameter in
  the iteration for the Green's function for the semi-infinite
  lead. Only the magnitude of the parameter should be given with this
  keyword. Default: $0.02\imath$}

\subkeydefinition{transport}{epsilon\_end}{control.in}
{
  \noindent
  Usage: \keyword{transport} \subkeyword{transport}{epsilon\_end} \option{number} \\[1.0ex]
  Purpose: Ending value for the regularizing complex parameter in
  the iteration for the Green's function for the semi-infinite
  lead. Only the magnitude of the parameter should be given with this
  keyword. Default: $0.0001\imath$}

\subkeydefinition{transport}{fermi\_level\_fix}{control.in}
{
  \noindent
  Usage: \keyword{transport} \subkeyword{transport}{fermi\_level\_fix} \\[1.0ex]

  Purpose: Set the potential reference of the leads to their Fermi
  level. Since in the transport calculations the leads are calculated
  first and only after that incorporated into the system there is a
  question of a potential reference. Currently two options are
  available. The default option is to use the average of the energy
  levels of the lowest orbitals of the lead atoms. The second option
  invoked by this keyword is to align the Fermi levels of the leads
  between the lead calculations and the transport calculation.
  
  Note that since the calculation of the lead is separate from the
  transport calculation in general the distance from the energy of the
  lowest lying orbital to the Fermi level of the calculation is not
  necessarily the same. This means that the leads in the lead
  calculation are not in the same environment as in the transport
  calculation and bringing them together introduces an alignment
  problem for the potential. From the transport calculation point of
  view this implies that a gate voltage is introduced into the
  system. In the transport calculation the value of this gate voltage
  is printed. In the tunneling results the energy is always
  referenced to the Fermi level of the transport calculation. Default:
  \texttt{.false.}}
