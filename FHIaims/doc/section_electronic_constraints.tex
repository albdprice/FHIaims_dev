\section{Electronic constraints}
\label{Sec:constraint}

Most production calculations only require the converged ground state
of a calculation, but in some cases, a deliberate deviation from the
Born-Oppenheimer surface is desired. For example it may be desirable
to fix the spin state of a spin-polarized calculation using a defined
\keyword{multiplicity}. 

More generally, intuitive chemical concepts
may suggest the localization of a fixed given spin moment or number of
electrons in one part of a system, and a different spin moment or
number in another part. Examples include enforcing definite charges on 
ions in a system, or a desired spin moment on one particular atom.

The latter idea of partitioning different numbers of electrons or spin
moments in \emph{space} thought of as divided into different atoms is
inherently ambiguous. Nonetheless, this is a classic intuitive picture
of chemistry, and, at least in the limit of well-separated system
parts, becomes exact.

For \emph{non-periodic} geometries, FHI-aims implements the possibility
of constrained calculations to enforce predefined electron numbers in
different regions and/or spin channels. We follow the prescription of
Behler et al. \cite{Behler05,Behler07}, which assigns electrons to different
atoms according to a Mulliken-like analysis, i.e., by partitioning the
occupied Kohn-Sham eigenstates according to the occupation of
different atom-centered basis functions. 

For details regarding the method, we refer to the original references,
but we add here a clear word of caution. The implemented partitioning
is well-defined when the relevant parts of the system are far apart,
and/or when relatively small, confined basis sets are employed. For
large, overlapping basis sets, electrons may be \emph{spatially}
located at one atom even though they are \emph{numerically} assigned
to the basis functions of a different atom. In other words, the
procedure becomes more ambiguous as the basis size and completeness
increase. Contrary to the usual paradigm of electronic structure
theory, it is not meaningful to converge constrained DFT calculations
to the basis set limit \emph{if} the individual pieces of the system
are not very well separated.

To set up an electronic constraint, the only keywords normally
required are \keyword{constraint\_region} in \texttt{geometry.in} and
\keyword{constraint\_electrons} in \texttt{control.in}. The remaining
keywords documented below are normally required only for experimental
purposes or troubleshooting.

For the purpose of simulating electronic excitations, either from 
electronic core levels (XPS) or from valence levels (UPS), the keywords 
\keyword{force\_occupation\_basis} and \keyword{force\_occupation\_projector}
enforce electron occupations of specific atomic basis states or Kohn-Sham states, 
respectively.

\newpage

\subsection*{Tags for \texttt{geometry.in}:}

\keydefinition{constraint\_region}{geometry.in}
{
  \noindent
  Usage: \keyword{constraint\_region} \option{number} \\[1.0ex]
  Purpose: Assigns the immediately preceding \keyword{atom} to the
    region labelled \option{number}. \\[1.0ex]
  \option{number} is the integer number of a spatial region, which
    must correspond to a region defined by keyword
    \keyword{constraint\_electrons} in file
    \texttt{control.in}. Default: 1.
}
To divide up space into regions for the purpose of enforcing an
electronic constraint, each atom in the structure is included in a 
\keyword{constraint\_region}.

Simple example of an Na-Cl dimer (\texttt{geometry.in}):
\begin{verbatim}
  atom 0. 0. 0. Na
    constraint_region 1
  atom 0. 0. 3. Cl
    constraint_region 2
\end{verbatim}
assigns Na to the first region and Cl to the second region of a constrained calculation.

The special case of only one region (e.g., for a fixed spin moment
calculation) needs no explicit \keyword{constraint\_region}
labels. Note that, apart from an explicit setup by keyword
\keyword{constraint\_electrons}, the case of an \emph{integer} fixed
spin moment for the \emph{whole} system (all atoms) can also be called
by the shortcut \keyword{multiplicity}. 

\newpage

\subsection*{Tags for general section of \texttt{control.in}:}

\keydefinition{constraint\_debug}{control.in}
{
  \noindent
  Usage: \keyword{constraint\_debug} \option{flag} \\[1.0ex]
  Purpose: If set, provides extra output that monitors the convergence
    of the local constraint potentials used to enforce the requested
    constraint. \\[1.0ex]
  \option{flag} is a logical string, either \texttt{.true.} or
  \texttt{.false.} Default: \texttt{.false.} \\
}

\keydefinition{constraint\_electrons}{control.in}
{
  \noindent 
  Usage: \keyword{constraint\_electrons} \option{region} \texttt{n\_1}
    [\texttt{n\_2}] \\[1.0ex]
  Purpose: Fixes the number of electrons to \texttt{n\_1} (in the
    spin-polarized case, to \texttt{n\_1} in the spin-up channel and
    \texttt{n\_2} in the spin-down channel) for a given
    \option{region}. \\[1.0ex]
  \option{region} is an integer number, corresponding to one of the
    regions listed as \keyword{constraint\_region} in
    \texttt{geometry.in}. \\
  \option{n\_1} is the number of electrons in the corresponding
    \option{region} (the number of spin-up electrons in the case of a
    spin-polarized calculation). \\
  \option{n\_2} is the number of spin-down electrons in the
    corresponding \option{region} (only needed in the spin-polarized
    case). \\
}
This is the central keyword that can be used to define a strict
constraint on the electron numbers associated (i) with the orbitals of a
given subset of atoms (``region'') and / or (ii) with a given spin channel.
See the \keyword{multiplicity} keyword for a shortcut for fixed spin
moment calculations with an integer spin multiplicity. 

\keydefinition{constraint\_it\_lim}{control.in}
{
  Usage: \keyword{constraint\_it\_lim} \option{number} \\[1.0ex]
  Purpose: For the determination of the constraint potentials in
    different \keyword{constraint\_region}s, sets the maximum number
    of internal iterations before the search for a converged value is
    aborted. \\[1.0ex]
  \option{number} is an integer number. Default: 200. \\
}
The method to determine the constraint potentials that enforce the
local electron / spin constraint is set by \keyword{constraint\_mix};
for more than one active \keyword{constraint\_region}, this
determination is always iterative. Keyword
\keyword{constraint\_it\_lim} sets the maximum number of iterations
before this search is aborted in case of failed convergence (or too
ambitious accuracy requirements from keyword
\keyword{constraint\_precision}). 

\keydefinition{constraint\_precision}{control.in}
{
  \noindent
  Usage: \keyword{constraint\_precision} \option{tolerance} \\[1.0ex]
  Purpose: Sets the precision with which each requested local
    constraint on the electron count must be fulfilled. \\[1.0ex]
  \option{tolerance} is a small positive real number. Default:
  10$^{-6}$. \\
}

\keydefinition{constraint\_mix}{control.in}
{
  \noindent
  Usage: \keyword{constraint\_mix} \option{factor1} [\option{factor2}]
    \\[1.0ex]
  Purpose: Mixing factors for the iteratively determined constraint
    potentials. \\[1.0ex] 
  \option{factor1} is a mixing factor for the iteratively determined
    constraint potentials (for spin-polarized calculations, the
    spin-up mixing factor). \\
  \option{factor2} is the mixing factor for spin-down constraint
    potentials in the case of spin polarized calculations. \\
}
Only meaningful for non-standard settings of
\keyword{mixer\_constraint}, irrelevant for the standard
\texttt{bfgs} case. 

\keydefinition{ini\_linear\_mixing\_constraint}{control.in}
{
  \noindent
  Usage: \keyword{ini\_linear\_mixing\_constraint} \option{number}
    \\[1.0ex]
  Purpose: If keyword \keyword{mixer\_constraint} is a Pulay mixer,
    initial linear mixing for a few iterations can be requested
    first. \\[1.0ex]
  \option{number} is the number of initial linear iterations. \\
}
Only meaningful for non-standard settings of
\keyword{mixer\_constraint}, irrelevant for the standard
\texttt{bfgs} case. 

\keydefinition{mixer\_constraint}{control.in}
{
  \noindent
  Usage: \keyword{mixer\_constraint} \option{type} \\[1.0ex]
  Purpose: Sets the iterative algorithm to determine constraint
    potentials. \\[1.0ex]
  \option{type} is a string. Default: \texttt{bfgs} \\
}
This flag has nothing to do with electron density mixing or the
electronic self-consistency loop. Instead, this defines the process to
determine constraint potentials that enforce the requested electron
number constraints, \emph{if} the keyword \keyword{constraint\_electrons} was
invoked. This process happens in each  
s.c.f. iteration after the Hamiltonian matrix is known, in the
course of solving the Kohn-Sham eigenvalue problem.

If more than one constraint regions are requested, determining the
constraint potentials to enforce the local constraint on the electron
numbers is an iterative process. Technically, FHI-aims supports three
different algorithms for \option{type} :
\begin{itemize}
  \item \texttt{bfgs} : The default. A BFGS algorithm that optimizes
    the constraint potentials to their nearest local optimum. Nothing
    else should be used unless for experimental purposes. 
  \item \texttt{linear} : Linear mixing algorithm to determine the
    constraint potentials.
  \item \texttt{pulay} : Pulay-type mixing algorithm to determine the
    constraint potentials. 
\end{itemize}
Under normal circumstances, the \keyword{mixer\_constraint} keyword
should not be needed explicitly.

\keydefinition{n\_max\_pulay\_constraint}{control.in}
{
  \noindent
  Usage: \keyword{n\_max\_pulay\_constraint} \option{number} \\[1.0ex]
  Purpose: If the \texttt{pulay} mixer is selected for
    \keyword{mixer\_constraint}, sets the number of iterations to be
    mixed. \\[1.0ex]
  \option{number} is the number of mixed iterations. Default: 8. \\
}
Only meaningful for non-standard settings of
\keyword{mixer\_constraint}, irrelevant for the standard
\texttt{bfgs} case. 

\keydefinition{force\_occupation\_basis}{control.in}
{
  \noindent
  Usage: \keyword{force\_occupation\_basis} \option{i\_atom} \option{spin} \option{basis\_type} \option{basis\_n} \option{basis\_l} \option{basis\_m}
   \option{occ\_number} \option{max\_KS\_state} \\[1.0ex]
  Purpose: Flag originally programmed to compute core-hole spectroscopy simulations (for a short how-to cf. \keyword{force\_occupation\_projector}). 
           In practice, it constrains the occupation of a specific
           energy level of an specific atom, being also able to ``break the symmetry'' 
           of an atom.  \\[1.0ex]
  \option{i\_atom} is the number of the atom on which the occupancy is constrained, as it is listed in the
                   geometry.in file.\\[1.0ex] 
  \option{spin} is the spin channel (e.g., 1 if only one spin channel). \\[1.0ex] 
  \option{basis\_type} is the type of basis which is used to force
                       the occupation of the orbital (set it to \texttt{atomic}). \\[1.0ex]
  \option{basis\_n} is the main quantum number for
                    the state of interest.\\[1.0ex]
  \option{basis\_l} is the orbital momentum quantum number for the state of interest.\\[1.0ex]
  \option{basis\_m} is the projection of the orbital momentum onto the z-axis (-1, 0, or 1 for a p state).\\[1.0ex]
  \option{occ\_number} is the occupation constraint for the chosen state.\\[1.0ex]
  \option{max\_KS\_state} is the number of the highest energy Kohn-Sham state 
                    in the system that will be included in the constraint.\\[1.0ex]  
}
  Example: \\[1.0ex]
  \texttt{force\_occupation\_basis 1 1 atomic 2 1 1 1.3333 6}

  This choice will constrain the overall occupation of a given basis function 
  (not Kohn-Sham state!) in the system to 1.3333 electrons.

  The basis function in question resides on the first atom (number 1)
  as listed in \texttt{geometry.in}. The first spin channel is
  constrained. 

  Since we are interested in constraining an atomic-like orbital, we
  choose one that is part of the minimal basis (type ``atomic''). 

  In fact, we let the constraint act on a 2$p$ level, $m$=1 (defined
  by the sequence ``2 1 1'').

  Only Kohn-Sham orbitals up to the 6th state (counted in the overall
  system) will be included in the constraint. In general, it is a good
  idea to constrain the orbital in question out of the occupied space,
  i.e., choose a value for \texttt{max\_KS\_state} that indicates a
  state above the Fermi level.

  There is a small problem here: We need to define the occupation of a
  ``basis function,'' but   really, we here have a non-orthogonal
  basis. Strictly speaking, only the Kohn-Sham states    in the system
  have a well-defined occupation. What to do? 

  One thing we could use (and we do) are Mulliken occupation numbers
  of the basis functions. These are formally always well defined. In
  practice, however, as the overall basis set becomes larger and
  larger and approaches (over-)completeness, Mulliken occupations
  become less and less meaningful because other basis functions are
  not exactly orthogonal to the one used in our projection, and can
  ``restore'' the component that was originally constrained away. 

  Either way, we use Mulliken occupations, assuming that the atomic
  core basis functions are sufficiently compact and practically
  orthogonal to everything else.  

  This assumption will work well for localized basis functions
  such as the 1$s$ levels of most elements. As a rule, the constraint
  is expected to be less and less unique if applied to more
  delocalized basis functions -- there can even be multiple
  different self-consistent constrained solutions for the same formal
  constraint. For instance, Si 2$p$ basis functions can exhibit this
  problem if the basis sets on the surrounding atoms -- which overlap
  with the Si atom -- become too large. Here, a smaller basis set
  (tier 1) can indeed be the way to keep a qualitatively meaningful
  Mulliken-type constraint.


\keydefinition{force\_occupation\_projector}{control.in}
{
  \noindent
  Usage: \keyword{force\_occupation\_projector} \option{KS\_state} \option{spin} \option{occ} \option{KS\_start} \option{KS\_stop} \\[1.0ex]
  Purpose: This keyword enforces the occupation \option{occ} in \option{KS\_state}
     of spin channel \option{spin}. Between different SCF steps the overlap of this state 
     with states \option{KS\_start} to \option{KS\_stop} is being checked and the 
     constraint is changed correspondingly if the main character of the state changes. \\
}

  Example: \\[1.0ex]
  \texttt{force\_occupation\_projector  8  1 0.0 6 10} \\
  \texttt{force\_occupation\_projector  9  2 1.0 6 10} \\

This enforces 0.0 occupation in state 8 of spin channel 1 
and 1.0 occupation for state 9 of spin channel 2. KS states 
between 6 and 10 are checked for overlap with state 8 and 9 
of previous SCF steps. If 8/1 was occupied and 9/2 was unoccupied 
in the ground state, this corresponds to a triplet excitation 8$\rightarrow$9.

To simulate XPS energies with $n$ inequivalent atoms of the same species (called excitation centre in the following) a total of $n+1$ single runs is required: One ground state calculation and one \keyword{force\_occupation\_basis}/\keyword{force\_occupation\_projector} calculation for each of the excitation centres.

The ground state calculation should use the \keyword{restart\_write\_only} or the \keyword{restart} flag to create a restart file that is needed for the \keyword{force\_occupation\_basis} or alternatively the \keyword{force\_occupation\_projector} run. Therefore the geometry.in and all other parameters (except the charge) such as basis sets have to match. In practise it is often beneficial for the interpretation of the XP spectra to use \keyword{output cube eigenstate} to have an idea of the localization of the different core levels. 

For the simulation of the XPS spectra typically a full core hole is introduced at the excitation centre (for example in ref.\ \cite{Diller2014}). Relying on initial state effects alone (i.e., defining the ionization energies using ground state eigenvalues) neglects the screening of the core hole by valence electrons \cite{Lizzit2001} and is therefore not a good approximation for XPS. For example, to force the occupation of eigenstate 3 to 1.0, where states 1-4 are (near-)degenerate or at least very similar in energy and type: 

  \texttt{force\_occupation\_projector 3 1 1.0 1 4 } \\

Results in the following occupation (the introduction of the core hole leads to a re-ordering):

\texttt{
\begin{tabular}{cccc}
State & Occupation & Eigenvalue [Ha] & Eigenvalue [eV]\\
      1   &    1.00000     &    -16.148150     &    -439.41353 \\
      2   &    2.00000     &    -14.162982     &    -385.39434 \\
      3   &    2.00000     &    -14.162981     &    -385.39432 \\
      4   &    2.00000     &    -14.091396     &    -383.44640 \\
\end{tabular}}

Note that \keyword{charge} was set to 1 to take into account the reduced electron number and a restart from the ground state run was made using \keyword{restart}.

XPS energies can then be calculated as the difference of the total energy obtained in the ground state calculation and the total energy of the core-hole excited simulation (corresponding to the definition of the ionization energy). That means that for each excitation center an ionization energy is calculated. For the core level shifts only relative energy differences are relevant, which are already directly reflected in the differences of total energies of the core-hole excited states. If, however, absolute energies are of interest, note that experiments are referenced to either the vacuum level or the Fermi level, and that simulations including an extended surface might differ by the workfunction from those for isolated molecules. The ionization energies can then by broadened with Gaussian functions of same amplitude (assuming no preferential direction, especially valid for $1s$ spectra) and summed up to obtain the total XPS spectrum.

Simulating NEXAFS spectra can be less straightforward as there are different approximations to account for the core hole and the excited electron. One possibility is to use the transition potential approach \cite{Triguero1998}, where instead of a full core only half a core hole is used, i.e., $n = 0.5$ in one spin channel. Independently from the approximation used for the core hole: To obtain the dipole matrix elements that give information about the transition probability the flag \keyword{compute\_dipolematrix} needs to be used. Note that to use this option the FHI-aims binary has to be compiled enabling hdf5, as the output is a hdf5 container containing eigenvalues and matrix elements. It is recommended to include additional \keyword{empty\_states}, depending on the amount of unoccupied states you want to probe. In this case the ground state calculation has already to include the same number of empty states, otherwise a restart is not possible.


\keydefinition{force\_occupation\_smearing}{control.in}
{
  \noindent
  Usage: \keyword{force\_occupation\_smearing} \option{smearing\_width} \\[1.0ex]
  Purpose: If keyword is set, the occupation constraints are enforced in form of gaussians 
  with a width of \option{smearing\_width} instead of delta peaks. This applies to
  orbitals within an energy range of -+ 3*\option{smearing\_width} \\[1.0ex]
  Default: No Smearing at all. \\
}

  Example: \\[1.0ex]
  \texttt{force\_occupation\_smearing  0.05} \\

This keyword helps to converge systems with state degeneracies, which are 
constrained by force\_occupation\_projector. Specifically 
when calculating electronic excited states in the frontier orbital regime, many state 
degeneracies can occur. If one of two degenerate states is constrained to a different 
than the ground state occupation, convergence can be hindered. If this happens, 
this keyword can enable convergence for the price of a minimally different final occupation 
and therefore also a small error in excitation energy. One should be very careful with 
this keyword and only employ it if convergence can not be reached without.

WARNING: It is very easy to generate reandom numbers when using this keyword. The 
smearing value should never exceed 0.15
