\section{Calculation of vibrational and phonon frequencies}
\label{Sec:vib}

Vibrations (for non-periodic structures) and phonons (for periodic
structures) can be computed in FHI-aims via script-based finite
difference approaches. Each tool first does all necessary DFT
calculations and then calls a routine to set up and diagonalize the
Hessian matrices. 

We support vibrations in non-periodic systems and phonons in periodic
systems using two different types of infrastructure:
\begin{itemize}
  \item Infrastructure for vibrations in non-periodic systems
    (clusters or molecules). The relevant perl scripts and source code
    for this step for the entire finite-difference calculation are
    included with FHI-aims. To compile the tools, change into the
    \emph{src/} directory, from which you also compiled the original
    FHI-aims. With the same Makefile settings as for the main program,
    type either \option{make vibrations}, \option{make
    vibrations.mpi}, or (in most cases) \option{make vibrations.scalapack.mpi}.
  \item Infrastructure for phonons in periodic systems, based on the GPL
    \texttt{phonopy} tool (not distributed with FHI-aims
    itself). Phonopy is a python-based utility to generate and process
    finite-diffence generated phonon-related quantities from a variety
    of codes. The infrastructure for FHI-aims support in phonopy was
    mainly written by J\"org Meyer (now at Universiteit Leiden) and
    contributed to the GPL'd phonopy project. We note that 
    \texttt{phonopy} and its dependencies must be installed separately
    (in addition to FHI-aims) on your system. \textbf{Only phonopy releases
    with version numbers <= 1.12.8.4 are currently compatible with FHI-aims}.
  \item A third level of infrastructure for vibrations and phonons in
    non-periodic and periodic systems based on density functional
    perturbatio theory (DFPT) is nearing readiness and usable to some
    degree. However, parts of this implementation are not yet
    optimized. The DFPT implementation should therefore still be
    considered experimental at this time. 
\end{itemize}


\emph{Note that the former, native phonon infrastucture in FHI-aims is
  no longer needed. The previous make targets \option{phonons} and
    \option{phonons.mpi} still exist, but we no longer recommend them.
    Phonopy is, simply, more stable and general.}

The infrastructure for non-periodic vibrations is covered in the next
section, followed by a section on phonons in periodic systems using
phonopy. 

\subsection*{Vibrations by finite differences within FHI-aims}

For the FHI-aims vibrations target, we recommend the
\texttt{vibrations.scalapack.mpi} make target. This will create two 
files in the FHI-aims binary directory. One is the actual
diagonalization subprogram, the other one the
\option{perl}-wrapper. \textbf{You will need to edit the header of the
  \option{perl}-wrapper scripts} and give three key parameters: The
absolute location of the FHI-aims binary directory
(\option{\$AIMS\_BINDIR}), the aims executable to be used
(\option{\$EXE}), and the calling command for the executable
(\option{\$CALLING\_PREFIX} and \option{\$CALLING\_SUFFIX}). The
\option{\$CALLING\_PREFIX} is required for specification of 
parallel runs, for example 

\option{\$CALLING\_PREFIX = mpirun -np 2}

for a two-core parallel run. In addition, some parallel environments
(most notably IBM's \option{poe} require that the number of cores is
specified after the executable is called, for those instances you can
set \option{\$CALLING\_PREFIX}. Note that the postprocessing routines
for the vibrations are also parallel. 

Having compiled the vibrations script, running it is
very simple. We use the same water molecule that was relaxed in the example section
\ref{Sec:example-Etot}. This test case is
contained in the directory \option{testcases/H2O-vibrations}. 

We are here calculating a second derivative of the energy from the
numerical change on the forces arising from small finite
displacements $\delta$, in the following way:
\begin{equation}
  \frac{\partial E}{\partial x_i \partial x_j} = \frac{\boldF_i(x_j+\delta)-\boldF_i(x_j-\delta)}{2 \delta}
\end{equation}
Here, $E$ is the total energy as a function of all atomic coordinates
$x_i$ ($i$=1,...,3$M$ for $M$ individual atoms), and $\boldF_i$ are the
components of the analytical first derivatives (forces) on each atom
(again, $i$=1,...,3$M$) due to a displacement of coordinate another
$x_j$ by an amount plus or minus $\delta$. 

Since we are interested in the Hessian in a local minimum of the potential
energy surface (ideally, $\boldF_i(\{x_j\})$=0 for all $i$ at the
local minimum geometry $\{x_j\}$), this expression is the difference
of two force values that are themselves already near zero. The smaller
the displacement $\delta$, the smaller the absolute magnitude of the
forces to be subtracted, giving greater weight to any residual
numerical noise in the calculation. The larger the displacement
$\delta$, the larger does the difference become, but at the same time,
we will also move out of the region of the potential energy surface
that is exactly harmonic, and introduce systematic errors into the
Hessian. We are thus faced with a tradeoff between choosing a value
$\delta$ that is large enough to be free of any potential numerical
noise, yet small enough to avoid large systematic errors due to real
anharmonicities. The default value for $\delta$ is 0.0025 {\AA}, but
this value should be checked explicitly in any practical calculation.
 
Since the vibrational frequencies depend strongly on the accuracy
of the forces, and on any near-zero residual forces on the fully
relaxed geometry itself, we apply the following changes to the
earlier file \option{control.in}: 
\begin{itemize}
\item We use the \emph{tight} species defaults for elements H and
  O. Among other things, these defaults prescribe denser integration
  grids than \emph{light} and thus lead to more accurate
  forces. Obviously, the previous optimum geometry from the
  \emph{light} settings should be postrelaxed to the exact minimum for
  the modified \emph{tight} settings, and our script does this
  automatically (but check the output just in case).
\item We increase the accuracy to which the forces are converged in
  the self-consistency cycle, \keyword{sc\_accuracy\_forces}, to
  \option{1E-5}. \emph{However}, be sure to check that the remaining
  settings, especially \keyword{sc\_accuracy\_eev} are accurate
  enough. You do \emph{not} want your FHI-aims calculation to spend
  more than 1-2 s.c.f. cycles per geometry on actually converging the
  forces, because each s.c.f. iteration with forces can be up to a
  factor 10 more expensive than an s.c.f. iteration during which the
  forces are not yet checked.
\item We change the \keyword{relax\_geometry} \option{bfgs}
  convergence criterion to
  \option{1E-4}. This is, again, a rather harsh requirement, and one
  should verify (especially for large molecules) that the
  relaxation algorithm is not spending large amounts of time on
  failed relaxation steps near the minimum, probing any residual
  numerical noise of integration grids etc. rather than following an
  actual, smooth potential energy surface.
\end{itemize}
These new settings contain all that is necessary to get close enough
to the actual minimum. Again, the convergence settings quoted above
should normally be fine, but in case of doubt, check. If, for some
reason, those stringent criteria can not be reached exactly, it is
still preferable to obtain a slightly less converged Hessian within an
amount of CPU time that is actually finite.

In order to run the calculation of the vibrations, change to the
directory containing the input files \texttt{control.in} and
\texttt{geometry.in} and run the aims.vibration.*.pl script in the
directory. That's it. The calculation should take no more than a few
minutes to complete. It first relaxes the structure to its minimum
configuration and then applies six finite displacements to each of the
atoms: 18 single-point calculations in total for the water
molecule. The result (using the default settings for $\delta$) can be
visited in subdirectory \texttt{test\_default}.

The output stream contains all of the important data, which is
additionally saved in a number of temporary and output files. The file
basic.vib.out contains the output of the normal mode
analysis, while the file basic.xyz contains the actual eigenmodes
(vibrations), which can be read by a number of molecular viewers. 

A short note on the actual physical output basic.vib.out: The
frequencies are given in cm$^{1}$, the corresponding zero point
energies in eV, and their cumulative total. It is important to ensure
that the first six eigenmodes are close to zero, which means that the
structure in question actually corresponds to a minimum on the
potential energy surface. FHI-aims does not do an \emph{a priori}
reduction of the translational and rotational eigenmodes, which allows
an explicit check on the quality of the strucure.

In the default version (no command line options specified), the
vibration script automatically assumes a jobname \option{basic}. This
name, as well as the finite difference displacement $\delta$, can be
changed on the command line, leading to the following complete calling
form of the script:

\option{aims.vibrations.*.pl $\langle$jobname$\rangle$ delta}

$\langle$jobname$\rangle$ is a name of choice that will be
appended to all output files produced during the run. One can thus
run the same job with different settings $\delta$ multiple times in
the same directory, starting from the same input files.

The second parameter, \option{delta}, is the finite
displacement $\delta$ used for calculating the Hessian matrix. Its default is
0.0025 \AA, which has been doing a good job for all the cases we are
aware of, but still, please verify that it works for your particular
system. (See the brief discussion above). 

For example, for our test case, subdirectory
\texttt{test\_delta\_0.001} contains a second example run with the
following command line:

\option{aims.vibrations.*.pl test\_0.001 0.001}

For the default settings, our resulting frequencies look like this:
{\footnotesize
\begin{verbatim}
  Mode number      Frequency [cm^(-1)]   Zero point energy [eV]   IR-intensity [D^2/Ang^2]
            1             -12.66044778              -0.00078485                 1.89406419
            2              -0.15559990              -0.00000965                 0.00266735
            3              -0.00081379              -0.00000005                 0.00000000
            4               0.04885082               0.00000303                 0.00002821
            5               6.58934967               0.00040849                 0.00000000
            6               7.01167236               0.00043467                 5.41560478
            7            1592.96295435               0.09875111                 1.63341394
            8            3708.86739272               0.22992045                 0.04298854
            9            3813.74446188               0.23642200                 1.07616306
\end{verbatim}
}

In contrast, the smaller value $\delta$=0.001~{\AA} produces the
following results:
{\footnotesize
\begin{verbatim}
  Mode number      Frequency [cm^(-1)]   Zero point energy [eV]   IR-intensity [D^2/Ang^2]
            1              -5.13719038              -0.00031847                 1.89407621
            2              -0.06616895              -0.00000410                 0.00323346
            3              -0.00073155              -0.00000005                 0.00000000
            4               0.01938144               0.00000120                 0.00002697
            5               2.47172391               0.00015323                 0.00000000
            6               2.70760068               0.00016785                 5.41498492
            7            1593.00852915               0.09875393                 1.63340808
            8            3708.82166272               0.22991761                 0.04299176
            9            3813.70936996               0.23641982                 1.07615430
\end{verbatim}
}

So, in this case, the rather tight relaxation and force convergence
settings do produce a set of translations and rotations that are
closer to zero for $\delta$=0.001~{\AA} than the default.

However, once again be warned that for larger molecules the same harsh
convergence criteria may not be applicable, and a too small value
$\delta$ can in fact inflate any residual noise. So, a small value
$\delta$=0.001~{\AA} or even less should not be blindly expected to improve
numerical accuracy. 

It should also be noted that the ``physical'' vibrational frequencies
7-9, above the six translations and rotations, remain completely unimpressed
by the change of $\delta$, either way. This is generally true, and
would even remain true for a yet larger value $\delta$=0.005~{\AA},
which would also be a reasonable default choice.


In the following you find the description of tags defining the output of free energies.
\keydefinition{vibrations}{control.in}
{ 
Usage: \keyword{vibrations} \option{[subkeywords and their options]}\\[1.0em]
  Purpose: allows to choose temperature/pressure range for output of free energies.
  It has the subkeywords \subkeyword{vibrations}{free\_energy},
  \subkeyword{vibrations}{trans\_free\_energy},
  as detailed below.\\ }

  \subkeydefinition{vibrations} {free\_energy}{control.in}
  {Usage: \keyword{vibrations} \subkeyword{vibrations}{free\_energy} \option{Tstart Tend Tpoints} \\[1.0em]
  Purpose: governs the calculation of the free energies of vibration and rotation (rigid rotor approximation)\\}
The calculation of the free energy of vibration and rotation (rigid rotor approximation) can be requested with this keyword. They are calculated between the temperatures \option{Tstart} (K) and \option{Tend} (K) with a total of \option{Tpoints} steps. The implemented equations read: 
\begin{equation}
   F_{\text{vib}}(T) = \sum_i^{3N-6} \left[ \frac{\hbar \omega_i}{2} + k_B T \ln \left(1-e^{\frac{-\hbar\omega_i }{k_B T}} \right)\right] \quad ,
\end{equation}
with N being the number of atoms in the molecule and $\omega_i$ depicting the vibrational frequencies.
\begin{equation}
   F_\text{rot}(T) = -\frac{3}{2} k_B T \ln \left[ \frac{2k_BT}{\hbar^2} (I_A I_B I_C)^{1/3} \pi^{1/3}\right] \quad ,
\end{equation}
where $I_A$, $I_B$, and $I_C$ depict the moments of inertia of the molecule.

  \subkeydefinition{vibrations}{trans\_free\_energy}{control.in}
  {Usage: \keyword{vibrations} \subkeyword{vibrations}{trans\_free\_energy} \option{pstart pend ppoints} \\[1.0em]
  Purpose: governs the calculation of the translational free energy\\}
The calculation of the free energy of translation can be requested with this keyword. 
It only works if \keyword{vibrations} \subkeyword{vibrations}{free\_energy} is used as well.
The free energy of translation is calculated within the temperature range given with
\keyword{vibrations} \subkeyword{vibrations}{free\_energy} and between pressures \option{pstart} (Pa) and
\option{pend} (Pa) with \option{ppoints} steps. The equation reads:
\begin{equation}
   F_\text{trans} = -k_B T \left[ \ln \frac{k_B T}{p} + 1 + \ln(\frac{m k_B T}{2\pi \hbar^2})^{3/2}\right] \quad,
\end{equation}
with $p$ denoting the partial pressure.

\subsubsection*{Harmonic Raman spectra}

A third optional argument is available in the vibrations script, which allows for the calculation of harmonic Raman spectra.
In order to obtain such spectra, one should type the following,

\option{aims.vibrations.*.pl $\langle$jobname$\rangle$ delta polar} \,.

The polarizabilities are obtained with DFPT, and the derivatives with respect to the displacement are then calculated
in a similar way as for IR spectra.
All the formulae can be found in the following reference: Neugebauer et al., J. Comput. Chem. 23: 895-910, 2002.
In particular, Eq.~(43) is used to produce the final Raman intensities.
Keep in mind that Raman spectra obtained in such fashion are much more time-consuming than IR spectra, the latter requiring only DFT.
Note that Raman spectra for solids can also be calculated in this manner (only $\Gamma$-point is involved).

\subsection*{Vibrations by density-functional perturbation theory (DFPT) within FHI-aims}
\keydefinition{DFPT vibration}{control.in}
{
Usage: \keyword{DFPT vibration} \option{[subkeywords and their options]}\\[1.0em]
  Purpose:  Allows to calculate vibrations using density-functional perturbation theory, use Acoustic Sum Rule (ASR) to get Hessian matrix, do not use moving-grid-effect.\\


Usage: \keyword{DFPT vibration\_with\_moving\_grid\_effect} \option{[subkeywords and their options]}\\[1.0em]
  Purpose: give the results for vibrations with moving-grid-effect, do not use ASR for Hessian matrix.\\ 

Usage: \keyword{DFPT vibration\_without\_moving\_grid\_effect} \option{[subkeywords and their options]}\\[1.0em]
  Purpose: give the results for vibrations without moving-grid-effect, ONLY served as comparison with vibration\_with\_moving\_grid\_effect.\\ }



\keydefinition{DFPT vibration\_reduce\_memory}{control.in}
{
Usage: \keyword{DFPT vibration\_reduce\_memory} \option{[subkeywords and their options]}\\[1.0em]
  Purpose: Allows to calculate vibrations density-functional perturbation theory by using nearly the same memory as DFT.  At present, functionals LDA, PBE are supported, relativistic is also supported. It should be noted that PBE and PBE+TS is supported only for DFPT cycle (first-order-H), but not for Hessian.  Only linear-mix (no Pulay-mixer) can be used for DFPT vibration\_reduce\_memory at present.\\ }

Here is an example,  the following need to be added to control.in:
\begin{verbatim} 
DFPT vibration_reduce_memory
DFPT_mixing   0.2       #default is 0.2
DFPT_sc_accuracy_dm   1E-6  # default is 1.0d-6
\end{verbatim}  



\subsection*{Polarrizability by density-functional perturbation theory within FHI-aims}
\keydefinition{DFPT polarizability}{control.in}
{
Usage: \keyword{DFPT polarizability} \option{[subkeywords and their options]}\\[1.0em]
  Purpose: Allows to calculate polarizability for cluster systems using density-functional perturbation theory.\\  For "DFPT polarizability", functionals LDA, PBE, HF(RI-V) are supported, relativistic is also supported.  }
  
Here is an example for using DFPT polarizability,  the following need to be added to control.in:
\begin{verbatim} 
DFPT polarizability
DFPT_mixing   0.5             #default is 0.2
DFPT_sc_accuracy_dm   1.0d-6  # default is 1.0d-3
dfpt_pulay_steps 6            # default is 8
\end{verbatim}  



\subsection*{Phonon(real space) by density-functional perturbation theory within FHI-aims}
\keydefinition{DFPT phonon}{control.in}
{
Usage: \keyword{DFPT phonon} \option{[subkeywords and their options]}\\[1.0em]
  Purpose: Allows to calculate phonon (real space method) for PBC systems using density-functional perturbation theory. This method could get force constants using real space method and give the phonon band structures. At present, only functionals LDA without relativistic is supported.\\ }

Here is an example for using DFPT phonon,  the following need to be added to control.in:
\begin{verbatim} 
DFPT phonon
DFPT_mixing   0.5             #default is 0.2
DFPT_sc_accuracy_dm   1.0d-6  # default is 1.0d-3
dfpt_pulay_steps 6            # default is 8
\end{verbatim}  
  

\subsection*{Phonon(reciprocal space) by density-functional perturbation theory within FHI-aims}
\keydefinition{DFPT phonon\_reduce\_memory}{control.in}
{
Usage: \keyword{DFPT phonon\_reduce\_memory} \option{[subkeywords and their options]}\\[1.0em]
  Purpose: Allows to calculate phonon (reciprocal space method) at q point for PBC systems using density-functional perturbation theory. At present, this keyword only works to get dynamic matrix at q = 0.  This feature is under developing. At present, functionals LDA, PBE are supported, relativistic is also supported. It should be noted that PBE and PBE+TS is supported only for DFPT cycle (first-order-H), but not for Hessian.\\}

Here is an example for using DFPT phonon\_reduce\_memory,  the following need to be added to control.in:
\begin{verbatim} 
DFPT phonon_reduce_memory
DFPT_mixing   0.5             #default is 0.2
DFPT_sc_accuracy_dm   1.0d-6  # default is 1.0d-3
dfpt_pulay_steps 6            # default is 8
\end{verbatim}  



\subsection*{Phonon calculations in general}

To calculate the phonons, we use the so-called
\emph{direct} or \emph{supercell} method, based on the approximations
that the interactions necessary to calculate the phonons are
finite [see e.g. Ref. \cite{Parlinski97} and references therein
for details of the method]. 

A phonon calculation proceeds as follows. A single (user-specified)
unit cell is extended a number of times in the direction of all three
basis vectors. A finite displacement technique as for the vibrations is then
used to gather the necessary force response, which then makes up a
basic force constant matrix. All finite displacement structures are
investigated for their symmetry (within the octahedral group), such as
to minimize the computations required and also to obtain properly
symmetrized forces where applicable. Finally, the force constant matrix is
diagonalized for a number of selected reciprocal lattice points
(i.e. along the requested bands) and the eigenvalues give the phonon
spectra. 

Three things are really important:
\begin{itemize}
\item {\bf Converge your supercell size.} The minimum sensible cell
  is $2\times 2\times 2$. It might be tempting to restrict a
  calculation to this size, but the results will almost certainly be
  insufficient. 
\item {\bf The file geometry.in must contain only the primitive 1x1x1
  cell of your lattice.} \texttt{phonopy-FHI-aims} constructs its own  
  supercell. If, by accident, your \option{geometry.in} already
  contains a $3\times3\times3$ supercell and you request phonons for a
  $3\times3\times3$ supercell on top of that, you will end up with a 
  $(9\times 9\times 9)$ supercell calculation. 
\item Do not forget to run the actual FHI-aims calculations with the
  right \keyword{k\_grid} for the supercell -- not for the primitive
  cell. If your primitive cell requires a $(20\times 20\times 20)$
  $k$-space grid, a $(5\times 5\times 5)$ supercell only requires
  \keyword{k\_grid} \texttt{4 4 4}. If, by accident, you use a
  $(20\times 20\times 20)$ $k$-grid for the $(5\times 5\times 5)$
  supercell instead, you will pay with a disproportionately large
  amoung of computer time, 125 times more time than what you should have
  been using. 
\end{itemize}

\emph{Notes: (i)
  FHI-aims does not (yet) provide support for long-range electrostatic
  fields ionic compounds, which leads to splittings of the LO and TO
  branches at the $\Gamma$ point. Phonopy supports this, but providing
  the necessay derivatives within FHI-aims is a work in progress. (ii)
  Tight s.c.f. convergence (\keyword{sc\_accuracy\_rho} etc.)} is a
  strong requirement for phonons. If you see, e.g., non-zero
  frequencies at the $\Gamma$ point, check your s.c.f. convergence.


\subsection*{phonopy and FHI-aims}
\textbf{WARNING:} \textit{The Phonopy 2.x releases are not fully compatible with FHI-aims. 
We are currently working on that. Earlier versions, especially all releases with version numbers <= 1.12.8.4 
are fully compatible. If you erroneously installed a newer version, you can easily downgrade
again by typing:} \texttt{pip install --force-reinstall phonopy==1.12.8.4}


There are roughly four steps to a calculation with phonopy:
\begin{itemize}
  \item On a convenient machine, e.g., your desktop, prepare an input
    directory with the usual \texttt{control.in} and
    \texttt{geometry.in} files. You should only specify the primitive
    cell in \texttt{geometry.in}, all the rest may be handled by the
    \keyword{phonon} keyword in \texttt{control.in} .
  \item Change to that directory and run the \texttt{phonopy-FHI-aims}
    command. This creates the supercell input geometry files with
    displacements for the force calculations in FHI-aims.
  \item Copy over the supercell input geometries and the
    \texttt{control.in} file to your favorite production computer and
    run FHI-aims. Do not forget to specify a $k$-grid that is
    appropriate for the \text{supercell} -- else, if you specify an
    unnecessarily dense $k$-grid, you will pay a high price. Run the
    individual FHI-aims calculations.
  \item Copy the FHI-aims output files for each displacement back to
    the first machine. Run \texttt{phonopy-FHI-aims} again. This
    should produce the output you were looking for.
\end{itemize}

To use the \texttt{phonopy-FHI-aims} infrastructure,
\begin{itemize}
\item[] python ($>=2.5$) http://www.python.org
\item[] numpy http://numpy.scipy.org
\item[] and a C compiler, e.g., gcc 
\end{itemize}
are required. The necessary symmetry analysis routines, wich are
taken from {\it spglib} (https://atztogo.github.io/spglib/), are already
included in the {\it phonopy} package. To obtain plots via the integrated 
plotting functionality,
\begin{itemize}
\item[] matplotlib http://matplotlib.sourceforge.net/
\end{itemize}
has to be installed. The latter dependency is, however, not compulsory
for running a calculation. 

Since \texttt{phonopy-FHI-aims} completely decouples FHI-aims
calculations and phonon pre- and post-processing operations, all the
necessary packages can easily be installed via the package management
system of a normal Linux desktop machine, e.g., for Ubuntu linux:\\  
{\it apt-get install python python-numpy python-matplotlib}

\textbf{The details:}

\texttt{phonopy-FHI-aims} supports the \keyword{phonon} sub-keywords
and output styles discussed above. When called from the command 
line in a directory containing a correct \option{geometry.in} and
\option{control.in} file, it generates $n=1,..,N$ subdirectories \\ 
{\it
  phonopy-FHI-aims-displacement-n}. \\ 
Each of these directories contains
correct input files for one specific displacement. The calculations
for the generated inputs need to be done manually (and concurrently if
desired) on any available computing machinery. The output has
to be stored in the respective subdirectory in a file named {\it
  basename.out}, e.g., {\it
  phonopy-FHI-aims-displacement-01/phonopy-FHI-aims-displacement-01.out}. 

Once the FHI-aims output files are available (at the right places), rerunning 
\texttt{phonopy-FHI-aims} in the same directory as before will produce
the output (band structure, DOS, thermodynamic properties) which has been
requested in control.in. The calculated data is written to ASCII files
which include comments describing their contents. Some plots are also
generated in .pdf files if matplotlib is available. 

Additionally, \texttt{phonopy-FHI-aims} supports some
advanced features 
that are still experimental at this point:
\begin{itemize}
\item Corrections for the long-range electrostatic fields in ionic crystals
  ($\Gamma$ point splitting of optical modes) can be accounted
  for. The required Born effective charges will be available from
  FHI-aims soon, but not yet as of this writing (version
  081213). Please check back with us if needed.
\item The force constants for the whole supercell can be written out (a
  prerequisite to use the thermodynamic integration routines for anharmonic
  free energy calculations).
\end{itemize}
Additional information can be found in the README text file provided at
http://www.fhi-berlin.mpg.de/aims/utilities.html .


The usual \texttt{control.in} and \texttt{geometry.in} files are read
by both \texttt{phonopy-FHI-aims} and \texttt{FHI-aims} itself. 

\subsection*{The phonon keyword}

We finally document the \keyword{phonon} keyword that is part of the
usual input file \texttt{control.in}. 

\keydefinition{phonon}{control.in}
{
  Usage: \keyword{phonon} \option{[subkeywords and their options]}\\[1.0em]
  Purpose: Main driver keyword for the phonon calculation script. has
  the subkeywords \subkeyword{phonon}{supercell},
  \subkeyword{phonon}{band}, \subkeyword{phonon}{dos},
  \subkeyword{phonon}{displacement}, as detailed below.\\ }

\subkeydefinition{phonon}{band}{control.in}
{ Usage: \keyword{phonon} \subkeyword{phonon}{band} \option{kstart1
    kstart2 kstart3 kend1 kend2 kend3 npoints startname endname} \\[1.0em]
  Purpose: To define a phonon band to be plotted explicitly. The
  remaining syntax is kept the same as for the \keyword{output} option
  \subkeyword{output}{band}: the first six parameters describe the
  starting and ending point in RELATIVE \emph{k}-coordinates,
  \option{npoints} is the number of points in the band, followed by
  the names of the special points for starting and ending the band. \\}

\subkeydefinition{phonon}{displacement}{control.in}
{ Usage: \keyword{phonon} \subkeyword{phonon}{displacement}
  \option{delta} \\[1.0em]
  Purpose: To define the absolute displacement of each atom for the
  finite difference calculation. \\[1.0em]
  Default: 0.001 \AA\\}

\subkeydefinition{phonon}{dos}{control.in}
{ Usage: \keyword{phonon} \subkeyword{phonon}{dos} \option{fstart fend
  fpoints broad qdensity}\\[1.0em]
  Purpose: governs the output of the phonon density of states. \\ }
This keyword defines the DOS output along a frequency
axis in THz. \option{fstart}, \option{fend}, and \option{broad} give
the starting and ending frequency as well as the broadening in units
of THz, while \option{fpoints} is the number of points. Unlike the
output option \subkeyword{output}{dos}, here we also have to specify a
$k$-point density \option{qdensity}, which does the BZ integration on
a \option{qdensity}$\times$\option{qdensity}$\times$\option{qdensity}
grid. Normally, the dynamic matrix only contains a very small number
of elements (3 in the example described below), which means that one
can chose \option{qdensity} quite high in order to get a
well-converged density of states. Notice that this value might be changed by the program
if the DOS and $c_v$ calculations are requested at the same time. In
that case, the higher of the two values for \option{qdensity}
specified for the two calculations is chosen in order to minimize the
computational effort. See also the \subkeyword{phonon}{free\_energy}
description. 

\subkeydefinition{phonon}{free\_energy}{control.in}
{ Usage: \keyword{phonon} \subkeyword{phonon}{free\_energy} \option{Tstart Tend
  Tpoints qdensity}\\[1.0em]
  Purpose: governs the calculation of the phonon free energy, the
  internal energy, and the specific heat for a given unit cell. \\ } 
The phonon free energy and some derived quantities can be requested
with this keyword. They are 
calculated between the temperatures \option{Tstart} and \option{Tend}
with a total of \option{Tpoints} steps. A $k$-point density \option{qdensity} is
also required, as the calculation involves an integration over the
Brillouin zone. Notice that this value might be changed by the program
if the DOS and free energy calculations are requested at the same
time. In that case, the higher of the two values for \option{qdensity}
specified for the two calculations is chosen in order to minimize the
computational effort. See also the \subkeyword{phonon}{dos}
description. 

Please refer to phonopy for the meaning of the exact quantities that
are written out.

\subkeydefinition{phonon}{frequency\_unit}{control.in}
{ Usage: \keyword{phonon} \subkeyword{phonon}{frequency\_unit} \option{unit}\\[1.0em]
  Purpose: Allows to specify the frequency output for the phonons by
  setting \option{unit} to either \option{cm\^{}-1} or to
  \option{THz}. Default is \option{cm\^{}-1} \\}

\subkeydefinition{phonon}{supercell}{control.in}
{ Usage: \keyword{phonon} \subkeyword{phonon}{supercell} \option{n1}
  \option{n2} \option{n3}
  \\[1.0em]
  Purpose: To define a $n1\times n2\times n3$ supercell in which to
  approximate the calculation of the dynamic matrix, based on the
  lattice vectors provided in \option{geometry.in}. }
The supercell keyword can also be used to prescribe a general (rotated)
supercell in matrix notation. For instance,

\texttt{phonon supercell -1 1 1  1 -1 1  1 1 -1}

corresponds to a supercell matrix
\begin{equation*}
\begin{pmatrix}  
-1 & 1 & 1 \\  1 & -1 & 1 \\  1 & 1 & -1 
\end{pmatrix}  \, .
\end{equation*}

\subkeydefinition{phonon}{symmetry\_thresh}{control.in}
{Usage: \keyword{phonon} \subkeyword{phonon}{symmetry\_thresh}
  \option{thresh} \\[1.0em]
Purpose: To define the maximally allowed coordinate difference (in
\AA) for the phonon symmetry checker to decide that two atoms are at
the same location. Default: 10$^{-6}$\\}
This keyword should not have to be changed from its default.\\


