\section{Kinetic energy, scalar relativity, and spin-orbit coupling}
\label{Sec:rel}

For elements beyond approximately $Z$=30, relativistic effects near
the nucleus cannot be neglected in an all-electron treatment---both
for core, and for valence electrons. For the purposes of ``everyday'' matter,
the full theory is given by Dirac's four-component Equation, but in the
``practice'' of materials physics and chemistry, we still tend to think in
terms of Schr\"odinger-like objects. A true four-component treatment is not
implemented in FHI-aims, but the following standard levels of approximation
are available: 
\begin{itemize}
  \item Non-relativistic kinetic energy (one or two collinear \keyword{spin}
    components of the Kohn-Sham orbitals)
  \item Scalar-relativistic kinetic energy expression (one or two collinear
    spin components). 
  \item Perturbative spin-orbit coupling, a single correction step to the
    Kohn-Sham eigenvalues based on the Kohn-Sham orbitals from a
    non-relativistic or scalar-relativistic s.c.f. cycle. \emph{Perturbative
      spin-orbit coupling in FHI-aims is a new feature. Importantly,
      changes to the total  
      energy beyond the sum-of-eigenvalues are not included, and total energy 
      gradients (forces) are also unavailable.}
\end{itemize}

\subsection*{Scalar relativity and spin-orbit coupling}

While the non-relativistic level of theory is exactly
defined and will be the same in any first-principles implementation (at a
complete basis set, all-electron level anyway), there are many different
versions of scalar-relativistic approximations which can yield
considerably different \emph{total} energies for different systems. Their
unifying feature is that any two scalar-relativistic methods should still
yield the same energy \emph{differences} for properties that concern
valence electrons: Binding energies, valence eigenvalues, etc. 

The recommended level of scalar relativity in FHI-aims is the
so-called ``atomic ZORA'' approximation, as defined specifically in
Equations (55) and (56) of Ref. \cite{Blum08}. It is important to
refer to this specific definition since there are other variants of
ZORA (``zero-order regular approximation'') in the literature and in
other codes, including variants also called ``atomic ZORA'' but
following a different mathematical definition. 

The keyword needed to use this level of theory is

    \keyword{relativistic} atomic\_zora scalar

That's it.

The ``atomic ZORA'' level of theory as implemented in FHI-aims
has held up extremely well in large, high-accuracy benchmarks of
scalar-relativistic total-energy based properties
\cite{Lejaeghereaad3000} as well as energy band structures
\cite{Huhn2017_SOC}. It works for the right mathematical reasons. It
can, in principle, be used across the entire periodic table (there
should be no need to resort to non-relativistic calculations except
for benchmarking purposes). 

In addition, a non-selfconsistent treatment of spin-orbit coupling for
band structures, densities of states, absorption properties and for the
independent-particle dielectric response is also available and can be
used in addition to (on top of) scalar relativistic calculations using
the atomic ZORA. This is described in detail in
Ref. \cite{Huhn2017_SOC}, including a simple discussion of relativistic
treatments in general and of how the ``atomic ZORA'' and the
spin-orbit coupling formalism on top of it are related.

The keyword to add post-scf spin-orbit coupling is

\keyword{include\_spin\_orbit}

That's it.

More details follow below, but here are three additional important
point:
\begin{itemize}
  \item Never mix results from different scalar relativistic
    treatments in total-energy differences. Absolute total-energy
    differences between different relativistic treatments can be very
    large because the deep core state energies change.
  \item The absolute core level energies in the ``atomic ZORA''
    approximation are far away from measured core level energies that
    would appear in experiment or in the actual Dirac
    equation. However, the relative core level shifts (differences)
    between different chemical systems are still reliable.
 \item FHI-aims also includes another relativistic treatment called
   ``scaled ZORA'' but this seems to be slightly less accurate and
   does not have support for forces or stresses or any other use
   cases. We do not recommend to use ``scaled ZORA'' any more
   (``atomic ZORA'' simply seems to do the better job).
\end{itemize}

\subsection*{More details on spin-orbit coupling}

Spin-orbit coupling (SOC) is a simple consequence of transforming Dirac's
equation to a (two-component) Schr\"odinger-like form. This leads to
an approximate ``spin-orbit-coupled'' Hamiltonian of the form
\begin{equation}
  \hat{H} = \hat{t}_{SR} + \hat{v} + \hat{v}_{SOC},
\end{equation}
where $\hat{t}_{SR}$ is the usual scalar relativistic kinetic energy
operator (e.g., atomic ZORA), $\hat{v}$ is the local or non-local
potential, and $\hat{v}_{SOC}$ is the spin-orbit coupling operator,
\begin{equation}
	v_{SOC}=\frac{i}{4c^{2}}\bm{\sigma} \cdot \bm{p} v \times \bm{p}.
\end{equation}

FHI-aims currently implements a treatment of spin-orbit coupling which 
adds spin-orbit coupling corrections to the Kohn-Sham eigenvalues, band 
structures, and densities of states in a single evaluation \emph{after} 
the scalar-relativistic s.c.f. cycle has converged.  This means that it 
is a post-processed implementation of spin-orbit coupling.  It works in 
the Hilbert space of calculated scalar-relativistic eigenstates, as 
opposed to the ``full'' space spanned by the computational basis set, 
to dramatically reduce the problem size.  This is known as the 
``second-variational'' method.  It only calculates and diagonalizes the 
spin-orbit-coupled Hamiltonian once;  therefore, the resulting 
spin-orbit-coupled eigenstates are non-self-consistent.

Applying the SOC operator as a correction to scalar-relativistic
eigenvectors is quantitatively accurate (to a few 0.01 eV for valence band
structures) for elements below Xe ($Z$=54) when combined with atomic ZORA.
\emph{For heavy elements (approximately Au and beyond) this level of
theory is only qualitatively accurate.  It captures the majority of the 
SOC effect, but quantitative deviations above 0.1 eV for band structures 
must be expected. Similarly, the corrections for any core levels would 
require one to go beyond non-self-consistent SOC.}

Since this implementation of spin-orbit coupling operates in the Hilbert 
space spanned by the calculated scalar-relativistic eigenvectors, for 
accurate high-lying bands one must include sufficiently many unoccupied 
states.  This may be done by setting \keyword{empty\_states} to a higher
value or, if you are feeling particularly paranoid, setting the \\ 
\keyword{calculate\_all\_eigenstates} keyword to include all possible 
eigenstates.  It is the opinion of the authors that this is only relevant 
for materials containing Au and heavier elements.

Full details on the implementation of spin-orbit coupling in FHI-aims, 
as well as a derivation of the spin-orbit-coupled Hamiltonian from the
Dirac equation and a detailed benchmark of the effect of spin-orbit 
coupling on band structures, may be found in Ref.~\cite{Huhn2017_SOC}  
When publishing results using spin-orbit coupling in FHI-aims, please 
remember to cite this reference.

\subsubsection{Which Parts of FHI-aims Support Spin-Orbit Coupling?}

The spin-orbit coupling implementation started in 2014.  FHI-aims has been 
in development since 2003.  While we are actively working on wrapping
spin-orbit coupling's tentacles around as much of FHI-aims as possible,
due to the sheer size of the code base much of the functionality in
FHI-aims does not have spin-orbit coupling support.  Enabling 
the SOC keyword will do nothing for functionality that has not been
modified to support SOC, and the code will return scalar-relativistic 
values.  A \textbf{partial} list of functionality that does support SOC 
and will output spin-orbit-coupled values is
\begin{itemize}
\item Band structure calculations
\item Densities of state calculations, both interpolated and 
	non-interpolated
\item Mulliken analyses
\item Atom/species-projected densities of state
\item Dielectric functions and absoprtion coefficients
\item Orbital cube plotting
\end{itemize}
The best way to determine whether a particular method supports
spin-orbit coupling is to look at its manual entry.

One advantage of post-processed SOC is that one still has access to 
scalar-relativistic values, as the spin-orbit-coupled values are 
generated from the scalar-relativistic values.  Physical insight 
may be gained by comparing scalar-relativistic and spin-orbit-coupled 
values against one another.  For example, strong spin-orbit splitting 
of eigenstate is a dead giveaway that it contains p-orbitals for a 
heavy species.  When spin-orbit coupling is enabled, FHI-aims will output 
both scalar-relativistic and spin-orbit-coupled values whenever this is 
computationally feasible.  For methods supporting spin-orbit coupling 
that output results to files, the files containing the scalar-relativistic 
values will have an additional suffix ".no\_soc" to distinguish them from 
the spin-orbit-coupled values.

\newpage

\subsection*{Tags for general section of \texttt{control.in}:}

\keydefinition{include\_spin\_orbit}{control.in}
{
  \noindent
  Usage: \keyword{include\_spin\_orbit} \option{method} \\[1.0ex]
  Purpose:  Include the effects of spin-orbit coupling, when supported, in 
  post-processed features of FHI-aims.  When using spin-orbit coupling in 
  your calculation, please cite Ref.~\cite{Huhn2017_SOC} \\[1.0ex] 
  \option{method} The method for including spin-orbit coupling.  At present, 
  only \option{type} \option{non\_self\_consistent} is suitable for 
  production-level calculations. \\[1.0ex] 
}

\emph{Note: While FHI-aims also prints out a corrected total-energy
  expression based on the SOC-corrected eigenvalues, do not use this 
  value. It is experimental.} 

\keydefinition{compute\_kinetic}{control.in}  
{
  \noindent
  Usage: \keyword{compute\_kinetic} \\[1.0ex]
  Purpose: \emph{Experimental} - for test purposes, allows to compute the kinetic energy via
    the product of the kinetic energy matrix and the density matrix
    \\[1.0ex]
}
This flag is presently kept for test purposes only (the electronic
kinetic energy is separately computed and printed for each scf
iteration anyway) but may be useful for some future modifications.

\keydefinition{override\_relativity}{control.in}
{
  \noindent
  Usage: \keyword{override\_relativity} \option{flag} \\[1.0ex]
  Purpose: If explicitly set, allows to override the stop enforced by
  the code when physically questionable \keyword{relativistic}
  settings are used.  \\[1.0ex]
  \option{flag} is a logical expression, either \texttt{.true.} or
    \texttt{.false.} Default: \texttt{.false.} \\
}
For example, this will allow you to run a physically incorrect
calculation of heavy elements (think Au) with Schr\"odinger's
expression for the kinetic energy, instead of a scalar relativistic
treatment. The results will be wrong, so this flag should only be set
for test purposes. When set, the code assumes that the user must know
what they are doing.

\keydefinition{relativistic}{control.in}
{
  \noindent
  Usage: \keyword{relativistic} \option{r-type} \option{s-type}
    [\option {threshold}] \\[1.0ex]
  Purpose: Specifies the level of relativistic treatment in the
    calculation. \\[1.0ex]
  \option{r-type} is a string, specifying the basic approximation
    made. \\
  \option{s-type} is a string, specifying whether a scalar treatment
    is desired (currently, only the \texttt{scalar} option is
    supported). \\ 
  \option{threshold} is a small positive real number, allowing to
    reduce some integration effort. \\
}
Detailed expressions for the scalar relativistic treatments available
here are given in Ref. \cite{Blum08}. We here only repeat the salient
options and expressions. Possible options for \option{r-type} are:
\begin{itemize}
  \item \texttt{none} : Non-relativistic kinetic energy. In this case,
        \option{s-type} and \option{threshold} need not be provided. 
  \item \texttt{atomic\_zora} : Atomic ZORA approximation as described
    in Ref. \cite{Blum08}. \option{threshold} need not be provided. 
    This is the currently recommended option for energy differences and valence 
    and unoccupied eigenvalues.
  \item \texttt{zora} The ZORA approximation is used throughout the
    s.c.f. cycle, followed by a ``scaled ZORA'' \cite{Lenthe94} post-processing step
    (rescaling of all eigenvalues). \textbf{WARNING:} Do not rely on
    intermediate, simple ZORA total energies, but only on the final,
    rescaled total energies instead! ZORA (unscaled) is not the same as 
    ``atomic ZORA'' and cannot be trusted. We also no longer recommend
    scaled ZORA values since there is no clear advantage. Just use
    \texttt{atomic\_zora} unless there is need to do otherwise.
\end{itemize}
Forces are only provided for \texttt{none} and \texttt{atomic\_zora}.

\textbf{Remember to never take energy differences between calculations
  performed with different ``relativistic'' settings.}

We recommend to simply use \texttt{atomic\_zora} for all calculations,
unless there is a particular need to stay with \keyword{relativistic}
\texttt{none}. 

If you really do want to use scaled ZORA (the case of \texttt{zora}
keyword), the \option{threshold} option is required.
It specifies the threshold value above which the difference between
the sum-of-free-atoms ZORA expression and that for the actual
potential during the s.c.f. cycle will be calculated. In areas of
shallow potentials, where both expressions are substantially similar,
this saves the extra integration effort associated with ZORA. For
(very!) safe settings, \option{threshold} may be set to 10$^{-12}$; in
our experience, also 10$^{-9}$ does not lead to any noticeable
accuracy loss. 

Default is \texttt{none} if all elements in the structure have $Z<$20
(no heavier than Ca). For all heavier elements, an explicit
\keyword{relativistic} setting is required. For 11$<Z<$20, the setting
\texttt{none} will be accepted, but a
warning will be issued. For $Z$>20, choosing \texttt{none} will cause
the code to stop with an error message in order to avoid accidental
calculations with incorrect relativity. If a non-relativistic
calculation is still desired, for example for test purposes, this
``stop'' can be disabled by setting the flag
\keyword{override\_relativity}---but use this only if you know what
you are doing.

Future version of FHI-aims will simply employ \texttt{atomic\_zora} as
the default level of relativity.
