\section{Electrostatic (Hartree) potential}
\label{Sec:Hartree}

This section describes the method used to compute the
electrostatic (Hartree) potential in FHI-aims. For a more exhaustive
description, please refer to Ref. \cite{Blum08}. 

Some central equations are repeated here in detail since, as a result, the
calculation of the Hartree potential can be heavily customized by many
analytically available accuracy / cutoff thresholds, given below.

For production calculations, it is important to note that \emph{our standard
accuracy thresholds in the Hartree potential are numerically sound,
and usually do not require an explicit customization}. The only
parameter which should be explicitly set is the angular momentum up to
which the atom-centered partitioned charge density is expanded,
\subkeyword{species}{l\_hartree} below. 

As pointed out in
Ref. \cite{Blum08}, our experience is that energy differences are
usually well converged for $l_\text{hartree}$=4, and total energy
convergence at the level of a few meV is reached at
$l_\text{hartree}$=6. Only in exceptional cases should different
settings be required. 

---

At the beginning of a calculation, we first compute the electrostatic
potential associated with the initial superposition of free-atom
densities, $\sum_\text{at}
n_\text{at}^\text{free}(|\boldr-\boldR_\text{at}|)$:
\begin{equation}
  v^\text{es,free}(\boldr)= \sum_\text{at}
  v_\text{at}^\text{es,free}(|\boldr-\boldR_\text{at}|) 
\end{equation}
This is the largest part of the Hartree potential, but is always
accurately known from the solution of spherical free atoms on a dense
logarithmic grid.

For a given electron density $n(r)$ \emph{during} the s.c.f. cycle, we
then only ever compute the electrostatic potential associated with the
\emph{difference} to the superposition of free atoms, $\delta
v_\text{es}(\boldr)$, based on
\begin{equation}
  \delta n(\boldr) = n(r) - \sum_\text{at}
  n_\text{at}^\text{free}(|\boldr-\boldR_\text{at}|)
\end{equation}

$\delta n(\boldr)$ is first split up into a sum of \emph{partitioned,
  atom-centered} charge multipoles,
\begin{equation}
  \label{Eq:mp}
  \delta\tilde{n}_{\text{at},lm}(r) =
  \int_{r=|\boldr-\boldR_\text{at}|} d^2\Omega_\text{at}
  p_\text{at}(\boldr)\cdot\delta n(\boldr)\cdot Y_{lm}(\Omega_\text{at})
\end{equation}
(the sum of all partition functions at every point is always
unity). Due to the finite extent of $\delta n(\boldr)$ and
$p_\text{at}(\boldr)$ (both are controlled by the
\subkeyword{species}{cut\_free\_atom} keyword), the range of each
component $\delta\tilde{n}_{\text{at},lm}(r)$ is also bounded.

The Hartree potential components $\delta\tilde{v}_{\text{at},lm}(r)$
are then determined on a dense, one-dimensional logarithmic grid,
using classical electrostatics. The resulting
$\delta\tilde{v}_{\text{at},lm}(r)$ are then numerically tabulated,
and evaluated elsewhere using cubic spline interpolation.

For \emph{cluster} systems, it is important to note that the finite
extent of $\delta\tilde{n}_{\text{at},lm}(r)$ implies that the
numerically tabulated part of $\delta\tilde{v}_{\text{at},lm}(r)$ can
also be kept finite. Outside this ``multipole radius'',
$\delta\tilde{n}_{\text{at},lm}(r)$=0, and
$\delta\tilde{v}_{\text{at},lm}(r)$ falls off analytically as 
\begin{equation}\label{Eq:moments}
  \delta\tilde{v}_{\text{at},lm}(r) = m_{\text{at},lm}/r^{l+1} \quad .
\end{equation}
Instead of a spline evaluation, faraway atoms can thus be analytically 
accounted for using tabulated, constant \emph{multipole moments}
$m_{\text{at},lm}$. High-$l$ components can be analytically cut off as they
approach zero at large distances. 

In this approach, the effort to create the complete Hartree potential
on the entire grid is determined by tabulating the contribution from
\emph{every} atom on \emph{every} grid point,
\begin{equation}\label{Eq:ves-assembly}
  v_\text{es}(\boldr) = v^\text{es,free}(\boldr) + \sum_{\text{at},lm}
  \delta\tilde{v}_{\text{at},lm}(|\boldr-\boldR_\text{at}|)
  Y_{lm}(\Omega_\text{at}) \quad .
\end{equation}
The scaling is thus close to $O(N^2)$ with system size, albeit
reduced by high-$l$ multipole components falling off towards large
distances.

For \emph{periodic} systems, essentially the same equations hold,
except that the Hartree potentials associated with the atom-centered 
charge densities $\delta\tilde{n}_{\text{at},lm}(r)$ are here
additionally split into a short-ranged real-space part, and a smooth,
long-ranged reciprocal-space part (Ewald's method), by splitting
\begin{equation}\label{Eq:Ewald}
  \frac{1}{r} = \frac{\erf(r/r_0)+\erfc(r/r_0)}{r} 
\end{equation}
(and similar for components of higher angular momentum).
The summation of long-range tails thus happens in reciprocal space,
using Fourier transforms. As a result, the scaling of this effort is
no longer $O(N^2)$, but rather approaches $O(N \mathrm{ln} N)$ in
Fourier transforms.

The parameter $r_0$ can be very important to determine the efficiency of the 
actual evaluation of the Hartree potential in periodic systems; it can be 
set in \texttt{control.in} using the
\keyword{Ewald\_radius} keyword. The keyword is
adaptive to some extent but especially for slab systems or 2D systems
with large vacuum regions, specifying the value of
\keyword{Ewald\_radius} by hand can lead to
significant performance improvements. (FHI-aims can accommodate very
large vacuum regions, e.g., 100 {\AA}, efficiently if this parameter is
set correctly.) 

The cutoff reciprocal space momentum for the Fourier part of the
electrostatic potential, $|\boldG_\text{max}|$, is estimated using a
small threshold parameter $\eta$:
\begin{equation}\label{Eq:Fourier}
  |\boldG_\text{max}|^{l_\text{max}^\text{es}-2} \cdot
  \frac{1}{\boldG_\text{max}^2} \cdot \exp(-\frac{r_0^2
  \boldG_\text{max}^2}{4}) = \frac{\eta}{10\cdot 4\pi} \, .
\end{equation}
Our default choice for $\eta$ (in atomic units, i.e., those used internally in
the code) is $\eta=5\cdot 10^{-7}$, but this is somewhat overconverged, and a
larger threshold value is probably sufficient for most situations. \emph{Note}
that Eq. (\ref{Eq:Fourier}) is slightly modified compared to the version given
in Ref. \cite{Blum08}.



\subsection{Non-periodic Ewald method}
\label{Sec:Hartree-non-periodic-ewald}

For large, finite systems (more than 200 atoms) it is possible to use the
so-called `non-periodic Ewald method' in aims (keyword
\keyword{use\_hartree\_non\_periodic\_ewald}). The basic idea of this method is
to use interpolation to reduce the effort for calculting the Hartree term.
Specifically, the method consists in computing the electrostatic potential not
on the fine interpolation grid points but firstly on a coarse Cartesian grid.
Subsequently, the values of the potential on the coarse grid are interpolated to
the fine integration grid. If the Cartesian grid is suffiently coarse, time is
saved because of the reduced number of potential computations.

We use an envenly spaced, Cartesian grid with a certain grid width. Due to this
fixed grid width, special attention has to be paid to the near-atom regions
where the electron density and hence also the potential oscillates strongly.
This problem can be solved by using the Ewald decomposition which was originally
developed for periodic systems. Ewald's method aims at separating large and
small scales by adding and subtracting charge spheres with Gaussian radial shape
to a lattice of monopoles. In terms of the potential, this yields $\bar{q}/r = [
  \bar{q}/r - \Omega(r) ] + \Omega(r)$ for each monopole, where $\bar{q} := q/(4
\pi \epsilon_0)$ and $q$ is the monopole charge. The function $\Omega(r) =
\bar{q}\erf(r/r_0)/r$ is the potential of a Gaussian charge sphere with width
parameter $r_0$. The first part of the decomposition $\bar{q}/r - \Omega(r)$
decays quickly with increasing $r$ so that this part is calculated in real
space, while the second part $\Omega(r)$ decays quickly in Fourier space so that
it is calculated there. The two parts are often referred to as `short range' and
`long range' part. However, this is somewhat misleading because the second part
is actually defined in whole space. For this reason, we call the first part
`localized' and the second part `extended'.

We can translate the classical Ewald decomposition to our case of a finite
system by calculating the smooth extended part $\Omega(r)$ on the coarse
Cartesian grid, with subsequent interpolation to the fine integration grid
points. In addition, we have to calculate the localized part in the vicinity of
the nuclei where we cannot save any computational time [actually some time is
lost since we have to compute $\Omega(r)$ there, too].

In the classical Ewald method, Gaussian spheres are an excellent choice as
auxiliary charges due to the quick convergence of both the localized part in
real space and the extended part in Fourier space. However in our case, where we
interpolate in real space, Gaussian spheres are not necessarily a proper choice.
Therefore optimized charge distributions obtained from a variational method by
W. J\"urgens are used.

In order to reduce the number of grid points, we allow the Cartesian grid to
have arbitrary orientation. More specifically, we are looking for a rectangular
cuboid that covers all integration grid points but with minimum volume. This
problem is solved approximately by using a common procedure that is based on
principle component analysis.



\newpage



\subsection*{Tags for general section of \texttt{control.in}:}

\keydefinition{adaptive\_hartree\_radius\_th}{control.in}
{
  \noindent
  Usage: \keyword{adaptive\_hartree\_radius\_th} \option{threshold} \\[1.0ex]
  Purpose: Determines the distance beyond which an analytical
    component $\delta\tilde{v}_{\text{at},lm}(r)$ of the
    \emph{periodic} (Ewald!) real-space Hartree 
    potential for a given atom is considered zero. \\[1.0ex]
  \option{threshold} is a small positive real number. Default:
    10$^{-8}$. \\
}
Usually, this tag need not be modified from the default. Long-range
multipole components $\delta\tilde{v}_{\text{at},lm}(r)$ of the
real-space (Ewald!) Hartree potential are not evaluated for distances
where $\delta\tilde{v}_{\text{at},lm}(r)<$\option{threshold}. This tag
provides similar functionality as the \keyword{multipole\_threshold} tag
for the cluster case (numerically different due to the absence of
$\erf(r/r_0)$ in the cluster case).

\keydefinition{compensate\_multipole\_errors}{control.in}
{
  \noindent
  Usage: \keyword{compensate\_multipole\_errors} \option{flag} \\[1.0ex]
  Purpose: If true, introduces a compensating normalization and 
    density to eliminate the effects of small charge integration errors 
    in the long-range Hartree potential. \\[1.0ex]
  \option{flag} is either \texttt{.true.} or \texttt{.false.}. Default:
    \texttt{.true.}. \texttt{.false.} only if a DFPT calculation (this includes:
    \keyword{calculate\_friction} and \keyword{magnetic\_response}) is requested. \\
}
This keyword is especially useful when assessing the electrostatic potential
far away from a structure, e.g., when calculating a surface dipole correction 
(for asymmetric slabs) or work function. 
See \keyword{use\_dipole\_correction} or \keyword{evaluate\_work\_function} for
details on these methods.

In general, keyword \keyword{compensate\_multipole\_errors} makes sure that the
long-range charge components of the Hartree potential are exactly those expected 
from the calculated (and normalized) electron density. Any small spurious non-zero
components that are solely due to integration errors on a finite integration grid.

%It may indeed be reasonable to use this keyword as the default in any 
%calculation, especially with ``light'' integration grids. It is not the
%general default because of a single test case where the harmonic vibrational 
%frequencies of a large molecule deteriorated slightly compared to the
%standard (uncompensated) case. However, in general, total energies with 
%\keyword{compensate\_multipole\_errors} appear to be slightly more accurate
%on balance than without. 

\keydefinition{Ewald\_radius}{control.in}
{
  \noindent
  Usage: \keyword{Ewald\_radius} \option{value}
    \\[1.0ex]
  Purpose: Governs the Ewald-type short-range / long-range splitting
    of the Coulomb potential in Eq. (\ref{Eq:Ewald}). \\[1.0ex]
  \option{value} : Either a string \texttt{automatic}, or the 
    range separation parameter $r_0$ in Eq. (\ref{Eq:Ewald})
    (in bohr). Default: \texttt{automatic}. \\
}
May also be specified as \keyword{hartree\_convergence\_parameter} or
\keyword{ewald\_radius}.

Necessary for periodic boundary conditions only. May be changed from
the default, but should not be set too small or too large (the compensating
Gaussian charge density of the Ewald method must cancel the actual charge
outside a radius that is still inside the partition table / integration 
grid for every atom.) 

This parameter is performance critical especially for slab
calculations (2D material or surface) with a large vacuum region.

If the string \texttt{automatic} is chosen, then the parameter $r_0$ is
set according to an empirically determined function as follows:
\begin{equation}
  r_0 = A_0 \cdot (v - A_1)^{1/3} \quad ,
\end{equation}
subject to the limiting conditions 2.5~bohr$\le r_0 \le$5.0~bohr. 
Here, $v$ is the specific volume (unit cell volume divided by number of atoms),
and $A_0$=1.47941~bohr and $A_1$=1.85873 {\AA}$^3$ are empirically determined
parameters. 

The chosen empirical form was tested and adapted for a slab model of a
2D material with a vacuum region up to 50 {\AA}. For such systems,
this choice entails a significant performance improvement; and for
larger vacuum regions, even larger choices than $r_0$=5.0~bohr are
possible. However, the same empirical relation may not be optimal for
moderately dense solids (such as GaAs), where smaller choices of
$r_0$ can perform better. Overall, the optimum choice of $r_0$ would
be to adapt it on the fly over the course of a given calculation, but
implementing such an adaptive algorithm has not yet been done.

For a yet more
refined choice, further testing would be necessary, as well as a dependence on
\keyword{hartree\_fourier\_part\_th} (which is not yet incorporated).

\keydefinition{ewald\_radius}{control.in}
{
  \noindent
  Usage: \keyword{ewald\_radius} \option{value}
    \\[1.0ex]
  Purpose: Governs the Ewald-type short-range / long-range splitting
    of the Coulomb potential in Eq. (\ref{Eq:Ewald}). \\[1.0ex]
  \option{value} : Either a string \texttt{automatic}, or the 
    range separation parameter $r_0$ in Eq. (\ref{Eq:Ewald})
    (in bohr). Default: \texttt{automatic}. \\
}
This keyword has exactly the same meaning as the \keyword{Ewald\_radius} kewyord.

\keydefinition{hartree\_convergence\_parameter}{control.in}
{
  \noindent
  Usage: \keyword{hartree\_convergence\_parameter} \option{value}
    \\[1.0ex]
  Purpose: Governs the Ewald-type short-range / long-range splitting
    of the Coulomb potential in Eq. (\ref{Eq:Ewald}). \\[1.0ex]
  \option{value} : Either a string \texttt{automatic}, or the 
    range separation parameter $r_0$ in Eq. (\ref{Eq:Ewald})
    (in bohr). Default: \texttt{automatic}. \\
}
This keyword has exactly the same meaning as the \keyword{Ewald\_radius} kewyord.

\keydefinition{hartree\_fp\_function\_splines}{control.in}
{
  \noindent
  Usage: \keyword{hartree\_fp\_function\_splines} \option{.true. /
    .false.} \\[1.0ex]
  Purpose: Switches on the splining of the Greens functions for the
  long-range Hartree multipole decomposition in periodic boundary
  conditions. This accelerates the calculation of the Hartree
  potential in large unit cells. \\[1.0ex]
  Default: \option{.true.}
}

\keydefinition{hartree\_fourier\_part\_th}{control.in}
{
  \noindent
  Usage: \keyword{hartree\_fourier\_part\_th}
    \option{threshold} \\[1.0ex]
  Purpose: \emph{Implicitly} determines the required reciprocal space
    cutoff momentum $|\boldG_\text{max}|$ for the Fourier summation of
    the long-range electrostatic potential (Ewald). \\[1.0ex]
  \option{threshold} is a real positive small number [$\eta$ in
    Eq. (\ref{Eq:Fourier})]. Default: 5$\cdot$10$^{-7}$ (in atomic
    units) . \\
}
See Eq. (\ref{Eq:Fourier}). Usually, this tag need not be modified from the
default. Necessary for periodic boundary conditions only.  

\keydefinition{hartree\_partition\_type}{control.in}
{
  \noindent
  Usage: \keyword{hartree\_partition\_type}   \option{type} \\[1.0ex]
  Purpose: Specifies which kind of partition function
    $p_\text{at}(\boldr)$ is used to split $\delta n(\boldr)$ into
    atom-centered pieces. \\[1.0ex] 
  Restriction: Presently, \option{type} should have the same
    value as specified for integration using the
  \keyword{partition\_type} keyword. \\[1.0ex]
  \option{type} : A string that specifies which kind of partition
    table is used. Default: \texttt{stratmann\_sparse} \\ 
}
Usually, this tag need not be modified from the default. The same
options are available as for the \keyword{partition\_type} keyword
(partition functions for three-dimensional integrands). See
  \keyword{partition\_type} for details. 

\keydefinition{hartree\_radius\_threshold}{control.in}
{
  \noindent
  Usage: \keyword{hartree\_radius\_threshold} \option{threshold} \\[1.0ex]
  Purpose: Technical criterion to ensure the inclusion of atoms with a
    potentially finite real-space Hartree potential component in periodic
    boundary conditions.  \\[1.0ex]
  \option{threshold} is a small positive real number. Default:
  10$^{-10}$. \\
}
Usually, this tag need not be modified from the default. Necessary for
periodic boundary conditions only. For each atom, determines a safe
real space outer radius based on $\erf(r_\text{outer}/r_0) <
\mathtt{threshold}$. This is then used to determine which atoms need
be included in the second term (sum over atoms) of
Eq. (\ref{Eq:ves-assembly}). 

\keydefinition{legacy\_monopole\_extrapolation}{control.in}
{
  \noindent
  Usage: \keyword{legacy\_monopole\_extrapolation} \option{flag} \\[1.0ex]
  Purpose: Specifies how the monopole ($l=0$) part of the partitioned charge
  density is extrapolated to $r=0$ before transforming to a logarithmic grid
  to integrate the radial Hartree potential.  If \texttt{.true.}, use the
  legacy variant, and an improved extrapolation otherwise.
  \\[1.0ex]
  \option{flag} is a Boolean.  Default: \texttt{.false.}.\\%
}%
The effect is generally very small, but for \texttt{light} grids, this can
have some impact on total energies.


\keydefinition{l\_hartree\_far\_distance}{control.in}
{
  \noindent
  Usage: \keyword{l\_hartree\_far\_distance} \option{value} \\[1.0ex]
  Purpose: Sets a maximum angular momentum beyond which the components of the
  analytic long-range Hartree potential will not be computed. \\[1.0ex]
  \option{value} is an integer number. Default: 10. \\
}
Usually, this tag need not be modified from the default. In
Eq. (\ref{Eq:moments}), the multipole moments $m_{\text{at},lm}$ are
determined by an explicit integration of the finite real-space density
component $\delta\tilde{n}_{\text{at},lm}(r)$. However, for very high $l$,
even spuriously small density components (10$^{-10}$ or lower) may be
artificially weighted up in $m_{\text{at},lm}$; on a finite integration grid,
$m_{\text{at},lm}$ becomes prone to numerical noise. Capping the evaluation of
such high-$l$ components increases stability, but can be undone through
  \keyword{l\_hartree\_far\_distance} if required.

\keydefinition{multip\_moments\_threshold}{control.in}
{
  \noindent
  Usage: \keyword{multip\_moments\_threshold} \option{threshold} \\[1.0ex]
  Purpose: Implicitly defines the maximum angular momentum for which
    the analytical multipole components are non-zero at all. \\[1.0ex]
  \option{threshold} is a small positive real number. Default:
  10$^{-10}$. \\
}
Usually, this tag need not be modified from the default. Used only in
the periodic case. If $m_{\text{at},lm}/r_\text{mp}<$\option{threshold} for all
$l\ge l_\text{thr}$, all analytical components beyond $l_\text{thr}$
are considered zero in the real-space and Fourier parts of the
long-range potential. $r_\text{mp}$ is the radius determined by
\keyword{multip\_radius\_threshold}. 

\keydefinition{multip\_moments\_rad\_threshold}{control.in}
{
  \noindent
  Usage: \keyword{multip\_moments\_rad\_threshold} \option{threshold}
    \\[1.0ex]
  Purpose: Defines the outer radius of the density
    components $\delta\tilde{n}_{\text{at},lm}(r)$ for the purpose of
    determining the far-field moments $m_{\text{at},lm}$. \\[1.0ex]
  \option{threshold} is a small positive real number. Default:
    10$^{-10}$. \\
}
Usually, this tag need not be modified from the default. The outer
radius is set where
$|\delta\tilde{n}_{\text{at},lm}(r)|<$\texttt{threshold}. The actual
$m_{\text{at},lm}$ are then determined by inward integration from
this point, using the standard relations of classical electrostatics.

\keydefinition{multip\_radius\_free\_threshold}{control.in}
{
  \noindent
  Usage: \keyword{multip\_radius\_free\_threshold} \option{threshold}
    \\[1.0ex]
  Purpose: Technical criterion to define the outermost charge radius
    of the spherical free atom density $n_\text{at}^\text{free}$
    \\[1.0ex]
  \option{threshold} is a small non-negative real number. Default: 0.0 \\
}
Usually, this tag need not be modified from the default. 

The free-atom radius inside the code is set to the radius where
$n_\text{at}^\text{free}(r)$ becomes smaller than
\option{threshold}. Note that the actual extent of the free-atom
charge can be influenced by the \subkeyword{species}{cut\_free\_atom} keyword, and
has ramifications not just for the electrostatic potential, but also
for the initial charge density, and the partition functions for all
integrals. 

\keydefinition{multip\_radius\_threshold}{control.in}
{
  \noindent
  Usage: \keyword{multip\_radius\_threshold} \option{threshold}
    \\[1.0ex]
  Purpose: Determines the (per-atom) radius outside of which the
    analytical multipoles $m_{\text{at},lm}$ are used to construct the
    Hartree potential $v_\text{es}(\boldr)$ \\[1.0ex]
  \option{threshold} is a small positive real number. Default: 10$^{-12}$.\\
}
Usually, this tag need not be modified from the default. The outer
radius is set where \emph{all}
$\delta\tilde{n}_{\text{at},lm}(r)<$\texttt{threshold} for a given
atom. At a given integration point $\boldr$, $v_\text{es}(\boldr)$ is
assembled by evaluating Eq. (\ref{Eq:ves-assembly}). The second part
(sum over atoms) is evaluated separately for each atom, and atoms
outside the radius defined by \keyword{multip\_radius\_threshold}, the
$lm$ summation is performed using the analytical expression.

\keydefinition{multipole\_threshold}{control.in}
{
  \noindent
  Usage: \keyword{multipole\_threshold} \option{threshold} \\[1.0ex]
  Purpose: Cluster case only -- determines the distance beyond which
    an analytical component $\delta\tilde{v}_{\text{at},lm}(r)$ of the
    real-space Hartree potential is considered zero. \\[1.0ex]
  \option{threshold} is a small positive real number. Default:
    10$^{-10}$. \\
}
Usually, this tag need not be modified from the default. Long-range
Hartree potential components $\delta\tilde{v}_{\text{at},lm}(r)$ are
  not evaluated for distances where
  $\delta\tilde{v}_{\text{at},lm}(r)<$\option{threshold}. This tag
  provides similar functionality as the 
  \keyword{adaptive\_hartree\_radius\_th} tag for the periodic case
  (numerically different due to the absence of $\erf(r/r_0)$ in the
  cluster case). 

\keydefinition{normalize\_initial\_density}{control.in}
{
  \noindent
  Usage: \keyword{normalize\_initial\_density} \option{flag} \\[1.0ex]
  Purpose: If true, normalizes the initial electron density to reproduce the
   exact intended number of electrons when integrated on the three-dimensional,
   overlapping atom-centered integration grid of FHI-aims. \\[1.0ex]
  \option{flag} is either \texttt{.true.} or \texttt{.false.}. Default:
    \texttt{.true.}  \\
}
This keyword only normalizes the initial density. It should always be an exact
subset of the functionality provided by \keyword{compensate\_multipole\_errors}.
If used in conjunction with collinear spin and a geometry optimization (or
\keyword{sc\_init\_iter}), no subsequent renormalizations are performed, except
for runs which use a \keyword{fixed\_spin\_moment}.

The default for \keyword{normalize\_initial\_density} was set to \texttt{.false.} 
before August, 2017.

\keydefinition{set\_vacuum\_level}{geometry.in}
{
  \noindent
  Usage: \keyword{set\_vacuum\_level} \option{z-coordinate} \\[1.0ex]
  Purpose: Surface slab calculations only -- defines a $z$-axis value that is
  deeply within the vacuum layer.  \\[1.0ex] \option{z-coordinate} is a
  $z$ coordinate value in the vacuum layer.  \\ 
}
%
In the case of periodic surface slab calculations, this value defines the
reference $z$ coordinate that is used to define the work function (keyword
\keyword{evaluate\_work\_function}) and/or the location of a dipole correction
(electrostatic potential step) to offset a potential electrostatic dipole
formed by a non-symmetric slab (keyword \keyword{use\_dipole\_correction}).  
As a requirement, the surface must be parallel to the $xy$ plane. The chosen
\option{z-coordinate} must be located deep in the vacuum, as far away as
possible from any surface. 

\keyword{set\_vacuum\_level} \texttt{auto} can be used instead to
determine the vacuum level on its own. 

If \keyword{use\_dipole\_correction} or
\keyword{evaluate\_work\_function} are specified, omitting the keyword
\keyword{set\_vacuum\_level} causes FHI-aims to automatically
determine a suitable $z$. 

However, a vacuum plane will only be determined if the nearest
atom is more than 6 {\AA} away from the vacuum level. Determining a
surface dipole for distances for which basis functions or charge
densities could overlap might lead to errors. Since
FHI-aims does allow one to use rather large vacuum spacings at low (if any)
computational overhead, the calculation will stop for too small vacuum
spacings and alert the user. 



\keydefinition{use\_dipole\_correction}{control.in}
{
  \noindent
  Usage: \keyword{use\_dipole\_correction}  \\[1.0ex]
  Purpose: Surface slab calculations only -- compensates a potential dipole
  field of non-symmetric slabs by an electrostatic potential step in the
  vacuum region.
  \\[1.0ex]
  Restriction: When specified for a charged periodic system,
    this keyword is currently disabled (see below). \\
}
If set, this option introduces an electrostatic potential step in the vacuum
region of a surface slab calculation, to compensate for a potential surface
dipole. The surface must be parallel to the $xy$ plane (perpendicular to the
$z$ direction). The $z$ location of the surface dipole must be provided by
hand, by specifying the \keyword{set\_vacuum\_level} keyword.

In practice, the dipole correction calculates the gradient of only 
the long-range Hartree potential term of the Ewald sum (which is evaluated in
reciprocal space). If the gradients on both sides of the vacuum level do not
agree to better than 10 \% (i.e., the potential is not linear in this range),
the dipole correction is not computed, and a warning is issued instead.

However, it must be possible to find a vacuum plane $z$, where the
surface dipole is compensated, that is further than 6 {\AA} away from
the nearest atom. Otherwise, the calculation will stop and alert the user.

The reason is that a surface dipole cannot be safely determined for
vacuum spacings for which basis functions or charge densities could
overlap. This can lead to errors. Note that FHI-aims does allow one to use
rather large vacuum spacings at low (if any) computational overhead.

%If a dipole correction is requested, keyword
%\keyword{compensate\_multipole\_errors} is automatically switched
%on by default. This will change total energies slightly compared to
%the uncompensating case, and -- we believe -- even for the better. 

\emph{Attention:} This keyword is currently disabled for charged periodic
systems. The Coulomb potential of a charged surface slab will reach
far into the vacuum, apparently leading to a completely arbitrary
dipole correction as a result. (The dipole correction will simply
flatten out the potential wherever it is asked 
to do so, but for a charged surface, the residual Coulomb potential should not
be flat.) 

In order to alert user to the problem, the code presently stops
with a warning. If you know what you are doing, the pertinent stop (one line)
can always be commented out---if the code is recompiled, the method will be
applied, even though the physical relevance of the result is
uncertain. Charged periodic calculations with a
vacuum region are physically questionable for very different reasons
in any case; we recommend to find a different workaround with explicit
charges whenever that is possible.

\emph{Note} that for very large surface slabs, this keyword might cause 
instabilities in the SCF cycle. If you suspect this to be the case and remove 
\keyword{use\_dipole\_correction} from your \texttt{control.in}
%, remember 
%to set the keyword \keyword{compensate\_multipole\_errors}: 
%\keyword{compensate\_multipole\_errors} is currently automatically enabled 
%by default and changes the total  
%energies slightly compared to the uncompensating case; we believe for 
%the better.


\keydefinition{use\_hartree\_non\_periodic\_ewald}{control.in}
{
  \noindent
  Usage: \texttt{use\_hartree\_non\_periodic\_ewald} \option{.true.} \\
  or:    \texttt{use\_hartree\_non\_periodic\_ewald} \option{gridspacing}
                                                        \option{value} \\
  or:    \texttt{use\_hartree\_non\_periodic\_ewald} \option{.false.} \\[1.0ex]
  %
  Purpose: This option is \emph{experimental} and applies only to non-periodic
  calculations. In this case, the Hartree potential is decomposed according to
  Ewald's method.  \\[1.0ex]
}
This method accelerates the calculation of the Hartree term in case of large
systems (more than 200 atoms) by using Ewald's decomposition combined with
spatial interpolation, see section \ref{Sec:Hartree-non-periodic-ewald}. The
method can be switched on by using option ``\option{.true.}''. In this case, a
default grid spacing of 0.6\,\r{A} ($=60\,$pm) is used for the Cartesian grid.
Other values for the grid spacing can be chosen with option
``\option{gridspacing} \option{value}''. If this option is used, the method is
switched on and the grid spacing is set to \option{value} in \r{A} ($=100\,$pm).
Finally, the method can be switched off with option ``\option{.false.}''.
However, since this is the default behaviour, it is not necessary to switch off
the method explicitely.


\newpage



\subsection*{Subtags for \emph{species} tag in \texttt{control.in}:}

\subkeydefinition{species}{l\_hartree}{control.in}
{
  \noindent
  Usage: \subkeyword{species}{l\_hartree} \option{value} \\[1.0ex]
  Purpose: For a given species, specifies the angular momentum
    expansion of the atom-centered charge density multipole for the
    electrostatic potential. \\[1.0ex]
  \option{value} is an integer number which gives the highest angular
    momentum component used in the multipole expansion of $\delta
    n(\boldr)$ into $\delta\tilde{n}_{\text{at},lm}(r)$ for the
    present species. \emph{Must be specified.} \\
}
As pointed out in
Ref. \cite{Blum08}, our experience is that energy differences are
usually well converged for $l_\text{hartree}$=4, and total energy
convergence at the level of a few meV is reached at
$l_\text{hartree}$=6. Only in exceptional cases should different
settings be required. 
