\section{Hartree-Fock, hybrid functionals, $GW$, \emph{et al.}: All the details}
\label{Sec:auxil}

The basic keywords to invoke different exchange-correlation methods
(ground state and excited states) are given described in
Sec. \ref{Sec:xc}. Usually, invoking the relevant keyword together with
the normal infrastructure required to run FHI-aims should be
sufficient to produce a correct, converged result. 

\emph{For Hartree-Fock and hybrid functionals, particularly for their
  periodic implementations, see also the dedicated next section,
  Sec. \ref{Sec:periodic_hf}.}  

For methods that rely explicitly on a two-electron Coulomb operator
(Hartree-Fock, hybrid functionals, $GW$, MP2, RPA, etc.) and/or a
frequency-dependent response function ($GW$, RPA, ...), some
considerable numerical trickery enters the computation in order to
keep it efficient yet manageable for practical purposes. We
hope to provide resilient, system-independent default settings, 
but there is a lot of freedom beyond those defaults to either
tighten up things or speed up calculations (at the price of
reduced accuracy).

The present section describes all numerical settings for the
aforementioned exchange-correlation treatments. 

Specifically, we describe:
\begin{itemize}
  \item All settings that relate to the setup of the (over-)complete
    \emph{auxiliary basis} that expands the products of pairs of basis
    functions into a separate basis to represent the Coulomb operator
  \item All settings that rely to the frequency grid and analytic
    continuation from the imaginary to the real axis in $GW$ related
    methods. 
\end{itemize}
Even if you do not know what this is all about, you \emph{should} know
that the ``auxiliary basis'' is determined as an overcomplete basis,
and superfluous basis functions are then reduced out by a threshold
criterion, using singular value decomposition. This threshold is a 
value to be tested in case something unexpected happens.

There are three key references that provide the technical background
for these sections:
\begin{enumerate}
  \item Principle of how we calculate the two-electron Coulomb
    operator by so-called resolution of identity: Xinguo Ren \emph{et
    al.} (2012), New J. Phys. 14, 053020, Ref. \cite{Ren12a}. The
    approach summarized below and implemented in Keyword
    \keyword{RI\_method} \texttt{V} is still the default for any
    non-periodic many-body perturbation calculations \emph{beyond} DFT
    (i.e., for MP2, RPA, $GW$, etc.)
  \item Localized resolution of identity (Keyword \keyword{RI\_method}
    \texttt{LVL}), which is used by default for all Hartree-Fock and
    hybrid functional calculations, and which is the only option for
    any periodic calculations including the two-electron Coulomb
    operator: Arvid Ihrig \emph{et al.} (2015), New J. Phys. 17,
    093020, Ref. \cite{Ihrig2015}.
  \item Linear-scaling and periodic implementation of periodic
    Hartree-Fock and hybrid functionals based on \keyword{RI\_method} 
    \texttt{LVL}, described in Sergey Levchenko \emph{et al.} (2015),
    Comput. Phys. Commun. 192, 60-69, Ref. \cite{Levchenko2015}.
\end{enumerate}

\textbf{Mathematical background:}

Any feature beyond standard DFT (e.g., HF, hybrid functional,
MP2, $GW$, etc) requires the two-electron Coulomb repulsion integrals, and 
in FHI-aims an additional auxiliary basis set is introduced to deal with 
them. By utilizing the auxiliary basis functions the $N_\text{basis}^4$ many 
4-center integrals are reduced to $N_\text{basis}^2\cdot N_\text{aux}$
many 3-center integrals and  $N_\text{aux}^2$ many 2-center integrals (
where $N_\text{basis}$ and $N_\text{aux}$ are the numbers of regular basis
functions and auxiliary basis functions, respectively). There are different
ways to do so, and here we describe two versions of these, namely, the
 ``V" and ``SVS" \cite{Vahtras93} versions which have been implemented in 
this code. In the ``V" version, the 4-center integrals are approximated by
  \begin{equation}
   (ij|i^\prime j^\prime) \approx \sum_{\mu\nu} (ij|\mu)V_{\mu\nu}^{-1} 
   (\nu|i^\prime j^\prime),
  \label{RI_V}
  \end{equation}
and in the ``SVS" one,
  \begin{equation}
   (ij|i^\prime j^\prime) \approx \sum_{\mu\nu} \sum_{\mu^\prime\nu^\prime}
 (ij\mu)S_{\mu\mu^\prime}^{-1}V_{\mu^\prime\nu^\prime}S_{\nu^\prime\nu}^{-1} 
   (\nu i^\prime j^\prime),
  \label{RI_SVS}
  \end{equation}
where $i,j,i^\prime,j^\prime, \dots$ denote the regular basis functions 
and $\mu,\nu, \dots$ denote the auxiliary basis functions. 
Here $V_{\mu\nu}$ is the Coulomb repulsion integral between two auxiliary basis 
functions, and $S_{\mu\nu}$ is the corresponding overlap integral. 
$(ij\mu)$ and $(ij|\mu)$ are the overlap and Coulomb repulsion between the 
regular basis orbital product 
$\phi_i\phi_j$ and the auxiliary basis function $P_\mu$ respectively.
Eq. (\ref{RI_V}) and (\ref{RI_SVS}) are often refered to as
resolution of identity (Refs. \cite{Boys59,Alsenoy88,Vahtras93,Eichkorn95} and
others).  In practice satisfactory accuracy can be gained with 
an auxiliary basis size $N_\text{aux}$ of 4-5 times of
$N_\text{basis}$. In addition, we implement a modified localized
version of RI-V known as ``RI-LVL'', described in
Ref. \cite{Ihrig2015} and also in Sec. \ref{Sec:periodic_hf}.

How is the auxiliary basis functions constructed? In FHI-aims, it is  built 
up as the ``on-site'' pair products of the regular basis functions (hence
the auxiliary basis is also called the ``product basis'' in this context). 
These products are then orthonormalized at each atom using the gram-Schmidt 
method. These auxiliary basis functions are hence atom-centered numeric 
functions with a radial function times spherical harmonics 
 $P_{\mu}({\bf r}) = \frac{\xi(r)_{\text{at},n,l}}{r}Y_{lm}(\vartheta,\varphi)$. 
The radial part of the auxiliary basis function is 
formally linked to that of the regular basis functions by
  \begin{equation}
    \{\xi_{\text{at},n,l}(r)\} = \{u_{\text{at},n_1,l_1}(r)u_{\text{at},n_2,l_2},
     |l_1-l_2| \le l \le |l_1+l_2|  \}.
   \label{auxil_radial}
  \end{equation}
In Eq. (\ref{auxil_radial}) we make it clear that the set of auxiliary
basis functions centered on certain atom  originates from the pair
products of regular basis functions  
centered on the the same atom. The angular momentum of the auxiliary basis
and those of the two constituent regular basis satisfy the triangular 
true. The number of auxiliary basis function for a give $l$ 
(enumerated by $n$) is
controlled by the allowed pairs of regular basis functions, and the accuracy 
threshold in the Gram-Schmidt orthormalization. The process is
described in Ref. \cite{Ren12a} and in more detail with an
illustrating figure in Ref. \cite{Ihrig2015} (open access). The
parameters that controls the construction of the auxiliary basis
functions can be found below. 

For self-energy calculatons a proper frequency grid is needed. For instance,
in the $GW$ approximation, the self-energy is given by
  \begin{equation}
    \Sigma({\bf r, r^\prime};i\omega) = \frac{i}{2\pi} \int d\omega^\prime
     G({\bf r, r^{\prime}};i\omega+i\omega^\prime) 
    W({\bf r,r^\prime};i\omega^\prime)
   \label{GW_selfenergy}
  \end{equation}
In FHI-aims, the self-energy is first calculated in the imaginary frequency
grid, and then analytically continued to the real frequency grid. One popular 
way of performing the analytical continuation is to model the self-energy with
a multi-pole expression \cite{Rojas95}, namely,
    \begin{equation} 
       \Sigma(i\omega) \approx A_0 + \sum_n \frac{A_n}{i\omega - B_n},
      \label{multipole_fitting}
    \end{equation}
where $n$ is the number of poles, and $A_n$ and $B_n$ are complex numbers. 
Eq. (\ref{multipole_fitting}) is used to fitted the calculated self-energy
on imagninary axis using the non-linear least square fitting algorithm. Once the
the the parameters $A_n$ and $B_n$ are obtained that give the best fitting,
the self-energy on the real frequency axis can be obtained by
    \begin{equation}
       \Sigma(\omega) \approx A_0 + \sum_n \frac{A_n}{\omega - B_n}.
       \label{eq:multipole-real}
    \end{equation}
In practice $n=2$ (the so-called two-pole fitting) is often found to
give good performance. 

A Pade approximation based variant with more poles is also implemented
in FHI-aims. In the Pade approximation, the self-energy is 
parameterized as
  \begin{equation}
       \displaystyle
       \Sigma(i\omega) =
       \dfrac{a_1}{1+\dfrac{a_2(i\omega-i\omega_1)}{1+\dfrac{a_3(i\omega-i\omega_2)}{\cdots}}} .
      \label{eq:pade}
  \end{equation}
For a given, chosen set of calculated self-energy data points
$\{i\omega_n, \Sigma(i\omega_n)\}$ with $n=1,\cdots, N$, the $N$
complex parameters  
$a_1, \cdots, a_N$ can be uniquely determined. The self-energy on the real
frequency axis is then obtained by replacing $i\omega$ by $\omega$ in   
Eq.~(\ref{eq:pade}). We note that the Pade approximation given by
(\ref{eq:pade}) can be interpreted as a multipole expression, with
the number of poles $N_{pole} = N-1$.

The type of the analytical continuation used in $GW$
(Eqs. (\ref{eq:multipole-real}) or (\ref{eq:pade})) is determined
using the \keyword{anacon\_type} and \keyword{n\_anacon\_par}
keywords. The number of frequency points used on the imaginary axis
(this determines the accuracy of the input used for fitting the
expressions Eqs. (\ref{eq:multipole-real}) or (\ref{eq:pade})) can be
set using the \keyword{frequency\_points} keyword. The frequency grid
used is a modified Gauss-Legendre grid that ranges from zero to
infinity. Our experience suggests that highly accurate results for molecules
(few meV accuracy for electronic excitations, compared to exact
expressions for the self-energy on the real axis) 
can be obtained using a 16-parameter Pade approximation with 200
frequency points.\cite{GW100} However, the numerical accuracy is then much higher
(and much more costly) than the accuracy of the underlying $GW$
approximation itself, and thus somewhat reduced defaults are set in
the code (see keyword descriptions below).

More importantly, the Pade approximation is also numerically less
stable than the two-pole approximation. This means that, for some
systems with a complicated pole structure of the self-energy, the Pade
fit might not converge for certain eigenvalues. The results must
therefore be inspected carefully, even if (for normal light-element
molecules and valence-like states) its accuracy can be much higher
than the two-pole approximation. This is, in fact, not a simple
implementation issue but rather one that goes back to the mathematical
structure of the true self-energy.

The analytical continuation becomes increasingly inaccurate for 
deeper states since the structure of the self-energy is typically more
complicated, i.e., has more poles. The fitted pole models fail to
represent these more complicated structures, see Ref.~\cite{Golze2018}.
In these cases, the self-energy should be calculated on the real-frequency
axis using the contour deformation technique, which is controlled by
 the keyword \keyword{contour\_def\_gw}. 


%\newpage

\subsection*{Tags for general section of \texttt{control.in}:}

\keydefinition{hf\_version}{control.in}
{ \noindent
  Usage: \keyword{hf\_version} \option{version} \\[1.0ex]
  Purpose: Allows to switch between the standard density-matrix based
    setup of the Fock operator, and an orbital based exchange
    operator. \\[1.0ex]
  \option{version} is a number, either 0 (alias \texttt{density\_matrix}) or
  1 (aliases \texttt{eigencoefficients} and \texttt{overlap}). Default: 0 \\
}
The exchange operator can be constructed either by summing over all
states \emph{first} to construct the density matrix (\option{version}
0), of by a straightforward setup of orbitals and computation of the
exchange operator only then (\option{version} 1).

Both versions are pure matrix algebra. For small systems (few states),
\option{version} 1 is vastly more efficient, but towards large systems
(many states) an efficiency crossover clearly favors the density matrx
based update.

By default, FHI-aims relies on the density-matrix based update always, because
this is more memory efficient, but for small systems (atoms) with lots of
basis functions, this choice should be reconsidered.  Please note that both
versions should work seamlessly with Pulay mixing.

\keydefinition{anacon\_type}{control.in}
{ \noindent
  Usage: \keyword{anacon\_type} \option{string} \\[1.0ex]
  Purpose: Specifies type of analytical continuation for the self-energy
   (we calculate the self-energy on the imaginary frequency axis, and
   hence need to continue it to the real axis) \\ 
  \option{string} is a string that indicates the self-energy type,
  either 'two-pole' or 'pade'. Default: No default (must be set by the
  user if the self-energy is required).
}\\[1.0ex]
  \begin{itemize}
     \item \option{string} = 'two-pole' or '0' : The normal two-pole
       fitting (Eq (\ref{eq:multipole-real})).
    \item \option{string} = 'pade' or '1' : Pade approximation (Eq. (\ref{eq:pade})).
  \end{itemize}
The number of parameters in either approximation can be set using the
keyword \keyword{n\_anacon\_par}.

\emph{Note} that the \keyword{anacon\_type} only makes sense if 
\keyword{qpe\_calc} or \keyword{sc\_self\_energy} is set, i.e., the
post-processing-type self-energy calculation is required.

This keyword must be set in control.in if the self-energy on the real
axis is needed (usually, for $GW$).
In past versions of FHI-aims, the code would set a silent default for
\keyword{anacon\_type} if a self-energy calculation for the real axis
was required. This is no longer the case in
present versions. Users must make this choice explicitly in
'control.in'; if that is not the case, the code will stop with a
(hopefully gentle and instructive) warning message.

The reason is that the choice of the analytical continuation used
can have a noticeable effect on the accuracy of $GW$-calculated
eigenvalues. The two-pole approximation is well established, but less
accurate than the Pade approximation when the latter works.\cite{GW100} On the
other hand, systems with a complicated self-energy structure can lead
to numerical problems with the Pade approximation that can result in
seemingly random values for certain predicted quasiparticle
eigenvalues (this can be tested, for instance, by modifying the
\keyword{frequency\_points} keyword and tracking the results). 

\keydefinition{freq\_grid\_type}{control.in}
{ \noindent
  Usage: \keyword{freq\_grid\_type} \option{value} \\[1.0ex]
  Purpose: If set, specifies the type of the grid for the imaginary frequency.
  \begin{itemize}
    \item \option{value} = 0 : Normal Gauss-Legendre grid ranging from 0
     to a maximum frequency, specified by the keyword \keyword{maximum\_frequency}.
    \item \option{value} = 1 : Modified Gauss-Legendre grid ranging from 0 to positive infinity.
    \item \option{value} = 2 : Logrithmic grid ranging from 0.01 a.u. to a maximum value specified by
       \keyword{maximum\_frequency}.
  \end{itemize}
  Default: \option{value}=1
}\\[1.0ex]

\emph{Note} that the \keyword{freq\_grid\_type} only makes sense if 
\keyword{qpe\_calc} or \keyword{sc\_self\_energy} is set, i.e., the
post-processing-type self-energy calculation is required.

\keydefinition{n\_anacon\_par}{control.in}
{ \noindent
  Usage: \keyword{n\_anacon\_par} \option{value} \\[1.0ex]
  Purpose: If set, specifies the number of parameters used in the two-pole
  fitting (Eq.~(\ref{multipole_fitting})) or Pade approximation (Eq~(\ref{eq:pade})).
  The default value for \keyword{n\_anacon\_par} is 5 if \keyword{anacon\_type} is
  set to 'two-pole' (two-pole fitting), and 16 if
  \keyword{anacon\_type} is set to 'pade' (Pade approximation).  
}\\[1.0ex]

\emph{Note} that the \keyword{n\_anacon\_par} only makes sense if 
\keyword{qpe\_calc} or \keyword{sc\_self\_energy} is set, i.e., the
post-processing-type self-energy calculation is required.

\keydefinition{contour\_def\_gw}{control.in}
{ \noindent
  Usage: \keyword{contour\_def\_gw} \option{state$_{\textnormal{\option{start,$\alpha$}}}$}
         \option{state$_{\textnormal{\option{end,$\alpha$}}}$ \option{state$_{\textnormal{\option{start,$\beta$}}}$}  \option{state$_{\textnormal{\option{end,$\beta$}}}$}}\\[1.0ex]
  Purpose: If set, specifies the range of states for which the $GW$ quasiparticle 
  energies are computed with the contour deformation, see Ref.~\cite{Golze2018}
  for a description of the implementation. The range can be specified for the $\alpha$ and $\beta$
  spin channel separately. Giving the range for the $\beta$ channel is optional. If not specified, the range set for
  $\alpha$ will be also used for the $\beta$ channel. For spin-unpolarized calculations specify only the
  $\alpha$ channel.\par
  The quasiparticle energies for 
  the other states are computed with the analytic continuation, i.e., the parameters
  \keyword{anacon\_type} and \keyword{n\_anacon\_par} should be set as well. 
  Setting \keyword{frequency\_points} is mandatory (200 grid points is a 
  solid choice).
}

\keydefinition{contour\_spin\_channel}{control.in}
{ \noindent
  Usage: \keyword{contour\_spin\_channel} \option{integer (1 or 2)}  \\[1.0ex]
  Purpose: If specified, restricts the contour deformation to a certain
  spin channel. If not given, QP energies for both channels will be calculated.
}

\keydefinition{contour\_eta}{control.in}
{ \noindent
  Usage: \keyword{contour\_eta} \option{real}  \\[1.0ex]
  Purpose: Specifies the broadening parameter $\eta$ used for the 
  contour deformation. It might be useful to set this parameter to 
  higher values (e.g 0.002 a.u.) when printing the self-energy or 
  spectral function. Otherwise the default setting
  ensures numerical accuracy and stability. \\
  Default: 0.001~a.u.
}

\keydefinition{contour\_restart}{control.in}
{ \noindent
  Usage: \keyword{contour\_restart} \option{task}  \\[1.0ex]
  Purpose: The iteration of the QP equation can become expensive
  for large systems. A restart is possible.\\[1.0ex]
  \option{task} is a string, specifying the desired restart task. \\[1.0ex]
  Available options for \option{task} are:
  }
   \begin{itemize}
  \item \texttt{write} : Writes restart info to file \texttt{contour\_gw\_qp\_energies.dat}.
  \item \texttt{read} : Reads restart info from \texttt{contour\_gw\_qp\_energies.dat} and 
   continues the QP iteration cycle.
  \item \texttt{read\_and\_write} : Performs what the \texttt{write} and
    \texttt{read} options do. If the restart files do not exist, the
    code will still proceed normally.
\end{itemize}


\keydefinition{full\_cmplx\_sigma}{control.in}
{ \noindent
  Usage: \keyword{full\_cmplx\_sigma} \option{boolean}  \\[1.0ex]
  Purpose: Technical keyword for the contour deformation. If set to .true., 
  the complex broadening term $i\eta$ enters also the integral term of the self-energy.
  This requires more grid points in the frequency integration, i.e.,
  \keyword{frequency\_points} should be set to 2000. Including $i\eta$
  is not necessary when calculating the quasiparticle energies, but gets rid of
  (the very small) unphysical steps in the spectral function at the
  KS/HF energies.
   \\
  Default: .false.
}

\keydefinition{gw\_zshot}{control.in}
{ \noindent
  Usage: \keyword{gw\_zshot} \option{boolean}  \\[1.0ex]
  Purpose: If set to .true., the $Z$-factor is calculated
  and the quasiparticle equation is not calculated iteratively,
  but linearized using a Taylor expansion. Less exact than the
  iterative solution. 
   \\
  Default: .false.
}

\keydefinition{gw\_hedin\_shift}{control.in}
{ \noindent
  Usage: \keyword{gw\_hedin\_shift} \option{state}  \\[1.0ex]
  Purpose: If set, the poor-man's self-consistency proposed
  by Hedin \cite{Hedin1999} is enabled. The Hedin shift is referenced to a particular state, 
  typically the HOMO. This procedure can be also considered as
  fixing the zero of the energy scale in a $G_0W_0$ calculation employing 
  an overall energy shift $\Delta E$. When including this shift, the starting point
  dependence is significantly reduced, similar to an ev-sc$GW_0$ calculation.
  The computational overhead is negligible. 
   \\
  Default: .false.
}

\keydefinition{print\_self\_energy}{control.in}
{ \noindent
  Usage: \keyword{print\_self\_energy} \option{state} \option{freq$_{\textnormal{\option{start}}}$}
  \option{freq$_{\textnormal{\option{end}}}$} \\[1.0ex]
  Purpose: If set, the diagonal self-energy matrix elements for the requested
  state are printed in the given frequency range that should be given in eV.
  Setting the frequency range is optional.
}

\keydefinition{auxil\_basis}{control.in}
{ \noindent
  Usage: \keyword{auxil\_basis} \option{type} \\[1.0ex]
  Purpose: Specifies the type of auxiliary basis used in the
  ``beyond-DFT'' calculation. 
    \\[1.0ex]
  \option{type} is a string, which can be set either as \option{full}
  or \option{opt}.   Default: \texttt{full}. Here is a brief explanation.
  \begin{itemize}
     \item  \texttt{full} : The auxiliary basis is constructed as
     the ``on-site" pair products of the regular basis functions. 
     The allowed pair products are controlled by the parameters 
     \subkeyword{species}{max\_n\_prodbas} and
     \subkeyword{species}{max\_l\_prodbas} (see later).
     These pair products
     are then orthonormalized using Gram-Schmidt procedure for each atom. 
     \item  \texttt{opt} : The auxiliary basis is obtained
     from an optimization procedure, and must be specified by
     hand in \texttt{control.in} -- in the same spirit as the basis
     sets used in standard Gaussian-bases RI-MP2 calculations.
  \end{itemize}
}

\keydefinition{default\_prodbas\_acc}{control.in}%
{ \noindent %
  Usage: \keyword{default\_prodbas\_acc} \option{threshold} \\[1.0ex]
  Purpose: Specifies the default for \subkeyword{species}{prodbas\_acc}\\[1.0ex]
  \option{threshold} is a real value, defining the onsite threshold for the
  auxiliary basis construction.\\
  Default: 10$^{-4}$ for \keyword{RI\_method} \texttt{lvl}, depends on species
  (\subkeyword{species}{species\_z}) otherwise.  
}

See \subkeyword{species}{prodbas\_acc} for more details. Default settings are:
\begin{itemize}
  \item 10$^{-2}$ for $Z \le$10 (light elements)
  \item 10$^{-3}$ for 10$< Z \le$18
  \item 10$^{-4}$ for $Z >$18 (all heavier elements)
\end{itemize}
The old default (version 042811 and earlier) was simply 10$^{-2}$ for
all elements. For light elements, this setting produces accurate total
energies and energy differences to the sub-meV level in all our
tests. For heavier elements, significant inaccuracies could happen in
atomic total energies. These inaccuracies would cancel out in energy
differences; to guarantee total energy accuracy as well, we now set
significantly tighter defaults for \subkeyword{species}{prodbas\_acc}
in this range (alas, also more expensive, both in time and memory use).

\keydefinition{default\_max\_l\_prodbas}{control.in}%
{ \noindent %
  Usage: \keyword{default\_max\_l\_prodbas} \option{value} \\[1.0ex]
  Purpose: Specifies the default for
  \subkeyword{species}{max\_l\_prodbas}\\[1.0ex]
  Default: 20 for \keyword{RI\_method} \texttt{lvl}, 5 for \keyword{RI\_method} \texttt{V} and
    nuclear charge $Z<=54$, and 6 for \keyword{RI\_method} \texttt{V} and $Z>54$.  }

\keydefinition{default\_max\_n\_prodbas}{control.in}%
{ \noindent %
  Usage: \keyword{default\_max\_n\_prodbas} \option{value} \\[1.0ex]
  Purpose: Specifies the default for
  \subkeyword{species}{max\_n\_prodbas}\\[1.0ex]
  Default: 20 for \keyword{RI\_method} \texttt{lvl}, 6 otherwise.
}

\keydefinition{frequency\_points}{control.in}
{ \noindent
  Usage: \keyword{frequency\_points} \option{value} \\[1.0ex]
  Purpose: If set, specifies the number of (imaginary) frequency points for the self-energy 
  calculation.\\[1.0ex]  
%  Default: \option{value}=80. \\[1.0ex]
  The default value for the frequency points depends on the choice of the analytical continutation type.
  For two-pole fitting (\keyword{anacon\_type}='two-pole'), the default value for \keyword{frequency\_points} value is 40;
  for the Pade approximation (\keyword{anacon\_type}='pade') the default value for \keyword{frequency\_points} is 100.
}
\emph{Note} that the \keyword{frequency\_points} only makes sense if 
\keyword{qpe\_calc} or \keyword{sc\_self\_energy} is set, i.e., the post-processing-type self-energy calculation 
is required.


\keydefinition{maximum\_frequency}{control.in}
{ \noindent
  Usage: \keyword{maximum\_frequency} \option{value} \\[1.0ex]
  Purpose: If set, specifies the maximal (imaginary) frequency value for the 
  self-energy self-consistent calculation.  The unit for \option{value} here is Hartree.\\[1.0ex]  
}
\emph{Note} that the \keyword{maximum\_frequency} only makes sense when the \keyword{freq\_grid\_type} is set to be 0 or 2, i.e., when
the stdandard Gauss-Legendre grid or logrithmic grid is used. For \keyword{freq\_grid\_type}$=$0, the default value for \keyword{maximum\_frequency} is
10 Hartree; for \keyword{freq\_grid\_type}$=$2, the default value is 5000 Hartree. However, when self-consistent $GW$ is involked (both sc$GW$ and sc$GW_0$), the
default value for \keyword{maximum\_frequency} is 7000 Hartree.

\keydefinition{maximum\_time}{control.in}
{ \noindent
  Usage: \keyword{maximum\_time} \option{value} \\[1.0ex]
  Purpose: If set, specifies the maximal (imaginary) time value for the
  self-consistent self-energy calculation.  The unit for \option{value} here is Hartree$^{-1}$. The default value
  is 1000 a.u..\\[1.0ex]
}
\emph{Note} that the \keyword{maximum\_time} only makes sense if
\keyword{sc\_self\_energy} is set, i.e., the self-consistent self-energy calculation
is required.

\keydefinition{n\_poles}{control.in}
{ \noindent
  Usage: \keyword{n\_poles} \option{value} \\[1.0ex]
  Purpose: If set, specifies the number of poles (i.e. the number of functions of the form 
  $f_i(\omega) = 1/(b_i +i\omega)$ ) adopted in the pole-based
  computation of the Fourier transform in self-consistent $GW$-type calculations. \\[1.0ex]
}
\emph{Note} that the \keyword{n\_poles} only makes sense if
\keyword{sc\_self\_energy} is set, i.e., the post-processing-type self-consistent self-energy calculation
is required.

\keydefinition{pole\_max}{control.in}
{ \noindent
  Usage: \keyword{pole\_max} \option{value} \\[1.0ex]
  Purpose: If set, specifies the position in the (imaginary) frequency axis of the 
  largest poles (i.e. the largest $b_i$ coefficient in $f_i(\omega) = 1/(b_i +i\omega)$ ) 
  used in computation of the Fourier transform in self-consistent $GW$-type calculations. 
   The unit for \option{value} here is Hartree. \\[1.0ex]
}
\emph{Note} that the \keyword{pole\_max} only makes sense if
\keyword{sc\_self\_energy} is set, i.e., the post-processing-type self-consistent self-energy calculation
is required.

\keydefinition{pole\_min}{control.in}
{ \noindent
  Usage: \keyword{pole\_min} \option{value} \\[1.0ex]
  Purpose: If set, specifies the position in the (imaginary) frequency axis of the 
  smallest poles (i.e. the smallest $b_i$ coefficient in $f_i(\omega) = 1/(b_i +i\omega)$ ) 
  used in computation of the Fourier transform in self-consistent $GW$-type calculations. 
   The unit for \option{value} here is Hartree. \\[1.0ex]
}
\emph{Note} that the \keyword{pole\_min} only makes sense if
\keyword{sc\_self\_energy} is set, i.e., the post-processing-type self-consistent self-energy calculation
is required.


\keydefinition{prodbas\_nb}{control.in}
{ \noindent
  Usage: \keyword{prodbas\_nb} \option{nb} \\[1.0ex]
  Purpose: For very large scale beyond-GGA calculations, the distribution of
  auxiliary basis functions among the CPUs becomes problematic because each
  CPU only gets very few.  The default Scalapack distribution is more taylored
  to efficient calculations and distributes these functions in chunks of
  finite size.  In massively parallel runs, often each CPU gets either one or
  two of these chunks, leading to bad memory distribution.  This can be
  circumvented, possibly sacrificing some performance, by setting the chunk
  size \option{nb} to ``1''.
  % 
  \\[1.0ex]  
  \option{nb} is the chunk size of auxiliary basis functions.
  \\
  \texttt{Default:} $\mathrm{min}\,(16,
  \lfloor N_{\mathrm{aux}}/N_{\mathrm{proc}}\rfloor)$.
}

\keydefinition{prodbas\_threshold}{control.in}
{ \noindent
  Usage: \keyword{prodbas\_threshold} \option{threshold} \\[1.0ex]
  Purpose: Prevent the possible ill-conditioning of the auxiliary basis, similar
  to \keyword{basis\_threshold} for the regular basis.  \\[1.0ex]  
  \option{threshold} is a small positive real number for the eigenvalue of the
   Coulomb matrix for the auxiliary basis. \texttt{Default:} 10$^{-5}$. 
}

  The the auxiliary basis functions centered on different atoms are nonorthogonal,
  and the possible linear dependence (with certain accuracy) between them 
  and the resultant behavior has to be carefully eliminated. This is achieved by
  setting the cutoff threshold for the auxiliary basis \keyword{prodbas\_threshold}.
  From many test calculations, it is found that the reliable value for 
  \option{threshold} here are between 10$^{-5}$ to 10$^{-3}$. Within this window,
   the total energy may still have some noticable change, but the energy difference 
   is usually negligible. It is suggested that the user should play with this
   value if he/she is not sure about his/her result.


\keydefinition{RI\_method}{control.in}
{ \noindent
  Usage: \keyword{RI\_method} \option{type} \\[1.0ex]
  Purpose: Specifies the version of the resolution of identity used in the 
      beyond-DFT calculations. Here \option{type} is a string, with
      possible options listed below.
      \\[1.0ex]
  Default: Non-periodic: \option{type}=\texttt{lvl} for Hartree-Fock
  and hybrid functional calculations, which can be used for geometry
  relaxations. \option{type}=\texttt{v} for MP2, RPA and $GW$
  etc. calculations that require unoccupied states.  This gives better
  accuracy for a given auxiliary basis, but is usually more expensive
  and provides no geometry relaxation at present. Periodic: 
  \option{type}=\texttt{LVL\_fast}, which provides the necessary
  reduction to what is essentially an $O(N)$ framework. \\
}

Different options for the \option{type} option include:
\begin{itemize}
  \item \texttt{svs} (for the ``SVS'' version, i.e.,
    Eq. (\ref{RI_SVS}))
  \item \texttt{v} (for the ``V'' version, i.e. Eq. (\ref{RI_V}))
  \item \texttt{lvl} (for the ``LVL'' version)
  \item \texttt{LVL}, \texttt{LVL\_fast}, or \texttt{lvl\_fast} are
    all synonymous with option \texttt{lvl}
  \item \texttt{lvl\_full} implements a non-linear scaling version of
    the ``LVL'' approach for total energy evaluations only
  \item \texttt{lvl\_2nd} implements the so-called ``robust Dunlap
    correction'' for total energies only, which essentially follows up
    an s.c.f. calculation with an additional RI-V-like step. In our
    experience, similar results are better accomplished without this
    correction (see also Ref. \cite{Ihrig2015}).
\end{itemize}

Rules of thumb:
\begin{itemize}
  \item \option{type}=\texttt{V} is the preferred method for
    non-periodic calculations. For formal reasons, this is clearly the
    most accurate version. However, neither a periodic version nor
    gradients (forces and relaxations) are implemented at present.
  \item \option{type}=\texttt{LVl\_fast} is the preferred version for
    periodic calculations, as well as for non-periodic Hartree-Fock and
    hybrid functional calculations with
    forces and relaxation. It scales as $O(N)$ and greatly limits
    the memory use compared to RI-V. \keyword{RI\_method}
    \texttt{LVL\_fast} localizes the expansion of the Coulomb
    potential of basis function products to two centers, as described
    in Sec. \ref{Sec:periodic_hf} and in
    Refs. \cite{Ihrig2015,Levchenko2015}. It also relies extensively
    on screening near-zero elements of the density matrix and of the
    Coulomb operator. On the other hand, the localized
    version has slightly larger errors than RI-V for hybrid
    functionals, and \emph{significantly} larger errors for correlated
    methods like MP2 or RPA. These can be remedied by adding a few
    extra functions to the construction of the auxiliary basis set
    using the \keyword{for\_aux} keyword, as described in detail in
    Ref. \cite{Ihrig2015}.
  \item At present, \emph{do not use RI-LVL for MP2 and RPA} unless
    you know what you are doing. It is possible to repair their
    accuracy in RI-LVL, by increasing the size of the auxiliary basis
    set using the \keyword{for\_aux} keyword, as described in detail in
    Ref. \cite{Ihrig2015}, but please see the figures and benchmarks
    in that reference before trying.
  \item \option{type}=\texttt{LVL\_full} is a slower, non-screened version
    of LVL for non-periodic systems. Very useful as a reference to
    make sure.
  \item \option{type}=\texttt{SVS} is here only for testing
    purposes. This is the naive, purely overlap based version of RI
    and should not be used in production.
\end{itemize}

There is an additional option for non-periodic systems,
\option{lvl\_2nd}, which adds a correction term to the nonlocal
exchange energy to make it ``robust'' in the Dunlap sense 
\cite{Dunlap10-RIreview}, that is, the error in the energy is quadratic in
the error in the product expansion.

\keydefinition{sbtgrid\_lnrange}{control.in}
{ \noindent
     Usage: \keyword{sbtgrid\_lnrange} \option{lnrange} \\[1.0ex] 
     Purpose: for \keyword{use\_logsbt}, set the range of the logarithmic grid
     (in logarithmic units).
     Default is $45$, which corresponds to nearly twenty orders of magnitude.
     Please note that the range should be larger than intuitively guessed
     because the range is the same both in real and reciprocal space (for
     algorithmic reasons) and the tails in reciprocal space have to be captured.
}

\keydefinition{sbtgrid\_lnr0}{control.in}
{ \noindent
     Usage: \keyword{sbtgrid\_lnr0} \option{lnr0} \\[1.0ex] 
     Purpose: for \keyword{use\_logsbt},set the onset of the logarithmic grid
     in real space.  The default is $-38$.
}

\keydefinition{sbtgrid\_lnk0}{control.in}
{ \noindent
     Usage: \keyword{sbtgrid\_lnk0} \option{lnk0} \\[1.0ex] 
     Purpose: for \keyword{use\_logsbt}, set the onset of the logarithmic grid
     in Fourier space.  The default is $-25$.
}

\keydefinition{sbtgrid\_N}{control.in}
{ \noindent
     Usage: \keyword{sbtgrid\_N} \option{N} \\[1.0ex] 
     Purpose: for \keyword{use\_logsbt}, set the number of logarithmic grid
     points both in real and Fourier space.  The default is $4096$.
}

For the accuracy, the density of points $N/\mathtt{lnrange}$ is relevant.
Additionally, one has to make sure that the tails in real and Fourier space
are properly included.


\keydefinition{state\_lower\_limit}{control.in}
{ \noindent
  Usage: \keyword{state\_lower\_limit} \option{value} \\[1.0ex]
  Purpose: If set, specifies the lowest single-particle eigenstate to be included 
  for the quasiparticle calculation. \\[1.0ex]  
}
\emph{Note} that the \keyword{state\_lower\_limit} only makes sense if 
\keyword{qpe\_calc} is set, i.e., the post-processing-type self-energy calculation 
is required.

\keydefinition{state\_upper\_limit}{control.in}
{ \noindent
  Usage: \keyword{state\_upper\_limit} \option{value} \\[1.0ex]
  Purpose: If set, specifies the highest single-particle eigenstate to be included 
  for the quasiparticle calculation. \\[1.0ex]  
}
\emph{Note} that the \keyword{state\_upper\_limit} only makes sense if 
\keyword{qpe\_calc} is set, i.e., the post-processing-type self-energy calculation 
is required.

\keydefinition{time\_points}{control.in}
{ \noindent
  Usage: \keyword{time\_points} \option{value} \\[1.0ex]
  Purpose: If set, specifies the number (imaginary) time points for the self-energy
  calculation.\\[1.0ex]
  Default: \option{value}=80. \\[1.0ex]
}
\emph{Note} that the \keyword{frequency\_points} only makes sense if
\keyword{qpe\_calc} or \keyword{sc\_self\_energy} is set, i.e., the post-processing-type self-energy calculation
is required.

\keydefinition{use\_logsbt}{control.in}
{ \noindent
     Usage: \keyword{use\_logsbt} \option{bool} \\[1.0ex] 
     Purpose: If set, the two-center integrals are calculated by
     one-dimensional integrations in Fourier-space. In the case of
     \keyword{RI\_method} \texttt{LVL}, also the three-center
     integrals are computed by this method.\\[1.0ex]
     Default: \texttt{.true.} \\
}
\keyword{use\_logsbt} \texttt{.true.} is faster and more accurate than
the alternative.

The algorithm for the overlap and Coulomb integrals is described by Talman in
\cite{Talman03-MCI}.  It uses an efficient spherical Bessel transform on a
logarithmic radial grid
\cite{Talman09-numSBT,Hamilton00-fftlog,Hamilton00-fftlog-www} to obtain the
Fourier transform of the auxiliary basis functions and calculates the
integrals in Fourier space.



\keydefinition{use\_ovlp\_swap}{control.in}
{ \noindent
     Usage: \keyword{use\_ovlp\_swap} \\[1.0ex] 
     Purpose: if set, the atomic orbital (AO) based 3-index overlap matrix 
     (``ovlp\_3fn" in the source code) is written to the disk before transforming 
     to the molecular orbital (MO) based 3-index overlap matrix (``O\_2bs1HF" for 
     HF calculations and ``ovlp\_3KS" for $GW$ calculations in the source code).
     This avoids the double allocation of both the AO-based 3-index integral matrix 
     and MO-based 3-index integral matrix at the same time and thus reduces the
     memory cost by about a factor of two.
}
\newpage

\subsection*{Subtags for \emph{species} tag in \texttt{control.in}:}

\subkeydefinition{species}{aux\_gaussian}{control.in}
{
  \noindent
  Usage: \subkeyword{species}{aux\_gaussian} \option{L} \option{N}
    \option{[alpha]} \\
    \hspace*{0.1\textwidth} [ alpha\_1  coeff\_1 ] \\
    \hspace*{0.1\textwidth} [ alpha\_2  coeff\_2 ] \\
    \hspace*{0.1\textwidth} [ ... ] \\
    \hspace*{0.1\textwidth} [ alpha\_N  coeff\_N ]
    \\[1.0ex]
  Purpose: For \keyword{auxil\_basis} \texttt{opt}, adds a Gaussian
    basis function to the auxiliary basis set for the Coulomb
    operator. \\[1.0ex]
  \option{L} is an integer number, specifying the angular momentum \\ 
  \option{N} is an integer number, specifying how many primitive Gaussians
    comprise the present radial function \\
  \option{alpha} : \emph{If} \option{N}=1, this is the exponent
    defining a primitive Gaussian function [in bohr$^{-2}$]. \\
  \option{alpha\_i} \option{coeff\_i} : 
    \emph{If} \option{N}$>$1, $i=1,\dots,N$ additional lines specify
    exponents $\alpha_i$ and expansion coefficients $g_i$ for a
    non-primitive linear combination of Gaussians.\\
}
See the description of \subkeyword{species}{gaussian} basis functions;
Gaussian basis functions in the \emph{auxiliary} basis use essentially
the same infrastructure.

\subkeydefinition{species}{for\_aux}{control.in}
{
 \noindent
  Usage: \subkeyword{species}{for\_aux} \option{basis} \option{options}    
  \\[1.0ex]
   Purpose: Add a extra basis function to constructor of auxiliary basis function.
  \\[1.0ex] 
   \option{Basis} is a either \subkeyword{species}{hydro} or \subkeyword{species}{ionic} 
    basis function keyword and \option{options} are options for that basis function.
  \\[1.0ex] 
}
Adds extra basis function ONLY to the construction of the auxiliary
basis set used to expand the Coulomb operator (resolution of identity,
see keyword \keyword{RI\_method}). In particular, the accuracy of 
\keyword{RI\_method} \texttt{LVL} can be increased by adding extra
high-angular momentum radial functions to the auxiliary basis set. The
improvement becomes less and less relevant as the orbital basis set
itself increases. For instance, there may be a noticeable change for
\emph{tier 1}, but much less or not at all for \emph{tier 2}. 
Currently supports only \subkeyword{species}{hydro}  and
\subkeyword{species}{ionic} basis functions. 

For a detailed description with benchmarks, please see
Ref. \cite{Ihrig2015}. 

Here is an example for the use of the \subkeyword{species}{for\_aux},
which was obtained by altering the ``light'' settings of the C atom:

\small
\begin{verbatim}
[...]
#  "First tier" - improvements: -1214.57 meV to -155.61 meV
     hydro 2 p 1.7
     hydro 3 d 6
     hydro 2 s 4.9
#  "Second tier" - improvements: -67.75 meV to -5.23 meV
  for_aux     hydro 4 f 9.8
#     hydro 3 p 5.2
#     hydro 3 s 4.3
  for_aux     hydro 5 g 14.4
#     hydro 3 d 6.2
[...]
\end{verbatim}
\normalsize

Adding those two high-angular momentum functions does improve the
quality of the \keyword{RI\_method} \texttt{LVL} noticeably, at the
price of more CPU time. On the other hand ... ``light'' settings are
specifically chosen because not everything is completely converged. In
fact, the orbital basis set error itself may be larger than the
accuracy gained by amending the two-electron Coulomb operator
expansion. 

\subkeydefinition{species}{prodbas\_acc}{control.in}
{
  \noindent
  Usage: \subkeyword{species}{prodbas\_acc} \option{threshold} \\[1.0ex]
  Purpose: Technical cutoff criterion for on-site orthonormalization
    of auxiliary radial functions. Here \option{threshold} is a small 
    positive real value. Default: See below. \\
}
To construct the set of auxiliary basis functions, the radial functions for 
a single species are ``on-site" Gram-Schmidt orthonormalized. If 
 the norm of the function after
  orthonormalization is smaller than \option{threshold}, that function
  is omitted.

The present default values are:
\begin{itemize}
  \item 10$^{-2}$ for $Z \le$10 (light elements)
  \item 10$^{-3}$ for 10$< Z \le$18
  \item 10$^{-4}$ for $Z >$18 (all heavier elements)
\end{itemize}
The old default (version 042811 and earlier) was simply 10$^{-2}$ for
all elements. For light elements, this setting produces accurate total
energies and energy differences to the sub-meV level in all our
tests. For heavier elements, significant inaccuracies could happen in
atomic total energies. These inaccuracies would cancel out in energy
differences; to guarantee total energy accuracy as well, we now set
significantly tighter defaults for \subkeyword{species}{prodbas\_acc}
in this range (alas, also more expensive, both in time and memory use).

For simplicity, it is also possible to use
\keyword{default\_prodbas\_acc} to set a global value of
\subkeyword{species}{prodbas\_acc} across all elements. 

\emph{Note} that the \subkeyword{species}{prodbas\_acc} should not be confused with
the \keyword{prodbas\_threshold}. The former is used when constructing the
auxiliary basis functions for each species, whereas the latter is used to 
deal with the ill-conditioning behavior of the Coulomb repulsion and/or 
the overlap matrix between the set of auxiliary basis functions 
for the whole systems.

\subkeydefinition{species}{max\_n\_prodbas}{control.in}
{
  \noindent
  Usage: \subkeyword{species}{max\_n\_prodbas} \option{value} \\[1.0ex]
  Purpose: Specifies the maximal principal quantum number for the \textit{regular}
  basis function to be included in the \textit{auxiliary (product)} basis construction. 
  \\[1.0ex]
  \option{value} is a positive integer number here.
} 

\emph{Note} that \subkeyword{species}{max\_n\_prodbas} has an effect only when
\keyword{auxil\_basis} is setted to \texttt{full}.

\subkeydefinition{species}{max\_l\_prodbas}{control.in}
{
  \noindent
  Usage: \subkeyword{species}{max\_l\_prodbas} \option{value} \\[1.0ex]
  Purpose: Specifies the maximal angular quantum number for the
  \textit{auxiliary (product)} basis function. Any possible auxiliary basis
  with an angular momentum higher than \subkeyword{species}{max\_l\_prodbas}
  is excluded.  \\[1.0ex]
  \option{value} is a positive integer number here.
}

\emph{Note} that \subkeyword{species}{max\_l\_prodbas} controls the 
 \textit{auxiliary} basis whereas \subkeyword{species}{max\_n\_prodbas} 
  controls only directly the \textit{regular} basis. \subkeyword{species}{max\_l\_prodbas} 
 has an effect regardless whether \keyword{auxil\_basis} is setted to \texttt{full}
 or \texttt{opt}.


