\chapter{Compiler- / platform-specific installation parts}
\label{appendix_compilers}

All that remains is to build a FHI-aims binary from the
source code. This part depends on your specific compiler,
and thus on the selected platform (we have tested Linux, AIX,
MacOS, and BlueGene so far). 

We have tested a number of compilers, but there are many out there. Please
notify us if you are successful with an install using a new compiler /
compiler version.

For efficiency, you should provide BLAS / LAPACK installations that go along
with your compiler. When building a parallel version, you must also specify
the location of your mpi compiler, and set the variable ``MPIF'' (location of
MPI header files) to be empty (the MPI compiler finds this automatically).

You may also leave the BLAS / LAPACK specifications away to obtain a reference
version of FHI-aims without \emph{any} external dependencies. This is highly
useful for testing, but the performance for real (large) systems will be very
slow. Don't consider this for production if you can.

\section{All architectures: GNU based compilers}

Wouldn't it be nice if there were one compiler which supported all possible
platforms -- and we didn't have to worry about different compiler flags,
compiler options, and -- most notably -- compiler \emph{bugs} on different
systems any more?

In the C / C++ world, this has been achieved by the gcc (gnu compiler 
collection) free software suite of compilers, but for many years, a viable
free f90 compiler was lacking. As of now, this has fortunately
changed: There are now two viable options that I know of, both stemming from
the same root (and an unfortunate and highly unproductive split between two
disagreeing developer groups). 

\begin{itemize}
  \item \emph{gfortran} is the compiler which is officially shipped with gcc
    4.x. However, to my knowledge it remains buggy, but at the very
    least not easy to use (I was unable to install a working binary version at
    an early stage. Thus, I do not recommend this compiler as of now, but feel
    free to notify me if you think that I am wrong. 
  \item \emph{g95} can be found at www.g95.org ; a current working binary can
    always be installed and downloaded from there with ease. This is
    also a free product based on gcc, but unless you are exceptionally curious,
    I recommend to read the developer's blog and simply download the latest
    binary version, instead of trying to obtain the source. As of this writing,
    all current g95 versions simply work for me.
\end{itemize}

Finally: NOTE that g95 is not yet a production quality compiler. Its execution
times lag behind commercial compilers. For instance, FHI-aims is faster with 
Intel's ifort by a factor of $\approx$2 for me. But, g95 is a great
reference tool, and also provides some very useful debugging output when
needed. I recommend to test FHI-aims with g95 on every new platform, simply to
obtain a reference binary which works out of the box. 

\textbf{Installation instructions:}
\begin{itemize}
  \item Go to the source directory (\texttt{./src}). Type \texttt{make
    clean}. 
  \item Change to the directory \texttt{./src/external}; type \texttt{make
    -f Makefile.g95}. 
  \item Return to the \texttt{./src} directory and edit the file
    \texttt{Makefile}: uncomment the lines ``FC'', ``FFLAGS'', and
    ``FFLAGS\_NO'' for your selected compiler (g95) -- leave the other lines
    (``CBFLAGS'', ...) untouched!
  \item Type \texttt{make} ; all source code should be compiled this way, and
    a binary \texttt{aims.<version>.x} should be created in the
    \texttt{./bin} directory.
\end{itemize}

\section{x86 Linux: Intel Fortran (ifort)}

I have tested ifort version 8.1, 9.0, 9.1 on 32 bit and 64 bit Linux. 

\textbf{Installation instructions:}
\begin{itemize}
  \item Go to the source directory (\texttt{./src}). Type \texttt{make
    clean}. 
  \item Edit the file
    \texttt{Makefile}: uncomment the lines ``FC'', ``FFLAGS'', and
    ``FFLAGS\_NO'' for your selected compiler (ifort) -- leave the other lines
    (``CBFLAGS'', ...) untouched!
  \item Type \texttt{make} ; all source code should be compiled this way, and
    a binary \texttt{aims.<version>.x} should be created in the
    \texttt{./bin} directory.
\end{itemize}

\section{x86 Linux: Portland Group Fortran (pgf)}

I tested an evaluation version of the Portland Group compiler (pgf
version 5.2) at an early stage, but no longer have access to this compiler. At
the time, the resulting code was as fast as Intel's ifort, on a Pentium IV
CPU. 

\textbf{Installation instructions:}
\begin{itemize}
  \item Go to the source directory (\texttt{./src}). Type \texttt{make
    clean}. 
  \item Edit the file
    \texttt{Makefile}: uncomment the lines ``FC'', ``FFLAGS'', and
    ``FFLAGS\_NO'' for your selected compiler (pgf90) -- leave the other lines
    (``CBFLAGS'', ...) untouched!
  \item Type \texttt{make} ; all source code should be compiled this way, and
    a binary \texttt{aims.<version>.x} should be created in the
    \texttt{./bin} directory.
\end{itemize}

\section{Power architecture (MacOS/IBM): The xlf compiler}

The xlf compiler is the commercially available compiler for MacOS, and (I
understand) generally for IBM's Power-based architecture. I have only played
with a demo version of xlf on an Apple Powerbook G4; while I never produced
any fully working, code, this was likely due to an undetected bug which did
not affect either ifort or g95. That bug is since removed, hence I see no
reason why xlf should not work. 

\textbf{Installation instructions:}
\begin{itemize}
  \item Go to the source directory (\texttt{./src}). Type \texttt{make
    clean}. 
  \item Edit the file
    \texttt{Makefile}: uncomment the lines ``FC'', ``FFLAGS'', and
    ``FFLAGS\_NO'' for your selected compiler (xlf) -- leave the other lines
    (``CBFLAGS'', ...) untouched!
  \item Regarding ``FC'', be sure to replace the compiler path for f95 (which
    is given explicitly) with the valid compiler path on your platform. 
  \item Type \texttt{make} ; all source code should be compiled this way, and
    a binary \texttt{aims.<version>.x} should be created in the
    \texttt{./bin} directory.
\end{itemize}

