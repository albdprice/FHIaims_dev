\documentclass[11pt,letterpaper]{article}
\special{papersize=8.5in,11in}
%\renewcommand\baselinestretch{1.05}


%%%%%%%%%%%%%%%%%%%%%%%%%%%%%%%%%%%%
\pagestyle{plain}                   
%%%%%%%%%% EXACT 1in MARGINS %%%%%%%
\setlength{\textwidth}{6.5in}     %%
\setlength{\oddsidemargin}{0in}   %%
\setlength{\evensidemargin}{0in}  %%
\setlength{\textheight}{8.85in}   %%
\setlength{\topmargin}{0in}       %%
\setlength{\headheight}{0in}      %%
\setlength{\headsep}{0in}         %%
\setlength{\footskip}{.15in}      %%
%%%%%%%%%%%%%%%%%%%%%%%%%%%%%%%%%%%%
\renewcommand{\refname}{References\hfil}
%%%%%%%%%%%%%%%%%%%%%%%%%%%%%%%%%%%%

\usepackage{amsmath,amssymb}
\makeatletter
\g@addto@macro\normalsize{%
  \setlength\abovedisplayskip{0pt}
  \setlength\belowdisplayskip{0pt}
  \setlength\abovedisplayshortskip{0pt}
  \setlength\belowdisplayshortskip{0pt}
}
\makeatother
\usepackage{amsthm}
\usepackage[comma, square, numbers, sort&compress]{natbib}
\makeatletter
\newcommand*{\citensp}[2][]{%
  \begingroup
  \let\NAT@mbox=\mbox
  \let\@cite\NAT@citenum
  \let\NAT@space\NAT@spacechar
  \let\NAT@super@kern\relax
  \renewcommand\NAT@open{[}%
  \renewcommand\NAT@close{]}%
  \citep[#1]{#2}%
  \endgroup
}
\makeatother

\usepackage{latexsym}
\usepackage{indentfirst}
\usepackage{mathrsfs}
%\usepackage[notref, notcite]{showkeys}
\usepackage{graphicx}
\usepackage{wrapfig}
\usepackage{mdwlist}
%\usepackage{hyperref}
\usepackage[compact]{titlesec}
\titlespacing{\section}{1pt}{*1}{*1}
\titlespacing{\subsection}{0.5pt}{*0.5}{*0.5}
\usepackage{enumitem}
\setitemize{noitemsep,topsep=2pt,parsep=0pt,partopsep=0pt}
\setenumerate{noitemsep,topsep=2pt,parsep=0pt,partopsep=0pt}

%\usepackage{ulem}

\usepackage{xcolor}
\providecommand{\red}[1]{\textcolor{red}{\textit{#1}}}


\newcommand{\veps}{\varepsilon}
\newcommand{\eps}{\epsilon}

\newcommand{\BV}{\mathrm{BV}}
\newcommand{\CB}{\mathrm{CB}}
\newcommand{\DG}{\mathrm{DG}}
\newcommand{\KS}{\mathrm{KS}}
\newcommand{\FD}{\mathrm{FD}}
\newcommand{\eff}{\mathrm{eff}}

\newcommand{\abs}[1]{\lvert#1\rvert}
\newcommand{\norm}[1]{\lVert#1\rVert}
\newcommand{\average}[1]{\langle#1\rangle}
\newcommand{\bra}[1]{\langle#1\rvert}
\newcommand{\ket}[1]{\lvert#1\rangle}

\newcommand{\ud}{\,\mathrm{d}}
\newcommand{\RR}{\mathbb{R}}

\newcommand{\mc}[1]{\mathcal{#1}}
\newcommand{\ms}[1]{\mathscr{#1}}
\newcommand{\mf}[1]{\mathsf{#1}}

\newcommand{\per}{\mathrm{per}}

\newcommand{\xc}{\mathrm{xc}}
\newcommand{\ext}{\mathrm{ext}}

\newcommand{\TT}{\mathrm{T}}
\newcommand{\Or}{\mathcal{O}}

\newcommand{\boldV}{\mbox{\boldmath$V$}}
\newcommand{\boldW}{\mbox{\boldmath$W$}}
\newcommand{\boldw}{\mbox{\boldmath$w$}}

\newcommand{\bolda}{\mbox{\boldmath$a$}}
\newcommand{\boldf}{\mbox{\boldmath$f$}}
\newcommand{\boldR}{\mbox{\boldmath$R$}}
\newcommand{\boldT}{\mbox{\boldmath$T$}}
\newcommand{\boldPsi}{\mbox{\boldmath$\Psi$}}
\newcommand{\boldPhi}{\mbox{\boldmath$\Phi$}}
\newcommand{\boldchi}{\mbox{\boldmath$\chi$}}
\newcommand{\bolds}{\mbox{\boldmath$\sigma$}}
\newcommand{\boldF}{\mbox{\boldmath$F$}}
\newcommand{\boldH}{\mbox{\boldmath$H$}}
\newcommand{\boldS}{\mbox{\boldmath$S$}}
\newcommand{\boldA}{\mbox{\boldmath$A$}}
\newcommand{\boldB}{\mbox{\boldmath$B$}}
\newcommand{\boldL}{\mbox{\boldmath$L$}}
\newcommand{\boldQ}{\mbox{\boldmath$Q$}}
\newcommand{\boldI}{\mbox{\boldmath$I$}}
\newcommand{\boldc}{\mbox{\boldmath$c$}}
\newcommand{\boldlambda}{\mbox{\boldmath$\lambda$}}
\newcommand{\boldrho}{\mbox{\boldmath$\rho$}}
\newcommand{\FHIaims}{\texttt{FHI-aims}}
\newcommand{\erf}{\mathop{\mathrm{erf}}}
\newcommand{\erfc}{\mathop{\mathrm{erfc}}}
\newcommand{\at}{\text{at}}

\begin{document}

\begin{flushleft} 
\noindent\Large\textbf{FHI-aims File Format Description: geometry.in}
\end{flushleft}

\section{Overview}

FHI-aims requires two input files, which should contain all
information needed to run a first-principles calculation. 

All technical settings should be specified in a single file
\texttt{control.in}. 

A second input file, \texttt{geometry.in},
specifies the system geometry (placement and type of
atoms, unit cell vectors, etc.) and any information that is linked
directly to the system geometry, to the system's physical environment,
or to specific atoms. No other technical information should be part of 
\texttt{geometry.in}. In particular, $k$-space related information is
provided to the code in a generic format as a part of
\texttt{control.in}, in fractional coordinates of the reciprocal space
vectors, not of \texttt{geometry.in}. 

In its simplest form, \texttt{geometry.in} is a generic format
specification for files conveying the atomic structure of molecules or
solids. This basic form is thus functionally equivalent to
other, code-agnostic file specifications containing atomic structure
information, such as .xyz, .cif, or .pdb . Like these other file
formats, \texttt{geometry.in} can contain technical information beyond
the actual atomic structure. 

The present document focuses on aspects related to the basic atomic
structure information (including information on specified initial spin
moments, which may result in physically different outcomes for the
structure in question). A full listing of all keywords related to
\texttt{geometry.in} may be found in the FHI-aims code manual.

\section{Examples}

Any molecular geometry in FHI-aims requires only cartesian atomic
coordinates of the position of a nucleus (in {\AA}) and of the atomic
species (usually, the chemical element) found at that position. No
other information is required. Any \texttt{geometry.in} file that does
not include a unit cell specification is automatically considered to
correspond to the definition of an isolated molecule.

The \texttt{geometry.in} file for an isolated N$_2$ molecule might look like
this:

\begin{verbatim}
  atom 0. 0. 0.      N
  atom 0. 0. 1.0976  N
\end{verbatim}
 
Periodic systems are defined by specifying three lattice repeat vectors
in cartesian units ({\AA}). The \texttt{geometry.in} file for the
primitive two-atom cell of a GaAs periodic crystal might look like
this: 

\begin{verbatim}
  lattice_vector  2.826650         2.826650         0.000000
  lattice_vector  0.000000         2.826650         2.826650
  lattice_vector  2.826650         0.000000         2.826650

  atom  0.000000         0.000000         0.000000  Ga
  atom  1.413325         1.413325         1.413325  As
\end{verbatim}

The previous version specifies the cartesian atomic coordinates in
{\AA} (\texttt{atom} keyword). Alternatively, a specification in 
fractional coordinates (units of the lattice vectors) is possible as
well: 

\begin{verbatim}
  lattice_vector  2.826650         2.826650         0.000000
  lattice_vector  0.000000         2.826650         2.826650
  lattice_vector  2.826650         0.000000         2.826650

  atom_frac  0.               0.               0.        Ga
  atom_frac  0.25             0.25             0.25      As
\end{verbatim}

The relation between lattice vectors $\{\bolda_i\}$, cartesian
atomic coordinates $\{\boldR_I\}$ of atom $I$ and fractional atomic
coordinates $\{\boldf_I\}$ is
\begin{equation}
  R_{I,j} = \sum_k f_{I,k} a_{k,j}
\end{equation}
The index $j$ denotes the cartesian coordinates of the $k$th lattice
vector in the order provided in \texttt{geometry.in}.

\section{Format and Keywords}

In \texttt{geometry.in}, whitespace separates keywords and their
values. Empty lines are ignored and comment lines can be inserted
starting with the '\#' character. The order of lines is generally
arbitrary, with the exception of keywords that pertain to a specific
atom. Keywords that pertain to a specific atom (such as an
\texttt{initial\_moment} spin moment specification or a constraint on
the atomic position) must be specified after the \texttt{atom} or
\texttt{atom\_frac} line of that atom, and before the \texttt{atom} or
\texttt{atom\_frac} line indicating the next atom. \\[1.0ex]

\noindent
\textbf{Keywords descriptions}: \\
(restricted to the keywords that do not refer to code-specific
technical features)
\\[1.0ex]

\texttt{atom} \\[1.0ex]
{
  \noindent
  Usage: \texttt{atom} \texttt{x} \texttt{y} \texttt{z}
  \texttt{species\_name} \\[1.0ex]
  Purpose: Specifies the initial location and type of an atom. \\[1.0ex]
  \texttt{x}, \texttt{y}, \texttt{z} are real numbers (in \AA) which
  specify the atomic position. \\[1.0ex]
  \texttt{species\_name} is a string descriptor which names the element on
    this atomic position; it must match with one of the species descriptions
    given in \texttt{control.in}. \\[1.0ex]
}

\texttt{atom\_frac} \\[1.0ex]
{
  \noindent
  Usage: \texttt{atom\_frac} \texttt{$n_1$} \texttt{$n_2$} \texttt{$n_3$}
  \texttt{species\_name} \\[1.0ex]
  Purpose: Specifies the initial location and type of an atom in fractional coordinates. \\[1.0ex]
  \texttt{$n_i$} is a real multiple of lattice vector $i$.
  \texttt{species\_name} is a string descriptor which names the element on
    this atomic position; it must match with one of the species descriptions
    given in \texttt{control.in}. \\[1.0ex]
}
Fractional coordinates are only meaningful in periodic calculations. \\[1.0ex]

\texttt{lattice\_vector} \\[1.0ex]
{
  \noindent
  Usage: \texttt{lattice\_vector} \texttt{x} \texttt{y} \texttt{z} \\[1.0ex]
  Purpose: Specifies one lattice vector for periodic boundary conditions. \\[1.0ex]
  \texttt{x}, \texttt{y}, \texttt{z} are real numbers (in \AA) which
  specify the direction and length of a unit cell vector. \\[1.0ex]
}
If up to three lattice vectors are specified, FHI-aims automatically assumes
periodic boundary conditions in those directions. \emph{Note} that the
order of lattice vectors matters, as the order of $k$ space divisions (given
in \texttt{control.in}) depends on it! \\[1.0ex]

\texttt{initial\_moment} \\[1.0ex]
{
  \noindent
  Usage: \texttt{initial\_moment} \texttt{moment} \\[1.0ex]
  Purpose: Allows to place an initial spin moment on an \texttt{atom} in
    file \texttt{geometry.in}. \\[1.0ex]
  \texttt{moment} is a real number, referring to the electron
  difference $N^\uparrow - N^\downarrow$ on that site. Default: Zero, 
  unless \texttt{default\_initial\_moment} is set explicitly. \\[1.0ex]
}
The \texttt{initial\_moment} keyword always applies to the immediately preceding
\texttt{atom} specified in input file
\texttt{geometry.in}. The moment is introduced by using a
spin-polarized instead of an unpolarized spherical free-atom density
on that site in the initial superposition-of-free-atoms density. Note
that initial charge densities are generated by the functional
  specified with \texttt{xc} 
for DFT-LDA/GGA, but refer to \texttt{pw-lda} densities for all other
functionals (hybrid functionals, Hartree-Fock, ...).


\end{document}
